Thus far we have used the Coq type \li{nat}, which represents
a number~$n$ using~$n$ bits.
In contrast, the binary system represents~$n$ in $\lfloor \log_{2} n \rfloor + 1$ bits.
Coq comes with a built-in binary type \li{N}, 
which consists
of constructors \li{N0} and \li{Npos}. The latter, \li{Npos}, unfolds to \li{positive}:

\begin{lstlisting}
Inductive positive : Set :=
  | xI : positive -> positive 
  | xO : positive -> positive
  | xH : positive.
\end{lstlisting}

Constructor \li{xH} represents $1$, and constructors \li{xO} and \li{xI} represent
appending $0$ and $1$ respectively.
By always starting with $1$, \li{positive} bypasses
the issue of disambiguating \emph{e.g.} the strings \li{011} and
\li{00011}, which represent the same number but pose
a minor technical challenge.
To represent $0$, the type \li{N} simply uses the separate constructor \li{N0}.

In both \li{nat} and \li{N}, addition/subtraction of $b$-bit
numbers is $\Theta(b)$, while multiplication is $\Theta \big(b^2\big)$.
In general, arithmetic operations are often faster when the inputs
are encoded in binary. 
In this section we show that this advantage also extends to our techniques.

Our codebase has binary versions of
	\href{https://github.com/inv-ack/inv-ack/blob/7270e64a2600b771f2b1b1b151f7d13fb2ae6c97/bin_repeater.v\#L78-L87}{\color{blue}hyperoperations},
	\href{https://github.com/inv-ack/inv-ack/blob/7270e64a2600b771f2b1b1b151f7d13fb2ae6c97/bin_applications.v\#L30-L36}{\color{blue}inverse hyperoperations},
	\href{https://github.com/inv-ack/inv-ack/blob/7270e64a2600b771f2b1b1b151f7d13fb2ae6c97/bin_repeater.v\#L157-L175}{\color{blue}Ackermann}, and
	\href{https://github.com/inv-ack/inv-ack/blob/7270e64a2600b771f2b1b1b151f7d13fb2ae6c97/bin_inv_ack.v\#L335-L342}{\color{blue}inverse Ackermann}.
Here we show how to compute inverse Ackermann for binary inputs in
$\Theta(b)$ time, where $b$ is the bitlength,
\emph{i.e.} logarithmic time in the input magnitude.
As before, we present an intuitive sketch here and put
full proofs in 
Appendix~D
 of the extended version of this paper~\cite{extendedinvack}.
% Appendix~\ref{apx:time_analysis_bin}.

\begin{rem}
Although we do not prove it here, our general binary inverse hyperoperations are $O(b^2)$ time,
since Ackermann and base-2 hyperoperations benefit from $\Theta(1)$ division via bitshifts,
whereas general division is $O(b^2)$.
\end{rem}

\subsection{Countdown and contractions in binary}

\renewcommand{\Tleb}{\runtime_{\li{N.leb}}}
\renewcommand{\Tsucc}{\runtime_{\li{N.succ}}}

Although the theoretical \emph{countdown} is independent of the encoding
of its inputs, its Coq definition needs to be adjusted to allow for inputs
in \li{N}. The first step is to translate the arguments of
\li{countdown\_worker} from \li{nat} to \li{N}. Budget \li{b} must
remain in \li{nat} so it can serve as Coq's termination argument,
but all other \li{nat} arguments should be changed
to \li{N}, and functions on \li{nat} to functions on \li{N}.
% Linked by A
%\begin{lstlisting}
%`\href{https://github.com/inv-ack/inv-ack/blob/7270e64a2600b771f2b1b1b151f7d13fb2ae6c97/bin_countdown.v#L104-L109}{Fixpoint bin\_cdn\_wkr}` (f : N -> N) (a n : N) (b : nat) : N :=
%  match b with O => 0 | S b' =>
%    if (n <=? a) then 0 else 1 + bin_cdn_wkr f a (f n) b'
%  end.
%\end{lstlisting}
% Edited by Linh
\begin{lstlisting}
`\href{https://github.com/inv-ack/inv-ack/blob/7270e64a2600b771f2b1b1b151f7d13fb2ae6c97/bin_countdown.v#L104-L109}{\color{blue}Fixpoint bin\_cdn\_wkr}` f a n b : N :=
  match b with O => 0 | S b' =>
    if (n <=? a) then 0
      else 1 + bin_cdn_wkr f a (f n) b'
  end.
\end{lstlisting}

Determining the budget for \li{bin\_countdown\_to} is tricky.
A naïve approach is to use the built-in \li{nat} translation of \li{n},
\emph{i.e.} \li{N.to_nat n}. This is untenable as the translation alone
takes exponential time \emph{viz} the length of $n$'s representation.
We need a linear-time budget calculation for countdowns
of oft-used functions like $\lambda n.(n-2)$.

The key is to focus on functions that can bring their arguments below a threshold via repeated application in
\emph{logarithmic} time, thus allowing a log-sized budget for
\li{bin\_cdn\_wkr}. Simply shrinking by~$1$ is no longer good enough;
we need to halve the argument on every application as shown below:
\begin{defn} \label{defn: bin-contraction}
	$f\in \contract$ is
	\href{https://github.com/inv-ack/inv-ack/blob/7270e64a2600b771f2b1b1b151f7d13fb2ae6c97/bin_countdown.v#L37-L41}{\color{blue}\emph{binary strict above}}
	$a\in \mathbb{N}$ if $\forall n > a, f(n) \le \lfloor \frac{n + a}{2} \rfloor$.
\end{defn}
\noindent The key advantage of binary strict contractions is that if a contraction $f$ is binary strict above some $a$, then \lb
we know that $\forall n > a, \forall k.~f(n) \le \left\lfloor \frac{n - a}{2^k} \right\rfloor + a$.
Therefore, within $\lfloor \log_2 (n - a) \rfloor + 1$ applications of $f$, the result will become equal to or less than~$a$. We can choose this number as the budget for \li{bin\_cdn\_wkr} to successfully reach the countdown value before terminating.
Note that this budget is simply the length of the binary representation
of \li{n - a}, which we calculate using our function
\href{https://github.com/inv-ack/inv-ack/blob/7270e64a2600b771f2b1b1b151f7d13fb2ae6c97/bin_prelims.v#L135-L143}{\color{blue}\li{nat\_size}}. % Linked by A
The Coq definition of countdown on \li{N} is:
\hide{
\begin{lstlisting}
`\href{https://github.com/inv-ack/inv-ack/blob/7270e64a2600b771f2b1b1b151f7d13fb2ae6c97/bin_prelims.v#L135-L143}{\color{blue}Definition nat\_size}` (n : N) : nat :=
  match n with
  | 0 => 0%nat
  | Npos p => let fix nat_pos_size (x : positive) : nat :=
                  match x with
                  | xH => 1%nat
                  | xI y | xO y => S (nat_pos_size y)
                  end
              in nat_pos_size p
  end.
\end{lstlisting}
Note that \li{nat_size} outputs $0$ on $0$, and on any positive number $m$ the size of its binary representation, thus equals to $\lfloor \log_{2} m \rfloor + 1$.
}% end hide
% Linked by A
%\begin{lstlisting}
%`\href{https://github.com/inv-ack/inv-ack/blob/7270e64a2600b771f2b1b1b151f7d13fb2ae6c97/bin_countdown.v#L111-L112}{Definition bin\_countdown\_to}` (f : N -> N) (a n : N) : N :=
%  bin_cdn_wkr f a n (nat_size (n - a)).
%\end{lstlisting}
% Edited by Linh
\begin{lstlisting}
`\href{https://github.com/inv-ack/inv-ack/blob/7270e64a2600b771f2b1b1b151f7d13fb2ae6c97/bin_countdown.v#L111-L112}{\color{blue}Definition bin\_countdown\_to}` f a n :=
  bin_cdn_wkr f a n (nat_size (n - a)).
\end{lstlisting}

\noindent The following is the binary version of Lemma~\ref{lem: cdt-runtime}:
%\begin{lem} \label{lem: cdt-runtime-bin}
%	$\forall n \in \li{N}$, if $f$ is a binary strict contraction above $a$,
%	\begin{equation*}
%	\runtime_{\cdt{f}{a}}(n) \le \sum_{i=0}^{\cdt{f}{a}(n) - 1} \hspace{-6pt}
%	\runtime_f\big(f^{(i)}(n)\big) \ + \ (\log_2a + 3)\left(\cdt{f}{a}(n) + 1\right) \ + \ 2\log_2n \ + \ \log_2\cdt{f}{a}(n)
%	\end{equation*}
%\end{lem}
% Edited by Linh
\begin{lem} \label{lem: cdt-runtime-bin}
	$\forall n \in \li{N} \ \forall a\in \li{N}$, if $f$ is a binary strict contraction above $a$,
	\begin{equation*}
	\begin{aligned}
	\runtime_{\cdt{f}{a}}(n) \ \le \ \sum_{i=0}^{\cdt{f}{a}(n) - 1} \hspace{-6pt}
	\runtime_f\big(f^{(i)}(n)\big) \ & + \ (\log_2a + 3)\left(\cdt{f}{a}(n) + 1\right) \\ 
  & + \ 2\log_2n \ + \ \log_2\cdt{f}{a}(n)
	\end{aligned}
	\end{equation*}
\end{lem}

%Substituting $a=1$ into \eqref{eq: cdt-runtime-bin} shows that \Cref{lem: inv-ack-hier-runtime} still holds.
%Similar to \Cref{sect: hardcode-lvl2}, the use of binaries is not immediately effective since the first level. We delve deeper into the hierarchy by \emph{hardcoding the $3^{\text{th}}$ level} and starts from there. Now
%\begin{equation*}
%\forall n, n+2  < n+3 < 2(n+2) \iff \forall n,
%\lfloor \log_2(n+2) \rfloor < \lceil \log_2(n+3) \rceil \le \lfloor \log_2(n+2) \rfloor + 1
%\end{equation*}
%This shift from floor to ceiling enables a direct computation, since $\lceil \log_2(n+3) \rceil$ can now be seen as the size of $(n+2)$'s binary representation.
%\begin{lstlisting}
%Definition alpha3 (n: N) : N := N.size (n+2) - 3.
%\end{lstlisting}
%Let $n\ge 2$. The computation of \li{N.size(n)} takes time equal to itself, so the above definition gives $\runtime(\alpha_3, n) \le 2\log_2n$. Fix an $i\ge 3$ and suppose $\runtime(\alpha_i, n) \le C_i\log_2n$. By \Cref{lem: inv-ack-hier-runtime},
\subsection{Inverse Ackermann in $\bigO\left(\log_2 n\right)$}
Our new Coq definition computes countdown
only for strict binary contractions. Fortunately, starting
from $n = 2$, the inverse hyperoperations $a\angle{n}b$ when $a\ge 2$
and the inverse Ackermann hierarchy $\alpha_n$ are all strict binary contractions.
We can construct these hierarchies by hardcoding their
first three levels and recursively building higher levels with \li{bin\_countdown\_to}.
Furthermore,
analagously to the optimization for \li{nat} discussed in~\S\ref{sect: hardcode-lvl2}, we hardcode an additional level.
% Linked by A
%\begin{lstlisting}
%`\href{https://github.com/inv-ack/inv-ack/blob/7270e64a2600b771f2b1b1b151f7d13fb2ae6c97/bin_inv_ack.v#L55-L60}{Fixpoint bin\_alpha}` (m : nat) (x : N) : N :=
%  match m with
%  | 0%nat => x - 1          | 1%nat => x - 2
%  | 2%nat => N.div2 (x - 2) | 3%nat => N.log2 (x + 2) - 2
%  | S m'  => bin_countdown_to (bin_alpha m') 1 (bin_alpha m' x)
%  end.
%\end{lstlisting}
% Edited by Linh
\begin{lstlisting}
`\href{https://github.com/inv-ack/inv-ack/blob/7270e64a2600b771f2b1b1b151f7d13fb2ae6c97/bin_inv_ack.v#L55-L60}{\color{blue}Fixpoint bin\_alpha}` (m : nat) (x : N) : N :=
  match m with
  | 0%nat => x - 1          
  | 1%nat => x - 2
  | 2%nat => N.div2 (x - 2) 
  | 3%nat => N.log2 (x + 2) - 2
  | S m'  => bin_countdown_to
               (bin_alpha m') 1 (bin_alpha m' x)
  end.
\end{lstlisting}
Note that for all $x$, $\li{N.div2}(x - 2) = \left\lfloor \frac{x - 2}{2} \right\rfloor = \left\lceil \frac{x - 3}{2} \right\rceil$ and $\li{N.log2}(x + 2) - 2 = \left\lfloor \log_2(x+2) \right\rfloor - 2 = \left\lceil \log_2(x+3) \right\rceil - 3$, so the above Coq definition is correct.

%\begin{thm} \label{thm: inv-ack-runtime-bin}
%	$\forall i,~\forall n,~\runtime_{\alpha_i}(n) \le 2\log_2n + \left(3\cdot 2^i - 3i - 13\right)\log_2\log_2n + 3i$.
%\end{thm}
% Edited by Linh
\begin{thm} \label{thm: inv-ack-runtime-bin}
	% For all $i$ and $n$,
	\begin{equation*}
	\forall i,n.~\runtime_{\alpha_i}(n) \le 2\log_2n + \left(3\cdot 2^i - 3i - 13\right)\log_2\log_2n + 3i.
	\end{equation*}
\end{thm}

\noindent For any level of the Ackermann hierarchy, this theorem demonstrates
a linear computation time up to the size of the representation of the input, \emph{i.e.} logarithmic time up to its magnitude $n$:
$\runtime_{\alpha_i}(n) = \bigO\big(\log_2n + 2^i\log_2\log_2n \big)$.

Moving on to inverse Ackermann itself, we follow a style nearly identical to that
in~\S\ref{subsec: inv_ack_hier}. For the worker, we simply translate to
\li{N}, keeping the budget in \li{nat} as described earlier.
The inverse Ackermann has an extra hardcoded level.
% Linked by A
%\begin{lstlisting}
%`\href{https://github.com/inv-ack/inv-ack/blob/7270e64a2600b771f2b1b1b151f7d13fb2ae6c97/bin_inv_ack.v#L326-L333}{Fixpoint bin\_inv\_ack\_wkr}` (f : N -> N) (n k : N) (b : nat) : N :=
%  match b with 0%nat  => k | S b' => if n <=? k then k else
%    let g := (bin_countdown_to f 1) in
%      bin_inv_ack_wkr (compose g f) (g n) (N.succ k) b'
%  end.
%
%`\href{https://github.com/inv-ack/inv-ack/blob/7270e64a2600b771f2b1b1b151f7d13fb2ae6c97/bin_inv_ack.v#L335-L342}{Definition bin\_inv\_ack}` (n : N) : N :=
%  if (n <=? 1) then 0 else if (n <=? 3) then 1
%    else if (n <=? 7) then 2 else
%      let f := (fun x => N.log2 (x + 2) - 2) in
%        bin_inv_ack_wkr f (f n) 3 (nat_size n).
%\end{lstlisting}
% Edited by Linh
\begin{lstlisting}
`\href{https://github.com/inv-ack/inv-ack/blob/7270e64a2600b771f2b1b1b151f7d13fb2ae6c97/bin_inv_ack.v#L326-L333}{\color{blue}Fixpoint bin\_inv\_ack\_wkr}` f n k b :=
  match b with 0%nat => k | S b' =>
    if n <=? k then k else
      let g := (bin_countdown_to f 1) in
        bin_inv_ack_wkr
          (compose g f) (g n) (N.succ k) b'
  end.

`\href{https://github.com/inv-ack/inv-ack/blob/7270e64a2600b771f2b1b1b151f7d13fb2ae6c97/bin_inv_ack.v#L335-L342}{\color{blue}Definition bin\_inv\_ack}` n :=
  if (n <=? 1) then 0 else if (n <=? 3) then 1
    else if (n <=? 7) then 2 else
      let f := (fun x => N.log2 (x + 2) - 2) in
        bin_inv_ack_wkr f (f n) 3 (nat_size n).
\end{lstlisting}
% Fixpoint bin_inv_ack_worker (f : N -> N) (n k : N) (b : nat) : N :=
%   match b with
%   | 0%nat  => k
%   | S b' => if n <=? k then k
%             else let g := (bin_countdown f 1) in
%                  bin_inv_ack_worker (compose g f) (g n) (N.succ k) b'
%   end.

% Definition bin_inv_ack (n : N) : N :=
%   if (n <=? 1) then 0
%   else if (n <=? 3) then 1
%        else if (n <=? 7) then 2
%             else let f := (bin_alpha 3) in
%                  bin_inv_ack_worker f (f n) 3 (nat_size n).
\noindent Note that, for $n > 7$, $n < \Ack\big(\lfloor \log_2n \rfloor + 1\big)$ $= \Ack\big(\li{nat_size}(n)\big)$, so a budget of $\li{nat_size}(n)$ suffices.
We show the
\href{https://github.com/inv-ack/inv-ack/blob/7270e64a2600b771f2b1b1b151f7d13fb2ae6c97/bin_inv_ack.v#L437-L472}
{\color{blue}correctness} and
\href{https://github.com/inv-ack/inv-ack/blob/195209ba895061fc51368c3c46c1d8760f05df50/bin_test_runtime_ocaml.ml#L359-L363}
{\color{blue}benchmark}
of \li{bin_inv_ack} in our codebase. Figure~\ref{fig:standalone_binary} 
shows a standalone binary-specific computation of the inverse 
Ackermann function that takes logarithmic 
time up to the input's magnitude.
This is simply an assimilation of the code snippets we have already discussed in 
this section, and serves as a 
binary translation of Figure~\ref{fig:standalone} shown earlier.

\begin{figure}
\lstset{style=myTinyStyle}
\begin{lstlisting}
Require Import Omega Program.Basics.
Open Scope N_scope.

`\href{https://github.com/inv-ack/inv-ack/blob/7270e64a2600b771f2b1b1b151f7d13fb2ae6c97/inv_ack_standalone.v#L56-L64}{\color{blue}Definition nat\_size}` n :=
  match n with
  | 0 => 0%nat
  | Npos p =>
      let fix nat_pos_size (x : positive) : nat :=
        match x with xH => 1%nat
        | xI y | xO y => S (nat_pos_size y) end
      in nat_pos_size p
  end.

`\href{https://github.com/inv-ack/inv-ack/blob/7270e64a2600b771f2b1b1b151f7d13fb2ae6c97/bin_countdown.v#L104-L109}{\color{blue}Fixpoint bin\_cdn\_wkr}` f a n b : N :=
  match b with O => 0 | S b' =>
    if (n <=? a) then 0
      else 1 + bin_cdn_wkr f a (f n) b'
  end.

`\href{https://github.com/inv-ack/inv-ack/blob/7270e64a2600b771f2b1b1b151f7d13fb2ae6c97/inv_ack_standalone.v#L75-L76}{\color{blue}Definition bin\_countdown\_to}` f a n := 
bin_cdn_wkr f a n (nat_size (n - a)).

`\href{https://github.com/inv-ack/inv-ack/blob/7270e64a2600b771f2b1b1b151f7d13fb2ae6c97/inv_ack_standalone.v#L97-L104}{\color{blue}Fixpoint bin\_inv\_ack\_wkr}` f n k b :=
  match b with 0%nat => k | S b' => 
    if n <=? k then k else 
      let g := (bin_countdown_to f 1) in
        bin_inv_ack_wkr
          (compose g f) (g n) (N.succ k) b'
  end.

`\href{https://github.com/inv-ack/inv-ack/blob/7270e64a2600b771f2b1b1b151f7d13fb2ae6c97/inv_ack_standalone.v#L111-L116}{\color{blue}Definition bin\_inv\_ack}` n :=
  if (n <=? 1) then 0 else if (n <=? 3) then 1
    else if (n <=? 7) then 2 else 
      let f := (fun x => N.log2 (x + 2) - 2) in
        bin_inv_ack_wkr f (f n) 3 (nat_size n).
\end{lstlisting}
\caption{A log-time Coq computation of the inverse Ackermann function for inputs represented in binary, \emph{i.e.} \li{N}.}
\label{fig:standalone_binary}
\end{figure}
\lstset{style=myStyle}

As in Theorem~\ref{thm: inv-ack-hardcode-correct}, the time complexity $\runtime_\alpha(n)$ is the sum of each component's runtime:
%\begin{equation*}
%\runtime_\alpha(n) =
%\begin{aligned}
%& \ \runtime_{\alpha_3}(n)
%+ \sum_{k = 3}^{\alpha(n) - 1} \hspace{-5pt} \runtime_{\cdt{\alpha_k}{1}\ }(\alpha_k(n))
%+ \sum_{k = 3}^{\alpha(n)}\Tleb\left(\alpha_k(n), k\right)
%+ \sum_{k = 3}^{\alpha(n) - 1} \hspace{-5pt} \Tsucc(k)\\
%& \ + \ \runtime_{\li{nat_size}}(n)
%\ + \Tleb(n, 1) \ + \Tleb(n, 3) \ + \Tleb(n, 7)
%\end{aligned}
%\end{equation*}
% Edited by Linh
\begin{equation*}
\begin{aligned}
\runtime_{\alpha_3}(n)
& \ + \sum_{k = 3}^{\alpha(n) - 1} \hspace{-5pt} \runtime_{\cdt{\alpha_k}{1}\ }(\alpha_k(n))
+ \sum_{k = 3}^{\alpha(n)}\Tleb\left(\alpha_k(n), k\right) \\
& \ + \sum_{k = 3}^{\alpha(n) - 1} \hspace{-5pt} \Tsucc(k)
\ + \ \runtime_{\li{nat_size}}(n) \\
& \ + \Tleb(n, 1) \ + \ \Tleb(n, 3) \ + \ \Tleb(n, 7)
\end{aligned}
\end{equation*}
With reference to Lemmas~48 (Appendix~C), 53 and 54 (Appendix~D), we have:
% With reference to Lemmas \ref{lem: compose-runtime} (Appendix~C), \ref{lem: leb-runtime-bin} and \ref{lem: succ-runtime-bin} \lb (Appendix~D), we have:
In the second summand, \lb $\runtime_{\cdt{\alpha_k}{1}\ }(\alpha_k(n)) = \runtime_{\alpha_{k+1}}(n) - \runtime_{\alpha_k}(n)$ for each $k$ by Lemma~48.
% \ref{lem: compose-runtime}.
By Lemma~53, 
% By Lemma~\ref{lem: leb-runtime-bin}, 
each $\Tleb$ in the third summand is $\Theta\left(\log_2k\right)$, totalling $\bigO\big(\alpha(n)\log_2\alpha(n)\big) = \text{o}\big(\log_2n\big)$.
The fourth summand is $\Theta(\alpha(n)) = \text{o}\big(\log_2n\big)$ by 
Lemma~54. 
% Lemma~\ref{lem: succ-runtime-bin}. 
The remaining items total $\Theta(\log_2n)$. Thus, $\forall n\ge 8$:
%\begin{equation*}
%\begin{aligned}
%\runtime_\alpha(n)
%& = \runtime_{\alpha_3}(n)
%+ \sum_{k = 3}^{\alpha(n) - 1} \big(\runtime_{\alpha_{k+1}}(n) - \runtime_{\alpha_{k}}(n)\big) + \Theta\left(\log_2 n\right)
%= \runtime_{\alpha_{\alpha(n)}}(n) + \Theta\big(\log_2 n\big) \\
%& = \bigO\big(\log_2n + 2^{\alpha(n)}\log_2\log_2n \big) + \Theta\big(\log_2n\big)
%= \Theta\big(\log_2n\big)
%\end{aligned}
%\end{equation*}
% Edited by Linh
\begin{equation*}
\begin{aligned}
\runtime_\alpha(n)
& = \runtime_{\alpha_3}(n)
+ \sum_{k = 3}^{\alpha(n) - 1} \big(\runtime_{\alpha_{k+1}}(n) - \runtime_{\alpha_{k}}(n)\big) + \Theta\left(\log_2 n\right) \\
& = \runtime_{\alpha_{\alpha(n)}}(n) + \Theta\big(\log_2 n\big) \\
& = \bigO\big(\log_2n + 2^{\alpha(n)}\log_2\log_2n \big) + \Theta\big(\log_2n\big) \\
& = \Theta\big(\log_2n\big)
\end{aligned}
\end{equation*}

\renewcommand{\Tleb}{\runtime_{\li{leb}}}
\renewcommand{\Tsucc}{\runtime_{\li{succ}}}

%We bound the RHS by bounding $\runtime\left(\alpha_{\alpha(n)}(n)\right)$, and $\runtime\left(\alpha_i(n)\right)$ in general. Note that by $\alpha_2(n) = \left\lfloor \frac{n - 2}{2} \right\rfloor$, we have $\runtime(\alpha_2, n) \le 2\log_2 n$ for all $n$.
%Note that this bound can potentially be improved by further tightening the above inequalities. Although we do not obtain an exact asymptotic runtime similar
%to Theorem~\ref{thm: inv-ack-hier-runtime-improved}, since this bound of $4^{\alpha(n)}\log_2n$ is strictly larger than the lower bound of $\log_2n$, it is still extremely small and can be bounded by simpler expressions such as $(\log_2n)^2$.
%Our result is an improved version of the inverse Ackermann function which runs in time $\bigO\big(4^{\alpha(n)}\log_2n\big)$.
%\begin{equation*}
%\alpha(n) \triangleq \begin{cases}
%0 & \text{if } n\le 1\\ 1 & \text{if } 2\le n\le 3 \\ 2 & \text{if } 4\le n\le 7 \\
%\W\alpha\left(\alpha_3,
%\alpha_3(n), 3, n\right) & \text{if } n\ge 8
%\end{cases} \quad \text{ where } \alpha_3 \triangleq \lambda m. \big(\lfloor \log_2(m+2) \rfloor - 2\big)
%\end{equation*} 