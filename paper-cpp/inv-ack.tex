%\documentclass[runningheads]{llncs}
% \documentclass[sigplan,10pt,review,anonymous]{acmart}
% \settopmatter{printfolios=true,printccs=false,printacmref=false}

%% For double-blind review submission, w/ CCS and ACM Reference
%\documentclass[sigplan,review,anonymous]{acmart}\settopmatter{printfolios=true}
%% For single-blind review submission, w/o CCS and ACM Reference (max submission space)
%\documentclass[sigplan,review]{acmart}\settopmatter{printfolios=true,printccs=false,printacmref=false}
%% For single-blind review submission, w/ CCS and ACM Reference
%\documentclass[sigplan,review]{acmart}\settopmatter{printfolios=true}
%% For final camera-ready submission, w/ required CCS and ACM Reference
\documentclass[sigplan,screen]{acmart}


%%% If you see 'ACMUNKNOWN' in the 'setcopyright' statement below,
%%% please first submit your publishing-rights agreement with ACM (follow link on submission page).
%%% Then please update our instructions page and copy-and-paste the NEW commands into your article.
%%% Please contact us in case of questions; allow up to 10 min for the system to propagate the information.
%%%
%%% The following is specific to CPP '20 and the paper
%%% 'Inverting the Ackermann Hierarchy'
%%% by Linh Tran, Anshuman Mohan, and Aquinas Hobor.
%%%
\setcopyright{acmlicensed}
\acmPrice{15.00}
\acmDOI{10.1145/3372885.3373837}
\acmYear{2020}
\copyrightyear{2020}
\acmISBN{978-1-4503-7097-4/20/01}
\acmConference[CPP '20]{Proceedings of the 9th ACM SIGPLAN International Conference on Certified Programs and Proofs}{January 20--21, 2020}{New Orleans, LA, USA}
\acmBooktitle{Proceedings of the 9th ACM SIGPLAN International Conference on Certified Programs and Proofs (CPP '20), January 20--21, 2020, New Orleans, LA, USA}



%% Copyright information
%% Supplied to authors (based on authors' rights management selection;
%% see authors.acm.org) by publisher for camera-ready submission;
%% use 'none' for review submission.
% \setcopyright{none}
%\setcopyright{acmcopyright}
%\setcopyright{acmlicensed}
%\setcopyright{rightsretained}
%\copyrightyear{2018}           %% If different from \acmYear

%% Bibliography style
\bibliographystyle{ACM-Reference-Format}
%% Citation style
%\citestyle{acmauthoryear}  %% For author/year citations
%\citestyle{acmnumeric}     %% For numeric citations
%\setcitestyle{nosort}      %% With 'acmnumeric', to disable automatic
%% sorting of references within a single citation;
%% e.g., \cite{Smith99,Carpenter05,Baker12}
%% rendered as [14,5,2] rather than [2,5,14].
%\setcitesyle{nocompress}   %% With 'acmnumeric', to disable automatic
%% compression of sequential references within a
%% single citation;
%% e.g., \cite{Baker12,Baker14,Baker16}
%% rendered as [2,3,4] rather than [2-4].


%%%%%%%%%%%%%%%%%%%%%%%%%%%%%%%%%%%%%%%%%%%%%%%%%%%%%%%%%%%%%%%%%%%%%%
%% Note: Authors migrating a paper from traditional SIGPLAN
%% proceedings format to PACMPL format must update the
%% '\documentclass' and topmatter commands above; see
%% 'acmart-pacmpl-template.tex'.
%%%%%%%%%%%%%%%%%%%%%%%%%%%%%%%%%%%%%%%%%%%%%%%%%%%%%%%%%%%%%%%%%%%%%%


%% Some recommended packages.
\usepackage{booktabs}   %% For formal tables:
%% http://ctan.org/pkg/booktabs
\usepackage{subcaption} %% For complex figures with subfigures/subcaptions
%% http://ctan.org/pkg/subcaption

\usepackage{amssymb}
\usepackage{amsmath}
\usepackage{appendix}

%% The amsthm package provides extended theorem environments

% \usepackage{amsthm}
% AM commented out because it clashes with proof, etc

%% The lineno packages adds line numbers. Start line numbering with
%% \begin{linenumbers}, end it with \end{linenumbers}. Or switch it on
%% for the whole article with \linenumbers.
\usepackage{lineno}

%% Use package enumitem to align enumeration and itemization
\usepackage{enumitem}
\usepackage{listings}
\usepackage{courier}           % for the courier font (optional)
\usepackage{multicol}          % for two equations side by side
\usepackage[justification=centering]{caption}
\usepackage{xcolor}
\usepackage{stmaryrd}
\usepackage{hyperref}
%\usepackage{cleveref}
\usepackage{hieroglf}
\usepackage{scalerel}
\usepackage{tikz}
\usetikzlibrary{tikzmark}
\usepackage{pgfplots}
\usepackage{diagbox}
\usepackage{array,multirow,graphicx}

\hypersetup{ % play with these to change the look of hyperlinks
    colorlinks=true,
    linkcolor=blue,
    filecolor=magenta,
    urlcolor=blue,
    citecolor=blue,
}

\newcommand{\coq}{\scalebox{.6}{\textpmhg{\Ha}}}

% required by LNCS
\renewcommand\UrlFont{\color{blue}\rmfamily}
\colorlet{red}{red!80!black}
\colorlet{green}{green!50!black}

%% NEW COMMANDS =============================================

\lstdefinestyle{myStyle}{
%	language=Coq,
    keywords={Prop,Type,Inductive,Require,Import,Definition,Fixpoint,match,with,end,let,in,fix,if,then,else},
	basicstyle=\normalfont\footnotesize\tt,
    keywordstyle=\color{green}, % Blue clashes with the cyan links. Change if you want.
	stepnumber=1,
	tabsize=2,
    numbers=none,
    numberstyle=\tiny,
    numbersep=5pt,
	showspaces=false,
    escapechar=`,
	showstringspaces=false
}
%basicstyle=\fontsize{10}{11}\selectfont\ttfamily,

\lstdefinestyle{myTinyStyle}{
%   language=Coq,
    keywords={Inductive,Require,Import,Definition,Fixpoint,match,with,end,let,in,fix,if,then,else},
    basicstyle=\normalfont\fontsize{8.2}{8.5}\tt,
    keywordstyle=\color{green}, % Blue clashes with the cyan links. Change if you want.
    stepnumber=1,
    tabsize=2,
    numbers=none,
    numberstyle=\tiny,
    numbersep=5pt,
    showspaces=false,
    escapechar=`,
    showstringspaces=false
}

\lstdefinestyle{isaStyle}{
%	language=Coq,
    keywords={theory,lemma,code,by,imports,begin,primrec,where,end,definition,if,then,else,fun},
	basicstyle=\normalfont\footnotesize\tt,
    keywordstyle=\color{green}, % Blue clashes with the cyan links. Change if you want.
	stepnumber=1,
	tabsize=2,
    numbers=none,
    numberstyle=\tiny,
    numbersep=5pt,
	showspaces=false,
    escapechar=`,
	showstringspaces=false
}

\lstset{style=myStyle}
\makeatletter
\newlength{\@mli}
\newcommand{\mli}[1]{%
  \settowidth{\@mli}{\lstinline/#1/}
  \hspace{-.5ex}\begin{minipage}[t]{\@mli}\lstinline/#1/\end{minipage}}
\makeatother
\newcommand{\li}[1]{\ifmmode\mbox{\mli{#1}}\else\mbox{\lstinline/#1/}\fi}

\newcommand{\cdt}[2]{#1\mathrel{\mathchoice{\mkern-5mu}{\mkern-5mu}{\mkern-2mu}{\mkern-2mu}} ^{\mathcal{C}}_{#2} \mathrel{\mathchoice{\mkern-3.5mu}{\mkern-3.5mu}{\mkern-2mu}{\mkern-2mu}}}

\newcommand{\cdw}[2]{#1\mathrel{\mathchoice{\mkern-5mu}{\mkern-5mu}{\mkern-2mu}{\mkern-2mu}} ^{\mathcal{CW}}_{#2} \mathrel{\mathchoice{\mkern-3.5mu}{\mkern-3.5mu}{\mkern-2mu}{\mkern-2mu}}}

\newcommand{\cds}[2]{#1\mathrel{\mathchoice{\mkern-5mu}{\mkern-5mu}{\mkern-2mu}{\mkern-2mu}} ^{\mathcal{CS}}_{#2} \mathrel{\mathchoice{\mkern-3.5mu}{\mkern-3.5mu}{\mkern-2mu}{\mkern-2mu}}}

\newcommand{\rf}[2]{{#1\mathrel{\mathchoice{\mkern-5mu}{\mkern-5mu}{\mkern-2mu}{\mkern-2mu}} ^{\mathcal{R}}_{#2} \mathrel{\mathchoice{\mkern-3.5mu}{\mkern-3.5mu}{\mkern-2mu}{\mkern-2mu}}}}

\newcommand{\invhyper}{I}
\newcommand{\lb}{\linebreak}
\renewcommand{\angle}[1]{\langle #1\rangle}

\newcommand{\W}{\mathcal{W}}
\newcommand{\contract}{\text{\scshape{cont}}}
\newcommand{\repeatable}{\text{\scshape{rept}}}

\newcommand{\divc}{\text{divc}}
\newcommand{\logc}{\text{logc}}
\newcommand{\logstarc}{\text{logc*}}

\newcommand{\runtime}{\mathcal{T}}
\newcommand{\Texp}{\runtime_{\li{exp}}}
\newcommand{\Tmul}{\runtime_{\li{mul}}}
\newcommand{\Tleb}{\runtime_{\li{leb}}}
\newcommand{\Tsucc}{\runtime_{\li{succ}}}
\newcommand{\bigO}{\text{O}}

\newcommand{\Ack}{\ensuremath{\mathcal{A}}}
\newcommand{\CA}{\ensuremath{\text{C}}}
%\renewcommand{\S}{\ensuremath{\text{S}}}
\newcommand{\CR}{\ensuremath{\text{CR}}}
\newcommand{\CRH}{\ensuremath{\text{CRH}}}
\newcommand{\wt}[1]{\ensuremath{\widetilde{#1}}}
\newcommand{\IAR}{\ensuremath{\text{IAR}}}
\newcommand{\IARH}{\ensuremath{\text{IARH}}}

\newcommand\hide[1]{}
\newenvironment{centermath}
 {\begin{center}$\displaystyle}
 {$\end{center}}
%\renewcommand{\note}[2][polish]{{\color{red} #2}{\marginpar{\tiny \color{blue} #1}}}
\newcommand{\note}[2][polish]{{\color{red} #2}{\marginpar{\tiny \color{blue} #1}}}
\renewcommand{\implies}{\Rightarrow}
\renewcommand{\iff}{\Leftrightarrow}

%\makeatletter
%\newtheorem*{rep@theorem}{\rep@title}
%\newcommand{\newreptheorem}[2]{%
%	\newenvironment{rep#1}[1]{%
%		\def\rep@title{#2 \ref{##1}}%
%		\begin{rep@theorem}}%
%		{\end{rep@theorem}}}
%\makeatother

% %\theoremstyle{plain}
\newtheorem{defn}{Definition}
\newtheorem{ex}{Example}
\newtheorem{prop}[defn]{Proposition}
\newtheorem{col}[defn]{Corollary}
\newtheorem{lem}[defn]{Lemma}
\newtheorem{hypo}{Hypothesis}
% %\theoremstyle{definition}
\newtheorem{thm}[defn]{Theorem}
\newtheorem{rem}[defn]{Remark}

%\newreptheorem{thm}{Theorem}

\newenvironment{usethmcounterof}[1]{%
	\renewcommand\thedefn{\ref{#1}}\thm}{\endthm\addtocounter{defn}{-1}}

\newenvironment{uselemcounterof}[1]{%
	\renewcommand\thedefn{\ref{#1}}\lem}{\endlem\addtocounter{defn}{-1}}


%% DOCUMENT =================================================

%\bibliographystyle{plainurl}% the mandatory bibstyle

% \title{A Functional Proof Pearl: \protect\\ Inverting the Ackermann Hierarchy}

\begin{document}
	
%% Title information
\title[Inverting the Ackermann Hierarchy]{A Functional Proof Pearl: \protect\\ Inverting the Ackermann Hierarchy}     %% [Short Title] is optional;
%% when present, will be used in
%% header instead of Full Title.
% \titlenote{with title note}             %% \titlenote is optional;
%% can be repeated if necessary;
%% contents suppressed with 'anonymous'
% \subtitle{Subtitle}                     %% \subtitle is optional
% \subtitlenote{with subtitle note}       %% \subtitlenote is optional;
%% can be repeated if necessary;
%% contents suppressed with 'anonymous'


%% Author information
%% Contents and number of authors suppressed with 'anonymous'.
%% Each author should be introduced by \author, followed by
%% \authornote (optional), \orcid (optional), \affiliation, and
%% \email.
%% An author may have multiple affiliations and/or emails; repeat the
%% appropriate command.
%% Many elements are not rendered, but should be provided for metadata
%% extraction tools.

%% Author with single affiliation.
\author{Linh Tran}
% \authornote{This work was conducted while at the National University of Singapore.}          %% \authornote is optional;
\affiliation{  
    \department{Department of Mathematics}
	\institution{Yale University}            %% \institution is required
}
\affiliation{  
    \department{School of Computing}
    \institution{National University of Singapore}            %% \institution is required
}

\author{Anshuman Mohan}
\affiliation{
	\department{Yale-NUS College and \protect \\ School of Computing}
	\institution{National University of Singapore}            %% \institution is required
}

\author{Aquinas Hobor}
\affiliation{
	\department{Yale-NUS College and \protect \\ School of Computing}
	\institution{National University of Singapore}            %% \institution is required
}

%% Abstract
%% Note: \begin{abstract}...\end{abstract} environment must come
%% before \maketitle command
\begin{abstract}
%	\vspace{-2em}
	We implement in Gallina a hierarchy of functions that calculate
	the upper inverses to the hyperoperation/Ackermann hierarchy.
	Our functions run in $\Theta(b)$ for inputs expressed
	in unary, and in $\bigO(b^2)$ for inputs expressed in binary \linebreak (where $b$ = bitlength).
	We use our inverses to define linear-time functions---$\Theta(b)$ for both unary-represented 
	and binary-\linebreak represented inputs---that compute the upper inverse of the 
	diagonal Ackermann %\linebreak[3]
	function~$\Ack(n)$. We show that these functions are consistent with the usual definition of the \linebreak inverse Ackermann function~$\alpha(n)$.
	\keywords{Ackermann  \and hyperoperations \and Coq}
\end{abstract}


%% 2012 ACM Computing Classification System (CSS) concepts
%% Generate at 'http://dl.acm.org/ccs/ccs.cfm'.
\begin{CCSXML}
<ccs2012>
<concept>
<concept_id>10003752.10003790.10003796</concept_id>
<concept_desc>Theory of computation~Constructive mathematics</concept_desc>
<concept_significance>500</concept_significance>
</concept>
<concept>
<concept_id>10003752.10003809</concept_id>
<concept_desc>Theory of computation~Design and analysis of algorithms</concept_desc>
<concept_significance>300</concept_significance>
</concept>
<concept>
<concept_id>10002950.10003624</concept_id>
<concept_desc>Mathematics of computing~Discrete mathematics</concept_desc>
<concept_significance>300</concept_significance>
</concept>
</ccs2012>
\end{CCSXML}

\ccsdesc[500]{Theory of computation~Constructive mathematics}
\ccsdesc[300]{Theory of computation~Design and analysis of algorithms}
\ccsdesc[300]{Mathematics of computing~Discrete mathematics}
%% End of generated code


%% Keywords
%% comma separated list
\keywords{Ackermann, hyperoperations, inverses, Coq}  %% \keywords are mandatory in final camera-ready submission


%% \maketitle
%% Note: \maketitle command must come after title commands, author
%% commands, abstract environment, Computing Classification System
%% environment and commands, and keywords command.
\maketitle

%\author{Anonymous Author(s)}
%\authorrunning{Anon.}

%\author{Linh Tran \and Anshuman Mohan \and Aquinas Hobor}
%\authorrunning{L. Tran, A. Mohan, A. Hobor}
% First names are abbreviated in the running head.
% If there are more than two authors, 'et al.' is used.
%
%\institute{National University of Singapore \\
%\email{\{tran,mohan,hobor\}@comp.nus.edu.sg}}

%\end{frontmatter}

%% \linenumbers

%% main text
\section{Overview}
\label{sec: overview}
\subsection*{The Ackermann and Inverse Ackermann functions}
\begin{frame}
\frametitle{The Ackermann and Inverse Ackermann functions}
\label{defn: ack}
	The Ackermann-P\'eter function is defined as follows:
	\begin{equation}
	\label{eq:ackermann}
	A(n, m) \triangleq \begin{cases}
	m + 1 & \text{ when } n = 0 \\
	A(n-1, 1) & \text{ when } n > 0, m = 0 \\
	A\big(n-1, A(n, m-1)\big) & \text{ otherwise}
	\end{cases}
	%A(m, n) \triangleq \begin{cases}
	%n + 1 & \text{ if } m = 0 \\
	%A(m-1, 1) & \text{ if } m > 0, n = 0 \\
	%A(m-1, A(m, n-1)) & \text{ if } m > 0, n > 0
	%\end{cases}
	\end{equation}
	The one-variable \emph{diagonal} Ackermann function is $\Ack(n)~\triangleq~A(n, n)$.\\[5pt]
	
	Its \emph{inverse}, $\alpha(n)$, is the smallest~$k$ for
	which~$n \le \Ack(k)$, \emph{i.e}:
	\begin{equation*}
	\alpha(n) \triangleq \min\left\{k\in \mathbb{N} : n \le \Ack(k)\right\}
	\end{equation*}
\end{frame}


\begin{frame}
\frametitle{Initial values for $\Ack(n)$ and $\alpha(n)$}
\begin{columns}[T]
	\begin{column}{0.5\textwidth}
		TODO: Value table for $\Ack(n)$
		
		Grows astronomically fast!
	\end{column}

  \begin{column}{0.5\textwidth}
  	TODO: Value table for $\alpha(n)$
  	
  	Grows astronomically slowlys!
  \end{column}
\end{columns}
\end{frame}


\begin{frame}
\frametitle{Computing $\alpha(n)$}
Despite growing extremely slowly, $\alpha(n)$ is hard to compute for large $n$ due to the explosive growth of $\Ack(k)$.

\bigskip

\textbf{The Naive Approach:} starting at $k=0$, calculate $\Ack(k)$,
compare~it~to~$n$, and increment $k$ until $n \le \Ack(k)$.

\bigskip

\textbf{Time complexity:} $\Omega(\Ack(\alpha(n)))$,
so \emph{e.g.} computing $\alpha(100) \mapsto^{*} 4$ in this way requires
$\Ack(4) = 2^{2^{2^{65536}}}-3$ steps!
\end{frame}

\subsection*{The hyperoperation/Ackermann hierarchy}

\begin{frame}
\frametitle{The hierarchy of Ackermann levels}
The Ackermann function is easy to define, but hard to
understand.

We see it as
a sequence of $n$-indexed functions $\Ack_n \triangleq \lambda b.A(n,b)$, where for each $n>0$, $\Ack_n$ is the result of applying the previous $\Ack_{n-1}$ $b$ times,

with a
\href{https://github.com/inv-ack/inv-ack/blob/7270e64a2600b771f2b1b1b151f7d13fb2ae6c97/repeater.v\#L161-L177}{\emph{kludge}}. %Linked by Linh

\bigskip

To better understand the Ackermann function as a hierarchy and this kludge, we explore the closely-related hyperoperations.

\end{frame}


\begin{frame}
\frametitle{The Ackermann hierarchy and hyperoperations}
TODO: Polish, add names of levels to this table without overflowing the page

\begin{table}[t]
	\begin{centermath}
		\begin{array}{c@{\hskip 0.5em}|@{\hskip 1em}c@{\hskip 1em}c@{\hskip 1em}|@{\hskip 1em}c@{\hskip 1em}c}
			n & a [n] b & \Ack_n(b) & a \angle{n} b & \alpha_n(b)\\
			\hline
			0 & 1 + b & 1 + b & b - 1 & b - 1 \\
			1 & a + b & 2 + b & b - a & b - 2 \\
			2 & a \cdot b & 2b + 3 & \left\lceil \frac{b}{a} \right\rceil & \left\lceil \frac{b-3}{2} \right\rceil \\
			3 & a^b & 2^{b + 3} - 3 & \left\lceil \log_a ~ b \right\rceil & \left\lceil \log_2 ~ (b + 3)\right\rceil - 3 \\
			[2pt]
			4 & \underbrace{a^{.^{.^{.^a}}}}_b & \underbrace{2^{.^{.^{.^2}}}}_{b+3} - 3 & \log^*_a ~ b & \log^*_2 ~ (b + 3) - 3
		\end{array}
	\end{centermath}
	\label{table: hyperop-ack-inv}
\end{table}

The kludge: $\Ack_n(b) = 2[n](b+3) - 3$ and $\alpha_n(b) = 2\angle{n}(b+3) - 3$.

\end{frame}


%\begin{frame}
%\frametitle{}
%\end{frame}
%
%
%\begin{frame}
%\frametitle{}
%\end{frame}

\section{Hyperoperations and Ackermann via Repeater}
\label{sec: countdown-repeater}
\subsection{The repeated application pattern}

\begin{frame}
\frametitle{Review: Repeated Application via \texttt{iter}}

Let $X$ be any set and any function $f: X\to X$. Define the notation:
\begin{equation*}
f^{(k)}(u) \triangleq ~ \underbrace{(f\circ f\circ \cdots \circ f)}_{\text{$k$ times}}(u),
\end{equation*}

\bigskip

Coq's standard library has \texttt{iter}. \quad $f^{(k)}(u) = \texttt{iter}(k, f, u).$

\bigskip

The next level in the hyperoperations/Ackermann hierarchy is the result of $b$ compositional applications of the current level to some \emph{initial value}.


%The following recursive rule applies:
%\begin{enumerate}
%	\item $f^{(0)}(u) = u$ (\emph{i.e.} applying $0$ times yields the identity).
%	\item $f^{(k+1)}(u) = f \left(f^{(k)}(u)\right)$.
%\end{enumerate}
%
%\bigskip
%
%Repeated application plays a vital role in both hyperoperations and Ackermann hierarchy.

\end{frame}


\begin{frame}
\frametitle{Formal Definition of Hyperoperation Hierarchy}
\begin{equation*}
\begin{array}{lrcl}
\textit{1. 0$^{\textit{th}}$ level: } & a[0]b & ~ \triangleq ~ & b + 1 \\[3pt]
\textit{2. Initial values: } & a[n+1]0 & ~ \triangleq ~ &
\small \begin{cases}
a & \text{when } n = 0 \\
0 & \text{when } n = 1 \\
1 & \text{otherwise}
\end{cases} \\[15pt]
\textit{3. Recursive rule: } \quad & a[n+1](b+1) & ~ \triangleq ~ & a[n]\big(a[n+1]b\big)
\end{array}
\end{equation*}

\begin{equation*}
\begin{aligned}
& a[n+1]b \\
& = ~ a[n]\big(a[n+1](b-1)\big) ~ = ~ a[n]\big(a[n](a[n+1](b-2))\big) \\
& = ~ \big(a[n]\circ a[n]\big)\big(a[n+1](b-2)\big) ~ = ~ \ldots \\
& = ~ \underbrace{\big( a[n]\circ a[n]\circ \cdots \circ a[n] \big)}_{b \text{ times}} \big(a[n+1]0\big)  ~ = ~ \big(a[n]\big)^{(b)}\underbrace{\big(a[n+1]0\big)}_{\text{initial value}}
\end{aligned}
\end{equation*}

\end{frame}


\begin{frame}
\frametitle{Formal Definition of Ackermann Hierarchy}
\begin{equation*}
\begin{array}{lrcl}
\textit{1. 0$^{\textit{th}}$ level: } & \Ack_0(b) & ~ \triangleq ~ & b + 1 \\[3pt]
\textit{2. Initial values: } & \Ack_{n+1}(0) & ~ \triangleq ~ & \Ack_n(1) \\[3pt]
\textit{3. Recursive rule: } \quad & \Ack_{n+1}(b+1) & ~ \triangleq ~ & \Ack_n\big(\Ack_{n+1}(b)\big)
\end{array}
\end{equation*}

\bigskip

\begin{equation*}
\begin{aligned}
& \Ack_{n+1}(b) \\
& = ~ \Ack_n\big(\Ack_{n+1}(b-1)\big) ~ = ~ \Ack_n\big(\Ack_n\big(\Ack_{n+1}(b-2)\big)\big) \\
& = ~ \big(\Ack_n\circ \Ack_n\big) \big(\Ack_{n+1}(b-2)\big) ~ = ~ \ldots \\
& = ~ \underbrace{\big( \Ack_n\circ \Ack_n\circ \cdots \circ \Ack_n \big)}_{b \text{ times}} \big(\Ack_{n+1}(0)\big)  ~ = ~ \big(\Ack_n\big)^{(b)}\underbrace{\big(\Ack_{n+1}(0)\big)}_{\text{initial value}}
\end{aligned}
\end{equation*}

\end{frame}


\subsection{The repeater operation}


\begin{frame}[fragile]
\frametitle{Notation: From \texttt{iter} to Repeater}
%The next level in the hyperoperations/Ackermann hierarchy is the result of $b$ compositional applications of the current level to an initial value.
%
%\bigskip
%
%We can abstract the concept of repeated application in a higher-order function called \emph{repeater}:
%
%\smallskip

Writing $a[n+1]b = \big(a[n]\big)^{(b)}\big(a[n+1]0\big)$ or $\Ack_{n+1}(b) = \big(\Ack_n\big)^{(b)}\big(\Ack_{n+1}(0)\big)$ is clumsy since $b$ is no longer the main argument.

\bigskip

\textbf{Repeater.} A rearranged form of \texttt{iter}, where $b$ becomes the main argument:
\impinline{$\rf{f}{u}(b) \triangleq \texttt{iter}(b, f, u) = f^{(b)}(u)$}.

\smallskip

Read $\rf{f}{a}$ as ``the \emph{repeater from} $a$ of $f$''.

\smallskip

%$\forall a\in \mathbb{N}, f: \mathbb{N}\to \mathbb{N}$, the
%\href{https://github.com/inv-ack/inv-ack/blob/7270e64a2600b771f2b1b1b151f7d13fb2ae6c97/repeater.v\#L32-L36}{\emph{repeater from}}
%$a$ of $f$, denoted by $\rf{f}{a}$ , is a function $\mathbb{N}\to \mathbb{N}$ such that $\rf{f}{a}(n) = f^{(n)}(a)$.

% Linked by A
\begin{lstlisting}
`\href{https://github.com/inv-ack/inv-ack/blob/7270e64a2600b771f2b1b1b151f7d13fb2ae6c97/repeater.v#L32-L36}{Fixpoint repeater\_from}` (f : nat -> nat) (a n : nat) : nat :=
match n with 0 => a | S n' => f (repeater_from f a n') end.
\end{lstlisting}

\smallskip

\begin{minipage}{0.25\linewidth}
	Drop $b$ to form \\ a recursive rule:
\end{minipage}
\quad 
$\begin{cases}
\displaystyle a[n+1] & = \rf{\big(a[n]\big)}{a[n+1]0} \\[5pt]
\Ack_{n+1} & = \rf{\big(\Ack_n\big)}{\Ack_{n+1}(0)}
\end{cases}$.

\end{frame}


\begin{frame}[fragile]
\frametitle{Hyperoperations via Repeater}
Without Repeater (via double recursion):
\begin{lstlisting}
`\href{https://github.com/inv-ack/inv-ack/blob/7270e64a2600b771f2b1b1b151f7d13fb2ae6c97/repeater.v#L51-L52}{Definition hyperop\_init}` (a n : nat) : nat :=
  match n with 0 => a | 1 => 0 | _ => 1 end.

`\href{https://github.com/inv-ack/inv-ack/blob/7270e64a2600b771f2b1b1b151f7d13fb2ae6c97/repeater.v#L55-L64}{Fixpoint hyperop\_original}` (a n b : nat) : nat :=
  match n with
  | 0    => 1 + b
  | S n' => let fix hyperop' (b : nat) := match b with
              | 0    => hyperop_init a n'
              | S b' => hyperop_original a n' (hyperop' b')
              end in hyperop' b
  end.
\end{lstlisting}

With Repeater:
% Linked by Anshuman
\begin{lstlisting} 
`\href{https://github.com/inv-ack/inv-ack/blob/7270e64a2600b771f2b1b1b151f7d13fb2ae6c97/repeater.v#L67-L71}{Fixpoint hyperop}` (a n b : nat) : nat :=
  match n with
  | 0    => 1 + b
  | S n' => repeater_from (hyperop a n') (hyperop_init a n') b
  end.
\end{lstlisting}
\end{frame}


\begin{frame}[fragile]
\frametitle{Ackermann via Repeater}
Without Repeater (via double recursion):
\begin{lstlisting}
`\href{https://github.com/inv-ack/inv-ack/blob/7270e64a2600b771f2b1b1b151f7d13fb2ae6c97/repeater.v#L123-L132}{Fixpoint ackermann\_original}` (m n : nat) : nat :=
  match m with
  | 0    => 1 + n
  | S m' => let fix ackermann' (n : nat) : nat := match n with
              | 0    => ackermann_original m' 1
              | S n' => ackermann_original m' (ackermann' n')
              end in ackermann' n
  end.
\end{lstlisting}

With Repeater:
% Linked by A
\begin{lstlisting}
`\href{https://github.com/inv-ack/inv-ack/blob/7270e64a2600b771f2b1b1b151f7d13fb2ae6c97/repeater.v#L135-L139}{Fixpoint ackermann}` (n m : nat) : nat :=
  match n with
  | 0    => S m
  | S n' => repeater_from (ackermann n') (ackermann n' 1) m
  end.
\end{lstlisting}
\end{frame}

%
%\begin{frame}
%\frametitle{}
%\end{frame}

\section{Inverses via Countdown}
\label{sec: countdown}
Many functions on $\mathbb{R}$ are bijections and thus have an intuituve inverse.  
Functions on~$\mathbb{N}$ are often non-bijections and so their inverses
do not come as naturally.

\subsection{Upper inverses, expansions, and repeatability}
%; indeed, simply defining an inverse can be a little subtle.
\begin{defn} \label{defn: inverse}
The \emph{upper inverse} of $F$, written $F^{-1}$, 
% {\color{red} $F^{-1}$, $F^{-1}_{\mathit{\shortuparrow}}$, $F^{-1}_{\upharpoonleft}$} 
is $\lambda n. \min\{m : F(m)\ge n\}$.\end{defn}
This is well-defined as long as $F$ is unbounded, 
\emph{i.e.} $\forall b.~\exists a.~ b \leq F(a)$.  
However, it only makes sense as an ``inverse'' if $F$ is strictly
increasing, \emph{i.e.} $\forall n,m.~n < m \Rightarrow F(n) < F(m)$, which is
a rough analogue of injectivity in the discrete domain.

We call this the \emph{upper} inverse because, for strictly increasing functions like
addition, multiplication, and exponentiation, the upper inverse is the ceiling of the 
corresponding inverse functions on $\mathbb{R}$. The following clarifies further:
\hide{
\color{red} It is reasonable to wonder about the floor.
\begin{defn} \label{defn: lower_inverse}
The converse \emph{lower inverse}, written $F^{-1}_{-}$,
is defined as $\max\{m : F(m)\le n\}$.
\end{defn}
Even if $F$ strictly increases, as $a[n]$ does for every $a\ge 2$, notice that the lower inverse will be undefined for $n < F(0)$, \emph{e.g.} $\{m : a[n]m \le 0 \} = \varnothing$ for $n\ge 3$.
Thus we focus on upper inverses (hereafter just ``inverses''), and discuss lower inverses in \cref{sec: discussion}. 
}%end hide 
\begin{thm}[a Galois connection] \label{thm: upp-inverse-rel}
	If $F:\mathbb{N}\to \mathbb{N}$ is increasing, then $f$ is the upper inverse of $F$ if and only if $\ \forall n, m.~ f(n)\le m \iff n \le F(m)$.
\end{thm}
\begin{proof}
Fix $n$. The sentence $\forall m.~ f(n)\le m \iff n\le F(m)$ implies: (1) $f(n)$ is a lower bound to $\{m: n \le F(m)\}$ and (2) $f(n)$ is itself in the set, since plugging in $m \triangleq f(n)$ yields $n\le F(f(n))$, which makes $f$ the upper inverse of $F$. Conversely, if $f$ is the upper inverse of $F$, we know $\forall m.~n\le F(m)\Rightarrow f(n)\le m$. Now, $\forall m \ge f(n)$.~$F(m)\ge F(f(n)) \ge n$ by increasing-ness, thus completing the proof.
\end{proof}
\begin{col}\label{col:inv_gives_id}
If $F$ is strictly increasing, then $F^{-1} \circ F$ is
the identity function.
\end{col}
\begin{proof}
By ($\Leftarrow$) of Theorem~\ref{thm: upp-inverse-rel}, $F(n) \le F(n)$ implies 
$(F^{-1} \circ F)(n) \le n$.  By ($\Rightarrow$) of the same theorem, $(F^{-1} \circ F)(n) \le (F^{-1} \circ F)(n)$ implies $F(n) \le F \big((F^{-1} \circ F)(n)\big)$. $F$ is strictly increasing, so $n \le (F^{-1} \circ F)(n)$.
\end{proof}

\begin{rem}
Theorem~\ref{thm: upp-inverse-rel} and Corollary~\ref{col:inv_gives_id} 
and are our key reasoning rules. Users employing our techniques 
in their work will use these to manipulate proof goals.
\end{rem}

\noindent Our inverses require increasing functions, 
and our definitions of hyperoperations/Ackermann use repeater.  
But if $F$, is strictly increasing, 
$\rf{F}{a}$ is not necessarily strictly increasing:  
\emph{e.g.} the identity function
$\li{id}$ is strictly increasing, but $\rf{\li{id}}{a}(n) = (\li{id} \circ \ldots \circ \li{id}) (a) = a$ is a constant function.  We need a little more.
\begin{defn}
A function $F:\mathbb{N}\to\mathbb{N}$ is an \emph{expansion} if $\forall n.~ F(n)\ge n$. Further, given $a\in \mathbb{N}$, an expansion $F$ is \emph{strict from} $a$ if \note{$\forall n \ge a.~ F(n)\ge n+1$}.
\end{defn}
If $a\ge 1$ and $F$ is an expansion \emph{strict from} $a$, we quickly get: 
$\forall n.~ \rf{F}{a}(n) = F^{(n)}(a) \ge a + n \ge 1 + n$. That is, $\rf{F}{a} \ $ is itself an expansion strict from $0$. 

\begin{defn} \label{rem: repeatable-subset}
$\forall a\ge 1$, if~$F$ is a strictly increasing function 
that is also a strict expansion from~$a$, then $\rf{F}{a}$ preserves 
repeatability.
We thus say that such a function $F$ is \emph{repeatable from} $a$, and denote the 
set of functions repeatable from $a$ as $\repeatable_a$. 
It is straightforward to see that $\forall s,t.~ s \le t \Rightarrow \repeatable_s \subseteq \repeatable_t $.
\end{defn}

\subsection{Contractions and the countdown operation}

Suppose $F \in \repeatable_a$ for some $a \ge 1$, and 
let $f \triangleq F^{-1}$, \emph{i.e.} the inverse of $F$.
Our goal is to use $f$ to compute an inverse to $\rf{F}{a}$.  
From the preceeding discussion we know that this inverse must exist, 
since $F \in \repeatable_a$ implies $\rf{F}{a} \in \repeatable_0$.  
For reasons that will be clear momentarily, we write this inverse as $\cdt{f}{a}$.  Now
fix $n$ and observe that $\forall m.~f^{(m)}(n)\le a \iff m \ge \cdt{f}{a}(n)$ 
since
\vspace{-0.5em}
\begin{equation} \label{eq: rf-upp-inv}
\begin{aligned}
\cdt{f}{a}(n)\le m & \iff n\le \rf{F}{a}(m) = F^{(m)}(a) \iff f(n)\le F^{(m-1)}(a) \\
& \iff f^{(2)}(n)\le F^{(m-2)}(a) \iff \ldots \iff f^{(m)}(n)\le a
\end{aligned}
\vspace{-0.5em}
\end{equation}
Setting $m = \cdt{f}{a}(n)$ gives us 
$f^{(\cdt{f}{a}(n))}(n) \le a$.  
Together these say that~$\cdt{f}{a}(n)$ is the smallest number of 
times~$f$ needs to be compositionally applied to $n$ before the result 
equals or passes below $a$.
In other words, count the length of the chain $\{n, f(n), f^{(2)}(n), \ldots\}$ that 
terminates as soon as we reach/pass $a$. 
This only works if each chainlink is strictly less than the previous, 
\emph{i.e.} $f$ is a \emph{contraction}.
\begin{defn} \label{defn: contracting}
	A function $f : \mathbb{N} \to \mathbb{N}$ is a \emph{contraction} if $\forall n.~ f(n) \le n$. Given an $a \ge 1$, a contraction $f$ is 
	\emph{strict above} $a$ if $\forall n > a.~n > f(n)$. We denote the set of contractions by $\contract$ and the set of contractions strict above $a$ by $\contract_a$. Analogously to our observation in 
	Definition \ref{rem: repeatable-subset}, $\forall s\le t.~ \contract_s \subseteq \contract_t$.
\end{defn}
What kinds of functions have contractive inverses? Expansions, naturally:
\begin{thm} \label{thm: expansion-inv-contraction}
$\forall a\in \mathbb{N}.~F\in \repeatable_a \Rightarrow F^{-1}_+ \in \contract_a$.
\end{thm}
\begin{proof}
$\forall n.~F(n)\ge n \Rightarrow n \ge F^{-1}_+(n)$, so $F^{-1}_+ \in
\contract$. Note, $n > a \iff n-1\ge a$.
Since $F\in \repeatable_a$, $F(n-1)\ge n$ holds. 
Next, $n-1\ge F^{-1}_+(n) \Rightarrow n > F^{-1}_+(n)$. 
\end{proof}
This clarifies the inverse relationship between expansions strict from some $a$ and contractions strict above that same $a$. The inverse of an expansion's 
repeater exists and can be built \note{from its own inverse}, in a method formalized as \emph{countdown}.
\begin{defn} \label{defn: informal-countdown} \label{eq: countdown}
Let $f\in \contract_a$. The \textit{countdown to} $a$ of $f$, written 
$\cdt{f}{a}(n)$, is the smallest number of times $f$ needs to be applied to 
$n$ for the answer to equal or go below $a$. \emph{i.e.}, 
$\cdt{f}{a}(n) \triangleq \min\{m: f^{(m)}(n)\le a \}$.
\end{defn}
Inspired by Equation~\ref{eq: rf-upp-inv}, we provide a neat, algebraically manipulable logical sentence equivalent to Definition~\ref{eq: countdown}, which is more useful later in our paper:
\begin{col} \label{col: cdt-repeat}
If $a \in \mathbb{N}$ and $f\in \contract_{a}$, then $\forall n, m.~ \cdt{f}{a}(n)\le m \iff f^{(m)}(n)\le a$.
\end{col}
\begin{proof}
	Fix $a$ and $n$. The interesting direction is $(\Rightarrow)$. Suppose $\cdt{f}{a}(n)\le m$, we get $f^{(m)}(n)\le f^{(\cdt{f}{a}(n))}(n)$ due to $f\in \contract$, and $f^{(\cdt{f}{a}(n))}(n)\le a$ due to Definition~\ref{eq: countdown}.
\end{proof}
%\begin{thm} \label{thm: upp-inv-cdt-rf}
%	For all $a\in \mathbb{N}$, if $f\in \contract_{a}$ is the upper inverse of $F: \mathbb{N}\to \mathbb{N}$, then $\cdt{f}{a}$ is the upper inverse of $\rf{F}{a}$.
%\end{thm}
Another useful result is the recursive formula for \emph{countdown}:
\begin{thm} \label{thm: cdt-recursion}
	For all $a\in \mathbb{N}$ and $f\in \contract_{a}$, $\cdt{f}{a}$ satisfies:
	\begin{equation*}
	\cdt{f}{a}(n) = \begin{cases}
	0 & \text{ if } n \le a \\ 1 + \cdt{f}{a}(f(n)) & \text{ if } n \ge a + 1
	\end{cases}
	\end{equation*}
\end{thm}
\begin{proof}
When $n \le a$, use Corollary~\ref{col: cdt-repeat} as follows: 
$n = f^{(0)}(n)\le a \iff \cdt{f}{a}(n)\le 0$. 
When $n\ge a+1$, proceed by antisymmetry. 
Define $m \triangleq \cdt{f}{a}(f(n))$, and note that 
Corollary~\ref{col: cdt-repeat} gives 
$\cdt{f}{a}(n)\le 1+m \iff f^{(1+m)}(n) \le a$.
Next, a simple expansion of the last clause gives $f^{(m)}(f(n)) \le a$,
and unfolding the definition of~$m$ shows that this last clause is true.
Now since $n\ge a+1$, we have $\cdt{f}{a}(n)\ge 1$ by the above. 
Define $p \triangleq \cdt{f}{a}(n) - 1$. 
It remains to show $\cdt{f}{a}(f(n))\le p \Rightarrow f^{(p)}(f(n))\le a \Rightarrow f^{(p+1)}(n)\le a$, which holds by unfolding $p$'s definition.
\end{proof}

%\begin{thm} \label{thm: cdt-contr-0}
%	For all $a\ge 1$, if $f\in \contract_{a}$, then $\cdt{f}{a}\ \in \contract_{c} \ \forall c$.
%\end{thm}
%\begin{proof}
%	Firstly we show that $\cdt{f}{a}$ is a contraction, namely showing $\cdt{f}{a}(n)\le n \ \forall n$, which has already been proved in the proof of \cref{thm: cdt-repeat}. To show $\cdt{f}{a}$ is strict from $1+c$, it suffices to show it is strict from $1$, equivalently $\cdt{f}{a}(n) \le n - 1 \ \forall n\ge 1$. By \cref{thm: cdt-repeat}, we need to show $f^{(n-1)}(n)\le a$. Assume the converse, then:
%	$$ n \ge 1 + f(n)\ge 2 + f(f(n)) \ge \cdots \ge (n-1) + f^{(n-1)}(n) $$
%	Thus $f^{(n-1)}(n)\le 1 \le a$, a contradiction. The theorem then follows.
%\end{proof}

\subsection{A structurally recursive computation of countdown}

The higher-order repeater function is well-defined for all input functions, 
even those not in $\repeatable_a$ (although for such functions it may not 
be useful), and so is easy to define
in Coq as shown in \S\ref{sec: countdown-repeater}. In contrast, a 
\emph{countdown} only exists for certain functions, most conveniently 
contractions, which makes it a little harder to encode into Coq. 
Our strategy is to define a \emph{countdown worker} which is written with only
structural recursion and is thus more palatable to Coq, and then 
prove that this worker computes the countdown when passed a contraction.

\begin{defn} \label{defn: countdown-worker} \label{lem: cdt-init}
For any $a\in \mathbb{N}$ and $f: \mathbb{N}\to \mathbb{N}$, the \emph{countdown worker} to $a$ of $f$ is a function $\cdw{f}{a}\ : \mathbb{N}^2\to \mathbb{N}$ such that:
\begin{equation*}
\cdw{f}{a}(n, b) = \begin{cases}
0 & \text{if } b = 0 \vee n\le a \\ 1 +\cdw{f}{a}(f(n), b-1) & \text{if } b \ge 1 \wedge n > a
\end{cases}
\end{equation*}
\end{defn}
The worker operates on two arguments: 
the \emph{true argument} $n$, for which we want to simulate 
\emph{countdown to} $a$, 
and the \emph{budget} $b$, the maximum number of times we shall attempt 
to compositionally apply $f$ on the input before giving up. 
If the input goes below or equals $a$ after $k$ applications, \emph{i.e.} $f^{(k)}(n)~\le~a$, we return the count $k$. If the budget is exhausted (\emph{i.e.} $b = 0$) while the result is still above $a$, we fail by returning the original budget. This definition is admittedly inelegant, but the point is that it can clearly be written as a Coq \li{Fixpoint}:
\begin{lstlisting}
`\href{https://github.com/inv-ack/inv-ack/blob/6099297c6ab0e16d14b037fb5ed600c4d22818f6/countdown.v#L94-L100}{Fixpoint cdn\_wkr}` (a : nat) (f : nat -> nat) (n b : nat) : nat :=
match b with 0 => 0 | S b' => 
  match (n - a) with 0 => 0 | _ => S (cdn_wkr a f (f n) b') end
end.
\end{lstlisting}
Given an $f$ that is a contraction strict from $a$, 
and given a sufficient budget, \li{cdn_wkr} 
will compute the correct \emph{countdown} value.  
Careful readers may have noticed that \emph{budget} is similar to 
\emph{gas}, which we discussed in ~\S\ref{sec:incfuncinv} 
and dismissed for potentially 
being too computationally expensive to calculate. 
Budget is actually a refinement of gas because 
we support its use with a method to calculate it in constant time.
In fact, we will soon show that a budget of $n$ is sufficient. 
This lets us define \emph{countdown} in Coq for the first time:
\begin{defn} \label{defn: countdown}
Redefine $\cdt{f}{a}(n) \triangleq \cdw{f}{a}(n, n)$.
\begin{lstlisting}
`\href{https://github.com/inv-ack/inv-ack/blob/6099297c6ab0e16d14b037fb5ed600c4d22818f6/countdown.v#L104}{Definition countdown\_to}` a f n := cdn_wkr a f n n.
\end{lstlisting}
\vspace{-0.8em}
\end{defn}
%Before beginning, let us clarify that the definition of $\mathbb{N}$ and operations on $\mathbb{N}$ in Gallina follow the Presburger Arithmetic \cite{presburger}, which despite being weaker than Peano Arithmetic, is a decidable theory. The Coq standard library includes \texttt{Omega}, an extensive listing of provable facts about $\mathbb{N}$ in Presburger Arithmetic, including everything used in this paper, most notably the law of excluded middle for comparisons on $\mathbb{N}$:
%\begin{equation*}
%(n \le m) \vee (m + 1 \le n) \ \ \forall n, m
%\end{equation*}
%, which is provable without the actual law of excluded middle in classical logic. This enables us to prove all results in this paper with Coq's baseline intuitionistic logic. Readers can refer to the \Cref{appendix} for Coq versions of the proofs.
%in which the most operations agree with usual operations on $\mathbb{Z}$, except subtraction, which is defined as:
%\begin{lstlisting}
%Fixpoint sub (n m : nat) : nat :=
%match n with
%| 0 => n
%| S k => match m with
%| 0 => n
%| S l => sub k l end
%end.
%\end{lstlisting}
%Essentially $\li{sub} \ n \ m = \max\{n - m, 0\}$. We will use this subtraction for the rest of the paper.

%Firstly, we begin with a lemma asserting the existence of the countdown value itself. Although its existence is guaranteed by the well-ordering principle of $\mathbb{N}$, we will achieve better by proving it in intuitionistic logic.
%
%\begin{lem} \label{lem: contract-repeat-threshold}
%	For all $a, n\in\mathbb{N}$ and $f\in \contract_{a}$,
%	\begin{equation}
%	\exists m : \left(f^{(m)}(n) \le a \right) \wedge \left(f^{(l)}(n)\le a \Rightarrow m \le l \ \forall l \right)
%	\end{equation}
%\end{lem}
%\begin{proof}
%  Fix $n$ and observe that if $n\le a$, $m = 0$ is the desired choice since $ f^{(0)}(n) = n \le a \ \text{ and } \ 0 \le l \ \forall l $.
%	Consider only when $a\le n$, we can define $c$ such that $n = a + c$. We prove the following statement by induction on c:
%	\begin{equation*}
%	P(c) \triangleq \exists m : \left(f^{(m)}(n) \le n - c \right) \wedge \left(f^{(l)}(n)\le n - c \Rightarrow m \le l \ \forall l \right)
%	\end{equation*}
%	under assumptions $f\in \contract_{n-c}$ and $c\le n$.
%	\begin{enumerate}[leftmargin=*]
%		\item \textit{Base case.} The case $c = 0$ implies $n = a$, which has been proven above.
%		\item \textit{Inductive step.} Suppose $P(c)$ is proved with witness $m_c$. Note that the assumptions are now $f\in \contract_{n-c}$ and $c+1\le n$, there are two cases:
%		\begin{itemize}[leftmargin=*, label={--}]
%			\item $f^{\left(m_c\right)}(n) = n - c$. Then $f^{\left(m_c+1\right)}(n) \le n - c - 1$. Let $m_{c+1} = m_c + 1$, for all $l$:
%			\begin{equation*}
%			f^{(l)}(n)\le n - c - 1 < f^{\left(m_c\right)}(n) \Rightarrow l > m_c \Rightarrow l \ge m_{c+1}
%			\end{equation*}
%			\item $f^{\left(m_c\right)}(n) \le n - c - 1$. Let $m_{c+1} = m_c$, for all $l$:
%			\begin{equation*}
%			f^{(l)}(n)\le n - c - 1\le n - c \overset{P(c)}{\Rightarrow} l\ge m_{c} = m_{c+1}
%			\end{equation*}
%		\end{itemize}
%		In any cases, we can find a witness $m_{c+1}$ for $P(c+1)$. Thus the proof is complete by induction.\vspace*{-\baselineskip}
%	\end{enumerate}
%\end{proof}

% \begin{lem} 
% 	$\forall n\le a.~\cdw{f}{a} (n, b) = 0$.
% \end{lem}
% \begin{proof}
% This simple fact follows directly from Definition \ref{defn: countdown-worker}.
% \end{proof}
\noindent The computational~$\cdt{f}{a}(n)$
finds the same value as the theoretical 
Definition~\ref{defn: informal-countdown}:
\begin{thm} \label{thm: cdt-repeat}
	$\forall a \in \mathbb{N}.~ \forall f\in \contract_{a}$, we have 
%	\begin{equation} \label{eq: cdt-minimum}
$	\forall n.~ \cdt{f}{a}(n) = \min\left\{ i : f^{(i)}(n) \le a \right\} $.
\end{thm}
\begin{proof}[outline, full proof found in Appendix~\ref{apx:proof_correct_countdown_worker}]
%	Observe that given arguments $\left(f^{(i)}(n), b - i\right)$ \emph{s.t.} $i < b$ and $a < f^{(i)}(n)$, $\cdw{f}{a}$ will pass $\left(f^{(i+1)}(n), b - i - 1\right)$ to the next recursive call and add $1$ to the result.
%\hide{
	Assume the initial arguments $(n, b)$, let the original call $\cdw{f}{a}(n, b)$ be the $0^{th}$ call. A straightforward induction on $i$ shows that in the $i^{th}$ call, the arguments will be $\big(f^{(i)}(n), b-i\big)$, while an accumulated amount of $i$ will have been added to the final result, with the exception of the last call, which returns $0$ and adds nothing to the result.
	
	Suppose $b = n$, let $m \triangleq \min\big\{i : f^{(i)}(n)\le a\big\}$\footnote{We prove the existence 
		of the min in Coq’s intuitionistic logic \href{https://github.com/inv-ack/inv-ack/blob/6099297c6ab0e16d14b037fb5ed600c4d22818f6/countdown.v\#L125-L150}{here} in our codebase.}, then $m \le n$ since $f^{(n)}(n)\le a$. Thus before the budget is exhausted, the function reaches the $m^{th}$ call, which is the last recursive call since $\big(f^{(m)}(n), n - m\big)$ satisfies the terminating condition $f^{(m)}(n)\le a$. This last call adds $0$ to the accumulated result, which is $m$ at the moment. Therefore $\cdw{f}{a}(n, n) = m$.
%}%end hide
\hide{
	Observe that given arguments $\left(f^{(i)}(n), b - i\right)$ \emph{s.t.} $i < b$ and $a < f^{(i)}(n)$, $\cdw{f}{a}$ will pass $\left(f^{(i+1)}(n), b - i - 1\right)$ to the next recursive call and add $1$ to the result.
	\note[Can be further shortened for 2-3 lines, but I'll hold it for now]
	{A simple induction on $i$ confirms that $\forall n, \forall a, \forall b$ and $f\in \contract_{a}$,
	\begin{equation*}  %\label{eq: cdt-intermediate}
	\text{If } i < b \text{ and } a < f^{(i)}(n) \text{, then } \cdw{f}{a}(n, b) = 1 + i + \cdw{f}{a}\left(f^{(1+i)}(n), b - i - 1\right)
	\end{equation*}
	Let $m \triangleq \min\big\{i : f^{(i)}(n)\le a\big\}$\footnote{We prove the existence 
of the min in Coq’s intuitionistic logic \href{https://github.com/inv-ack/inv-ack/blob/6099297c6ab0e16d14b037fb5ed600c4d22818f6/countdown.v\#L125-L150}{here} in our codebase.}, the case $m = 0$ is trivial. If $m > 0$, substitute $i = m - 1$ in the above equation, we have: 
	\begin{equation*}
	\cdt{f}{a}(n) = \cdw{f}{a}(n, n) = m + \cdw{f}{a}\left(f^{(m)}(n), n-m\right) = m,
	\end{equation*}
where the last equality follows by applying Definition~\ref{defn: countdown-worker} given $f^{(m)}(n)\le a$}.
}%end hide
\end{proof}
We give an extended version of this proof in 
Appendix~\ref{apx:proof_correct_countdown_worker}, 
and mechanize it \note{here}.
Theorem~\ref{thm: cdt-repeat} and (\ref{eq: rf-upp-inv}) establish the correctness of the Coq definitions of \emph{countdown worker} and \emph{countdown}, thereby justifying our budget of $n$ and our unification of 
Definitions \ref{defn: informal-countdown} and \ref{defn: countdown}. We wrap everything together with the following theorem:
\begin{thm} \label{thm: cdt-inv-rf}
	$\forall F\in \repeatable_a.~f\triangleq F^{-1}_+$ satisfies $f \in \contract_{a}$ and $\displaystyle \cdt{f}{a} \ = \left(\rf{F}{a}\ \right)^{-1}_+$. Furthermore, if $a\ge 1$, then $\rf{F}{a}\ \in \repeatable_0$ and $\cdt{f}{a}\ \in \contract_0$.
\end{thm}
\begin{proof}
	By Theorem~\ref{thm: expansion-inv-contraction}, $f\triangleq F^{-1}_+ \in \contract_a$. 
	\eqref{eq: rf-upp-inv} and Corollary~\ref{col: cdt-repeat}
	then show $\cdt{f}{a}\ = \left(\rf{F}{a}\ \right)^{-1}_+$.
	Now if $a\ge 1$, a simple induction shows that $F^{(n)}(a)\ge a + n\ge 1 + n$, so $\rf{F}{a}\ \in \repeatable_0$, hence $\cdt{f}{a} \ = \left(\rf{F}{a}\ \right)^{-1}_+ \in \contract_0$ by Theorem~\ref{thm: expansion-inv-contraction}.
\end{proof}



\section{Inverting Hyperoperations and Ackermann}
\label{sec: inv-hyperop}
We now use \emph{countdown} to define the inverse
hyperoperation hierarchy, which features elegant new definitions of
division, $\log$, and $\log^{*}$.
We then modify this technique to arrive at the inverse
Ackermann hierarchy.

\subsection{Inverse hyperoperations, including \li{div}, \li{log}, \li{log*}}

\begin{defn} \label{defn: inv-hyperop}
	The inverse hyperoperations, written $a\angle{n}b$, are defined as:
%	\begin{equation}
%	a\angle{n}b \triangleq \begin{cases}
%	b - 1 & \text{if } n = 0 \\
%	\cdt{a\angle{n-1}}{a_n}(b) & \text{if } n \ge 1
%	\end{cases}
%	\ \ \text{ where } \ a_n = \begin{cases}
%	a & \hspace{-10pt}\text{ if } n = 1 \\
%	0 & \hspace{-10pt}\text{ if } n = 2 \\
%	1 & \hspace{-10pt}\text{ if } n \ge 3
%	\end{cases}
%	\end{equation}
% Edited by Linh
	\begin{equation}
  a\angle{n}b \triangleq \begin{cases}
  b - 1 & \hspace{-5pt}\text{if } n = 0 \\
  \cdt{a\angle{n-1}}{a_n}(b) & \hspace{-5pt}\text{if } n \ge 1
  \end{cases}
   \ \text{ where } \ a_n = \begin{cases}
   a & \hspace{-10pt}\text{ if } n = 1 \\
   0 & \hspace{-10pt}\text{ if } n = 2 \\
  1 & \hspace{-10pt}\text{ if } n \ge 3
  \end{cases}
\end{equation}
% linked by A
%\begin{lstlisting}
%`\href{https://github.com/inv-ack/inv-ack/blob/7270e64a2600b771f2b1b1b151f7d13fb2ae6c97/applications.v#L28-L33}{Fixpoint inv\_hyperop}` (a n b : nat) : nat :=
%  match n with 0 => b - 1 | S n' =>
%    countdown_to (hyperop_init a n') (inv_hyperop a n') b
%  end.
%\end{lstlisting}
% Edited by Linh
\begin{lstlisting}
`\href{https://github.com/inv-ack/inv-ack/blob/7270e64a2600b771f2b1b1b151f7d13fb2ae6c97/applications.v#L28-L33}{\color{blue}Fixpoint inv\_hyperop}` (a n b : nat) : nat :=
  match n with 0 => b - 1 | S n' =>
    countdown_to
      (hyperop_init a n') (inv_hyperop a n') b
  end.
\end{lstlisting}
\end{defn}
\hide{
Continuing with the notion of upper-inverses we have used thus far,
we aim for the \emph{ceiling} of division and $\log$ on real numbers, $\left\lceil b/a \right\rceil$ and $\left\lceil \log_ab \right\rceil$. The iterated logarithm is formally defined as the minimum number of times $\log$ needs to be iteratively
applied to the input for the result to become less than or equal to $1$.
Thus it is natural to define it using \emph{countdown}.

\begin{defn} \label{defn: divc} \label{defn: logc} \label{defn: log*}
\[
\divc(a, b) \! \triangleq \! \cdt{\left(\lambda b. (b{-}a)\right)}{0}(b) \quad
\logc(a, b) \! \triangleq \! \cdt{\left(\lambda b. \divc(a, b)\right)}{1}(b) \quad
\log^*(a, b) \! \triangleq \! \cdt{\left(\lambda b. \logc(a, b)\right)}{1}(b)
%\begin{array}{c}
%\divc(a, b) \triangleq \cdt{\left(\lambda b. (b-a)\right)}{0}(b) \qquad
%\logc(a, b) \triangleq \cdt{\left(\lambda b. \divc(a, b)\right)}{1}(b) \\
%\log^*(a, b) \triangleq \cdt{\left(\lambda b. \logc(a, b)\right)}{1}(b)
%\end{array}
\]
\begin{lstlisting}
Definition divc a b := countdown_to 0 (fun n => n - a) b.
Definition logc a b := countdown_to 1 (divc a) b.
Definition logstar a b := countdown_to 1 (logc a) b.
\end{lstlisting}
\end{defn}
}%end hide
\hide{Proving $\forall a, b \ge 1,~\divc(a, b) = \left\lceil b/a \right\rceil$ serves
to prove that all three definitions are correct, since that statement implies
the correctness of  $\logc$, which in turn implies the correctness of $\log^*$. We will prove this in \ref{blah},
together with the correctness of the entire hyperoperation hierarchy.
}
\noindent where the Curried $a\angle{n{-}1}$ denotes the single-variable function
$\lambda b.a\angle{n{-}1}b$.
We now show that $a\angle{n}$ is the inverse \lb to $a[n]$.
First note that $a\angle{0}\in \contract_0$.~Then:
% Linked by Linh
\begin{lem}
	\href{https://github.com/inv-ack/inv-ack/blob/7270e64a2600b771f2b1b1b151f7d13fb2ae6c97/applications.v#L48-L62}{\color{blue}\coq}
$\forall a.~\forall b.~a\angle{1}b = b - a$.
\end{lem}
\begin{proof}
Theorem~\ref{thm: cdt-recursion} applies because $a\angle{0} \in \contract_0 \subseteq \contract_{a}$, giving the intermediate step shown below.
Thereafter we have $a\angle{1}b = b - a$ by induction on $b$.
%\vspace{-0.8em}
%\begin{equation*}
%a\angle{1}b \quad = \quad \cdt{\left(a\angle{0}\right)}{a}(b) \quad = \quad \begin{cases}
%0 & \text{ if } b\le a \\ 1 + a\angle{1}(b - 1) & \text{ if } b\ge a+1 \end{cases} \\
%\end{equation*}
% Edited by Linh
\begin{equation*}
a\angle{1}b \ = \ \cdt{\left(a\angle{0}\right)}{a}(b) \ = \ \begin{cases}
0 & \text{ if } b\le a \\ 1 + a\angle{1}(b - 1) & \text{ if } b\ge a+1 \end{cases} \\
\end{equation*}
\end{proof}
\begin{col} \label{col: inv-hyperop-1-contr1}
$\forall a \ge 1, a\angle{1} \in \contract_1$.
\end{col}
\hide{\begin{col} \label{col: inv-hyperop-234}
{\color{magenta}$\forall a \in \mathbb{N}$}:
 {\color{blue} For all $a \in \mathbb{N}$},
\begin{enumerate}
	\item If $a\ge 1$, $a\angle{2}b = \divc(a, b) \ \forall b$.
	\item If $a\ge 2$, $a\angle{3}b = \logc(a, b) \ \forall b$ and $a\angle{4}b = \logstarc(a, b) \ \forall b$.
\end{enumerate}
\end{col}}%end hide
%Now we are ready to prove the correctness of the inverse hyperoperations.
\noindent \textbf{N.B.} $a\angle{n}b$ is a total function, but it is never actually used for $a = 0$ when $n \ge 2$ or for $a=1$ when $n \ge 3$. For the values
we do care about, we have our inverse:
% Linked by Linh
\begin{thm} \label{thm: inv-hyperop-correct}
	\href{https://github.com/inv-ack/inv-ack/blob/7270e64a2600b771f2b1b1b151f7d13fb2ae6c97/applications.v#L72-L98}{\color{blue}\coq}
When $n\le 1$, or $n \le 2 \wedge a\ge 1$, or $a\ge 2$, then
$a\angle{n} = \left(a[n]\right)^{-1}$.
\end{thm}
\begin{proof}
$\forall n \ge 2,~$ let $a_0 = a, a_1 = 0, a_n = 1$. Define $P$ and $Q$ as:
\begin{equation*}
P(n) \triangleq  \Big(a[n] \in \repeatable_{a_n}\Big)\ \text{ and } \
Q(n) \triangleq  \Big((a\angle{n} = \left(a[n]\right)^{-1} \Big).
\end{equation*}
We have three goals:
\begin{enumerate}
	\item $\forall a.~ Q(0) \wedge Q(1)$ \\ \vspace{-1.2em}
	\item $\forall a \ge 1.~ Q(2)$ \\ \vspace{-1.2em}
	\item $\forall a \ge 2.~ Q(n)$ \\ \vspace{-1.2em}
\end{enumerate}
	Note that \ \mbox{$\forall n.~ a\angle{n+1}$} $= \cdt{a\angle{n}}{a_n}$ ~and \ $a[n+1] = \rf{a[n]}{a_n}$. By Theorem~\ref{thm: cdt-inv-rf},
% Edited by Linh
%\vspace*{-1em}
%\noindent\begin{minipage}{.45\linewidth}
%\begin{equation}
%P(n) \Rightarrow Q(n) \Rightarrow Q(n+1) \label{eq: tmp-induction-1}
%\end{equation}
%\end{minipage}
%\begin{minipage}{.55\linewidth}
%\begin{equation}
%\qquad a_{n} \ge 1 \Rightarrow P(n) \Rightarrow P(n+1) \label{eq: tmp-induction-2}
%\end{equation}
%\end{minipage}
% Edited by Linh
	\begin{align}
	P(n) & \ \Rightarrow \ Q(n) \ \Rightarrow \ Q(n+1) \label{eq: tmp-induction-1} \\
	\qquad a_{n} \ge 1 & \ \Rightarrow \ P(n) \ \Rightarrow \ P(n+1) \label{eq: tmp-induction-2}
	\end{align}
Goal 1: $P(0) \iff \lambda b.(b+1)\in \repeatable_a$ and
$Q(0)\hspace{-0.3em}\iff\hspace{-0.3em}a\angle{0}\hspace{-0.3em}=\hspace{-0.3em}\left(a[0]\right)^{-1} \lb \iff (b-~1\le c \iff b\le c+1)$.
These are both straightforward, and $Q(1)$ holds by~(\ref{eq: tmp-induction-1}). 
Goal 2: we have $a \ge 1$, so
$P(1)$ holds by $P(0)$ and~(\ref{eq: tmp-induction-2}), and
$Q(2)$ holds by $Q(1)$ and~(\ref{eq: tmp-induction-1}).
Goal 3: we have $a\ge 2$, and using~(\ref{eq: tmp-induction-1}) and $Q(0)$
reduces the goal to $P(n)$. Using~(\ref{eq: tmp-induction-2}) and the fact that $\forall n \neq 1.~a_n\ge 1$, the goal reduces to $P(2)$. This unfolds to:
\begin{equation*}
a[2]\in \repeatable_0 \iff \forall b < c.~ab < ac \quad \wedge \quad \forall b \ge 1.~ab \ge b+1\text{,}
\end{equation*}
which is straightforward for $a\ge 2$. Induction on $n$ gives us the third goal.
\end{proof}
%With this theorem, we have proved the correctness of $\divc$, $\logc$ and $\log^*$, as well as the whole inverse hyperoperation hierarchy.
\begin{rem}
Three early hyperoperations are $a[2]b = ab$, $a[3]b = a^b$ and
$a[4]b = \! ^ba$, so, by Theorem~\ref{thm: inv-hyperop-correct}, we can define their inverses $\left\lceil b/a \right\rceil$, $\left\lceil \log_a b \right\rceil$, and $\log^*_a b$ as	
\href{https://github.com/inv-ack/inv-ack/blob/7270e64a2600b771f2b1b1b151f7d13fb2ae6c97/applications.v#L102-L113}{\color{blue}$a\angle{2}b$},	
\href{https://github.com/inv-ack/inv-ack/blob/7270e64a2600b771f2b1b1b151f7d13fb2ae6c97/applications.v#L115-L124}{\color{blue}$a\angle{3}b$}, and	
\href{https://github.com/inv-ack/inv-ack/blob/7270e64a2600b771f2b1b1b151f7d13fb2ae6c97/applications.v#L126-L128}{\color{blue}$a\angle{4}b$}.
Note that the functions $\log_a b$ and $\log^*_a b$ are not in the Coq Standard Library but are one-liners for us:

\begin{lstlisting}
Definition divc a b := inv_hyperop a 2 b.
Definition logc a b := inv_hyperop a 3 b.
Definition logstar a b := inv_hyperop a 4 b.
\end{lstlisting}
\end{rem}



\hide{
\begin{col}
	$\forall a \in \mathbb{N}$:
	\begin{enumerate}
		\item If $a\ge 1$, $\divc(a, b) = \left\lceil \frac{b}{a} \right\rceil$.
		\item If $a\ge 2$, $\logc(a, b) = \left\lceil \log_ab \right\rceil $ and $\logstarc(a, b) = \log^*_ab$.
	\end{enumerate}
\end{col}
\begin{proof}
	The result follows from \cref{col: inv-hyperop-234}, \cref{thm: inv-hyperop-correct} and the first few expansions of the hyperoperation hierarchy: $a[2]b = ab$, $a[3]b = a^b$ and $a[4]b = \ ^ba$.
\end{proof}

\begin{rem}
	\Cref{thm: cdt-recursion} leads to useful recursive formulas for $\divc$, $\logc$ and $\logstarc$:
	\begin{align*}
	\divc(a, b) & = 1 + \divc(a, b - a) & \text{ if } b > 0 &\text{, and } 0 \text{ otherwise}. \\
	\logc(a, b) & = 1 + \logc\left(a, \left\lceil b/a \right\rceil\right) & \text{ if } b > 1 &\text{, and } 0 \text{ otherwise}. \\
	\logstarc(a, b) & = 1 + \divc(a, \left\lceil \log_ab \right\rceil) & \text{ if } b > 1 &\text{, and } 0 \text{ otherwise}.
	\end{align*}
\end{rem}
}

\subsection{The inverse Ackermann hierarchy}
\label{subsec: inv_ack_hier}

% NEW METHOD FOR INVERSE ACKERMANN
Next, we want to use \emph{countdown} to build the \emph{inverse Ackermann hierarchy}, where each
level $\alpha_i$ inverts the level $\Ack_i$.
We know $\Ack_{i+1} = \rf{\Ack_i}{\Ack_i(1)}$\hspace{0.2em}, so the recursive
rule $\alpha_{i+1} \triangleq \cdt{\alpha_i}{\Ack_i(1)}$ is tempting.
But this approach is flawed because it still depends on $\Ack_i$.
Instead, we reexamine the inverse relationship: suppose $\alpha_i = \big(\Ack_i\big)^{-1}$ and $\alpha_{i+1} = \big(\Ack_{i+1}\big)^{-1}$. Then $\Ack_{i+1}(m) = \big(\Ack_{i}\big)^{(m)} \big(\Ack_{i}(1)\big)$. We then have:
\begin{equation} \label{eq: inv-ack-hier-derive}
\alpha_{i+1}(n)\le m \iff n\le \big(\mathcal{A}_i\big)^{(m+1)}(1) \iff \big(\alpha_i\big)^{(m+1)}(n) \le 1
\end{equation}
Equivalently, $\alpha_{i+1}(n) = \min\big\{m : \big( \alpha_i \big)^{(m+1)}(n)\le 1\big\}$, or $\alpha_{i+1}(n) = \cdt{\alpha_i}{1}\big(\alpha_i(n)\big)$. From \eqref{eq: inv-ack-hier-derive} we can thus define the inverse Ackermann hierarchy:
\begin{defn} \label{defn: inv-ack-hier}
	$ \alpha_i \triangleq \begin{cases}
	\lambda n.(n - 1) & \text{if } i = 0
	\\ \big(\cdt{\alpha_{i-1}}{1}\big)\circ \alpha_{i-1} & \text{if } i\ge 1 \end{cases}
$
\end{defn}
Abracadabra! We can express $\alpha$ using the hierarchy \emph{without referring to $\Ack$ itself}:
\begin{thm} \label{thm: inv-ack-new}
	%For all $n, k$, $n\le \Ack(k, k) \!\!\iff \!\! \alpha_k(n)\le k$. Thus
	$\forall n.~ \alpha(n) = \min\big\{k : \alpha_k(n)\le k \big\}$.
\end{thm}


\begin{table}[t]
	\begin{centermath}
		\begin{array}{c | c@{\hspace{1.5em}} c@{\hspace{1.5em}} c@{\hspace{1.5em}} c@{\hspace{1.5em}} c@{\hspace{1.5em}} c@{\hspace{1.5em}} c@{\hspace{1.5em}} c@{\hspace{1.5em}} c@{\hspace{1.5em}} c@{\hspace{1.5em}}}
			          $\diagbox[height=\line]{$\alpha_k$}{$n$}$ & 0\tikzmark{zero} & 1\tikzmark{one} & \tikzmark{twoa}2\tikzmark{two} & \tikzmark{threea}3\tikzmark{three} & \tikzmark{foura}4\tikzmark{four} & \tikzmark{fivea}5\tikzmark{five} & \tikzmark{sixa}6\tikzmark{six} & \tikzmark{sevena}7\tikzmark{seven} & \tikzmark{eighta}8\tikzmark{eight} & \tikzmark{ninea}9\tikzmark{nine} \\
			\hline
			\alpha_0 & 0\tikzmark{zero_succ} & 0\tikzmark{one_succ} & \tikzmark{two_fail}1 & \tikzmark{three_fail}2 & \tikzmark{four_fail1}3 & \tikzmark{five_fail1}4 & \tikzmark{six_fail1}5 & \tikzmark{seven_fail1}6 & \tikzmark{eight_fail1}7 & \tikzmark{nine_fail1}8 \\
			\alpha_1 & 0 & 0 & 0\tikzmark{two_succ} & 1\tikzmark{three_succ} & \tikzmark{four_fail2}2 & \tikzmark{five_fail2}3 & \tikzmark{six_fail2}4 & \tikzmark{seven_fail2}5 & \tikzmark{eight_fail2}6 & \tikzmark{nine_fail2}7 \\
			\alpha_2 & 0 & 0 & 0 & 0 & 1\tikzmark{four_succ} & 1\tikzmark{five_succ} & 2\tikzmark{six_succ} & 2\tikzmark{seven_succ} & \tikzmark{eight_fail3}3 & \tikzmark{nine_fail3}3 \\
			\alpha_3 & 0 & 0 & 0 & 0 & 0 & 0 & 1 & 1 & 1\tikzmark{eight_succ} & 1\tikzmark{nine_succ} \\
		\end{array}
	\end{centermath}
	\caption{Intuition for $\alpha(n)$ defined without $\Ack(n)$.}
	\label{table:inv_intuition}
\end{table}
  \begin{tikzpicture}[overlay, remember picture, yshift=.25\baselineskip, shorten >=.5pt, shorten <=.5pt]
    \draw [->, green] ({pic cs:zero}) to [bend left]({pic cs:zero_succ});
    \draw [->, green] ({pic cs:one}) to [bend left]({pic cs:one_succ});
   	\draw [->, red] ({pic cs:twoa}) to [bend right]({pic cs:two_fail});
   	\draw [->, green] ({pic cs:two}) to [bend left]({pic cs:two_succ});
   	\draw [->, red] ({pic cs:threea}) to [bend right]({pic cs:three_fail});
   	\draw [->, green] ({pic cs:three}) to [bend left]({pic cs:three_succ});
   	\draw [->, red] ({pic cs:foura}) to [bend right]({pic cs:four_fail1});
   	\draw [->, red] ({pic cs:foura}) to [bend right]({pic cs:four_fail2});
   	\draw [->, green] ({pic cs:four}) to [bend left]({pic cs:four_succ});
   	\draw [->, red] ({pic cs:fivea}) to [bend right]({pic cs:five_fail1});
   	\draw [->, red] ({pic cs:fivea}) to [bend right]({pic cs:five_fail2});
   	\draw [->, green] ({pic cs:five}) to [bend left]({pic cs:five_succ});
   	\draw [->, red] ({pic cs:sixa}) to [bend right]({pic cs:six_fail1});
   	\draw [->, red] ({pic cs:sixa}) to [bend right]({pic cs:six_fail2});
   	\draw [->, green] ({pic cs:six}) to [bend left]({pic cs:six_succ});
   	\draw [->, red] ({pic cs:sevena}) to [bend right]({pic cs:seven_fail1});
   	\draw [->, red] ({pic cs:sevena}) to [bend right]({pic cs:seven_fail2});
   	\draw [->, green] ({pic cs:seven}) to [bend left]({pic cs:seven_succ});
   	\draw [->, red] ({pic cs:eighta}) to [bend right]({pic cs:eight_fail1});
   	\draw [->, red] ({pic cs:eighta}) to [bend right]({pic cs:eight_fail2});
   	\draw [->, red] ({pic cs:eighta}) to [bend right]({pic cs:eight_fail3});
   	\draw [->, green] ({pic cs:eight}) to [bend left]({pic cs:eight_succ});
   	\draw [->, red] ({pic cs:ninea}) to [bend right]({pic cs:nine_fail1});
   	\draw [->, red] ({pic cs:ninea}) to [bend right]({pic cs:nine_fail2});
   	\draw [->, red] ({pic cs:ninea}) to [bend right]({pic cs:nine_fail3});
   	\draw [->, green] ({pic cs:nine}) to [bend left]({pic cs:nine_succ});
  \end{tikzpicture}

Intuitionally, this new definition of $\alpha$ aligns 
with the goal we laid out initially 
in Definition~\ref{defn: inv_ack}. For some $n$, we were trying to 
minimize $k$ such that $n \le \Ack(k)$. 
Now we have shown that this is equivalent to minimizing $k$ such that 
$\alpha_{k} (n) \le k$.

% \vspace{-3em}
Table~\ref{table:inv_intuition} gives further intuition for how this 
new definition works.
This table is populated exactly as suggested by 
Table~\ref{table: hyperop-ack-inv}:
$\alpha_{0}(n)$ is $n-1$, 
$\alpha_{1}(n)$ is $n-2$,
$\alpha_{2}(n)$ is $\left\lceil \frac{n-3}{2} \right\rceil$, and
$\alpha_{3}(n)$ is $\left\lceil \log_2 ~ (b + 3)\right\rceil - 3$.
Now for any $n$, we search down its column, which is indexed by $k$, 
looking for the smallest $k$ such that $\alpha_k(n) \le k$.

Consider $n = 8$. $\alpha_{0}(8) = 7$. Because $7 \not\le 0$, $\alpha_0$ is rejected.
Similarly $\alpha_{1}(8) = 6 \not\le 1$ and $\alpha_{2}(8) = 3 \not\le 2$
are rejected. Finally $\alpha_{3}(8) = 1$ is accepted because $1 \le 3$. 
Indeed, $\alpha(8) = 3$.
Similar reasoning applies to the other values for $n$. To aid intuition in this table,
we show show unsuccessful searches in {\color{red}red} and successful searches in 
{\color{green}green}.

Now all that remains is to provide a structurally-recursive function that computes $\alpha$.
%\begin{defn} \label{defn: inv-ack-worker}
%	The inverse Ackermann worker is a function $\alpha^{\W}$: %(\mathbb{N}\to \mathbb{N}) \times \mathbb{N}^3\to \mathbb{N}$ such that for all $n, k, b\in \mathbb{N}$ and $f:\mathbb{N}\to \mathbb{N}$:
%	\begin{equation} \label{eq: inv-ack-worker-recursion}
%	\alpha^{\W}(f, n, k, b) \triangleq \begin{cases}
%	k & \text{if } b = 0 \vee n\le k \\ \alpha^{\W}(\cdt{f}{1}\circ f , \cdt{f}{1}(n), k+1, b-1) & \text{if } b \ge 1 \wedge n \ge k+1
%	\end{cases}
%	\end{equation}
%\end{defn}
% Edited by Linh
\begin{defn} \label{defn: inv-ack-worker}
	The inverse Ackermann worker, written $\alpha^{\W}$, is a function from $\mathbb{N}^4$ to $\mathbb{N}$ defined as: %(\mathbb{N}\to \mathbb{N}) \times \mathbb{N}^3\to \mathbb{N}$ such that for all $n, k, b\in \mathbb{N}$ and $f:\mathbb{N}\to \mathbb{N}$:
	\begin{equation} \label{eq: inv-ack-worker-recursion}
	\begin{aligned}
	& \alpha^{\W}(f, n, k, b) \\ 
	& \triangleq \begin{cases}
	k & \text{if } b = 0 \vee n\le k \\ \alpha^{\W}(\cdt{f}{1}\circ f , \cdt{f}{1}(n), k+1, b-1) & \text{if } b \ge 1 \wedge n \ge k+1
	\end{cases}
		\end{aligned}
	\end{equation}
\end{defn}
\noindent Next, we show that this function computes the inverse Ackermann function when passed appropriate arguments.
\hide{
Given the arguments $\big(\alpha_i, \alpha_i(n), i, b - i\big)$ such that $\alpha_i(n) > i$ and $b > i$, $\alpha^{\W}$ takes on arguments $\big(\alpha_{i+1}, \alpha_{i+1}(n), i+1, b - (i+1)\big)$ at the next recursive call. Thus if $\alpha^{\W}$ is given a sufficient budget $b$, it will recursively transform the tuple $(\alpha_k, \alpha_k(n), k, b - k)$ until a point $k$ where $\alpha_k(n)\le k$, and will then return $k$. We now need to show that $\alpha^{\W}$ correctly computes $\alpha(n)$ given a reasonable budget.
The following theorem demonstrates that a budget of $n$ suffices.
}%end hide
\hide{
\noindent We are finally ready for a strategy to compute the inverse Ackermann function:
}%end hide
%formalizes a setting for $\W\alpha$ to work.
% Linked by Linh
\begin{thm} \label{thm: inv-ack-worker-correct}
	\href{https://github.com/inv-ack/inv-ack/blob/7270e64a2600b771f2b1b1b151f7d13fb2ae6c97/inv_ack.v#L199-L231}{\color{blue}\coq}
	$\forall n.~\alpha^{\W}\big(\alpha_0, \alpha_0(n), 0, n\big) = \alpha(n)$.
\end{thm}
\begin{proof}[Proof outline]
	When given the arguments $\big(\alpha_i, \alpha_i(n), i, b - i\big)$ such that $\alpha_i(n) > i$ and $b > i$, $\alpha^{\W}$ takes on the arguments $\big(\alpha_{i+1}, \alpha_{i+1}(n), i+1, b - (i+1)\big)$ at the next recursive call. A simple induction on $k$ then shows that if $k\le \min\big\{b, \alpha_k(n)\big\}$,
	\begin{equation} \label{eq: inv-ack-worker-intermediate}
	\alpha^{\W}\big(\alpha_0, \alpha_0(n), 0, b\big) = \alpha^{\W}\big(\alpha_k, \alpha_k(n), k, b-k\big)
	\end{equation}
	Let $m \triangleq \min\big\{k : \alpha_k(n) \le k \}$. Then $m\le n$ since $\alpha_n(n)\le n$. \eqref{eq: inv-ack-worker-intermediate} then implies:
	$$ \alpha^{\W}\big(\alpha_0, \alpha_0(n), 0, n\big) = \alpha^{\W}\big(\alpha_m, \alpha_m(n), m, n - m\big) = m = \alpha(n) $$
\end{proof}
\noindent We put a more involved mathematical proof of correctness in Appendix~\ref{apx:proof_correct_inv_ack_worker}, and a mechanized proof
% Linked by Linh
	\href{https://github.com/inv-ack/inv-ack/blob/7270e64a2600b771f2b1b1b151f7d13fb2ae6c97/inv_ack.v#L163-L231}{\color{blue}here}.
We thus have a (re-)definition of inverse Ackermann that is %computationally
definable via a Coq-accepted worker, \emph{i.e.} $\alpha(n) \triangleq \alpha^{\W}\big(\alpha_0, \alpha_0(n), 0, n\big)$. Below, we present the inverse
Ackermann function in Gallina.

% Linked by A
%\begin{lstlisting}
%`\href{https://github.com/inv-ack/inv-ack/blob/7270e64a2600b771f2b1b1b151f7d13fb2ae6c97/inv_ack.v#L155-L161} {Fixpoint inv\_ack\_wkr}` (f : nat -> nat) (n k b : nat) : nat :=
%  match b with 0 => 0 | S b' =>
%    if (n <=? k) then k else let g := (countdown_to f 1) in
%      inv_ack_wkr (compose g f) (g n) (S k) b
%  end.
%
%`\href{https://github.com/inv-ack/inv-ack/blob/7270e64a2600b771f2b1b1b151f7d13fb2ae6c97/inv_ack.v#L37-L41}{Fixpoint alpha}` (m x : nat) : nat :=
%  match m with 0 => x - 1 | S m' =>
%    countdown_to 1 (alpha m') (alpha m' x)
%  end.
%
%`\href{https://github.com/inv-ack/inv-ack/blob/7270e64a2600b771f2b1b1b151f7d13fb2ae6c97/inv_ack.v#L167}{Definition inv\_ack}` := inv_ack_wkr (alpha 0) (alpha 0 n) 0 n.
%\end{lstlisting}
% Edited by Linh
\begin{lstlisting}
`\href{https://github.com/inv-ack/inv-ack/blob/7270e64a2600b771f2b1b1b151f7d13fb2ae6c97/inv_ack.v#L155-L161} {\color{blue}Fixpoint inv\_ack\_wkr}` f n k b :=
  match b with 0 => 0 | S b' =>
    if (n <=? k) then k
      else let g := (countdown_to f 1) in
        inv_ack_wkr (compose g f) (g n) (S k) b
  end.

`\href{https://github.com/inv-ack/inv-ack/blob/7270e64a2600b771f2b1b1b151f7d13fb2ae6c97/inv_ack.v#L37-L41}{\color{blue}Fixpoint alpha}` m x :=
  match m with 0 => x - 1 | S m' =>
    countdown_to 1 (alpha m') (alpha m' x)
  end.

`\href{https://github.com/inv-ack/inv-ack/blob/7270e64a2600b771f2b1b1b151f7d13fb2ae6c97/inv_ack.v#L167}{\color{blue}Definition inv\_ack}` :=
  inv_ack_wkr (alpha 0) (alpha 0 n) 0 n.
\end{lstlisting}

\noindent Note that this is not the linear-time computation we presented in
Figure~\ref{fig:standalone}. We will arrive at that code via an improvement discussed in the next section.
% OLD METHOD FOR INVERSE ACKERMANN

%Using the Ackermann kludge from \cref{sec: overview}, we sketch a method to find the inverse Ackermann function from the inverse hyperoperations.
%\begin{thm} \label{thm: inv-hyperop-inv-ack}
%	For all $n, k$, $n\le \Ack(k, k) \iff 2\angle{k}(n+3)\le k+3$.
%\end{thm}
%\begin{proof}
%	$n\le \Ack(k, k) \iff n\le 2[k](k+3) - 3$. Now $2[k]$ is an expansion by \cref{thm: inv-hyperop-correct}'s proof, so $2[k](k+3) \ge k+3\ge 3$, so  $n\le 2[k](k+3) - 3 \iff n+3 \le 2[k](k+3)\iff 2\angle{k}(n+3)\le k+3$, again by \cref{thm: inv-hyperop-correct}.
%\end{proof}
%\Cref{thm: inv-hyperop-inv-ack} allows a simple method to compute the Inverse Ackermann function, based on the \emph{budget} idea in \emph{countdown worker}.
%\begin{defn} \label{defn: inv-ack-worker}
%	The \emph{inverse Ackermann worker} of $f$ is a function $\W\alpha\ : \mathbb{N}^4\to \mathbb{N}$ such that for all $n, k, b\in \mathbb{N}$ and $f:\mathbb{N}\to \mathbb{N}$:
%	$$ \W\alpha(f, n, k, b) = \begin{cases}
%	k - 3 & \text{if } b = 0 \vee f(n)\le k+3 \\ \W\alpha(\cdt{f}{a_k}\ , n, k+1, b-1) & \text{if } b \ge 1 \wedge f(n) > k+3
%	\end{cases} $$
%	where $a_0 = 2$, $a_1 = 0$ and $a_k = 1 \ \forall k\ge 2$.
%\end{defn}
%The function $\W\alpha$ takes a function $f$ and pretends it was the first level of the inverse hyperoperation hierarchy. It then keeps applying countdown to $f$, supposedly arrives at the $\text{m}^{\text{th}}$-level at the $\text{m}^{\text{th}}$-recursive step, until the budget $b$ is exhausted or $f(n)\le k+3$, i.e. \emph{early stopping}, and will output $k$. If at the beginning $k=3$ and $f = 2\angle{0}$, the early stopping condition becomes $2\angle{k}n \le k+3$, which when replace $n$ by $n+3$ gives what we need in \cref{thm: inv-hyperop-inv-ack}.
%The inverse Ackermann function can be defined as follows.
%\begin{thm} For all $n$, we have $\alpha(n) = \W\alpha(\lambda m.(m - 1), n+3, 3, n)$.
%\end{thm} 

\section{Time Bound of Our Inverses}
\label{sec: inv-ack}
\subsection{Implementations}
\begin{frame}
\frametitle{Implementation Idea}
\textbf{Redefinition.}
$
\alpha(n) = \min\{k: n\le A(k, k) \} \triangleq \min\{k: \alpha_k(n)\le k \}
$

\bigskip

\pause 
\textbf{The worker function.} $\begin{cases}
\text{Budget } b: & \hspace{-1em}\text{Maximum } b \text{ steps. } \\
\text{Step } i: & \hspace{-1em}\text{Compute } \alpha_i(n). \\
\text{Stops when: } & \hspace{-1em}\text{budget reaches } 0 \text{ or } \alpha_i(n) \le i.
\end{cases}$

%\textbf{The worker function.}
%\bigskip
%\begin{itemize}
%	\item Budget $b$: Maximum $b$ steps.
%	\item Step $i$: Computes $\alpha_i(n)$.
%	\item Stops when budget reaches $0$ or $\alpha_i(n) \le i$.
%\end{itemize}

\bigskip

\pause 
\textbf{Advantage.} $\alpha_{i+1}(n) = \cdt{\big(\alpha_i\big)}{1}\big(\alpha_i(n)\big)$\\
\smallskip
$\implies$ next step is computable from the previous.
\end{frame}


\begin{frame}[fragile]
\frametitle{The Worker Function}

\uncover<1->{$\displaystyle 	\alpha^{\W}(f, n, k, b) = \begin{cases}
k & \text{if } b = 0 \vee n\le k \\ \alpha^{\W}(\cdt{f}{1}\circ f , \cdt{f}{1}(n), k+1, b-1) & \text{if } b \ge 1 \wedge n \ge k+1
\end{cases}$}

\bigskip

\begin{overlayarea}{\linewidth}{3cm}
\begin{onlyenv}<2->
	\begin{lstlisting}
`\href{https://github.com/inv-ack/inv-ack/blob/7270e64a2600b771f2b1b1b151f7d13fb2ae6c97/inv_ack.v#L155-L161} {Fixpoint inv\_ack\_wkr}` (f : nat -> nat) (n k b : nat) : nat :=
match b with 0 => 0 | S b' =>
  if (n <=? k) then k else let g := (countdown_to f 1) in
                      inv_ack_wkr (compose g f) (g n) (S k) b
end.
\end{lstlisting}
\end{onlyenv}
\end{overlayarea}

\end{frame}



\begin{frame}
\frametitle{The Worker Function}

\textbf{Observation.}
\begin{equation*}
\begin{array}{lllllll}
Initial \ Arguments & & \big(\alpha_i, & \alpha_i(n), & i, & b - i \big) \\[5pt]
\pause
b - i > 0, \ \alpha_i(n) > i & \pause \to & \big(\cdt{\alpha_i}{1}\circ \alpha_i, & \cdt{\alpha_i}{1}(\alpha_i(n)), & i + 1, & b - i - 1\big) \\[5pt]
\pause
{\color{blue} b > i}, \ \hspace{15pt} \alpha_i(n) > i & \to  & \big({\color{blue} \alpha_{i+1}}, & {\color{blue} \alpha_{i+1}(n)}, & i + 1, & {\color{blue} b - (i + 1)}\big) \\
\end{array}
\end{equation*}

\pause
\textbf{Execution.}
\begin{equation*}
\begin{array}{l|lllllll}
Step & Initial \ Arguments &  &  \big(\alpha_0, & \alpha_0(n), & 0, & b - 0 \big) \\[3pt]
\pause
0 & b > 0, \ \alpha_0(n) > 0 & \pause \to & \big(\alpha_1, & \alpha_1(n), & 1, & b - 1\big) \\[3pt]
\pause
1 & b > 1, \ \alpha_1(n) > 1 & \pause \to  & \big(\alpha_2, & \alpha_2(n), & 2, & b - 2\big) \\[3pt]
\pause
\cdots & \cdots & \cdots & \multicolumn{4}{l}{\cdots} \\
%b > k-1, \ \alpha_{k-1}(n) > k - 1 & \to  & \big(\alpha_k, & \alpha_k(n), & k, & b - k\big) & k-1 \\
\pause
k & b > k, \ {\color{red} \alpha_{k}(n) \le k} & \pause \to  & \multicolumn{4}{l}{\impinline{$k = \alpha(n)$}}
\end{array}
\end{equation*}
\end{frame}



\begin{frame}
\frametitle{The Inverse Ackermann Function}

Any budget $b \ge \alpha(n)$ suffices, so we can choose $b := n$.

\bigskip

\pause
\textbf{First version:} $\bigO(n^2)$.
\pause
\begin{equation*}
\alpha(n) = \alpha^{\mathcal{W}}\big(\alpha_0, \alpha_0(n), 0, n\big)
= \alpha^{\mathcal{W}}\big(\lambda n(n - 1), n - 1, 0, n\big)
\end{equation*}

\smallskip

\pause
\textbf{Simple improvement:} $\bigO(n)$.
\pause
\begin{equation*}
\begin{aligned}
\alpha(n) & = \begin{cases}
\alpha^{\mathcal{W}}\big(\alpha_{{\color{red} 1}}, \alpha_{{\color{red} 1}}(n), {{\color{red} 1}}, n - {{\color{red} 1}}\big) & \hspace{3em}\text{ when } n > \Ack(0) \\
0 & \hspace{3em}\text{ when } n \le \Ack(0)
\end{cases} \\[5pt]
\pause & = \begin{cases}
\alpha^{\mathcal{W}}\big(\lambda n(n - 2), n - 2, 1, n - 1\big) & \text{ when } n > 1 \\
0 & \text{ when } n \le 1
\end{cases}
\end{aligned}
\end{equation*}

\end{frame}


\subsection{Time Complexity: $\bigO(n)$}

\begin{frame}
\frametitle{Runtime of Countdown}

\textbf{Sum by component in each recursive step.}
\begin{equation*}
\begin{array}{l|llllll}
Step & \multicolumn{2}{l}{Initial \ Arguments} &  & \big(f^{(0)}(n), & n - 0 \big) &  \\[3pt]
\pause 0    & n - 0 > 0,     & f^{(0)}(n) > a & \to & \big(f^{(1)}(n), & n - 1\big) & + 1 \\[3pt]
\pause 1    & n - 1 > 0, & f^{(1)}(n) > a & \to & \big(f^{(2)}(n), & n - 2\big) & + 2 \\[3pt]
\pause \cdots & \cdots & \cdots & \cdots & \cdots & \cdots & \cdots \\
k-1  & n - (k-1) > 0, & f^{(k-1)}(n) > a & \to & \big(f^{(k)}(n), & n - k\big) & + k \\[3pt]
\pause k & n - k \mathrel{?} 0, & {\color{red} f^{(k)}(n) \le a}  &\to & \multicolumn{2}{l}{0} &  \\[3pt] \hline
SUM & \pause \Theta(k) & \pause \Theta((a+1)k) \pause & &  \multicolumn{2}{c}{\sum_{i=0}^{k-1}\runtime_f\big(f^{i}(n)\big)} & \pause \Theta(k)
\end{array}
\end{equation*}

%TODO: Increase vertical space after \hline

\pause
Substitute $k = \cdt{f}{a}(n)$:
%\begin{equation*}
\imppar{$\displaystyle
\runtime_{\cdt{f}{a}}(n) = \sum_{i=0}^{\cdt{f}{a}(n)-1}\runtime_f\big(f^{i}(n)\big)
+ \Theta\left((a+1)\cdt{f}{a}(n)\right).
$}
%\end{equation*}
\end{frame}



\begin{frame}
\frametitle{Runtime of Each $\alpha_i$ - Asymptotic Bounds}

$\displaystyle
\alpha_{i+1} = \cdt{\big(\alpha_i\big)}{1}\circ \alpha_i
\quad \implies \quad \runtime_{\alpha_{i+1}}(n) = \runtime_{\alpha_i}(n) + \runtime_{\cdt{\alpha_i}{1}}\big(\alpha_i(n)\big)
$

\smallskip
\pause 
\imppar{
$\displaystyle
\runtime_{\alpha_{i+1}}(n) = \runtime_{\alpha_i}(n) + 
\sum_{i=0}^{\cdt{\alpha_i}{1}(\alpha_i(n)) - 1} \runtime_{\alpha_i}\left(\alpha_i^{(i+1)}(n)\right) + \Theta\left(\cdt{\alpha_i}{1}(\alpha_i(n)) \right)
$
}

\bigskip

\pause 
Substitute $\cdt{\alpha_i}{1}(\alpha_i(n))$ for $\alpha_{i+1}(n)$:
\bigskip

$ \displaystyle 
\runtime_{\alpha_{i+1}}(n)
  = \runtime_{\alpha_i}(n) + 
  \sum_{i=0}^{\alpha_{i+1}(n) - 1} \runtime_{\alpha_i}\left(\alpha_i^{(i+1)}(n)\right) + \Theta\left(\alpha_{i+1}(n) \right)
$

\pause 
\smallskip
\imppar{
$\displaystyle 
\runtime_{\alpha_{i+1}}(n)
	= \sum_{i=0}^{\alpha_{i+1}(n)} \runtime_{\alpha_i}\left(\alpha_i^{(i)}(n)\right) + \Theta\left(\alpha_{i+1}(n) \right)
$
}

\end{frame}


\begin{frame}
\frametitle{Runtime of Each $\alpha_i$ - Precise Bounds}

Countdown running time:
\begin{equation*}
\textstyle \runtime_{\cdt{f}{a}}(n) =
\sum_{i=0}^{\cdt{f}{a}(n) - 1} \runtime_f\left(f^{(i)}(n)\right)
+ (a + 2)\cdt{f}{a}(n) + f^{\left(\cdt{f}{a}(n)\right)}(n) + 1
\end{equation*}

\smallskip

\pause 
$\alpha_2$ and $\alpha_3$ running time: $\runtime_{\alpha_2}(n) \le 2n - 2$ and $\runtime_{\alpha_3}(n) \le 4n + 4$.

\bigskip 

\pause 
$\alpha_i$ running time:
\begin{equation*}
\textstyle \runtime_{\alpha_{i+1}}(n) \le \sum_{k=0}^{\alpha_{i+1}(n)}\runtime_{\alpha_i}\left( \alpha_i^{(k)}(n)\right) + 3\alpha_{i+1}(n) + 3
\end{equation*}

\smallskip

\pause 
\imppar{
\begin{minipage}{\linewidth}
	\textbf{Theorem.} $\forall i.~\runtime_{\alpha_i}(n) \le 4n + \left(19\cdot 2^{i-3} - 2i - 13\right)\log_2n + 2i = \bigO(n)$, when using $\alpha_1 \triangleq \lambda n. (n-2)$.
\end{minipage}}

\end{frame}



\begin{frame}
\frametitle{Runtime of Inverse Ackermann}

\begin{equation*}
\begin{array}{l|llllllll}
Step & \multicolumn{2}{l}{Initial \ Arguments} &  &  \big(\alpha_1, & \alpha_1(n), & 1, & b - 1 \big) \\[3pt]
\pause 1 & b - 1 > 0, & \alpha_1(n) > 1 & \to & \big(\alpha_2, & \alpha_2(n), & 2, & b - 2\big) \\[3pt]
2 & b - 2 > 0, & \alpha_2(n) > 2 & \to  & \big(\alpha_3, & \alpha_3(n), & 3, & b - 3\big) \\
\cdots & \cdots &\cdots & \cdots & \cdots & \cdots & \cdots \\
%b > k-1, \ \alpha_{k-1}(n) > k - 1 & \to  & \big(\alpha_k, & \alpha_k(n), & k, & b - k\big) & k-1 \\
k & b - k > 0, & {\color{red} \alpha_{k}(n) \le k} & \to  & \multicolumn{4}{l}{k = \alpha(n)} \\[3pt] \hline
\pause SUM & \pause \Theta(k) & \pause \Theta\big(\sum_{i=1}^{k}i\big) \pause & & \multicolumn{4}{l}{\sum_{i=1}^{k-1} \runtime_{\cdt{\alpha_i}{1}} (\alpha_i(n))} \\[3pt]
\pause = & \Theta(k) & \Theta(k^2) & & \multicolumn{4}{l}{\sum_{i=1}^{k-1} \big(\runtime_{\alpha_{i+1}}(n) - \runtime_{\alpha_i}(n) \big)}
\end{array}
\end{equation*}

\smallskip

\pause
Therefore, 
\imppar{$\displaystyle
\begin{aligned}
\runtime_{\alpha}(n)
& = \runtime_{\alpha_{\alpha(n)}}(n) - \runtime_{\alpha_1}(n) + \Theta\left(\alpha(n)^2\right) + \runtime_{\alpha_1}(n) \\
& = \bigO \left(n + 2^{\alpha(n)}\log_2n + \alpha(n)^2 \right) = \bigO (n)
\end{aligned}
$}

\end{frame}

\section{Inputs Encoded in Binary}
\label{sec: binary}
Thus far we have used the Coq type \li{nat}, which represents
a number~$n$ using~$n$ bits.
In contrast, the binary system represents~$n$ in $\lfloor \log_{2} n \rfloor + 1$ bits.
Coq comes with a built-in binary type \li{N}, 
which consists
of constructors \li{N0} and \li{Npos}. The latter, \li{Npos}, unfolds to \li{positive}:

\begin{lstlisting}
Inductive positive : Set :=
  | xI : positive -> positive 
  | xO : positive -> positive
  | xH : positive.
\end{lstlisting}

Constructor \li{xH} represents $1$, and constructors \li{xO} and \li{xI} represent
appending $0$ and $1$ respectively.
By always starting with $1$, \li{positive} bypasses
the issue of disambiguating \emph{e.g.} the strings \li{011} and
\li{00011}, which represent the same number but pose
a minor technical challenge.
To represent $0$, the type \li{N} simply uses the separate constructor \li{N0}.

In both \li{nat} and \li{N}, addition/subtraction of $b$-bit
numbers is $\Theta(b)$, while multiplication is $\Theta \big(b^2\big)$.
In general, arithmetic operations are often faster when the inputs
are encoded in binary. 
In this section we show that this advantage also extends to our techniques.

Our codebase has binary versions of
	\href{https://github.com/inv-ack/inv-ack/blob/7270e64a2600b771f2b1b1b151f7d13fb2ae6c97/bin_repeater.v\#L78-L87}{\color{blue}hyperoperations},
	\href{https://github.com/inv-ack/inv-ack/blob/7270e64a2600b771f2b1b1b151f7d13fb2ae6c97/bin_applications.v\#L30-L36}{\color{blue}inverse hyperoperations},
	\href{https://github.com/inv-ack/inv-ack/blob/7270e64a2600b771f2b1b1b151f7d13fb2ae6c97/bin_repeater.v\#L157-L175}{\color{blue}Ackermann}, and
	\href{https://github.com/inv-ack/inv-ack/blob/7270e64a2600b771f2b1b1b151f7d13fb2ae6c97/bin_inv_ack.v\#L335-L342}{\color{blue}inverse Ackermann}.
Here we show how to compute inverse Ackermann for binary inputs in
$\Theta(b)$ time, where $b$ is the bitlength,
\emph{i.e.} logarithmic time in the input magnitude.
As before, we present an intuitive sketch here and put
full proofs in 
Appendix~D of the extended version of this paper~\cite{extendedinvack}.
% Appendix~\ref{apx:time_analysis_bin}.

\begin{rem}
Although we do not prove it here, our general binary inverse hyperoperations are $O(b^2)$ time,
since Ackermann and base-2 hyperoperations benefit from $\Theta(1)$ division via bitshifts,
whereas general division is $O(b^2)$.
\end{rem}

\subsection{Countdown and contractions in binary}

\renewcommand{\Tleb}{\runtime_{\li{N.leb}}}
\renewcommand{\Tsucc}{\runtime_{\li{N.succ}}}

Although the theoretical \emph{countdown} is independent of the encoding
of its inputs, its Coq definition needs to be adjusted to allow for inputs
in \li{N}. The first step is to translate the arguments of
\li{countdown\_worker} from \li{nat} to \li{N}. Budget \li{b} must
remain in \li{nat} so it can serve as Coq's termination argument,
but all other \li{nat} arguments should be changed
to \li{N}, and functions on \li{nat} to functions on \li{N}.
% Linked by A
%\begin{lstlisting}
%`\href{https://github.com/inv-ack/inv-ack/blob/7270e64a2600b771f2b1b1b151f7d13fb2ae6c97/bin_countdown.v#L104-L109}{Fixpoint bin\_cdn\_wkr}` (f : N -> N) (a n : N) (b : nat) : N :=
%  match b with O => 0 | S b' =>
%    if (n <=? a) then 0 else 1 + bin_cdn_wkr f a (f n) b'
%  end.
%\end{lstlisting}
% Edited by Linh
\begin{lstlisting}
`\href{https://github.com/inv-ack/inv-ack/blob/7270e64a2600b771f2b1b1b151f7d13fb2ae6c97/bin_countdown.v#L104-L109}{\color{blue}Fixpoint bin\_cdn\_wkr}` f a n b : N :=
  match b with O => 0 | S b' =>
    if (n <=? a) then 0
      else 1 + bin_cdn_wkr f a (f n) b'
  end.
\end{lstlisting}

Determining the budget for \li{bin\_countdown\_to} is tricky.
A naïve approach is to use the built-in \li{nat} translation of \li{n},
\emph{i.e.} \li{N.to_nat n}. This is untenable as the translation alone
takes exponential time \emph{viz} the length of $n$'s representation.
We need a linear-time budget calculation for countdowns
of oft-used functions like $\lambda n.(n-2)$.

The key is to focus on functions that can bring their arguments below a threshold via repeated application in
\emph{logarithmic} time, thus allowing a log-sized budget for
\li{bin\_cdn\_wkr}. Simply shrinking by~$1$ is no longer good enough;
we need to halve the argument on every application as shown below:
\begin{defn} \label{defn: bin-contraction}
	$f\in \contract$ is
	\href{https://github.com/inv-ack/inv-ack/blob/7270e64a2600b771f2b1b1b151f7d13fb2ae6c97/bin_countdown.v#L37-L41}{\color{blue}\emph{binary strict above}}
	$a\in \mathbb{N}$ if $\forall n > a, f(n) \le \lfloor \frac{n + a}{2} \rfloor$.
\end{defn}
\noindent The key advantage of binary strict contractions is that if a contraction $f$ is binary strict above some $a$, then \lb
we know that $\forall n > a, \forall k.~f(n) \le \left\lfloor \frac{n - a}{2^k} \right\rfloor + a$.
Therefore, within $\lfloor \log_2 (n - a) \rfloor + 1$ applications of $f$, the result will become equal to or less than~$a$. We can choose this number as the budget for \li{bin\_cdn\_wkr} to successfully reach the countdown value before terminating.
Note that this budget is simply the length of the binary representation
of \li{n - a}, which we calculate using our function
\href{https://github.com/inv-ack/inv-ack/blob/7270e64a2600b771f2b1b1b151f7d13fb2ae6c97/bin_prelims.v#L135-L143}{\color{blue}\li{nat\_size}}. % Linked by A
The Coq definition of countdown on \li{N} is:
\hide{
\begin{lstlisting}
`\href{https://github.com/inv-ack/inv-ack/blob/7270e64a2600b771f2b1b1b151f7d13fb2ae6c97/bin_prelims.v#L135-L143}{\color{blue}Definition nat\_size}` (n : N) : nat :=
  match n with
  | 0 => 0%nat
  | Npos p => let fix nat_pos_size (x : positive) : nat :=
                  match x with
                  | xH => 1%nat
                  | xI y | xO y => S (nat_pos_size y)
                  end
              in nat_pos_size p
  end.
\end{lstlisting}
Note that \li{nat_size} outputs $0$ on $0$, and on any positive number $m$ the size of its binary representation, thus equals to $\lfloor \log_{2} m \rfloor + 1$.
}% end hide
% Linked by A
%\begin{lstlisting}
%`\href{https://github.com/inv-ack/inv-ack/blob/7270e64a2600b771f2b1b1b151f7d13fb2ae6c97/bin_countdown.v#L111-L112}{Definition bin\_countdown\_to}` (f : N -> N) (a n : N) : N :=
%  bin_cdn_wkr f a n (nat_size (n - a)).
%\end{lstlisting}
% Edited by Linh
\begin{lstlisting}
`\href{https://github.com/inv-ack/inv-ack/blob/7270e64a2600b771f2b1b1b151f7d13fb2ae6c97/bin_countdown.v#L111-L112}{\color{blue}Definition bin\_countdown\_to}` f a n :=
  bin_cdn_wkr f a n (nat_size (n - a)).
\end{lstlisting}

\pagebreak
\noindent The following is the binary version of Lemma~\ref{lem: cdt-runtime}:
%\begin{lem} \label{lem: cdt-runtime-bin}
%	$\forall n \in \li{N}$, if $f$ is a binary strict contraction above $a$,
%	\begin{equation*}
%	\runtime_{\cdt{f}{a}}(n) \le \sum_{i=0}^{\cdt{f}{a}(n) - 1} \hspace{-6pt}
%	\runtime_f\big(f^{(i)}(n)\big) \ + \ (\log_2a + 3)\left(\cdt{f}{a}(n) + 1\right) \ + \ 2\log_2n \ + \ \log_2\cdt{f}{a}(n)
%	\end{equation*}
%\end{lem}
% Edited by Linh
\begin{lem} \label{lem: cdt-runtime-bin}
	$\forall n \in \li{N} \ \forall a\in \li{N}$, if $f$ is a binary strict contraction above $a$,
	\begin{equation*}
	\begin{aligned}
	\runtime_{\cdt{f}{a}}(n) \ \le \ \sum_{i=0}^{\cdt{f}{a}(n) - 1} \hspace{-6pt}
	\runtime_f\big(f^{(i)}(n)\big) \ & + \ (\log_2a + 3)\left(\cdt{f}{a}(n) + 1\right) \\ 
  & + \ 2\log_2n \ + \ \log_2\cdt{f}{a}(n)
	\end{aligned}
	\end{equation*}
\end{lem}

%Substituting $a=1$ into \eqref{eq: cdt-runtime-bin} shows that \Cref{lem: inv-ack-hier-runtime} still holds.
%Similar to \Cref{sect: hardcode-lvl2}, the use of binaries is not immediately effective since the first level. We delve deeper into the hierarchy by \emph{hardcoding the $3^{\text{th}}$ level} and starts from there. Now
%\begin{equation*}
%\forall n, n+2  < n+3 < 2(n+2) \iff \forall n,
%\lfloor \log_2(n+2) \rfloor < \lceil \log_2(n+3) \rceil \le \lfloor \log_2(n+2) \rfloor + 1
%\end{equation*}
%This shift from floor to ceiling enables a direct computation, since $\lceil \log_2(n+3) \rceil$ can now be seen as the size of $(n+2)$'s binary representation.
%\begin{lstlisting}
%Definition alpha3 (n: N) : N := N.size (n+2) - 3.
%\end{lstlisting}
%Let $n\ge 2$. The computation of \li{N.size(n)} takes time equal to itself, so the above definition gives $\runtime(\alpha_3, n) \le 2\log_2n$. Fix an $i\ge 3$ and suppose $\runtime(\alpha_i, n) \le C_i\log_2n$. By \Cref{lem: inv-ack-hier-runtime},
\subsection{Inverse Ackermann in $\bigO\left(\log_2 n\right)$}
Our new Coq definition computes countdown
only for strict binary contractions. Fortunately, starting
from $n = 2$, the inverse hyperoperations $a\angle{n}b$ when $a\ge 2$
and the inverse Ackermann hierarchy $\alpha_n$ are all strict binary contractions.
We can construct these hierarchies by hardcoding their
first three levels and recursively building higher levels with \li{bin\_countdown\_to}.
Furthermore,
analagously to the optimization for \li{nat} discussed in~\S\ref{sect: hardcode-lvl2}, we hardcode an additional level.
% Linked by A
%\begin{lstlisting}
%`\href{https://github.com/inv-ack/inv-ack/blob/7270e64a2600b771f2b1b1b151f7d13fb2ae6c97/bin_inv_ack.v#L55-L60}{Fixpoint bin\_alpha}` (m : nat) (x : N) : N :=
%  match m with
%  | 0%nat => x - 1          | 1%nat => x - 2
%  | 2%nat => N.div2 (x - 2) | 3%nat => N.log2 (x + 2) - 2
%  | S m'  => bin_countdown_to (bin_alpha m') 1 (bin_alpha m' x)
%  end.
%\end{lstlisting}
% Edited by Linh
\begin{lstlisting}
`\href{https://github.com/inv-ack/inv-ack/blob/7270e64a2600b771f2b1b1b151f7d13fb2ae6c97/bin_inv_ack.v#L55-L60}{\color{blue}Fixpoint bin\_alpha}` (m : nat) (x : N) : N :=
  match m with
  | 0%nat => x - 1          
  | 1%nat => x - 2
  | 2%nat => N.div2 (x - 2) 
  | 3%nat => N.log2 (x + 2) - 2
  | S m'  => bin_countdown_to
               (bin_alpha m') 1 (bin_alpha m' x)
  end.
\end{lstlisting}
Note that for all $x$, $\li{N.div2}(x - 2) = \left\lfloor \frac{x - 2}{2} \right\rfloor = \left\lceil \frac{x - 3}{2} \right\rceil$ and $\li{N.log2}(x + 2) - 2 = \left\lfloor \log_2(x+2) \right\rfloor - 2 = \left\lceil \log_2(x+3) \right\rceil - 3$, so the above Coq definition is correct.

%\begin{thm} \label{thm: inv-ack-runtime-bin}
%	$\forall i,~\forall n,~\runtime_{\alpha_i}(n) \le 2\log_2n + \left(3\cdot 2^i - 3i - 13\right)\log_2\log_2n + 3i$.
%\end{thm}
% Edited by Linh
\begin{thm} \label{thm: inv-ack-runtime-bin}
	% For all $i$ and $n$,
	\begin{equation*}
	\forall i,n.~\runtime_{\alpha_i}(n) \le 2\log_2n + \left(3\cdot 2^i - 3i - 13\right)\log_2\log_2n + 3i.
	\end{equation*}
\end{thm}

\noindent For any level of the Ackermann hierarchy, this theorem demonstrates
a linear computation time up to the size of the representation of the input, \emph{i.e.} logarithmic time up to its magnitude $n$:
$\runtime_{\alpha_i}(n) = \bigO\big(\log_2n + 2^i\log_2\log_2n \big)$.

Moving on to inverse Ackermann itself, we follow a style nearly identical to that
in~\S\ref{subsec: inv_ack_hier}. For the worker, we simply translate to
\li{N}, keeping the budget in \li{nat} as described earlier.
The inverse Ackermann has an extra hardcoded level.
% Linked by A
%\begin{lstlisting}
%`\href{https://github.com/inv-ack/inv-ack/blob/7270e64a2600b771f2b1b1b151f7d13fb2ae6c97/bin_inv_ack.v#L326-L333}{Fixpoint bin\_inv\_ack\_wkr}` (f : N -> N) (n k : N) (b : nat) : N :=
%  match b with 0%nat  => k | S b' => if n <=? k then k else
%    let g := (bin_countdown_to f 1) in
%      bin_inv_ack_wkr (compose g f) (g n) (N.succ k) b'
%  end.
%
%`\href{https://github.com/inv-ack/inv-ack/blob/7270e64a2600b771f2b1b1b151f7d13fb2ae6c97/bin_inv_ack.v#L335-L342}{Definition bin\_inv\_ack}` (n : N) : N :=
%  if (n <=? 1) then 0 else if (n <=? 3) then 1
%    else if (n <=? 7) then 2 else
%      let f := (fun x => N.log2 (x + 2) - 2) in
%        bin_inv_ack_wkr f (f n) 3 (nat_size n).
%\end{lstlisting}
% Edited by Linh
\begin{lstlisting}
`\href{https://github.com/inv-ack/inv-ack/blob/7270e64a2600b771f2b1b1b151f7d13fb2ae6c97/bin_inv_ack.v#L326-L333}{\color{blue}Fixpoint bin\_inv\_ack\_wkr}` f n k b :=
  match b with 0%nat => k | S b' =>
    if n <=? k then k else
      let g := (bin_countdown_to f 1) in
        bin_inv_ack_wkr
          (compose g f) (g n) (N.succ k) b'
  end.

`\href{https://github.com/inv-ack/inv-ack/blob/7270e64a2600b771f2b1b1b151f7d13fb2ae6c97/bin_inv_ack.v#L335-L342}{\color{blue}Definition bin\_inv\_ack}` n :=
  if (n <=? 1) then 0 else if (n <=? 3) then 1
    else if (n <=? 7) then 2 else
      let f := (fun x => N.log2 (x + 2) - 2) in
        bin_inv_ack_wkr f (f n) 3 (nat_size n).
\end{lstlisting}
% Fixpoint bin_inv_ack_worker (f : N -> N) (n k : N) (b : nat) : N :=
%   match b with
%   | 0%nat  => k
%   | S b' => if n <=? k then k
%             else let g := (bin_countdown f 1) in
%                  bin_inv_ack_worker (compose g f) (g n) (N.succ k) b'
%   end.

% Definition bin_inv_ack (n : N) : N :=
%   if (n <=? 1) then 0
%   else if (n <=? 3) then 1
%        else if (n <=? 7) then 2
%             else let f := (bin_alpha 3) in
%                  bin_inv_ack_worker f (f n) 3 (nat_size n).
\noindent Note that, for $n > 7$, $n < \Ack\big(\lfloor \log_2n \rfloor + 1\big)$ $= \Ack\big(\li{nat_size}(n)\big)$, so a budget of $\li{nat_size}(n)$ suffices.
We show the
\href{https://github.com/inv-ack/inv-ack/blob/7270e64a2600b771f2b1b1b151f7d13fb2ae6c97/bin_inv_ack.v#L437-L472}
{\color{blue}correctness} and
\href{https://github.com/inv-ack/inv-ack/blob/195209ba895061fc51368c3c46c1d8760f05df50/bin_test_runtime_ocaml.ml#L359-L363}
{\color{blue}benchmark}
of \li{bin_inv_ack} in our codebase. Figure~\ref{fig:standalone_binary} 
shows a standalone binary-specific computation of the inverse 
Ackermann function that takes logarithmic 
time up to the input's magnitude.
This is simply an assimilation of the code snippets we have already discussed in 
this section, and serves as a 
binary translation of Figure~\ref{fig:standalone} shown earlier.

\begin{figure}
\lstset{style=myTinyStyle}
\begin{lstlisting}
Require Import Omega Program.Basics.
Open Scope N_scope.

`\href{https://github.com/inv-ack/inv-ack/blob/7270e64a2600b771f2b1b1b151f7d13fb2ae6c97/inv_ack_standalone.v#L56-L64}{\color{blue}Definition nat\_size}` n :=
  match n with
  | 0 => 0%nat
  | Npos p =>
      let fix nat_pos_size (x : positive) : nat :=
        match x with xH => 1%nat
        | xI y | xO y => S (nat_pos_size y) end
      in nat_pos_size p
  end.

`\href{https://github.com/inv-ack/inv-ack/blob/7270e64a2600b771f2b1b1b151f7d13fb2ae6c97/bin_countdown.v#L104-L109}{\color{blue}Fixpoint bin\_cdn\_wkr}` f a n b : N :=
  match b with O => 0 | S b' =>
    if (n <=? a) then 0
      else 1 + bin_cdn_wkr f a (f n) b'
  end.

`\href{https://github.com/inv-ack/inv-ack/blob/7270e64a2600b771f2b1b1b151f7d13fb2ae6c97/inv_ack_standalone.v#L75-L76}{\color{blue}Definition bin\_countdown\_to}` f a n := 
bin_cdn_wkr f a n (nat_size (n - a)).

`\href{https://github.com/inv-ack/inv-ack/blob/7270e64a2600b771f2b1b1b151f7d13fb2ae6c97/inv_ack_standalone.v#L97-L104}{\color{blue}Fixpoint bin\_inv\_ack\_wkr}` f n k b :=
  match b with 0%nat => k | S b' => 
    if n <=? k then k else 
      let g := (bin_countdown_to f 1) in
        bin_inv_ack_wkr
          (compose g f) (g n) (N.succ k) b'
  end.

`\href{https://github.com/inv-ack/inv-ack/blob/7270e64a2600b771f2b1b1b151f7d13fb2ae6c97/inv_ack_standalone.v#L111-L116}{\color{blue}Definition bin\_inv\_ack}` n :=
  if (n <=? 1) then 0 else if (n <=? 3) then 1
    else if (n <=? 7) then 2 else 
      let f := (fun x => N.log2 (x + 2) - 2) in
        bin_inv_ack_wkr f (f n) 3 (nat_size n).
\end{lstlisting}
\caption{A log-time Coq computation of the inverse Ackermann function for inputs represented in binary, \emph{i.e.} \li{N}.}
\label{fig:standalone_binary}
\end{figure}
\lstset{style=myStyle}

As in Theorem~\ref{thm: inv-ack-hardcode-correct}, the time complexity $\runtime_\alpha(n)$ is the sum of each component's runtime:
%\begin{equation*}
%\runtime_\alpha(n) =
%\begin{aligned}
%& \ \runtime_{\alpha_3}(n)
%+ \sum_{k = 3}^{\alpha(n) - 1} \hspace{-5pt} \runtime_{\cdt{\alpha_k}{1}\ }(\alpha_k(n))
%+ \sum_{k = 3}^{\alpha(n)}\Tleb\left(\alpha_k(n), k\right)
%+ \sum_{k = 3}^{\alpha(n) - 1} \hspace{-5pt} \Tsucc(k)\\
%& \ + \ \runtime_{\li{nat_size}}(n)
%\ + \Tleb(n, 1) \ + \Tleb(n, 3) \ + \Tleb(n, 7)
%\end{aligned}
%\end{equation*}
% Edited by Linh
\begin{equation*}
\begin{aligned}
\runtime_{\alpha_3}(n)
& \ + \sum_{k = 3}^{\alpha(n) - 1} \hspace{-5pt} \runtime_{\cdt{\alpha_k}{1}\ }(\alpha_k(n))
+ \sum_{k = 3}^{\alpha(n)}\Tleb\left(\alpha_k(n), k\right) \\
& \ + \sum_{k = 3}^{\alpha(n) - 1} \hspace{-5pt} \Tsucc(k)
\ + \ \runtime_{\li{nat_size}}(n) \\
& \ + \Tleb(n, 1) \ + \ \Tleb(n, 3) \ + \ \Tleb(n, 7)
\end{aligned}
\end{equation*}
With reference to Lemmas~47 (Appendix~C), 52 and 53 (Appendix~D), we have:
% With reference to Lemmas \ref{lem: compose-runtime} (Appendix~C), \ref{lem: leb-runtime-bin} and \ref{lem: succ-runtime-bin} \lb (Appendix~D), we have:
In the second summand, \lb $\runtime_{\cdt{\alpha_k}{1}\ }(\alpha_k(n)) = \runtime_{\alpha_{k+1}}(n) - \runtime_{\alpha_k}(n)$ for each $k$ by Lemma~47.
% \ref{lem: compose-runtime}.
By Lemma~52, 
% By Lemma~\ref{lem: leb-runtime-bin}, 
each $\Tleb$ in the third summand is $\Theta\left(\log_2k\right)$, totalling $\bigO\big(\alpha(n)\log_2\alpha(n)\big) = \text{o}\big(\log_2n\big)$.
The fourth summand is $\Theta(\alpha(n)) = \text{o}\big(\log_2n\big)$ by 
Lemma~53. 
% Lemma~\ref{lem: succ-runtime-bin}. 
The remaining items total $\Theta(\log_2n)$. Thus, $\forall n\ge 8$:
%\begin{equation*}
%\begin{aligned}
%\runtime_\alpha(n)
%& = \runtime_{\alpha_3}(n)
%+ \sum_{k = 3}^{\alpha(n) - 1} \big(\runtime_{\alpha_{k+1}}(n) - \runtime_{\alpha_{k}}(n)\big) + \Theta\left(\log_2 n\right)
%= \runtime_{\alpha_{\alpha(n)}}(n) + \Theta\big(\log_2 n\big) \\
%& = \bigO\big(\log_2n + 2^{\alpha(n)}\log_2\log_2n \big) + \Theta\big(\log_2n\big)
%= \Theta\big(\log_2n\big)
%\end{aligned}
%\end{equation*}
% Edited by Linh
\begin{equation*}
\begin{aligned}
\runtime_\alpha(n)
& = \runtime_{\alpha_3}(n)
+ \sum_{k = 3}^{\alpha(n) - 1} \big(\runtime_{\alpha_{k+1}}(n) - \runtime_{\alpha_{k}}(n)\big) + \Theta\left(\log_2 n\right) \\
& = \runtime_{\alpha_{\alpha(n)}}(n) + \Theta\big(\log_2 n\big) \\
& = \bigO\big(\log_2n + 2^{\alpha(n)}\log_2\log_2n \big) + \Theta\big(\log_2n\big) \\
& = \Theta\big(\log_2n\big)
\end{aligned}
\end{equation*}

\renewcommand{\Tleb}{\runtime_{\li{leb}}}
\renewcommand{\Tsucc}{\runtime_{\li{succ}}}

%We bound the RHS by bounding $\runtime\left(\alpha_{\alpha(n)}(n)\right)$, and $\runtime\left(\alpha_i(n)\right)$ in general. Note that by $\alpha_2(n) = \left\lfloor \frac{n - 2}{2} \right\rfloor$, we have $\runtime(\alpha_2, n) \le 2\log_2 n$ for all $n$.
%Note that this bound can potentially be improved by further tightening the above inequalities. Although we do not obtain an exact asymptotic runtime similar
%to Theorem~\ref{thm: inv-ack-hier-runtime-improved}, since this bound of $4^{\alpha(n)}\log_2n$ is strictly larger than the lower bound of $\log_2n$, it is still extremely small and can be bounded by simpler expressions such as $(\log_2n)^2$.
%Our result is an improved version of the inverse Ackermann function which runs in time $\bigO\big(4^{\alpha(n)}\log_2n\big)$.
%\begin{equation*}
%\alpha(n) \triangleq \begin{cases}
%0 & \text{if } n\le 1\\ 1 & \text{if } 2\le n\le 3 \\ 2 & \text{if } 4\le n\le 7 \\
%\W\alpha\left(\alpha_3,
%\alpha_3(n), 3, n\right) & \text{if } n\ge 8
%\end{cases} \quad \text{ where } \alpha_3 \triangleq \lambda m. \big(\lfloor \log_2(m+2) \rfloor - 2\big)
%\end{equation*} 

\section{Further Discussion}
\label{sec: discussion}
\newcommand{\ackt}{\ensuremath{\hat{\alpha}}}

\subsection{The Value of a Linear-Time Solution to the Hierarchy}

Our functions' linear runtimes can be understood in two distinct but
complementary ways.  A runtime less than the bitlength is impossible
without prior knowledge of the size of the input.  Accordingly, in
an information-theory or pure-mathematical sense, our definitions are
optimal up to constant factors.  And of course in practice, linear-time
solutions are highly usable in real computations.

Sublinear solutions are possible with \emph{a priori} knowledge about
the function and bounds on the inputs one will receive.
An extreme case is $\alpha(n)$, which has value $4$ for all practical
inputs greater than $61$. Accordingly,
this function can be inverted in $O(1)$ in practice.  That said, 
such solutions require external knowledge of the problems and
lookup tables within the algorithm to store precomputed
values, and thus fall more into the realm of engineering than mathematics. 

\begin{rem}
Rather surprisingly, for binary-encoded numbers, it appears we can invert the Ackermann function asymptotically faster than we can calculate the base-3 logarithm. 
\end{rem}

\subsection{Benchmarking our Inverses}

\begin{table}[t]
	\begin{centermath}
		\begin{array}{cl@{\hspace{8em}}c}
			& \li{n} & \text{time in }\li{OCaml} \\
			\cline{2-3}
			\cline{2-3}
			\parbox[t]{2mm}{\multirow{4}{*}{\rotatebox[origin=c]{90}{\li{n:nat}}}} 
			& 100 & 0.000020 \\
			& 1000 & 0.000926 \\
			& 10000 & 0.001450 \\
			& 100000 & 0.016406 \\	
			\cline{2-3}
			\parbox[t]{2mm}{\multirow{4}{*}{\rotatebox[origin=c]{90}{\li{n:N}}}} 
			& 2^{100} & 0.000012 \\
			& 2^{1000} & 0.000082 \\
			& 2^{10000} & 0.000959 \\
			& 2^{100000} & 0.014332 \\
		\end{array}
	\end{centermath}
	\caption{Benchmarking our results \\ (all times in seconds).}
	\label{table:benchmarking}
	\vspace{0em}
\end{table}

As shown in Table~\ref{table:benchmarking}, our functions are very rapid in practice.
Extracting our code to OCaml is straightforward: we import Coq's built-in \li{Extraction} module and then execute \li{Recursive Extraction inv_ack_linear.} We then benchmark our results using a machine running i5 2.4GHz with 8GB RAM.  The displayed times are
the average of three runs, and we suspect that garbage collection is the largest cause 
of variations in the wall-clock times.  Our functions run quickly
in Gallina as well, but exhibit unexpectedly rapid times when the inputs are 
large because Coq optimizes large \li{nat}s under the hood.

%we show the results of 
%extracting our linear-time inverse Ackermann functions to OCaml using Coq's built-in 
%\li{Extraction} utility and then running a few benchmarks.
%We made no further optimizations in OCaml, and the timings above
%are $3$-time averages of wallclock timings on 

\subsection{The Two-Parameter Inverse Ackermann Function}
Some authors~\cite{chazelle,tarjan} prefer a two-parameter inverse Ackermann function.
\begin{defn} \label{defn: 2para-alpha}
	The two-parameter inverse Ackermann function is defined as:
	\begin{equation} \label{eq: tmp-2para-alpha}
	\ackt (m, n) \triangleq \min\left\{i \ge 1 : \Ack\left(i, \left\lfloor \frac{m}{n} \right\rfloor \right)\ge \log_2n \right\}
	\end{equation}
\end{defn}
\noindent Note that $\ackt(n, n) \neq \alpha(n)$. However,
it is straightforward to modify our techniques to compute $\ackt(m, n)$.
\hide{This function arises from deep runtime analysis of the disjoint-set data structure. Tarjan \cite{tarjan} showed that, in the disjoint-set data structure, the time required $t(m,n)$ for a sequence of $m$ \textsc{\color{magenta}FIND}s intermixed with $n-1$ \textsc{\color{magenta}UNION}s (such that $m \geq n$) is bounded as: $k_{1}m\cdot\alpha(m,n) \leq t(m,n) \leq k_{2}m\cdot\alpha(m,n)$. In graph theory, Chazelle \cite{chazelle} showed that the minimum spanning tree of a connected graph with $n$ vertices and $m$ edges can be found in time $O(m\cdot\alpha(m,n))$. Computing this function is in fact easier than $\alpha(n)$, as when $m$ and $n$ are given, we are reduced to finding the minimum $i\ge 1$ such that $\Ack_i(s)\ge t$ for $s, t$ fixed, which can be done with the following variant of the \emph{inverse Ackermann worker}.
}
\begin{defn} \label{defn: 2para-inv-ack-worker}
	The {two-parameter inverse Ackermann worker},
	written $\ackt^{\W}$, is a function $\mathbb{N}^4\to \mathbb{N}$, defined by:
	\hide{$(\mathbb{N}\to \mathbb{N}) \times \mathbb{N}^3\to \mathbb{N}$ such that for all $n, k, b\in \mathbb{N}$ and $f:\mathbb{N}\to \mathbb{N}$:}
%	\begin{equation} \label{eq: 2para-inv-ack-worker-recursion}
%	\ackt^{\W}(f, n, k, b) = \begin{cases}
%	0 & \text{if } b = 0 \vee n\le k \\ 1 + \ackt^{\W}\big(\cdt{f}{1}\circ f , \cdt{f}{1}(n), k, b-1\big) & \text{if } b \ge 1 \wedge n \ge k+1
%	\end{cases}
%	\end{equation}
  \begin{equation} \label{eq: 2para-inv-ack-worker-recursion}
  \begin{aligned}
  & \ackt^{\W}(f, n, k, b) \\
  & \triangleq \begin{cases}
  0 & \text{if } b = 0 \vee n\le k \\ 1 + \ackt^{\W}\big(\cdt{f}{1}\circ f , \cdt{f}{1}(n), k, b-1\big) & \text{if } b \ge 1 \wedge n \ge k+1
  \end{cases}
  \end{aligned}
  \end{equation}
\end{defn}
%Similar to the one-parameter version, the following theorem establishes the correct setting for $\W\alpha_2$ to compute $\alpha(m, n)$.
%\begin{thm}
%	$\displaystyle \ackt(m, n) = 1 + \ackt^{\W}\left(\alpha_1, \alpha_1\big(\lceil\log_2n \rceil\big), \left\lfloor \frac{m}{n} \right\rfloor, \lceil\log_2n \rceil \right)$.
%\end{thm}
% Edited by Linh
\begin{thm} We can calculate $\ackt(m, n)$ from $\ackt^{\W}$:
	\begin{equation*}
	\displaystyle \forall m,n.~\ackt(m, n) = 1 + \ackt^{\W}\left(\alpha_1, \alpha_1\big(\lceil\log_2n \rceil\big), \left\lfloor \frac{m}{n} \right\rfloor, \lceil\log_2n \rceil \right).
	\end{equation*}
\end{thm}
\noindent We mechanize the above for both \href{https://github.com/inv-ack/inv-ack/blob/7270e64a2600b771f2b1b1b151f7d13fb2ae6c97/inv_ack.v#L245-L248}{\color{blue}unary} and \href{https://github.com/inv-ack/inv-ack/blob/7270e64a2600b771f2b1b1b151f7d13fb2ae6c97/bin_inv_ack.v#L222-L228}{\color{blue}binary} inputs.

\hide{
	\begin{proof}[Proof Sketch]
		It is easy to prove in a similar fashion to \cref{lem: inv-ack-worker-intermediate} that for all $n, b, k$ and $i$, if $\alpha_i(n) > k$ and $b > i$, then
		\begin{equation*}
		\W\alpha_2\big(\alpha_1, \alpha_1(b), k, b\big) = i + \W\alpha_2\big(\alpha_{i+1}, \alpha_{i+1}(n), k, b - i\big)
		\end{equation*}
		Now let $k \triangleq \lfloor m/n \rfloor$, $b \triangleq \lceil \log_2n \rceil$ and $l \triangleq \min\big\{i : \alpha_i(b)\le k\big\}$, which exists because $\Ack(i, \cdot)$ increases strictly with $i$. Then $\alpha(m, n) = \max{1, l}$. If $l = 0$, $\alpha_1(b) \le \alpha_0(b) \le k$, so $\W\alpha_2\big(\alpha_1, \alpha_1(b), k, b\big) = 0$, as desired. If $l \ge 1$,
		\begin{equation*}
		1 + \W\alpha_2\big(\alpha_1, \alpha_1(b), k, b\big)
		= 1 + l - 1 + \W\alpha_2\big(\alpha_l, \alpha_l(b), k, b-l\big) = l
		\end{equation*}
		Here $b\ge l$ due to the fact that $\Ack(b, k)\ge b$, so $\alpha_b(b)\le k$. This completes the proof.
\end{proof}}%end hide

\subsection{Implementing our Functions in Isabelle/HOL}

To show that our functions are independent of any 
Coq idiosyncrasies,
we implemented our functions in Isabelle/HOL.  Despite having 
no previous experience with Isabelle/HOL, it took us
only 2 hours to define our inverse Ackermann function.
Figure~\ref{fig:standalone_isabelle} is an Isabelle translation of 
Figure~\ref{fig:standalone}.

\lstset{style=isaStyle}
\begin{figure}
\begin{lstlisting}
theory inv_ack_standalone
  imports "HOL-Library.Log_Nat" HOL.Divides
begin

primrec cdn_wkr :: "(nat => nat) => nat => 
                   nat => nat => nat" where
  "cdn_wkr f a n 0 = 0" |
  "cdn_wkr f a n (Suc k) = 
    (if n <= a then 0 
  	           else Suc (cdn_wkr f a (f n) k))"

fun countdown_to :: "(nat => nat) => nat => 
                    nat => nat" where
  "countdown_to f a n = cdn_wkr f a n n"

primrec inv_ack_wkr :: "(nat => nat) => nat => 
                       nat => nat => nat" where
  "inv_ack_wkr f n k 0 = k" |
  "inv_ack_wkr f n k (Suc b) = 
  	(if n <= k then k
  	else let g = (countdown_to f 1) in
    inv_ack_wkr (g o f) (g n) (Suc k) b)"

fun inv_ack_linear :: "nat => nat" where
  "inv_ack_linear 0 = 0" |
  "inv_ack_linear (Suc 0) = 0" |
  "inv_ack_linear (Suc (Suc n)) = 
  	inv_ack_wkr (`$\lambda$` x. (x - 2)) n 1 (Suc n)"
 
end
\end{lstlisting}
\caption{A linear-time Isabelle computation of the inverse Ackermann function (inputs  in unary, \emph{i.e.} \li{nat}).}
\label{fig:standalone_isabelle}
\end{figure}

%\noindent Although we have not done the proofs about our functions in Isabelle, they only require first-order mathematics and so should pose no difficulties.

\begin{figure}
\begin{lstlisting}
definition ln :: "real => real" where
  "ln x = (THE u. exp u = x)"

definition log :: "[real,real] => real" where
  "log a x = ln x / ln a"

definition floorlog :: "nat => nat => nat" where
  "floorlog b a = (if a > 0 `$\wedge$` b > 1 
                   then nat `$\lfloor$`log b a`$\rfloor$` + 1 else 0)"

lemma compute_floorlog[code]:
  "floorlog b x = (if x > 0 `$\wedge$` b > 1 then 
              floorlog b (x div b) + 1 else 0)"
 by (* proof elided *)
\end{lstlisting}
\caption{The standard Isabelle technique for extractable discrete logarithm.}
\label{fig:isabelle_hack}
\end{figure}

Furthermore, our techniques are as \emph{applicable} to Isabelle as they are to Coq:
Isabelle offers \li{floorlog}, a discrete logarithm with arbitrary base, but 
\li{floorlog} is defined using~\li{ln}, the continous logarithm on $\mathbb{R}$.
This does not yield a computable function, and so the relevant 
Isabelle standard library
uses a ``hack'' (in the form of a \li{[code]} annotation) to generate a computable equivalent for extraction.
In Figure~\ref{fig:isabelle_hack} we present relevant extracts from the Isabelle standard libraries~\cite{isastan2013, isastan2019}.

This hack requires a pre-developed computational strategy 
for the \li{[code]} extraction-substitution lemma: there ain't no such thing 
as a free lunch~\cite{moonmistress}. Thus, \li{[code]}-extracting the inverse 
Ackermann would require the Isabelle/HOL functions presented in Figure~\ref{fig:standalone_isabelle} or some equivalent thereof. 
Further, we found no definitions of $\log^*$ 
or the inverse Ackermann function in the Isabelle documentation. Our technique 
gives directly extractable solutions to the entire hierarchy.
\lstset{style=myStyle}

% https://www.isa-afp.org/browser_info/current/AFP/IEEE_Floating_Point/Log_Nat.html
% https://isabelle.in.tum.de/website-Isabelle2013/dist/library/HOL/Log.html


\section{Related Work}
\label{sec:related}
\label{sec:related}

%\newcommand{\ackt}{\ensuremath{\hat{\alpha}}}

\subsection{The Value of a Linear-Time Solution to the Hierarchy}

Our functions' linear runtimes can be understood in two distinct but
complementary ways.  A runtime less than the bitlength is impossible
without prior knowledge of the size of the input.  Accordingly, in
an information-theory or pure-mathematical sense, our definitions are
optimal up to constant factors.  And of course in practice, linear-time
solutions are highly usable in real computations.

Sublinear solutions are possible with \emph{a priori} knowledge about
the function and bounds on the inputs one will receive.
An extreme case is $\alpha(n)$, which has value $4$ for all practical
inputs greater than $61$. Accordingly,
this function can be inverted in $O(1)$ in practice.  That said, 
such solutions require external knowledge of the problems and
lookup tables within the algorithm to store precomputed
values, and thus fall more into the realm of engineering than mathematics. 

\begin{rem}
Rather surprisingly, for binary-encoded numbers, it appears we can invert the Ackermann function asymptotically faster than we can calculate the base-3 logarithm. 
\end{rem}

\subsection{Benchmarking our Inverses}

\begin{table}[t]
	\begin{centermath}
		\begin{array}{cl@{\hspace{8em}}c}
			& \li{n} & \text{time in }\li{OCaml} \\
			\cline{2-3}
			\cline{2-3}
			\parbox[t]{2mm}{\multirow{4}{*}{\rotatebox[origin=c]{90}{\li{n:nat}}}} 
			& 100 & 0.000020 \\
			& 1000 & 0.000926 \\
			& 10000 & 0.001450 \\
			& 100000 & 0.016406 \\	
			\cline{2-3}
			\parbox[t]{2mm}{\multirow{4}{*}{\rotatebox[origin=c]{90}{\li{n:N}}}} 
			& 2^{100} & 0.000012 \\
			& 2^{1000} & 0.000082 \\
			& 2^{10000} & 0.000959 \\
			& 2^{100000} & 0.014332 \\
		\end{array}
	\end{centermath}
	\caption{Benchmarking our results \\ (all times in seconds).}
	\label{table:benchmarking}
	\vspace{0em}
\end{table}

As shown in Table~\ref{table:benchmarking}, our functions are very rapid in practice.
Extracting our code to OCaml is straightforward: we import Coq's built-in \li{Extraction} module and then execute \li{Recursive Extraction inv_ack_linear.} We then benchmark our results using a machine running i5 2.4GHz with 8GB RAM.  The displayed times are
the average of three runs, and we suspect that garbage collection is the largest cause 
of variations in the wall-clock times.  Our functions run quickly
in Gallina as well, but exhibit unexpectedly rapid times when the inputs are 
large because Coq optimizes large \li{nat}s under the hood.

%we show the results of 
%extracting our linear-time inverse Ackermann functions to OCaml using Coq's built-in 
%\li{Extraction} utility and then running a few benchmarks.
%We made no further optimizations in OCaml, and the timings above
%are $3$-time averages of wallclock timings on 

\subsection{The Two-Parameter Inverse Ackermann Function}
Some authors~\cite{chazelle,tarjan} prefer a two-parameter inverse Ackermann function.
\begin{defn} \label{defn: 2para-alpha}
	The two-parameter inverse Ackermann function is defined as:
	\begin{equation} \label{eq: tmp-2para-alpha}
	\ackt (m, n) \triangleq \min\left\{i \ge 1 : \Ack\left(i, \left\lfloor \frac{m}{n} \right\rfloor \right)\ge \log_2n \right\}
	\end{equation}
\end{defn}
\noindent Note that $\ackt(n, n) \neq \alpha(n)$. However,
it is straightforward to modify our techniques to compute $\ackt(m, n)$.
\hide{This function arises from deep runtime analysis of the disjoint-set data structure. Tarjan \cite{tarjan} showed that, in the disjoint-set data structure, the time required $t(m,n)$ for a sequence of $m$ \textsc{\color{magenta}FIND}s intermixed with $n-1$ \textsc{\color{magenta}UNION}s (such that $m \geq n$) is bounded as: $k_{1}m\cdot\alpha(m,n) \leq t(m,n) \leq k_{2}m\cdot\alpha(m,n)$. In graph theory, Chazelle \cite{chazelle} showed that the minimum spanning tree of a connected graph with $n$ vertices and $m$ edges can be found in time $O(m\cdot\alpha(m,n))$. Computing this function is in fact easier than $\alpha(n)$, as when $m$ and $n$ are given, we are reduced to finding the minimum $i\ge 1$ such that $\Ack_i(s)\ge t$ for $s, t$ fixed, which can be done with the following variant of the \emph{inverse Ackermann worker}.
}
\begin{defn} \label{defn: 2para-inv-ack-worker}
	The {two-parameter inverse Ackermann worker},
	written $\ackt^{\W}$, is a function $\mathbb{N}^4\to \mathbb{N}$, defined by:
	\hide{$(\mathbb{N}\to \mathbb{N}) \times \mathbb{N}^3\to \mathbb{N}$ such that for all $n, k, b\in \mathbb{N}$ and $f:\mathbb{N}\to \mathbb{N}$:}
%	\begin{equation} \label{eq: 2para-inv-ack-worker-recursion}
%	\ackt^{\W}(f, n, k, b) = \begin{cases}
%	0 & \text{if } b = 0 \vee n\le k \\ 1 + \ackt^{\W}\big(\cdt{f}{1}\circ f , \cdt{f}{1}(n), k, b-1\big) & \text{if } b \ge 1 \wedge n \ge k+1
%	\end{cases}
%	\end{equation}
  \begin{equation} \label{eq: 2para-inv-ack-worker-recursion}
  \begin{aligned}
  & \ackt^{\W}(f, n, k, b) \\
  & \triangleq \begin{cases}
  0 & \text{if } b = 0 \vee n\le k \\ 1 + \ackt^{\W}\big(\cdt{f}{1}\circ f , \cdt{f}{1}(n), k, b-1\big) & \text{if } b \ge 1 \wedge n \ge k+1
  \end{cases}
  \end{aligned}
  \end{equation}
\end{defn}
%Similar to the one-parameter version, the following theorem establishes the correct setting for $\W\alpha_2$ to compute $\alpha(m, n)$.
%\begin{thm}
%	$\displaystyle \ackt(m, n) = 1 + \ackt^{\W}\left(\alpha_1, \alpha_1\big(\lceil\log_2n \rceil\big), \left\lfloor \frac{m}{n} \right\rfloor, \lceil\log_2n \rceil \right)$.
%\end{thm}
% Edited by Linh
\begin{thm} We can calculate $\ackt(m, n)$ from $\ackt^{\W}$:
	\begin{equation*}
	\displaystyle \forall m,n.~\ackt(m, n) = 1 + \ackt^{\W}\left(\alpha_1, \alpha_1\big(\lceil\log_2n \rceil\big), \left\lfloor \frac{m}{n} \right\rfloor, \lceil\log_2n \rceil \right).
	\end{equation*}
\end{thm}
\noindent We mechanize the above for both \href{https://github.com/inv-ack/inv-ack/blob/7270e64a2600b771f2b1b1b151f7d13fb2ae6c97/inv_ack.v#L245-L248}{\color{blue}unary} and \href{https://github.com/inv-ack/inv-ack/blob/7270e64a2600b771f2b1b1b151f7d13fb2ae6c97/bin_inv_ack.v#L222-L228}{\color{blue}binary} inputs.

\hide{
	\begin{proof}[Proof Sketch]
		It is easy to prove in a similar fashion to \cref{lem: inv-ack-worker-intermediate} that for all $n, b, k$ and $i$, if $\alpha_i(n) > k$ and $b > i$, then
		\begin{equation*}
		\W\alpha_2\big(\alpha_1, \alpha_1(b), k, b\big) = i + \W\alpha_2\big(\alpha_{i+1}, \alpha_{i+1}(n), k, b - i\big)
		\end{equation*}
		Now let $k \triangleq \lfloor m/n \rfloor$, $b \triangleq \lceil \log_2n \rceil$ and $l \triangleq \min\big\{i : \alpha_i(b)\le k\big\}$, which exists because $\Ack(i, \cdot)$ increases strictly with $i$. Then $\alpha(m, n) = \max{1, l}$. If $l = 0$, $\alpha_1(b) \le \alpha_0(b) \le k$, so $\W\alpha_2\big(\alpha_1, \alpha_1(b), k, b\big) = 0$, as desired. If $l \ge 1$,
		\begin{equation*}
		1 + \W\alpha_2\big(\alpha_1, \alpha_1(b), k, b\big)
		= 1 + l - 1 + \W\alpha_2\big(\alpha_l, \alpha_l(b), k, b-l\big) = l
		\end{equation*}
		Here $b\ge l$ due to the fact that $\Ack(b, k)\ge b$, so $\alpha_b(b)\le k$. This completes the proof.
\end{proof}}%end hide

\subsection{Implementing our Functions in Isabelle/HOL}

To show that our functions are independent of any 
Coq idiosyncrasies,
we implemented our functions in Isabelle/HOL.  Despite having 
no previous experience with Isabelle/HOL, it took us
only 2 hours to define our inverse Ackermann function.
Figure~\ref{fig:standalone_isabelle} is an Isabelle translation of 
Figure~\ref{fig:standalone}.

\lstset{style=isaStyle}
\begin{figure}
\begin{lstlisting}
theory inv_ack_standalone
  imports "HOL-Library.Log_Nat" HOL.Divides
begin

primrec cdn_wkr :: "(nat => nat) => nat => 
                   nat => nat => nat" where
  "cdn_wkr f a n 0 = 0" |
  "cdn_wkr f a n (Suc k) = 
    (if n <= a then 0 
  	           else Suc (cdn_wkr f a (f n) k))"

fun countdown_to :: "(nat => nat) => nat => 
                    nat => nat" where
  "countdown_to f a n = cdn_wkr f a n n"

primrec inv_ack_wkr :: "(nat => nat) => nat => 
                       nat => nat => nat" where
  "inv_ack_wkr f n k 0 = k" |
  "inv_ack_wkr f n k (Suc b) = 
  	(if n <= k then k
  	else let g = (countdown_to f 1) in
    inv_ack_wkr (g o f) (g n) (Suc k) b)"

fun inv_ack_linear :: "nat => nat" where
  "inv_ack_linear 0 = 0" |
  "inv_ack_linear (Suc 0) = 0" |
  "inv_ack_linear (Suc (Suc n)) = 
  	inv_ack_wkr (`$\lambda$` x. (x - 2)) n 1 (Suc n)"
 
end
\end{lstlisting}
\caption{A linear-time Isabelle computation of the inverse Ackermann function (inputs  in unary, \emph{i.e.} \li{nat}).}
\label{fig:standalone_isabelle}
\end{figure}

%\noindent Although we have not done the proofs about our functions in Isabelle, they only require first-order mathematics and so should pose no difficulties.

\begin{figure}
\begin{lstlisting}
definition ln :: "real => real" where
  "ln x = (THE u. exp u = x)"

definition log :: "[real,real] => real" where
  "log a x = ln x / ln a"

definition floorlog :: "nat => nat => nat" where
  "floorlog b a = (if a > 0 `$\wedge$` b > 1 
                   then nat `$\lfloor$`log b a`$\rfloor$` + 1 else 0)"

lemma compute_floorlog[code]:
  "floorlog b x = (if x > 0 `$\wedge$` b > 1 then 
              floorlog b (x div b) + 1 else 0)"
 by (* proof elided *)
\end{lstlisting}
\caption{The standard Isabelle technique for extractable discrete logarithm.}
\label{fig:isabelle_hack}
\end{figure}

Furthermore, our techniques are as \emph{applicable} to Isabelle as they are to Coq:
Isabelle offers \li{floorlog}, a discrete logarithm with arbitrary base, but 
\li{floorlog} is defined using~\li{ln}, the continous logarithm on $\mathbb{R}$.
This does not yield a computable function, and so the relevant 
Isabelle standard library
uses a ``hack'' (in the form of a \li{[code]} annotation) to generate a computable equivalent for extraction.
In Figure~\ref{fig:isabelle_hack} we present relevant extracts from the Isabelle standard libraries~\cite{isastan2013, isastan2019}.

This hack requires a pre-developed computational strategy 
for the \li{[code]} extraction-substitution lemma: there ain't no such thing 
as a free lunch~\cite{moonmistress}. Thus, \li{[code]}-extracting the inverse 
Ackermann would require the Isabelle/HOL functions presented in Figure~\ref{fig:standalone_isabelle} or some equivalent thereof. 
Further, we found no definitions of $\log^*$ 
or the inverse Ackermann function in the Isabelle documentation. Our technique 
gives directly extractable solutions to the entire hierarchy.
\lstset{style=myStyle}

% https://www.isa-afp.org/browser_info/current/AFP/IEEE_Floating_Point/Log_Nat.html
% https://isabelle.in.tum.de/website-Isabelle2013/dist/library/HOL/Log.html



\subsection{Historical notes}
% This creates an unnumbered paragraph. ie a smaller, less flashy, header

%\marginpar{\tiny \color{blue} Multiplication, Division, Algorisms. Representations of numbers (Egyption fractions/Roman numerals/Decimal/Zero). Exponentiation, Logarithm, Tetration, Log*, ...   Hyperoperations, Knuth Arrows.  Inverses as a separate notation? Mechanizations of the above?}
The operations successor, predecessor, addition, and subtraction have
been integral to counting forever. The ancient Egyptian
number system used glyphs denoting $1$, $10$, $100$, \emph{etc.},
and expressed numbers using additive combinations of these.
The Roman system, which is still in use, is similar, but
it combines glyphs using both addition and subtraction. This buys brevity,
since \emph{e.g.} $9_{\text{roman}}$ is two characters, ``one less than ten'',
and not a series of nine $1$s.
The ancient Babylonian system was, like the modern Hindu-Arabic decimal system,
an \emph{algorism}: the place value of a glyph determined how many times it 
counted towards the number being represented.
The Babylonians operated in
base $60$, and so \emph{e.g.} a three-gylph number $abc_{\text{babylonian}}$ could
be parsed as $a \times 60^2 + b \times 60 + c$. Sadly they lacked
a radix point, and so
$a \times 60^3 + b \times 60^2 + c \times 60$, $a \times 60 + b + c \div 60$, 
\emph{etc.} were also reasonable interpretations. 
Incorporating multiplication and division bought great brevity: a number $n$ was 
represented in $\lfloor \log_{60}n \rfloor + 1$ glyphs.

%\subsection*{The Ackermann function and its inverse.}
% This creates an unnumbered subsection

\subsection{The Ackermann function and its inverse}
%\marginpar{\tiny \color{blue} Several variations. Original. Peter.
%Primitive recursive. Hilbert? Ackermann is used in CS. Formalizations that use or define it. The grit of sand. The bug.}
% https://projecteuclid.org/download/pdf_1/euclid.bams/1183512393
% https://www.cs.princeton.edu/~chazelle/pubs/mst.pdf
%{\color{magenta}A brief sentence explaining what a primitive recursive function is and
%why total computable functions tend to be primitive recursive.}
The three-variable Ackermann function was presented by
Wilhelm Ackermann as an example of a total computable function that
is not primitive recursive~\cite{ackermann}.
It does not have the higher-order
relation to repeated application and hyperoperation that we have been studying in
this paper. Those properties emerged thanks to refinements by Rózsa Péter~\cite{peter},
and it is her variant, usually called the Ackermann-Péter function, 
that computer scientists commonly care about.

%\marginpar{\tiny \color{blue}implemented with }
The inverse Ackermann
features in the time bound analyses of several algorithms.
Tarjan \cite{tarjan} showed that the union-find data structure
takes time $O(m\cdot\alpha(m,n))$ for a sequence of $m$ operations
involving no more than $n$ elements. 
%Tarjan \cite{tarjan} showed that the union-find data structure
%with the optimisations of \emph{path compression} and \emph{weighted union}
%takes time $O(m\cdot\alpha(n))$ for a sequence of $m$ operations
%involving no more than $n$ elements.
Chazelle \cite{chazelle} showed that the minimum spanning tree
of a connected graph with $n$ vertices and $m$ edges
can be found in time $O(m\cdot\alpha(m,n))$.

% http://gallium.inria.fr/~fpottier/publis/chargueraud-pottier-uf-sltc.pdf
% http://gallium.inria.fr/~agueneau/publis/gueneau-chargueraud-pottier-coq-bigO.pdf
% https://scholar.google.com.sg/scholar?start=0&hl=en&as_sdt=2005&sciodt=0,5&cites=6488308509111085774&scipsc=
Charguéraud and Pottier, later joined by Guéneau \cite{charpott,gueneauetal},
extended Separation Logic with ``time credits'',
formalized the~$O$ notation in Coq,
and verified the correctness
and the time complexity of the union-find data structure in Coq.
They formalized a version of the Ackermann function, but not its inverse. 
Others \cite{others2,others4,others3,others1}
have also checked bounds on the resources
used by programs formally in proof assistants such as Coq, Isabelle/HOL, and Why3.

{\color{red}The Coq standard library has linear-time definitions 
of division and base-$2$ discrete logarithm on \li{nat} and \li{N}.
The Mathematical Components library~\cite{MathComp} 
has \li{log} on \li{nat} with prime bases. 
To our knowledge, we are the first to generalize this 
problem, extend it both 
upwards and downwards in the natural hierarchy of functions, and
provide linear computations up to representation size.}

Every pearl starts with a grain of sand.  We had the benefit of two: 
Nivasch~\cite{nivasch} and Seidel~\cite{seidel}.
They proposed a definition of the inverse Ackermann essentially in terms of
the inverse hyperoperations.  Unfortunately, their technique is unsound, since it diverges from
the true Ackermann inverse when the inputs grow sufficiently large.  Our technique is verified in Coq.

\subsection{Conclusion}
We have implemented a hierarchy of functions that calculate the upper inverses
to the Ackermann/hyperoperation hierarchy and used these inverses
to compute the inverse of the diagonal Ackermann function $\Ack(n)$.
Our functions are structurally recursive, and are thus immediately accepted by Coq,
and we have shown that they run in linear time.
%, and that it is consistent with
%the usual definition of the inverse Ackermann function $\alpha(n)$.


%{\color{magenta}This is a direction
%we intend to explore to formally check the $O(n)$ bound
%of the inverse Ackermann function.}

%Cite: Ackermann, Peter, Tarjan, Chazelle, Pottier? Anything in HOL? Anything in SSReflect?

%\paragraph*{Alternative strategies}


% Other ways to skin the cat.
% - You can define division via mutual recursion (subtraction and division simultaenously).
% - The inverse ackerman-lite by Anshuman.
% - The automata technique.
% - Binary representations
% - Division by constant, etc. is simpler.
% - Custom termination metrics.  Gas.
% - Space, tail recursion, time?

%\paragraph*{Other?}

%% Acknowledgments
\begin{acks}
We thank Olivier Danvy for a helpful early-stage discussion about the 
difficulty of defining \li{div} in Coq.
This work was funded in part by
\grantsponsor{}{Yale-NUS College}{} grant~\mbox{\grantnum{}{R-607-265-322-121}}
and the
sponsors of the Crystal Center at National University of
Singapore. Any opinions, findings, conclusions, or recommendations expressed in 
this material are those of the authors and do not necessarily reflect the 
views of Yale-NUS College or the Crystal Centre.
\end{acks}

\appendix
\appendixpage

\section{Standalone Code for Linear-Time $\alpha$ in \li{nat}}
\label{apx:standalone_nat}
\begin{lstlisting}
Require Import Omega.
Require Import Program.Basics.

Fixpoint countdown_worker (f : nat -> nat) (a n k : nat) : nat :=
  match k with
  | 0    => 0
  | S k' => if (n <=? a) then 0 else
              S (countdown_worker f a (f n) k')
  end.

Definition countdown_to f a n := countdown_worker f a n n.

Fixpoint inv_ack_worker (f : nat -> nat) (n k b : nat) : nat :=
  match b with
  | 0    => k
  | S b' => match (n - k) with
            | 0   => k
            | _ => let g := (countdown_to f 1) in
                     inv_ack_worker (compose g f) (g n) (S k) b'
            end
  end.

Definition inv_ack_linear n :=
  match n with
  | 0 | 1 => 0
  | _     => let f := (fun x => x - 2) in inv_ack_worker f (f n) 1 (n - 1)
  end.
\end{lstlisting}

\section{Binary Numbers in Coq}
\label{apx:bin_in_coq}
Coq has support for binary numbers with the type \li{N}, which consists 
of constructors \li{0} and \li{positive}:
\begin{lstlisting}
Inductive positive : Set := 
  | xI : positive -> positive | xO : positive -> positive
  | xH : positive.
\end{lstlisting}
Constructor \li{xH} represents $1$, and constructors \li{xO} and \li{xI} represent 
appending $0$ and $1$ respectively. 
By always starting with $1$, \li{positive} dodges
the issue of disambiguating \emph{e.g.} the numbers \li{011} and 
\li{00011}, which represent the same number but pose
a minor technical challenge. 
To represent $0$, the type \li{N} simply provides a separate constructor \li{0}. 

\section{Proofs of time bounds lemmas on \li{nat}}
\label{apx:time_analysis}

\begin{proof}[Lemma \ref{lem: cdt-runtime-general}]
	Since $f\in \contract_{a}$, $\cdt{f}{a}\ $ is the minimum $k$ such that $f^{(k)}(n) \le a$. The execution of $\cdw{f}{a}\big(n, n\big)$ then takes $k+1$ recursive calls, where the $i^{th}$ call for $0\le i \le k$ takes the arguments $i, a$ and $n_i \triangleq f^{(i)}(n)$ from the previous call (or the initial argument when $i = 0$), and performs the following computations:
	\begin{enumerate}
		\item Compute $\li{leb}\left(n_i, a\right)$ for $\Tleb\left(f^{(i)}(n), a\right)$ steps
		\item If $\li{leb}\left(n_i, a\right) = \li{true}$, return $0$. Else proceed to the next step
		\item Compute $n_{i+1} \triangleq f(n_i) = f^{(i+1)}(n)$ for $\runtime_f\left(f^{(i)}(n)\right)$ steps
		\item Pass $n_{i+1}, i+1, a$ to the next recursive call and wait for it to return $k - i - 1$
		\item Add $1$ to the result for $\Tsucc(k-i-1)$ steps and return $k - i$
	\end{enumerate}
    Summing up the time of each call gives the desired expression for $\runtime_{\cdt{f}{a}}(n)$.
\end{proof}

\begin{proof}[Lemma \ref{lem: cdt-runtime}]
	Per Definition~\ref{defn: countdown} of countdown,

$\Tleb\left(f^{(i)}(n), a\right) = a + 1$ if $i < \cdt{f}{a}(n)$ and $f^{(i)}(n) + 1$ otherwise (\S\ref{apx:time_analysis}, Lemma~\ref{lem: sub-runtime}).  Thus,
the second summand in \eqref{eq: cdt-runtime-struct} is equal to $(a + 1)\cdt{f}{a}(n) + f^{\left(\cdt{f}{a}(n)\right)}(a) + 1$. Since $\Tsucc(i) = 1$ on \li{nat}, the third summand is equal to $\cdt{f}{a}(n)$, completing the desired formula.
%	Per Definition~\ref{defn: countdown-worker}, the computation makes $\cdt{f}{a}(n)$ recursive calls to $\W\cdt{f}{a}$ before terminating. At the $(i+1)^{\text{th}}$ call, two computations must take place: $n_i - a$, which takes $\Theta(a + 1)$ time, and $f(n_i) = n_{i+1}$, where $n_i \triangleq f^{(i)}(n)$ has been  computed by the $i$th call, and is greater than $a$.  The total time is then
%	\begin{equation*}
%	\begin{aligned}
%	\runtime\big(\cdt{f}{a}\ , n\big)
%	& = \sum_{i=0}^{\cdt{f}{a}(n) - 1} \left[\runtime\left(f, f^{(i)}(n)\right) + \Theta(a + 1)\right] \\
%	& = \sum_{i=0}^{\cdt{f}{a}(n) - 1} \runtime\left(f, f^{(i)}(n)\right) + \Theta\big((a + 1)\cdt{f}{a}(n)\big)
%	\end{aligned}
%	\end{equation*}
\end{proof}

\begin{proof}[Lemma \ref{lem:runtimealpha2}]
	$\alpha_0 = \lambda m.(m-1)$ so $\alpha_1 = \cdt{\big(\alpha_0\big)}{1}\circ \alpha_0 = \lambda m.(m - 2)$. By Lemma~\ref{lem: inv-ack-hier-runtime},
	\begin{equation*}
	\runtime_{\alpha_1}(n) \ge \textstyle \sum_{i=0}^{n-1} \runtime\big(\lambda m.(m-1), n - i\big) + 3(n - 2) + 1 = 4n - 5\text{,}
	\end{equation*}
	since $\runtime_{\alpha_0}(k) = 1$. Because $\alpha_2 = \cdt{\big(\alpha_1\big)}{1}\circ \alpha_1 = \lambda m.\left\lceil \frac{m-3}{2} \right\rceil$, again by Lemma~\ref{lem: inv-ack-hier-runtime},
	\begin{equation*}
	\runtime_{\alpha_2}(n)
	\ge \textstyle \sum_{i=0}^{\left\lceil \frac{n-3}{2} \right\rceil} \big(4(n-2i) - 5\big) + 3\left\lceil \frac{n-3}{2} \right\rceil + 1
	= \Theta\big(n^2\big)
	\end{equation*}
\end{proof}

\begin{proof}[Lemma \ref{lem: inv-ack-3-runtime}]
	It is easy to show that $\alpha_2^{(k)}(n) = \left\lfloor \frac{n+2}{2^k} \right\rfloor - 2$. Thus
	\begin{equation*}
	\begin{aligned}
		\runtime_{\alpha_3}(n)
		& \textstyle \le \ \sum_{k=0}^{\alpha_{3}(n)}\runtime_{\alpha_2}\left(\left\lfloor \frac{n+2}{2^k} \right\rfloor - 2\right) + 3\alpha_{3}(n) + 2 \\
		& \textstyle \le \ 2\sum_{k=0}^{\alpha_3(n)}\left(\frac{n+2}{2^k} - 3\right) + 3\alpha_3(n) + 2 \\
		& \le \ 4(n + 2) - 6(\alpha_3(n) + 1) + 3\alpha_3(n) + 2 \le 4n + 4.
	\end{aligned}
\end{equation*}
%	$\runtime_{\alpha_3}(n) \le $
%	\begin{equation*}
%\begin{array}{@{}l@{}}
%	\sum_{k=0}^{\alpha_{3}(n)}\runtime_{\alpha_2}\left(\left\lceil \frac{n+3}{2^k} \right\rceil - 3\right) + 3\alpha_{3}(n) + 2  \quad
%	\le \quad 2\sum_{k=0}^{\alpha_3(n)}\frac{n+3}{2^k} - 3\big(\alpha_3(n) + 1\big) + 3\alpha_3(n) + 2 \\
%	\le 4(n + 3) - 1 \le \quad 4n + 11 \\
%\end{array}
%	\end{equation*}
\end{proof}

\begin{proof}[Lemma \ref{lem:critlem2}]
	Define $S: \mathbb{R}_{\ge 0}\to \mathbb{R}_{\ge 0}$ where $S(x) \triangleq \sum_{k = 0}^{\log_2^*(x) - 1}\log_2^{(k)}x$. Clearly $S$ is strictly increasing and $\forall x>1, S(x) = n + S(\log_2 x)$. Using the fact $\log_2^*(x)\in \mathbb{N}$, we prove by induction on $k$ the statement $P(k) \triangleq$ $\forall x: \log_2^*(x) = k$, $S(x)\le 2x$.
	\begin{itemize}[leftmargin=*]
		\item \emph{Base case.} $P(0)$, $P(1)$ hold trivially and $P(2) = \forall x: 2 < x \le 4$, $x + \log_2x \le 2x$, which is equivalent to $\log_2x \le x$, which holds for all $x > 2$.
		\item \emph{Inductive case.} Assume $P(k-1)$ where $k\ge 3$. Fix any $x$ such that $\log_2^*(x)=~k$, then $x > 4$. The function $\lambda x.\frac{x}{\log_2x}$ is increasing on $[4, +\infty)$, so $\frac{4}{\log_24} \le \frac{x}{\log_2x}$ or $2\log_2x \le x$. Since $\log_2^*(x) = k$, we have $\log_2^*(\log_2x) = k-1$. By $P(k-1)$, $S(x) = x + S(\log_2x) \le x + 2\log_2x \le 2x$, which completes the proof.
	\end{itemize}
\end{proof}

\begin{proof}[Lemma \ref{lem: sum-alpha-repeat}]
%	Let the LHS be $S_i(n)$. Firstly, consider $i = 3$. Note that for $n\le 13$, $S_3(n) = 0$ and for $n\ge 14$, i.e. $\alpha_3(n)\ge 2$, $S_3(n) = \alpha_3(n) + S_3\big(\alpha_3(n)\big)$. The result thus holds for $n\le 13$. Suppose it holds for all $m < n$, where $n\ge 14$. Then
%	\begin{equation*}
%	S_3(n) \quad \le \quad \alpha_3(n) + 3\big\lceil \log_2(\alpha_3(n)) \big\rceil \quad \le \quad \big\lceil \log_2n \big\rceil + 3\big\lceil \log_2\log_2n \big\rceil
%	\end{equation*}
%	It is easy to prove \, $2\big\lceil \log_2\log_2n \big\rceil \le \big\lceil \log_2n \big\rceil$ by induction on $\big\lceil \log_2n \big\rceil$. Thus $S_3(n)~\le~3\big\lceil \log_2n \big\rceil$, as desired. Now for $i \ge 4$,
  $\forall i\ge 3, \forall n, \alpha_i(n)\le \log_2n$ and $\alpha_{i+1}(n)\le \log_2^*(n)-1$, therefore
	\begin{equation*}
%	S_i(n) \ = \
	\sum_{k=s}^{\alpha_{i+1}(n)} \log_2^{(l)}\alpha_i^{(k)}(n) \ \le \
%	\sum_{k=1}^{\log_2^*(n)-1} \log_2^{(l)}\alpha_i^{(k)}(n) \ \le \
	\sum_{k=s}^{\log_2^*(n)-1} \log_2^{(l+k)}(n) \ \le \
	2\log_2^{(l+s)} n
	\end{equation*}
%	Let $P(n) \triangleq 2\big\lceil \log_2\log_2n \big\rceil \le \big\lceil \log_2n \big\rceil$. It suffices to prove $P(n) \ \forall n$. Observe that $P(n)$ holds for $n\ge 4$.
\end{proof}
\begin{proof}[of Theorem~\ref{thm: inv-ack-hier-runtime-improved}]
	We have proved the result for $i = 0, 1, 2$. Let us proceed with induction on $i\ge 3$. The case $i = 3$ has been covered by Lemma~\ref{lem: inv-ack-3-runtime}. Let $M_i \triangleq 2^{i-3}19 - 2i - 13$ for each $i$ and suppose the result holds for $i\ge 3$. We have
	\begin{equation*}
	\begin{array}{@{}l@{}}
	 \runtime_{\alpha_{i+1}}(n) \le \sum_{k=0}^{\alpha_{i+1}(n)} \runtime_{\alpha_i}\big(\alpha_i^{(k)}(n)\big) + 3\alpha_{i+1}(n) + 2 \\[3pt]
	\le \sum_{k=0}^{\alpha_{i+1}(n)}\big(4\alpha_i^{(k)}(n) + M_i\log_2\alpha_i^{(k)}(n) + 2i \big) + 3\alpha_{i+1}(n) + 2 \\[3pt]
%	\le 4n + 2(i+1) + (2i + 3)\alpha_{i+1}(n) + 4\sum_{k=1}^{\alpha_{i+1}(n)}\alpha_i^{(k)}(n) + M_i\sum_{k=0}^{\alpha_{i+1}(n)}\log_2\alpha_i^{(k)}(n) \\[3pt]
  \le 4n + 2(i+1) + (2i + 3)\underbrace{\alpha_{i+1}(n)}_{\le \log_2n} + 4\underbrace{\textstyle \sum_{k=1}^{\alpha_{i+1}(n)}\alpha_i^{(k)}(n)}_{\le 2\log_2n} + M_i\underbrace{\textstyle \sum_{k=0}^{\alpha_{i+1}(n)}\log_2\alpha_i^{(k)}(n)}_{\le 2\log_2n} \\[3pt]
	\le 4n + 2(i+1) + (2M_i + 2i + 3 + 8)\log_2n
	= 4n + M_{i+1}\log_2n + 2(i+1)
%	\le 4n + M_i\left\lceil\log_2n\right\rceil + 5 + (M_i+2)\sum_{k=1}^{\alpha_{i+1}(n)}\alpha_i^{(k)}(n) + 6\alpha_{i+1}(n) \\
%	\le 2n + M_i\left\lceil\log_2n\right\rceil + 5 +
%	3(M_i + 2)\left\lceil\log_2n\right\rceil + 6\left\lceil\log_2n\right\rceil ~~
%	= ~~ 2n + (4M_i + 12)\left\lceil\log_2n\right\rceil + 5 \\
%	= 2n + M_{i+1}\left\lceil\log_2n\right\rceil + 5\text{, since $4M_i + 12 = 4^{i+2} - 16 + 12 = M_{i+1}$}.
	\end{array}
	\end{equation*}
\end{proof}

\begin{col}
	$\runtime(\alpha, n) = \Omega\big(n^2\big)$ per Definition~\ref{defn: inv-ack-worker}.
\end{col}


%% If you have bibdatabase file and want bibtex to generate the
%% bibitems, please use
%%
%\bibliographystyle{plainurl}% the mandatory bibstyle

% \bibliographystyle{ACM-Reference-Format}
\bibliography{inv-ack}

%% The Appendices part is started with the command \appendix;
%% appendix sections are then done as normal sections


%% else use the following coding to input the bibitems directly in the
%% TeX file.

%%\begin{thebibliography}{00}

%% \bibitem{label}
%% Text of bibliographic item

%%\end{thebibliography}
\end{document}
\endinput
%%
%% End of file `elsarticle-template-num.tex'.
