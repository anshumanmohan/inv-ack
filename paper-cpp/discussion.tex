\newcommand{\ackt}{\ensuremath{\hat{\alpha}}}

\subsection{The value of a linear-time solution to the hierarchy}

Our functions' linear runtimes can be understood in two distinct but
complementary ways.  A runtime less than the bitlength is impossible
without prior knowledge of the size of the input.  Accordingly, in
an information-theory or pure-mathematical sense, our definitions are
optimal up to constant factors.  And of course in practice, linear-time
solutions are highly usable in real computations.

Sublinear solutions are possible with \emph{a priori} knowledge about
the function and bounds on the inputs one will receive.
An extreme case is $\alpha(n)$, which has value $4$ for all practical
inputs greater than $61$. Accordingly,
this function can be inverted in $O(1)$ in practice.  That said, 
such solutions require external knowledge of the problems and
lookup tables within the algorithm to store precomputed
values, and thus fall more into the realm of engineering than mathematics. 

\subsection{The two-parameter inverse Ackermann function}
Some authors~\cite{chazelle,tarjan} prefer a two-parameter inverse Ackermann function.
\begin{defn} \label{defn: 2para-alpha}
	The two-parameter inverse Ackermann function is defined as:
	\begin{equation} \label{eq: tmp-2para-alpha}
	\ackt (m, n) \triangleq \min\left\{i \ge 1 : \Ack\left(i, \left\lfloor \frac{m}{n} \right\rfloor \right)\ge \log_2n \right\}
	\end{equation}
\end{defn}
Note that $\ackt(n, n)$ and the single-parameter $\alpha(n)$
are neither equal nor directly related, but
it is straightforward to modify our techniques to compute $\ackt(m, n)$.
\hide{This function arises from deep runtime analysis of the disjoint-set data structure. Tarjan \cite{tarjan} showed that, in the disjoint-set data structure, the time required $t(m,n)$ for a sequence of $m$ \textsc{\color{magenta}FIND}s intermixed with $n-1$ \textsc{\color{magenta}UNION}s (such that $m \geq n$) is bounded as: $k_{1}m\cdot\alpha(m,n) \leq t(m,n) \leq k_{2}m\cdot\alpha(m,n)$. In graph theory, Chazelle \cite{chazelle} showed that the minimum spanning tree of a connected graph with $n$ vertices and $m$ edges can be found in time $O(m\cdot\alpha(m,n))$. Computing this function is in fact easier than $\alpha(n)$, as when $m$ and $n$ are given, we are reduced to finding the minimum $i\ge 1$ such that $\Ack_i(s)\ge t$ for $s, t$ fixed, which can be done with the following variant of the \emph{inverse Ackermann worker}.
}
\begin{defn} \label{defn: 2para-inv-ack-worker}
	The {two-parameter inverse Ackermann worker}
	,written $\ackt^{\W}$, is a function $\mathbb{N}^4\to \mathbb{N}$, defined by:
	\hide{$(\mathbb{N}\to \mathbb{N}) \times \mathbb{N}^3\to \mathbb{N}$ such that for all $n, k, b\in \mathbb{N}$ and $f:\mathbb{N}\to \mathbb{N}$:}
%	\begin{equation} \label{eq: 2para-inv-ack-worker-recursion}
%	\ackt^{\W}(f, n, k, b) = \begin{cases}
%	0 & \text{if } b = 0 \vee n\le k \\ 1 + \ackt^{\W}\big(\cdt{f}{1}\circ f , \cdt{f}{1}(n), k, b-1\big) & \text{if } b \ge 1 \wedge n \ge k+1
%	\end{cases}
%	\end{equation}
  \begin{equation} \label{eq: 2para-inv-ack-worker-recursion}
  \begin{aligned}
  & \ackt^{\W}(f, n, k, b) \\
  & \triangleq \begin{cases}
  0 & \text{if } b = 0 \vee n\le k \\ 1 + \ackt^{\W}\big(\cdt{f}{1}\circ f , \cdt{f}{1}(n), k, b-1\big) & \text{if } b \ge 1 \wedge n \ge k+1
  \end{cases}
  \end{aligned}
  \end{equation}
\end{defn}
%Similar to the one-parameter version, the following theorem establishes the correct setting for $\W\alpha_2$ to compute $\alpha(m, n)$.
%\begin{thm}
%	$\displaystyle \ackt(m, n) = 1 + \ackt^{\W}\left(\alpha_1, \alpha_1\big(\lceil\log_2n \rceil\big), \left\lfloor \frac{m}{n} \right\rfloor, \lceil\log_2n \rceil \right)$.
%\end{thm}
% Edited by Linh
\begin{thm} For all $m$ and $n$,
	\begin{equation*}
	\displaystyle \ackt(m, n) = 1 + \ackt^{\W}\left(\alpha_1, \alpha_1\big(\lceil\log_2n \rceil\big), \left\lfloor \frac{m}{n} \right\rfloor, \lceil\log_2n \rceil \right).
	\end{equation*}
\end{thm}
We mechanize the above for both \href{https://github.com/inv-ack/inv-ack/blob/7270e64a2600b771f2b1b1b151f7d13fb2ae6c97/inv_ack.v#L245-L248}{unary} and \href{https://github.com/inv-ack/inv-ack/blob/7270e64a2600b771f2b1b1b151f7d13fb2ae6c97/bin_inv_ack.v#L222-L228}{binary} inputs in our codebase.

\hide{
	\begin{proof}[Proof Sketch]
		It is easy to prove in a similar fashion to \cref{lem: inv-ack-worker-intermediate} that for all $n, b, k$ and $i$, if $\alpha_i(n) > k$ and $b > i$, then
		\begin{equation*}
		\W\alpha_2\big(\alpha_1, \alpha_1(b), k, b\big) = i + \W\alpha_2\big(\alpha_{i+1}, \alpha_{i+1}(n), k, b - i\big)
		\end{equation*}
		Now let $k \triangleq \lfloor m/n \rfloor$, $b \triangleq \lceil \log_2n \rceil$ and $l \triangleq \min\big\{i : \alpha_i(b)\le k\big\}$, which exists because $\Ack(i, \cdot)$ increases strictly with $i$. Then $\alpha(m, n) = \max{1, l}$. If $l = 0$, $\alpha_1(b) \le \alpha_0(b) \le k$, so $\W\alpha_2\big(\alpha_1, \alpha_1(b), k, b\big) = 0$, as desired. If $l \ge 1$,
		\begin{equation*}
		1 + \W\alpha_2\big(\alpha_1, \alpha_1(b), k, b\big)
		= 1 + l - 1 + \W\alpha_2\big(\alpha_l, \alpha_l(b), k, b-l\big) = l
		\end{equation*}
		Here $b\ge l$ due to the fact that $\Ack(b, k)\ge b$, so $\alpha_b(b)\le k$. This completes the proof.
\end{proof}}%end hide

