
%\newcommand{\ackt}{\ensuremath{\hat{\alpha}}}

\subsection{The value of a linear-time solution to the hierarchy}

Our functions' linear runtimes can be understood in two distinct but
complementary ways.  A runtime less than the bitlength is impossible
without prior knowledge of the size of the input.  Accordingly, in
an information-theory or pure-mathematical sense, our definitions are
optimal up to constant factors.  And of course in practice, linear-time
solutions are highly usable in real computations.

Sublinear solutions are possible with \emph{a priori} knowledge about
the function and bounds on the inputs one will receive.
An extreme case is $\alpha(n)$, which has value $4$ for all practical
inputs greater than $61$. Accordingly,
this function can be inverted in $O(1)$ in practice.  That said, 
such solutions require external knowledge of the problems and
lookup tables within the algorithm to store precomputed
values, and thus fall more into the realm of engineering than mathematics. 

\subsection{The two-parameter inverse Ackermann function}
Some authors~\cite{chazelle,tarjan} prefer a two-parameter inverse Ackermann function.
\begin{defn} \label{defn: 2para-alpha}
	The two-parameter inverse Ackermann function is defined as:
	\begin{equation} \label{eq: tmp-2para-alpha}
	\ackt (m, n) \triangleq \min\left\{i \ge 1 : \Ack\left(i, \left\lfloor \frac{m}{n} \right\rfloor \right)\ge \log_2n \right\}
	\end{equation}
\end{defn}
Note that $\ackt(n, n)$ and the single-parameter $\alpha(n)$
are neither equal nor directly related, but
it is straightforward to modify our techniques to compute $\ackt(m, n)$.
\hide{This function arises from deep runtime analysis of the disjoint-set data structure. Tarjan \cite{tarjan} showed that, in the disjoint-set data structure, the time required $t(m,n)$ for a sequence of $m$ \textsc{\color{magenta}FIND}s intermixed with $n-1$ \textsc{\color{magenta}UNION}s (such that $m \geq n$) is bounded as: $k_{1}m\cdot\alpha(m,n) \leq t(m,n) \leq k_{2}m\cdot\alpha(m,n)$. In graph theory, Chazelle \cite{chazelle} showed that the minimum spanning tree of a connected graph with $n$ vertices and $m$ edges can be found in time $O(m\cdot\alpha(m,n))$. Computing this function is in fact easier than $\alpha(n)$, as when $m$ and $n$ are given, we are reduced to finding the minimum $i\ge 1$ such that $\Ack_i(s)\ge t$ for $s, t$ fixed, which can be done with the following variant of the \emph{inverse Ackermann worker}.
}
\begin{defn} \label{defn: 2para-inv-ack-worker}
	The {two-parameter inverse Ackermann worker}
	,written $\ackt^{\W}$, is a function $\mathbb{N}^4\to \mathbb{N}$, defined by:
	\hide{$(\mathbb{N}\to \mathbb{N}) \times \mathbb{N}^3\to \mathbb{N}$ such that for all $n, k, b\in \mathbb{N}$ and $f:\mathbb{N}\to \mathbb{N}$:}
%	\begin{equation} \label{eq: 2para-inv-ack-worker-recursion}
%	\ackt^{\W}(f, n, k, b) = \begin{cases}
%	0 & \text{if } b = 0 \vee n\le k \\ 1 + \ackt^{\W}\big(\cdt{f}{1}\circ f , \cdt{f}{1}(n), k, b-1\big) & \text{if } b \ge 1 \wedge n \ge k+1
%	\end{cases}
%	\end{equation}
  \begin{equation} \label{eq: 2para-inv-ack-worker-recursion}
  \begin{aligned}
  & \ackt^{\W}(f, n, k, b) \\
  & \triangleq \begin{cases}
  0 & \text{if } b = 0 \vee n\le k \\ 1 + \ackt^{\W}\big(\cdt{f}{1}\circ f , \cdt{f}{1}(n), k, b-1\big) & \text{if } b \ge 1 \wedge n \ge k+1
  \end{cases}
  \end{aligned}
  \end{equation}
\end{defn}
%Similar to the one-parameter version, the following theorem establishes the correct setting for $\W\alpha_2$ to compute $\alpha(m, n)$.
%\begin{thm}
%	$\displaystyle \ackt(m, n) = 1 + \ackt^{\W}\left(\alpha_1, \alpha_1\big(\lceil\log_2n \rceil\big), \left\lfloor \frac{m}{n} \right\rfloor, \lceil\log_2n \rceil \right)$.
%\end{thm}
% Edited by Linh
\begin{thm} For all $m$ and $n$,
	\begin{equation*}
	\displaystyle \ackt(m, n) = 1 + \ackt^{\W}\left(\alpha_1, \alpha_1\big(\lceil\log_2n \rceil\big), \left\lfloor \frac{m}{n} \right\rfloor, \lceil\log_2n \rceil \right).
	\end{equation*}
\end{thm}
We mechanize the above for both \href{https://github.com/inv-ack/inv-ack/blob/7270e64a2600b771f2b1b1b151f7d13fb2ae6c97/inv_ack.v#L245-L248}{\color{blue}unary} and \href{https://github.com/inv-ack/inv-ack/blob/7270e64a2600b771f2b1b1b151f7d13fb2ae6c97/bin_inv_ack.v#L222-L228}{\color{blue}binary} inputs in our codebase.

\hide{
	\begin{proof}[Proof Sketch]
		It is easy to prove in a similar fashion to \cref{lem: inv-ack-worker-intermediate} that for all $n, b, k$ and $i$, if $\alpha_i(n) > k$ and $b > i$, then
		\begin{equation*}
		\W\alpha_2\big(\alpha_1, \alpha_1(b), k, b\big) = i + \W\alpha_2\big(\alpha_{i+1}, \alpha_{i+1}(n), k, b - i\big)
		\end{equation*}
		Now let $k \triangleq \lfloor m/n \rfloor$, $b \triangleq \lceil \log_2n \rceil$ and $l \triangleq \min\big\{i : \alpha_i(b)\le k\big\}$, which exists because $\Ack(i, \cdot)$ increases strictly with $i$. Then $\alpha(m, n) = \max{1, l}$. If $l = 0$, $\alpha_1(b) \le \alpha_0(b) \le k$, so $\W\alpha_2\big(\alpha_1, \alpha_1(b), k, b\big) = 0$, as desired. If $l \ge 1$,
		\begin{equation*}
		1 + \W\alpha_2\big(\alpha_1, \alpha_1(b), k, b\big)
		= 1 + l - 1 + \W\alpha_2\big(\alpha_l, \alpha_l(b), k, b-l\big) = l
		\end{equation*}
		Here $b\ge l$ due to the fact that $\Ack(b, k)\ge b$, so $\alpha_b(b)\le k$. This completes the proof.
\end{proof}}%end hide



The Coq standard library has linear-time definitions
of division and base-$2$ discrete logarithm on \li{nat} and \li{N}.
The Mathematical Components library~\cite{MathComp}
has a discrete logarithm with arbitrary base, with inputs encoded in \li{nat}.
The Isabelle/HOL Archive of Formal Proofs~\cite{isastan2019} 
provides a definition of discrete logarithm
with arbitrary base along with a separTate computation strategy (\lstset{style=isaStyle}\li{[code]}\lstset{style=myStyle}).
None of these libraries provides a definition of iterated logarithm or
further members of the hierarchy.
Indeed, to our knowledge, we are the first to generalize this
problem in a proof assistant, extend it both
upwards and downwards in the natural hierarchy of functions, and
provide linear-time computations.

\subsection{Historical Notes}
The operations of successor, predecessor, addition, and subtraction have
been integral to counting forever. The ancient Egyptian
number system used glyphs denoting $1$, $10$, $100$, \emph{etc.},
and expressed numbers using additive combinations of these.
The Roman system is similar, but
combines glyphs using both addition and subtraction.
This buys brevity and readability,
since \emph{e.g.} $9_{\text{roman}}$ is two characters, ``one less than ten'',
and not a series of nine $1$s.
The ancient Babylonian system was the first to introduce 
\emph{algorisms}: the place value of a glyph succinctly expressed \emph{how many times} 
it counted towards the number being represented, along with associated techniques for addition, multiplication, \emph{etc.}
The Babylonians operated in
base $60$, and so \emph{e.g.} a three-gylph number $abc_{\text{babylonian}}$ could
be parsed as $a \times 60^2 + b \times 60 + c$. Sadly they lacked
a radix point, and so
$a \times 60^3 + b \times 60^2 + c \times 60$, $a \times 60 + b + c \div 60$,
\emph{etc.} were also reasonable interpretations, and the correct number had
to be inferred from context.
Using multiplication and division in numerical representations bought great concision: a number $n$ was
represented in $\lfloor \log_{60}n \rfloor + 1$ glyphs.
The modern Indo-Arabic decimal system is also an algorism, 
but operates in base $10$ and (thankfully) has a radix point.

A form of exponentiation was discussed by Euclid (\textasciitilde 300 BCE), and our modern
understanding of it, including the superscript notation,
came in 1637 thanks to René Descartes~\cite{descartes}. 
Similarly, logarithms were mentioned by 
Archimedes (\textasciitilde 250 BCE) and our modern understanding came in 1614 thanks to 
John Napier~\cite{napier}.

The three-variable Ackermann function was presented by Wilhelm Ackermann~\cite{ackermann} in 1928 as an example of a total computable function that is not primitive recursive.
In 1935, Rózsa Péter~\cite{peter} developed a two-argument variant of the Ackermann 
function, and it is her variant, often called the Ackermann-Péter function,
that computer scientists---the authors included---commonly care about.
In 1947, Reuben Goodstein~\cite{goodstein} showed that a variant of the Ackermann function 
can be used to place the natural sequence of functions (addition, multiplication,
exponentiation) in a systematic hierarchy of hyperoperations. 
This brought tetration, the fourth member of the sequence, into use.
In the 1980s, computer scientists started using the
iterated logarithm $\log^*$ in algorithmic analysis.

\subsection{Inverse Ackermann in Computer Science}
% Ackermann's original function does not have the higher-order
% relation to repeated application and hyperoperation that we have been studying in
% this paper. Those properties emerged thanks to refinements by ,

The inverse of the Ackermann function
features in the time bound analyses of several algorithms.
Tarjan~\cite{tarjan} showed that the union-find data structure
takes time \mbox{$O(m\cdot\alpha(m,n))$} for a sequence of $m$ operations
involving no more than $n$ elements.
Tarjan and van Leeuwen~\cite{tarjan2} later refined this to $O(m\cdot\alpha(n))$.
Chazelle~\cite{chazelle} showed that the minimum spanning tree
of a connected graph with $n$ vertices and $m$ edges
can be found in time $O(m\cdot\alpha(m,n))$.

Chargu\'eraud and Pottier~\cite{charpott}
verify the time complexity of union-find in Coq.
% Their analysis requires inverse Ackermann, and 
To this end they invert another variant of the Ackermann function 
(\emph{i.e.} not the Ackermann-Péter function we have 
studied). 
However, their inversion strategy 
relies on Hilbert's non-constructive~$\varepsilon$ operator 
to choose the necessary minimum (see Definition~\ref{defn: inv_ack}). 
% The~$\varepsilon$ operator blurs the distinction between \li{Prop} and \li{Type}, 
This collapses Coq's distinction between \li{Prop} and \li{Type}, meaning that
Chargu\'eraud and Pottier's 
inverse cannot be extracted to executable code.  Moreover, 
Coq tactics such as \li{compute} cannot reduce applications of
their inverse to concrete values.  In contrast, with our technique,
proving a goal such as \li{inv_ack_linear 100 = 4} is as simple as 
using \li{reflexivity}, and the computation is essentially instantaneous.

%Every pearl starts with a grain of sand.  We had the benefit of two:
%Nivasch~\cite{nivasch} and Seidel~\cite{seidel}.
%They proposed a definition of the inverse Ackermann essentially in terms of
%the inverse hyperoperations, \emph{i.e.} along the lines of our Theorem~\ref{thm: inv-ack-new}: $\forall n.~ \alpha(n) \stackrel{?}{=} \Theta(\min\big\{k : 2 \langle k \rangle n\le 3 \big\})$.  Unfortunately, their technique is unsound (it should be $\le k$), since it diverges from
%the true Ackermann inverse when the inputs grow sufficiently large; even adjusting for this, their technique is off by an additive constant.  Our technique is exact and verified in Coq.

\section{Conclusion}
We have implemented a hierarchy of functions that calculate the upper inverses
to the hyperoperation/Ackermann hierarchy and used these inverses
to compute the inverse of the diagonal Ackermann function~$\Ack(n)$.
Our functions run in~$\Theta(b)$ time on unary-represented inputs.  On
binary-represented inputs, our \mbox{base-2} hyperoperations and inverse Ackermann
run in~$\Theta(b)$ time as well, where~$b$ is bitlength; our general binary 
hyperoperations run in~$O(b^2)$.
Our functions are structurally recursive,
and thus easily satisfy Coq's termination checker.

Every pearl starts with a grain of sand.  We had the benefit of two:
Nivasch~\cite{nivasch} and Seidel~\cite{seidel}.
They proposed a definition of the inverse Ackermann essentially in terms of
the inverse hyperoperations.  However, their definition is approximate rather than exact (\emph{i.e.} it has the same asymptotic rate of growth as the canonical definition but is off by a bounded-but-varying constant factor) and they provide only a sketch of an algorithm.  Our technique is exact; our algorithm concrete, structurally recursive and asymptotically optimal; and our theory is verified in Coq.

%, and that it is consistent with
%the usual definition of the inverse Ackermann function $\alpha(n)$.


%{\color{magenta}This is a direction
%we intend to explore to formally check the $O(n)$ bound
%of the inverse Ackermann function.}

%Cite: Ackermann, Peter, Tarjan, Chazelle, Pottier? Anything in HOL? Anything in SSReflect?

%\paragraph*{Alternative strategies}


% Other ways to skin the cat.
% - You can define division via mutual recursion (subtraction and division simultaenously).
% - The inverse ackerman-lite by Anshuman.
% - The automata technique.
% - Binary representations
% - Division by constant, etc. is simpler.
% - Custom termination metrics.  Gas.
% - Space, tail recursion, time?

%\paragraph*{Other?} 