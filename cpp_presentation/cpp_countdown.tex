\subsection{Inverting Repeater with Countdown}

%\begin{frame}
%\frametitle{Repeatable Functions}
%
%Functions in the Ackermann and hyperoperation (when $a\ge 2$) hierarchies are all \textcolor{red}{\emph{repeatable function}}.
%
%\bigskip
%%
%%\pause
%%\textbf{Expansions.} A function $F:\mathbb{N}\to\mathbb{N}$ is an
%%\href{https://github.com/inv-ack/inv-ack/blob/7270e64a2600b771f2b1b1b151f7d13fb2ae6c97/increasing_expanding.v\#L80-L82}{\emph{expansion}} if $\forall n.~ F(n)\ge n$. Further, for $a\in \mathbb{N}$, an expansion $F$ is
%%\href{https://github.com/inv-ack/inv-ack/blob/7270e64a2600b771f2b1b1b151f7d13fb2ae6c97/increasing_expanding.v\#L84-L86}{\emph{strict from}} $a$ if ~$\forall n \ge a.~ F(n) > n$.
%%
%%\bigskip
%
%\pause
%\textbf{Repeatable functions:} A $F$ is \emph{repeatable} from $a$ if\\
%$F$ is strictly increasing and $F$ is an
%\href{https://github.com/inv-ack/inv-ack/blob/7270e64a2600b771f2b1b1b151f7d13fb2ae6c97/increasing_expanding.v\#L80-L82}{\emph{expansion}}
%that is
%\href{https://github.com/inv-ack/inv-ack/blob/7270e64a2600b771f2b1b1b151f7d13fb2ae6c97/increasing_expanding.v\#L84-L86}{\emph{strict from}}
%$a$, i.e. ~$\forall n \ge a.~ F(n) > n$.
%
%\bigskip
%\pause
%We extend our scope of study from functions in the hyperoperations and Ackermann hierarchies to repeatable functions.
%
%\bigskip
%\pause
%\textbf{Advantage.} If $F$ is repeatable from $a$, $F^{-1}$ makes sense and is total,
%\\ and $\rf{F}{a}$ is repeatable from $1$.
%
%\end{frame}
%
%
%
%\begin{frame}
%\frametitle{Inverting Repeater: The Idea of Countdown}
%
%The \href{https://github.com/inv-ack/inv-ack/blob/7270e64a2600b771f2b1b1b151f7d13fb2ae6c97/inverse.v\#L28-L45}{
%	\emph{upper inverse}} of $F$, written $F^{-1}$,
%% {\color{red} $F^{-1}$, $F^{-1}_{\mathit{\shortuparrow}}$, $F^{-1}_{\upharpoonleft}$}
%is \impinline{$\lambda n. \min\{m : F(m)\ge n\}$}.
%
%\smallskip
%
%\pause 
%\textbf{\href{https://github.com/inv-ack/inv-ack/blob/7270e64a2600b771f2b1b1b151f7d13fb2ae6c97/inverse.v\#L65-L77}{Logical equivalence} (more useful)}: If $F:\mathbb{N}\to \mathbb{N}$ is increasing, \\
%then $f = F^{-1}$ iff \impinline{$\forall n, m.~ f(n)\le m \iff n \le F(m)$}.
%
%\bigskip
%
%\textbf{Idea.} Build $\left(\rf{F}{a}\ \right)^{-1}$ from $f \triangleq F^{-1}$.
%	\begin{equation*}
%	\displayindent0pt
%	\displaywidth\textwidth
%	\begin{array}{ccll}
%	& \left(\rf{F}{a}\ \right)^{-1}(n)\le m \pause & \iff \ n \le \rf{F}{a}(m) \pause & \iff n \ \le F^{(m)}(a) \\
%  \pause \iff & f(n)\le F^{(m-1)}(a) \pause & \iff \ \cdots \pause &
%	\iff \ f^{(m)}(n) \le a 
%	\end{array}
%	\end{equation*}
%
%\pause
%\imppar{$\left(\rf{F}{a}\ \right)^{-1}(n)$ is \emph{the least} $m$ for which $f^{(m)}(n)\le a$.}
%
%\end{frame}



\begin{frame}[fragile]

\textbf{Goal.} Construct $\alpha_{n+1}$ from $\alpha_n$ and $a\angle{n+1}$ from $a\angle{n}$.
$$
\tikzset{
LA/.style = {
draw=red, % just to demonstrate, where LA is used
line width=1.5pt, -{Straight Barb[length=3pt]}},
LA/.default=1pt,
LB/.style = {
draw=blue, % just to demonstrate, where LA is used
line width=1.5pt, -{Straight Barb[length=3pt]}},
LB/.default=1pt,
invisible/.style={opacity=0},
visible on/.style={alt={#1{}{invisible}}},
alt/.code args={<#1>#2#3}{%
  \alt<#1>{\pgfkeysalso{#2}}{\pgfkeysalso{#3}}%
  }
}
\newcommand{\ard}{
  \ar[visible on=<2->, d, LB,
    "\ \text{\textbf{\textcolor{blue}{Repeater}}}"]}
\newcommand{\rd}{\ar[d, LA]}
\newcommand{\abd}{
  \ar[visible on=<3->, d, LA,
    "\ \text{\textbf{\textcolor{red}{???}}}"]}
\newcommand{\bd}{\ar[d, LB]}
\begin{tikzcd}[row sep = 13.5pt]
a[n]b & \Ack_n(b) & a\angle{n}b & \alpha_n(b) \\[-15pt] \hline
1 + b \ard & 1 + b \ard & b - 1 \abd & b - 1 \abd
\\
a + b \ard & 2 + b \ard & b - a \abd & b - 2 \abd
\\
a \cdot b \ard & 2b + 3 \ard & \left\lceil \frac{b}{a} \right\rceil \abd & \left\lceil \frac{b-3}{2} \right\rceil \abd
\\
a^b & 2^{b + 3} - 3 & \left\lceil \log_a b \right\rceil & \left\lceil \log_2 (b + 3)\right\rceil - 3
%\\
%\underbrace{a^{.^{.^{.^a}}}}_b & \underbrace{2^{.^{.^{.^2}}}}_{b+3} - 3 & \log^*_a ~ b & \log^*_2 ~ (b + 3) - 3
\end{tikzcd}
$$

\uncover<4->{\imppar{\textbf{Next Goal.} Construct the inverse of $\rf{F}{a}\ $ entirely from $F$'s inverse.}}

\end{frame}


%\begin{frame}
%\frametitle{Formalizing Inverse}
%
%If $F: \mathbb{N} \to \mathbb{N}$ is strictly increasing (holds for all $\Ack_n$ and $a[n]$ when $a\ge 2$),
%the \href{https://github.com/inv-ack/inv-ack/blob/7270e64a2600b771f2b1b1b151f7d13fb2ae6c97/inverse.v\#L28-L45}{
%	\emph{upper inverse}} of $F$, written $F^{-1}$,
%% {\color{red} $F^{-1}$, $F^{-1}_{\mathit{\shortuparrow}}$, $F^{-1}_{\upharpoonleft}$}
%is \impinline{$\lambda n. \min\{m : F(m)\ge n\}$}.
%
%\smallskip
%
%\textbf{Note.} Since $F$ is strictly increasing, $F^{-1}$ is a total function.
%
%\bigskip
%
%\pause 
%\textbf{\href{https://github.com/inv-ack/inv-ack/blob/7270e64a2600b771f2b1b1b151f7d13fb2ae6c97/inverse.v\#L65-L77}{Logical equivalence} (more useful)}:
%\impinline{$\forall n, m.~ F^{-1}(n)\le m \iff n \le F(m)$}.
%
%\bigskip
%
%\pause
%\textbf{Repeatable functions.} $F$ is \textcolor{red}{\emph{repeatable from}} $a$ if $F$ is strictly increasing and \textcolor{red}{\emph{expanding from}} $a$, i.e. $~\forall n.~F(n) \ge n~$ and $~\forall n \ge a.~F(n) > n$.
%
%\vspace{5pt}
%
%\pause
%\textbf{Advantage.} $\displaystyle \begin{cases}
%F \text{ is repeatable from } a \implies \rf{F}{a}\ \text{ is repeatable from } 0. \\[3pt]
%\forall n, \ \Ack_n \text{ is repeatable from } 0. \\[3pt]
%\forall n, \forall a \ge 2, \ a[n] \text{ is repeatable from } a[n+1](0).
%\end{cases}$
%
%\end{frame}


\begin{frame}
\frametitle{Formalizing Inverse}

\textbf{Assumption.} We assume $a\ge 2$ whenever $a[n]b$ is mentioned.

\bigskip

\pause

The \href{https://github.com/inv-ack/inv-ack/blob/7270e64a2600b771f2b1b1b151f7d13fb2ae6c97/inverse.v\#L28-L45}{
	\emph{upper inverse}} of $F$, written $F^{-1}$,
% {\color{red} $F^{-1}$, $F^{-1}_{\mathit{\shortuparrow}}$, $F^{-1}_{\upharpoonleft}$}
is \impinline{$\lambda n. \min\{m : F(m)\ge n\}$}.

\smallskip

\textbf{Remarks.}

$\bullet~$ If $F$ is strictly increasing, $F^{-1}$ is a total function.

$\bullet~$ $\Ack_n$ and $a[n]$ are strictly increasing for each $n$.




\bigskip

\pause 
\textbf{\href{https://github.com/inv-ack/inv-ack/blob/7270e64a2600b771f2b1b1b151f7d13fb2ae6c97/inverse.v\#L65-L77}{Logical equivalence} (more useful)}:
\impinline{$\forall n, m.~ F^{-1}(n)\le m \iff n \le F(m)$}.

\end{frame}



\begin{frame}
\frametitle{Inverting Repeater: The Idea of Countdown}

Assume F satisfies the 3 conditions of the theorem. \\ \smallskip
Then \impinline{$\forall n, m.~ F^{-1}(n)\le m \iff n \le F(m)$}.

\smallskip

Let us build $\left(\rf{F}{a}\ \right)^{-1}$ from $F^{-1}$.
\pause
	\begin{equation*}
	\displayindent0pt
	\displaywidth\textwidth
	\begin{array}{ccll}
	& \left(\rf{F}{a}\ \right)^{-1}(n)\le m \pause & \iff \ n \le \rf{F}{a}\ (m) \pause & \iff n \ \le F^{(m)}(a) \\[5pt]
  \pause \iff & F^{-1}(n)\le F^{(m-1)}(a) \pause & \iff \ \cdots \pause &
	\iff \ \left(F^{-1}\right)^{(m)}(n) \le a 
	\end{array}
	\end{equation*}

\pause
\imppar{$\left(\rf{F}{a}\ \right)^{-1}(n)$ is \emph{the least} $m$ for which $\left(F^{-1}\right)^{(m)}(n)\le a$.}

\end{frame}


%\begin{frame}
%\frametitle{Formalizing countdown}
%
%\textbf{Countdown.} The \textit{\textcolor{red}{countdown} to} $a$ of $f$, written~$\cdt{f}{a}(n)$, if exists, is the smallest number of times $f$ needs to be compositionally applied to
%$n$ for the answer to equal or go below $a$. \emph{i.e.},
%\begin{equation*}
%\cdt{f}{a}(n) \triangleq \min\{m: f^{(m)}(n)\le a \}.
%\end{equation*}
%
%\textbf{Contractions.} $f$ is a \emph{\textcolor{red}{contraction} strict above} a if \\
%$\forall n.~f(n)\le n~$ and $~\forall n > a.~f(n) < n$.
%
%\smallskip
%
%\pause
%If $f$ is a contraction strict from $a$, $\cdt{f}{a}$ is a total function.
%
%\bigskip
%
%\pause
%\textbf{Theorem.} If $F$ is repeatable from $a$,\\
%$F^{-1}$ is a contraction strict above $a$ and $\cdt{\left(F^{-1}\right)}{a}\ \ = \ \left(\rf{F}{a}\ \right)^{-1}$.
%
%\end{frame}


\begin{frame}
\frametitle{Formalizing countdown}

\textbf{Countdown.} The \textit{\textcolor{red}{countdown} to} $a$ of $f$, written~$\cdt{f}{a}(n)$, if exists, is the smallest number of times $f$ needs to be compositionally applied to
$n$ for the answer to equal or go below $a$. \emph{i.e.},
\begin{equation*}
\cdt{f}{a}(n) \triangleq \min\{m: f^{(m)}(n)\le a \}.
\end{equation*}

\pause
\textbf{Theorem.}
$~\displaystyle
\begin{cases}
F \text{ strictly increasing} \\ \forall n.~F(n)\ge n \\
\forall n > a.~F(n) > n
\end{cases}
\quad \pause \implies \quad
\begin{cases}
\cdt{\left(F^{-1}\right)}{a} \text{ is total} \\[3pt]
\cdt{\left(F^{-1}\right)}{a}\ \ = \ \left(\rf{F}{a}\ \right)^{-1}
\end{cases}
$

\bigskip
\pause
\textbf{Corollary.} $a[n+1]$ and $\alpha_{n+1}$ are respectively the\\ countdowns of $a[n]$ and $\alpha_n$ to some appropriate value.

\end{frame}


\begin{frame}[fragile]
\frametitle{A Coq Computation of Countdown}
\textbf{Idea.} %To compute $\cdt{f}{a}(n)$,
Compute $f^{k}(n)$ for $k=0, 1,\ldots$ until $f^{(k)}(n)\le a$. Return $k$.

\bigskip

\pause 
\textbf{Primary Issue.} Termination in Coq.
%Coq needs a known terminating point, i.e. an explicit decreasing argument. How to know when to terminate beforehand?

\bigskip

\textbf{Secondary issue.} $\cdt{f}{a}$ may not exist for general $f$.
%It's hard to define a Coq definition of Countdown for contractions only. Defining a computation for \emph{all functions} that actually computes Countdown on contractions is easier.

\bigskip

\pause 
\textbf{The worker.} $\begin{cases}
\text{Budget } b: & \hspace{-1em}\text{Maximum } b \text{ steps. } \\
\text{Step } i: & \hspace{-1em}\text{Compute } f^{(i)}(n). \\
\text{Stops when: } & \hspace{-1em}\text{budget reaches } 0 \text{ or } f^{(i)}(n) \le a.
\end{cases}$
\end{frame}


\begin{frame}[fragile]
\frametitle{A Coq Computation of Countdown}

\textbf{The worker.}
The \emph{countdown worker}
to $a$ of $f$ is a function $\cdw{f}{a}$~~:
\begin{equation*}
\cdw{f}{a}(n, b) = \begin{cases}
0 & \text{if } b = 0 \vee n\le a \\ 1 +\cdw{f}{a}(f(n), b-1) & \text{if } b \ge 1 \wedge n > a
\end{cases}
\end{equation*}

% Linked by A
\begin{lstlisting}
`\href{https://github.com/inv-ack/inv-ack/blob/7270e64a2600b771f2b1b1b151f7d13fb2ae6c97/countdown.v#L103-L108}{Fixpoint cdn\_wkr}` (a : nat) (f : nat -> nat) (n b : nat) : nat :=
  match b with 0 => 0 | S b' =>
    if (n <=? a) then 0 else S (cdn_wkr f a (f n) k')
end.
\end{lstlisting}

\bigskip

\pause 
\textbf{The Countdown.} Budget $b = n$ is sufficient.
\impinline{Redefine $\cdt{f}{a}(n) \triangleq \cdw{f}{a}(n, n)$.}
% Linked by A
\begin{lstlisting}
`\href{https://github.com/inv-ack/inv-ack/blob/7270e64a2600b771f2b1b1b151f7d13fb2ae6c97/countdown.v#L112}{Definition countdown\_to}` a f n := cdn_wkr a f n n.
\end{lstlisting}
\end{frame}


%\subsection*{}
%\begin{frame}
%\frametitle{Step 2: Inverting the Hierarchies via \textbf{Countdown}}
%\setbeamertemplate{enumerate items}[default]
%\setbeamercolor{local structure}{fg=black}
%
%\begin{enumerate}[\bfseries 1.]
%	\itemsep 3ex
%	\item<done@1->
%	\emph{What is inverse?} \textbf{Upper inverse and increasing functions}.
%	
%	\item<done@1->
%	\emph{Can Repeater preserve Invertibility?} \textbf{Repeatable functions}.
%	
%	\item<done@1->
%	\emph{Computing inverse with inverse:} \textbf{Contractions and Countdown}.
%	
%	\item<come@2->
%	\textbf{Invert each level in hyperoperations/Ackermann hierarchies}.
%\end{enumerate}
%\end{frame}


\subsection{Inverting the Hyperoperations/Ackermann Hierarchies}
\begin{frame}[fragile]
\frametitle{The Inverse Hyperoperation Hierarchy}
The inverse hyperoperations, written $a\angle{n}b$, are defined as:
\begin{equation*}
a\angle{n}b \; \triangleq \; \begin{cases}
b - 1 & \text{ if } n = 0 \\
\cdt{a\angle{n-1}}{a_n}(b) & \text{ if } n \ge 1
\end{cases}
\quad \text{ where } \ a_n = \begin{cases}
a & \hspace{-7pt}\text{ if } n = 1 \\
0 & \hspace{-7pt}\text{ if } n = 2 \\
1 & \hspace{-7pt}\text{ if } n \ge 3
\end{cases}
\end{equation*}
\vspace{-1em}
% linked by A
\pause 
\begin{lstlisting}
`\href{https://github.com/inv-ack/inv-ack/blob/7270e64a2600b771f2b1b1b151f7d13fb2ae6c97/applications.v#L28-L33}{Fixpoint inv\_hyperop}` (a n b : nat) : nat :=
  match n with 0 => b - 1 | S n' =>
    countdown_to (hyperop_init a n') (inv_hyperop a n') b
  end.
\end{lstlisting}

\smallskip

\pause 
Interesting individual levels:
$\href{https://github.com/inv-ack/inv-ack/blob/7270e64a2600b771f2b1b1b151f7d13fb2ae6c97/applications.v#L102-L113}{a\angle{2}b}
= \left\lceil b/a \right\rceil$,
$\href{https://github.com/inv-ack/inv-ack/blob/7270e64a2600b771f2b1b1b151f7d13fb2ae6c97/applications.v#L115-L124}{a\angle{3}b}
= \left\lceil \log_a b \right\rceil$, and
$\href{https://github.com/inv-ack/inv-ack/blob/7270e64a2600b771f2b1b1b151f7d13fb2ae6c97/applications.v#L126-L128}{a\angle{4}b}
= \log^*_a b$ (not currently in Coq standard library).
\end{frame}


\begin{frame}[fragile]
\frametitle{The Inverse Ackermann Hierarchy}

\textbf{Naive approach.} $\Ack_{i+1} = \rf{\big(\Ack_i\big)}{{\color<2->{red} \Ack_i(1)}}\ $.
Thus $\alpha_{i+1} \triangleq \cdt{\big(\alpha_i\big)}{{\color<2->{red} \Ack_i(1)}}\ $?\\ 

\smallskip

\pause
\textcolor{red}{\textbf{Flaw!}} $\alpha_{i+1}$ depends on {\color{red}$\Ack_i$}.

\bigskip

\pause 
\textbf{Observation.} $\Ack_{i+1}(n) = \Ack_i^{(n)}(\Ack_i(1)) = \Ack_i^{(n+1)}(1)$. Thus
\vspace{-0.7em}
\begin{equation*}
\begin{aligned}
\alpha_{i+1}(n) & = \min\left\{m : n\le \Ack_i^{(m+1)}(1)\right\} \pause 
= \min\left\{m : \big( \alpha_i \big)^{(m+1)}(n)\le 1\right\} \pause \\
& = \min\left\{m : \big( \alpha_i \big)^{(m)}\big(\alpha_i(n)\big)\le 1\right\} \pause
= \cdt{\big(\alpha_i\big)}{1}\big(\alpha_i(n)\big)
\end{aligned}
\end{equation*}
\pause 
\textbf{Redefine:}
$
\alpha_i \triangleq
\begin{cases}
\lambda n.(n - 1) & \text{when } i = 0
\\ \big(\cdt{\alpha_{i-1}}{1}\big)\circ \alpha_{i-1} & \text{when } i\ge 1
\end{cases}
$

\smallskip

\pause 
% Linked by A
\begin{lstlisting}
`\href{https://github.com/inv-ack/inv-ack/blob/7270e64a2600b771f2b1b1b151f7d13fb2ae6c97/inv_ack.v#L37-L41}{Fixpoint alpha}` (m x : nat) : nat :=
  match m with 0 => x - 1 | S m' =>
    countdown_to 1 (alpha m') (alpha m' x)
end.
\end{lstlisting}
\end{frame}
