\subsection{Using the Hierarchy to Implement $\alpha(n)$}
\begin{frame}
\frametitle{Implementation Idea}
\textbf{Redefinition.}
$
\alpha(n) = \min\{k: n\le A(k, k) \} \triangleq \min\{k: \alpha_k(n)\le k \}
$

\bigskip

\pause 
\textbf{The worker function.} $\begin{cases}
\text{Budget } b: & \hspace{-1em}\text{Maximum } b \text{ steps. } \\
\text{Step } i: & \hspace{-1em}\text{Compute } \alpha_i(n). \\
\text{Stops when: } & \hspace{-1em}\text{budget reaches } 0 \text{ or } \alpha_i(n) \le i.
\end{cases}$

%\textbf{The worker function.}
%\bigskip
%\begin{itemize}
%	\item Budget $b$: Maximum $b$ steps.
%	\item Step $i$: Computes $\alpha_i(n)$.
%	\item Stops when budget reaches $0$ or $\alpha_i(n) \le i$.
%\end{itemize}

\bigskip

\pause 
\textbf{Advantage.} $\alpha_{i+1}(n) = \cdt{\big(\alpha_i\big)}{1}\big(\alpha_i(n)\big)$\\
\smallskip
So the next step is computable from the previous.
\end{frame}


\begin{frame}[fragile]
\frametitle{The Worker Function}

\uncover<1->{$\displaystyle 	\alpha^{\W}(f, n, k, b) = \begin{cases}
k & \text{if } b = 0 \vee n\le k \\ \alpha^{\W}(\cdt{f}{1}\circ f , \cdt{f}{1}(n), k+1, b-1) & \text{if } b \ge 1 \wedge n \ge k+1
\end{cases}$}

\bigskip

\begin{overlayarea}{\linewidth}{3cm}
\begin{onlyenv}<2->
	\begin{lstlisting}
`\href{https://github.com/inv-ack/inv-ack/blob/7270e64a2600b771f2b1b1b151f7d13fb2ae6c97/inv_ack.v#L155-L161} {Fixpoint inv\_ack\_wkr}` (f : nat -> nat) (n k b : nat) : nat :=
match b with 0 => 0 | S b' =>
  if (n <=? k) then k else let g := (countdown_to f 1) in
                      inv_ack_wkr (compose g f) (g n) (S k) b
end.
\end{lstlisting}
\end{onlyenv}
\end{overlayarea}

\end{frame}



\begin{frame}
\frametitle{The Worker Function}

\textbf{Observation.}
\begin{equation*}
\begin{array}{lllllll}
Initial \ Arguments & & \big(\alpha_i, & \alpha_i(n), & i, & b - i \big) \\[5pt]
\pause
b - i > 0, \ \alpha_i(n) > i & \pause \to & \big(\cdt{\alpha_i}{1}\circ \alpha_i, & \cdt{\alpha_i}{1}(\alpha_i(n)), & i + 1, & b - i - 1\big) \\[5pt]
\pause
{\color{blue} b > i}, \ \hspace{15pt} \alpha_i(n) > i & \to  & \big({\color{blue} \alpha_{i+1}}, & {\color{blue} \alpha_{i+1}(n)}, & i + 1, & {\color{blue} b - (i + 1)}\big) \\
\end{array}
\end{equation*}

\pause
\textbf{Execution.}
\begin{equation*}
\begin{array}{l|lllllll}
Step & Initial \ Arguments &  &  \big(\alpha_0, & \alpha_0(n), & 0, & b - 0 \big) \\[3pt]
\pause
0 & b > 0, \ \alpha_0(n) > 0 & \pause \to & \big(\alpha_1, & \alpha_1(n), & 1, & b - 1\big) \\[3pt]
\pause
1 & b > 1, \ \alpha_1(n) > 1 & \pause \to  & \big(\alpha_2, & \alpha_2(n), & 2, & b - 2\big) \\[3pt]
\pause
\vdots & \vdots & \vdots & & \multicolumn{3}{l}{\vdots} \\
%b > k-1, \ \alpha_{k-1}(n) > k - 1 & \to  & \big(\alpha_k, & \alpha_k(n), & k, & b - k\big) & k-1 \\
\pause
k & b > k, \ {\color{red} \alpha_{k}(n) \le k} & \pause \to  & \multicolumn{4}{l}{\impinline{$k = \alpha(n)$}}
\end{array}
\end{equation*}
\end{frame}



\begin{frame}
\frametitle{The Inverse Ackermann Function}

Any budget $b \ge \alpha(n)$ suffices, so we can choose $b = n$.

\smallskip

\pause
\textbf{First version:}.
\pause
\begin{equation*}
\alpha(n) = \alpha^{\mathcal{W}}\big(\alpha_0, \alpha_0(n), 0, n\big)
= \alpha^{\mathcal{W}}\big(\lambda n(n - 1), n - 1, 0, n\big)
\end{equation*}

\pause
\textbf{Time Complexity.}

\vspace{5pt}

\begin{tabular}{l|l|l}
Computation & Time & Reason \\ \hline
$\topspace{2pt} \alpha_0(n) = n - 1$ & $\Theta(1)$ & Coq SL \\
$\topspace{2pt} \alpha_1(n) = n - 2$ & $\Theta(n)$ & Apply $\alpha_0$ for $n - 1$ times \\
$\topspace{2pt} \alpha_2(n) = \left \lceil \frac{n-3}{2} \right \rceil$ & $\Theta(n^2)$ & Apply $\alpha_2$ for $n/2$ times
\end{tabular}

\vspace{7pt}

\textcolor{red}{\textbf{Total time is at least $\mathbf{\Omega(n^2)}$!}}

\end{frame}


\begin{frame}
\frametitle{The Inverse Ackermann Function}

\textbf{Simple improvement.}
\pause
\begin{equation*}
\begin{aligned}
\alpha(n) & = \begin{cases}
\alpha^{\mathcal{W}}\big(\alpha_{{\color{red} 1}}, \alpha_{{\color{red} 1}}(n), {{\color{red} 1}}, n - {{\color{red} 1}}\big) & \hspace{3em}\text{ when } n > \Ack(0) \\
0 & \hspace{3em}\text{ when } n \le \Ack(0)
\end{cases} \\[5pt]
\pause \alpha(n) & = \begin{cases}
\alpha^{\mathcal{W}}\big(\lambda n(n - 2), n - 2, 1, n - 1\big) & \text{ when } n > 1 \\
0 & \text{ when } n \le 1
\end{cases}
\end{aligned}
\end{equation*}

\pause
Now, computing $\alpha_1(n) = n - 2$ takes time $\Theta(1)$,\\
so computing $\alpha_2(n)$ takes time $\Theta(n)$.

\bigskip
\pause
\textcolor{red}{\textbf{In fact, the total time is still $\mathbf{\Theta(n)}$.}}

\end{frame}


%\begin{frame}
%\frametitle{Time Complexity of $\alpha$: A Sketch}
%
%$
%{\color{green}\alpha_{i+1}} = \cdt{\big({\color{blue}\alpha_{i}}\big)}{1}\circ {\color{blue}\alpha_i} \quad \text{ and } \quad {\color{green} \alpha_{i+1}(n)} = \cdt{\big({\color{blue}\alpha_{i}}\big)}{1}\big({\color{red}\alpha_{i}(n)}\big).
%$
%
%\bigskip
%
%$ \displaystyle
%\begin{aligned}
%\alpha^{\mathcal{W}}\big({\color{blue}\alpha_i}, \ {\color{red}\alpha_i(n)}, \ i, \ b\big)
%& = \alpha^{\mathcal{W}}\left(
%\cdt{{\color{blue}\alpha_i}}{1}\circ {\color{blue}\alpha_i},  \
%\cdt{{\color{blue}\alpha_i}}{1}\big({\color{red}\alpha_i(n)}\big), \
%i+1, \ b - 1\right) \\
%& = \alpha^{\mathcal{W}}\left(
%{\color{green}\alpha_{i+1}}, \ \ \ \
%{\color{green}\alpha_{i+1}(n)}, \ \ \ \ \
%i+1, \ b - 1\right).
%\end{aligned}
%$
%
%\bigskip
%
%The recursive formula of $\alpha^{\mathcal{W}}$ contains the recursive formula of $\alpha_i$.
%
%%The computation $\alpha(n)$ up to step $i$ incorporates the computation of $\alpha_i(n)$, plus some minor computations.
%
%\bigskip
%
%Thus, most of $\alpha(n)$'s running time is to compute $\alpha_k(n)$, where $k \triangleq \alpha(n)$.
%
%\bigskip
%
%Similarly, for each $i$, most of $\alpha_i(n)$'s running time is to compute the chain $\alpha_i(n), \alpha_i^{(2)}(n), \alpha_i^{(3)}(n) \ldots, $ until reaching $1$.
%
%%\begin{equation*}
%%\begin{array}{l|lllllll}
%%Step & Initial \ Arguments &  &  \big(\alpha_0, & \alpha_0(n), & 0, & b - 0 \big) \\[3pt]
%%0 & b > 0, \ \alpha_0(n) > 0 & \to & \big(\alpha_1, & \alpha_1(n), & 1, & b - 1\big) \\[3pt]
%%1 & b > 1, \ \alpha_1(n) > 1 & \to  & \big(\alpha_2, & \alpha_2(n), & 2, & b - 2\big) \\[3pt]
%%\vdots & \vdots & \vdots & & \multicolumn{3}{l}{\vdots} \\
%%%b > k-1, \ \alpha_{k-1}(n) > k - 1 & \to  & \big(\alpha_k, & \alpha_k(n), & k, & b - k\big) & k-1 \\
%%k & b > k, \ \alpha_{k}(n) \le k & \to  & \multicolumn{4}{l}{k = \alpha(n)}
%%\end{array}
%%\end{equation*}
%
%\end{frame}



\begin{frame}
\frametitle{Time Complexity of $\alpha$: A Sketch}

Let $\runtime_f(n)$ denotes the running time of computing $f(n)$ given $f$ and $n$.
\pause
\begin{equation*}
\runtime_{\alpha_{i+1}}(n) \approx
\runtime_{\alpha_i}(n) + \runtime_{\alpha_i}\big(\alpha_i(n)\big)
+ \runtime_{\alpha_i}\big(\alpha_i^{(2)}(n)\big) + \ldots (\text{until 1})
\end{equation*}

\smallskip

\pause
Manual computation: $\runtime_{\alpha_3}(n)\le 4n + 4$.

\smallskip

\pause
Suppose $\runtime_{\alpha_i}(n) \le 4n + \bigO\big(\log_2n\big)$. \pause Then
\begin{equation*}
\begin{aligned}
& \runtime_{\alpha_{i+1}}(n) \ \lessapprox \ 4n + \bigO\big(\log_2n\big) + 4\log_2n + \bigO\big(\log_2^{(2)}n\big) + \ldots \\
& \quad = 4n + \bigO\left(\log_2n + \log_2^{(2)}n + \log_2^{(3)}n + \ldots\right)
 = 4n + \bigO\big(\log_2n\big)
\end{aligned}
\end{equation*}
\pause
Therefore, \imppar{$T_{\alpha}(n) \approx \runtime_{\alpha_{\alpha(n)}}(n) \lessapprox 4n + \bigO\big(\log_2n\big) \lessapprox \Theta(n)$} by induction.

\end{frame}
