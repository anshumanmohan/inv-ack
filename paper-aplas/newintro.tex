
Let us examine a function iterator, in the spirit of Knuth's up arrow 
\cite{knuth_up_arrow}: 

\begin{defn} \label{defn:repeater}
For a function $g: \mathbb{N} \to \mathbb{N}$, 
the \textit{repeater} of $g$, written $g^R$, is:
\begin{equation}
g^R(n) = \begin{cases}
1 & \text{if } n = 0 \\ 
g \left( g^R (n - 1) \right) & \text{otherwise.}
\end{cases}
\end{equation}
{\color{magenta}In other words, 
$\forall n, g^R(n) = g^{(n)}\left(g^R(0)\right) = g^{(n)}(1)$}.
\end{defn}

Noting that multiplication is just repeated addition, 
we can define multiplication by $n$ using $(\texttt{Nat.add n})^R$:

(Blah... currently off by one)

As Knuth demonstrates, we can go further, ascending a 
``ladder'' of repeated application, where the 
rungs are addition, multiplication, exponentiation, 
tetration, and so on.

Interestingly, with a little care, we can also work our way
downwards using \textit{shrinking} functions
(\emph{i.e.} $\forall m, f(m) < m$) and a subtly different 
function iterator:

\begin{defn} \label{defn:countdown}
For a function $f: \mathbb{N} \to \mathbb{N}$,
the \textit{countdown} of $f$, written $f^*$, is:
\begin{equation} \label{eq: countdown}
f^*(n) = \begin{cases}
0 & \text{if } n \le 1 \text{ or } n \le f(n) \\
1 + f^*(f(n)) & \text{otherwise.}
\end{cases}
\end{equation}
\end{defn}

We can now define division using \texttt{Nat.sub n}:

(Blah)

The standard library has a definition of \texttt{Nat.div2},
which is straightforward, and a more general definition 
of \texttt{Nat.div}, which is much more complicated. This 
complication comes from needing to prove termination. Our 
general definition is much cleaner because it dodges this 
termination argument. 


An obvious benefit of our pair of iterators is that they
intuitively emphasize the relationship between the multiplication 
and division. More importantly, they scale very well and allow us
to iterate further: $(\texttt{Nat.add n})^{R^R}$ is exponentiation, 
and $(\texttt{Nat.sub n})^{*^*}$ is logarithm.

