We provide a runtime analysis for the
inverse Ackermann function. First, we show that the
runtime of the version defined in Definition~\ref{thm: inv-ack-hier-correct}
is $O(n^2)$. We then provide a simple improvement that
cuts the runtime to $O(n)$.

%\footnote{We will be careful not to conclude that functions
%$f$ and $g$ are equal when they agree on all inputs but are computed with
%different pieces of code.} 
% momentarily forget about the axiom of extensionality and 
\subsection{Basic Time Analysis: $O(n^2)$}
To formalize the notion of runtime, we identify each function on 
$\mathbb{N}$ with its \emph{computation}, \emph{i.e.} the program that computes 
it in Coq \note{under a call-by-value strategy}.
\begin{defn}
 Given a function $f:\mathbb{N}\to\mathbb{N}$, the \emph{runtime} of $f$ on input $n\in \mathbb{N}$, denoted by $\runtime(f, n)$ is the number of computational steps it takes to compute~$f(n)$.
\end{defn}
\begin{lem} \label{lem: sub-runtime}
	$\runtime(\lambda n.(n - a), n) = \Theta(\min\{a, n\})$, per Coq's definition of~subtraction.
\end{lem}
\begin{lem} \label{lem: compose-runtime}
	$\forall f, g: \mathbb{N}\to \mathbb{N}.~\runtime(f\circ g, n) = \runtime(f, g(n)) + \runtime(g, n)$ per Coq's definition of functional composition.
\end{lem}
\begin{lem} \label{lem: cdt-runtime}
	Per Definition~\ref{defn: countdown}, $\forall a\ge 1, \forall n\le 1, \forall f\in \contract_{a}$.~$\runtime\big(\cdt{f}{a}\ , n\big) = 1$, and
	\begin{equation*}
	\forall n\ge 2.~\runtime\big(\cdt{f}{a}\ , n\big) = \sum_{i=0}^{\cdt{f}{a}(n) - 1} \runtime\left(f, f^{(i)}(n)\right) + \Theta\big((a + 1)\cdt{f}{a}(n)\big)
	\end{equation*}
\end{lem}
\begin{proof}
	Per Definition~\ref{defn: countdown-worker}, the computation makes $\cdt{f}{a}(n)$ recursive calls to $\W\cdt{f}{a}$ before terminating. At the $(i+1)^{\text{th}}$ call, two computations must take place: $n_i - a$, which takes $\Theta(a + 1)$ time, and $f(n_i) = n_{i+1}$, where $n_i \triangleq f^{(i)}(n)$ has been  computed by the $i$th call, and is greater than $a$.  The total time is then
	\begin{equation*}
	\begin{aligned}
	\runtime\big(\cdt{f}{a}\ , n\big)
	& = \sum_{i=0}^{\cdt{f}{a}(n) - 1} \left[\runtime\left(f, f^{(i)}(n)\right) + \Theta(a + 1)\right] \\
	& = \sum_{i=0}^{\cdt{f}{a}(n) - 1} \runtime\left(f, f^{(i)}(n)\right) + \Theta\big((a + 1)\cdt{f}{a}(n)\big)
	\end{aligned}
	\end{equation*}
\end{proof}

From Lemma~\ref{lem: compose-runtime} and Lemma~\ref{lem: cdt-runtime}, the following lemma follows easily:
\begin{lem} \label{lem: inv-ack-hier-runtime}
	Per Definition~\ref{defn: inv-ack-hier} and Definition~\ref{defn: countdown},
	\begin{equation*}
	\runtime\big(\alpha_{i+1}, n\big) = \sum_{k=0}^{\alpha_{i+1}(n)}\runtime\left(\alpha_i, \alpha_i^{(k)}(n)\right) + \Theta\big(\alpha_{i+1}(n)\big)
	\end{equation*}
\end{lem}
This lemma implies $\runtime\big(\alpha_{i+1}, n\big) \ge \runtime\big(\alpha_i, n\big)$. If there is some $i$ such that $\runtime\big(\alpha_i, n\big) = \Theta\big(n^2\big)$, each function in the hierarchy will take at least $\Omega\big(n^2\big)$ to compute \note{from $\alpha_i$}, thus making $\runtime(\alpha, n) = \Omega\big(n^2\big)$ per Definition~\ref{defn: inv-ack-worker}. The next lemma shows that $i = 2$ suffices.
\begin{lem}
	Per Definition~\ref{defn: inv-ack-hier}, $\runtime\big(\alpha_2, n\big) = \Theta\big(n^2\big)$
\end{lem}
\begin{proof}
	$\alpha_1 = \big(\cdt{\lambda m.(m-1)}{1}\big)\circ \big(\lambda m.(m-1)\big) = \lambda m.(m - 2)$. By Lemma~\ref{lem: inv-ack-hier-runtime},
	\begin{equation*}
	\runtime\big(\alpha_1, n\big) = \sum_{i=0}^{n-1} \runtime\big(\lambda m.(m-1), n - i\big) + \Theta\big((n-2)\big) = \Theta(n)\text{,}
	\end{equation*}
	since $\runtime\big(\lambda m.(m-1), k\big) = 1$. Because $\alpha_2 = \big(\cdt{\alpha_1}{0}\big)\circ \alpha_1 $, again by Lemma~\ref{lem: inv-ack-hier-runtime},
	\begin{equation*}
	\runtime\big(\alpha_2, n\big)
	= \sum_{i=0}^{\lceil (n-3)/2 \rceil} \runtime \big(\alpha_1, n-2i\big) + \Theta\left(\frac{n}{2}\right)
	= \Theta\left( \sum_{i=0}^{\lceil (n-3)/2 \rceil}(n - 2i) \right)
	= \Theta\big(n^2\big)
	\end{equation*}
\end{proof}
\begin{col}
	$\runtime(\alpha, n) = \Omega\big(n^2\big)$ per Definition~\ref{defn: inv-ack-worker}.
\end{col}

%This lemma implies $\runtime\big(\cdt{f}{a}\ , n\big) \ge \runtime(f, n)$. If for some $i$, $\runtime\big(2\angle{i}, n\big) = \Theta\big(n^2\big)$, the entire hierarchy will take at least $\Omega\big(n^2\big)$ from $2\angle{i}$, thus making $\runtime(\alpha, n) = \Omega\big(n^2\big)$ per \cref{defn: inv-ack-worker}. The next lemma shows $i = 2$ suffices.
%\begin{lem}
%	Per \cref{defn: inv-hyperop}, $\runtime\big(2\angle{2}, n\big) = \Theta\big(n^2\big) \ \forall n$.
%\end{lem}
%\begin{proof}
%	Since $2\angle{1} = \cdt{\lambda m.(m-1)}{2}\ \equiv \lambda m.(m - 2)$, by \cref{lem: cdt-runtime},
%	$$ \runtime\big(2\angle{1}, n\big) = \sum_{i=0}^{n-2} \runtime\big(\lambda m.(m-1), n - i\big) + \Theta\big(3(n-2)\big) = \Theta(n) $$
%	, since $\runtime\big(\lambda m.(m-1), k\big) = 1 \ \forall k$. Since $2\angle{2} = \cdt{2\angle{1}}{0}\ $, again by \cref{lem: cdt-runtime},
%	$$ \runtime\big(2\angle{2}, n\big)
%	= \sum_{i=0}^{\lceil n/2 \rceil} \runtime \big(2\angle{1}, n-2i\big) + \Theta\left(\frac{n}{2}\right)
%	= \Theta\left( \sum_{i=0}^{\lceil n/2 \rceil}(n - 2i) \right)
%	= \Theta\big(n^2\big) $$
%	The proof is complete.
%\end{proof}
%\begin{col}
%	$\runtime(\alpha, n) = \Omega\big(n^2\big)$ per \cref{defn: inv-ack-worker}.
%\end{col}
%Intuitively, the function $2\angle{1} \equiv \lambda n.(n-2)$ is responsible for dragging the whole hierarchy's performance due to one silly weakness: its does not know it will always output $n-2$ before beginning its computation, hence needs to tediously subtract $1$ until it goes below $2$. This observation leads to the next improvement.

\subsection{Hard-coding the second level: $O(n)$} \label{sect: hardcode-lvl2}

Intuitively, it is clear that the function $\alpha_2 \equiv \lambda n.(n-2)$
slows down the entire hierarchy's performance due to one silly weakness:
it does not know that it will always output $n-2$ before beginning its computation,
and so it tediously subtracts $1$ until \note{it goes below $2$.}
%This observation leads to the next improvement.

We can hardcode $\alpha_1$ as $\lambda n.(n-2)$ to reduce its runtime
from $\Theta(n)$ to $\Theta(1)$.
\begin{lstlisting}
Definition sub_2 (n : nat) : nat :=
  match n with | 0 => 0 | 1 => 1 | S (S n') => n' end.
\end{lstlisting}
Without loss of generality, let us henceforth assume that the constant factors in $\runtime(\alpha_1, n)$ and Lemma~\ref{lem: inv-ack-hier-runtime} are both $1$. The runtime for $\alpha_2$ is then
\begin{equation*}
\runtime\big(\alpha_2, n\big)
 = \sum_{i=0}^{\lceil (n-3)/2 \rceil} \runtime \big(\alpha_1, n-2i\big) + \left\lceil \frac{n-3}{2} \right\rceil =  2\left\lceil \frac{n-3}{2} \right\rceil
 < n
\end{equation*}
In fact, this allows every function in the hierarchy to be computed in linear time:
\begin{thm} \label{thm: inv-ack-hier-runtime-improved}
	$\runtime\big(\alpha_i, n\big) \le 2n + (4^{i+1} - 4)\lceil \log_2n\rceil + 5$.
\end{thm}
We need two crucial technical lemmas to prove this theorem.
\begin{lem} \label{lem: inv-ack-3-runtime}
	$\runtime\big(\alpha_3, n\big) \le 2n + 4$.
\end{lem}
\begin{proof}
	It is easy to show that $\alpha_2^{(k)}(n) = \left\lceil \frac{n+3}{2^k} \right\rceil - 3$. Thus $\runtime\big(\alpha_3, n\big) = $
	\begin{equation*}
\begin{array}{@{}l@{}}
	\sum_{k=0}^{\alpha_{3}(n)}\runtime\left(\alpha_2, \left\lceil \frac{n+3}{2^k} \right\rceil - 3\right) + \alpha_{3}(n)  \quad
	\le \quad \sum_{k=0}^{\alpha_3(n)}\left(\frac{n+3}{2^k} + 1\right) - 3\big(\alpha_3(n) + 1\big) + \alpha_3(n) \\
	\le \sum_{k=0}^{\alpha_3(n)}\frac{n+3}{2^k} - \alpha_3(n) - 2 \quad \le \quad 2(n+3) - 2 \quad = \quad 2n + 4 \\
\end{array}
	\end{equation*}
\end{proof}
\begin{lem} \label{lem: sum-alpha-repeat}
	$\forall i \ge 3$.~$\displaystyle \sum_{k=1}^{\alpha_{i+1}(n)} \alpha_i^{(k)}(n) \le 3\big\lceil \log_2n \big\rceil$.
\end{lem}
\begin{proof}
	Let the LHS be $S_i(n)$. Firstly, consider $i = 3$. Note that for $n\le 13$, $S_3(n) = 0$ and for $n\ge 14$, i.e. $\alpha_3(n)\ge 2$, $S_3(n) = \alpha_3(n) + S_3\big(\alpha_3(n)\big)$. The result thus holds for $n\le 13$. Suppose it holds for all $m < n$, where $n\ge 14$. Then
	\begin{equation*}
	S_3(n) \quad \le \quad \alpha_3(n) + 3\big\lceil \log_2(\alpha_3(n)) \big\rceil \quad \le \quad \big\lceil \log_2n \big\rceil + 3\big\lceil \log_2\log_2n \big\rceil
	\end{equation*}
	It is easy to prove \, $2\big\lceil \log_2\log_2n \big\rceil \le \big\lceil \log_2n \big\rceil$ by induction on $\big\lceil \log_2n \big\rceil$. Thus $S_3(n)~\le~3\big\lceil \log_2n \big\rceil$, as desired. Now for $i \ge 4$,
	\begin{equation*}
	S_i(n) \quad = \quad \sum_{k=1}^{\alpha_{i+1}(n)} \alpha_i^{(k)}(n) \quad \le \quad
	\sum_{k=1}^{\alpha_{i+1}(n)} \alpha_3^{(k)}(n) \quad \le \quad
	\sum_{k=1}^{\alpha_{4}(n)} \alpha_3^{(k)}(n) \quad \le \quad
	3\big\lceil \log_2n \big\rceil
	\end{equation*}
%	Let $P(n) \triangleq 2\big\lceil \log_2\log_2n \big\rceil \le \big\lceil \log_2n \big\rceil$. It suffices to prove $P(n) \ \forall n$. Observe that $P(n)$ holds for $n\ge 4$.
\end{proof}
\begin{proof}[of Theorem~\ref{thm: inv-ack-hier-runtime-improved}]
	We have proved the result for $i = 0, 1, 2$. Let us proceed with $i\ge 3$ by induction. The base case $i = 3$ has been covered by Lemma~\ref{lem: inv-ack-3-runtime}. Now suppose the result holds for $i\ge 3$, let $M_i \triangleq 4^{i+1}-4$ for each $i$, we have
	\begin{equation*}
	\begin{array}{@{}l@{}}
	 \runtime\big(\alpha_{i+1}, n\big) = \sum_{k=0}^{\alpha_{i+1}(n)} \runtime\left(\alpha_i, \alpha_i^{(k)}(n)\right) + \alpha_{i+1}(n) \\
	\le \sum_{k=0}^{\alpha_{i+1}(n)}\left(2\alpha_i^{(k)}(n) + M_i\left\lceil\log_2\left(\alpha_i^{(k)}(n)\right)\right\rceil + 5 \right) + \alpha_{i+1}(n) \\
	\le 2n + M_i\left\lceil\log_2n\right\rceil + 5 + (M_i+2)\sum_{k=1}^{\alpha_{i+1}(n)}\alpha_i^{(k)}(n) + 6\alpha_{i+1}(n) \\
	\le 2n + M_i\left\lceil\log_2n\right\rceil + 5 +
	3(M_i + 2)\left\lceil\log_2n\right\rceil + 6\left\lceil\log_2n\right\rceil ~~
	= ~~ 2n + (4M_i + 12)\left\lceil\log_2n\right\rceil + 5 \\
	= 2n + M_{i+1}\left\lceil\log_2n\right\rceil + 5\text{, since $4M_i + 12 = 4^{i+2} - 16 + 12 = M_{i+1}$}.
	\end{array}
	\end{equation*}
\end{proof}
By hardcoding $\alpha_1$ as $\lambda m.(m-2)$, the $i$th level of the inverse Ackermann hierarchy can be computed in $\Theta\big(n + 4^i\log n\big)$ time, \emph{i.e.} linear time $\Theta(n)$ for fixed $i$.

%\subsection{An improved inverse Ackermann computation: $O(n)$}
%Building on the previous improvement, we can improve the running time of $\alpha(n)$ per \Cref{defn: inv-ack-hier} by hard-coding the output when $n\le 1 = \Ack(0, 0)$, and starting $\W\alpha$ with $f := \alpha_1$ and $n := \alpha_1(n)$. In other words,
This also allows us to improve the runtime of $\alpha(n)$ per Definition~\ref{defn: inv-ack-hier} by hardcoding the output when $n\le 1 = \Ack(0, 0)$, and starting $\alpha^{\mathcal{W}}$ with $f = \alpha_1$ and $n = \alpha_1(n)$:
\begin{equation*}
\tilde{\alpha}(n) = \begin{cases}
0 & \text{ if } n \le 1 \\ \alpha^{\mathcal{W}}\big(\alpha_1, \alpha_1(n), 1, n-1\big) & \text{ if } n \ge 2
\end{cases}
\end{equation*}
For $n > 1$, $1\le \min\big\{n-1, \alpha_1(n)\big\}$. So
$\alpha^{\mathcal{W}}\big(\alpha_0, \alpha_0(n), 0, n\big) =
\alpha^{\mathcal{W}}\big(\alpha_1, \alpha_1(n), 1, n-~1\big)$.
Thus $\tilde{\alpha}(n) = \alpha(n)$. At each recursive step, the transition from $\alpha^{\mathcal{W}}\big(\alpha_k, \alpha_k(n), k, n-~k\big)$ to $\alpha^{\mathcal{W}}\big(\alpha_{k+1}, \alpha_{k+1}(n), k+1, n-k-1\big)$ consists of the following computations:
 \begin{enumerate}[label={(\arabic*)}]
	\item $\cdt{\alpha_k}{1}(x)$ given $x\triangleq \alpha_k(n)$, which takes time $\runtime\big(\alpha_{k+1}, n\big) - \runtime\big(\alpha_k, n\big)$ by Lemma~\ref{lem: compose-runtime}.
	\item $\alpha_k(n) - k$, which takes time $\Theta(k)$.
\end{enumerate}
The computation will terminate at $k = \alpha(n)$. Thus $\forall n\ge 1$,
\begin{equation} \label{eq: inv-ack-runtime-improved}
\begin{aligned}
\runtime\big(\tilde{\alpha}, n\big) & = \runtime\big(\alpha_1, n\big) + \sum_{k=1}^{\alpha(n) - 1}
\left[ \runtime\big(\alpha_{k+1}, n\big) - \runtime\big(\alpha_k, n\big)
\right] + \sum_{k=1}^{\alpha(n)}\Theta(k) \\
& = \runtime\left(\alpha_{\alpha(n)}, n\right) + \Theta\left(\alpha(n)^2\right)
= \Theta\left(2n + 4^{\alpha(n)}\log_2n + \alpha(n)^2\right) = \Theta(n)
\end{aligned}
\end{equation}
Therefore, $\tilde{\alpha}$ is able to compute $\alpha$ in linear time.
