We provide a runtime analysis for the
inverse Ackermann function. First, we show that the
runtime of the version defined in Definition~\ref{thm: inv-ack-hier-correct}
is $O(n^2)$. We then provide a simple improvement that
cuts the runtime to $O(n)$.

%\footnote{We will be careful not to conclude that functions
%$f$ and $g$ are equal when they agree on all inputs but are computed with
%different pieces of code.} 
% momentarily forget about the axiom of extensionality and 
\subsection{Basic Time Analysis: $O(n^2)$}
To formalize the notion of runtime, we identify each function on 
$\mathbb{N}$ with its \emph{computation}, \emph{i.e.} the program that computes 
it in Coq \note{under a call-by-value strategy}.
\begin{defn}
 For a function on $k$ variables, $f:\mathbb{N}^k\to\mathbb{N}$, the \emph{runtime} of $f$ on input $\big(n_1, n_2, \ldots, n_k\big)$, denoted by $\runtime_f\big(n_1, n_2, \ldots, n_k\big)$, is the number of basic computational steps it takes to compute $f(n_1, n_2, \ldots, n_k)$.
\end{defn}
The following two lemmas are properties of Coq functions with trivial proofs.
\begin{lem} \label{lem: sub-runtime}
	Let $\li{leb}(x, y)$ returns \li{true} if $x\le y$ and \li{false} otherwise. Then in the unary encoding system, \emph{i.e.} Coq type \li{nat}, $\runtime_{\li{leb}}(n, a) = \min\{a, n\} + 1$, and in the binary encoding system, \emph{i.e.} Coq type \li{N}, $\runtime_{\li{leb}}(n, a) = \lfloor \log_{2}\min\{a, n\} \rfloor + 1$ per Coq's definitions of \li{leb} on these types.
\end{lem}
\begin{lem} \label{lem: compose-runtime}
	In all encoding systems, $\runtime_{f\circ g}(n) = \runtime_f(g(n)) + \runtime_g(n)$ per Coq's definition of functional composition.
\end{lem}
The next lemma establishes the general running time structure of countdown.
\begin{lem} \label{lem: cdt-runtime-general}
	For all $a\in \mathbb{N}$ and $f\in \contract_{a}$,
	\begin{equation} \label{eq: cdt-runtime-struct}
		\runtime_{\cdt{f}{a}}(n) =
		\sum_{i=0}^{\cdt{f}{a}(n) - 1} \runtime_f\left(f^{(i)}(n)\right)
		+ \sum_{i=0}^{\cdt{f}{a}(n)}\Tleb\left(f^{(i)}(n), a\right)
		+ \sum_{i=0}^{\cdt{f}{a}(n) - 1}\Tsucc(i),
	\end{equation}
	where $\li{succ}(x) = x+1$.
\end{lem}
\begin{proof}
	Since $f\in \contract_{a}$, $\cdt{f}{a}\ $ is the minimum $k$ such that $f^{(k)}(n) \le a$. The execution of $\cdw{f}{a}\big(n, n\big)$ then takes $k+1$ recursive calls, where the $i^{th}$ call for $0\le i \le k$ takes the arguments $i, a$ and $n_i \triangleq f^{(i)}(n)$ from the previous call (or the initial argument when $i = 0$), and performs the following computations:
	\begin{enumerate}
		\item Compute $\li{leb}\left(n_i, a\right)$ for $\Tleb\left(f^{(i)}(n), a\right)$ steps.
		\item If $\li{leb}\left(n_i, a\right) = \li{true}$, return $0$. Else proceed to next step.
		\item Compute $n_{i+1} \triangleq f(n_i) = f^{(i+1)}(n)$ for $T_f\left(f^{(i)}(n)\right)$ steps.
		\item Pass $n_{i+1}, i+1, a$ to the next recursive call and waits for it to return $k - i - 1$.
		\item Add $1$ to the result for $\Tsucc(k-i-1)$ steps and returns $k - i$.
	\end{enumerate}
    Summing up the time of each call, we get the desired expression for $\runtime_{\cdt{f}{a}}(n)$.
\end{proof}

\begin{lem} \label{lem: cdt-runtime}
	Per Definition~\ref{defn: countdown} of countdown, in the unary encoding system, \emph{i.e.} the type \li{nat} in Coq, $\forall a\ge 1, \forall n, \forall f\in \contract_{a}$ we have
	\begin{equation*}
	\runtime_{\cdt{f}{a}}(n) =
	\sum_{i=0}^{\cdt{f}{a}(n) - 1} \runtime_f\left(f^{(i)}(n)\right)
	+ (a + 2)\cdt{f}{a}(n) + f^{\left(\cdt{f}{a}(n)\right)}(a) + 1.
	\end{equation*}
\end{lem}
\begin{proof}
	By definitions of \li{leb} and \li{succ} on \li{nat} and countdown, we have for all $i$ and $n$,
 $\Tleb\left(f^{(i)}(n), a\right) = a + 1$ if $i < \cdt{f}{a}(n)$ and $f^{(i)}(n) + 1$ otherwise, and $\Tsucc(i) = 1$. Therefore the third summand is equal to $\cdt{f}{a}(n)$ and the second summand in \eqref{eq: cdt-runtime-struct} is equal to $(a + 1)\cdt{f}{a}(n) + f^{\left(\cdt{f}{a}(n)\right)}(a) + 1$, thus forming the desired formula.
%	Per Definition~\ref{defn: countdown-worker}, the computation makes $\cdt{f}{a}(n)$ recursive calls to $\W\cdt{f}{a}$ before terminating. At the $(i+1)^{\text{th}}$ call, two computations must take place: $n_i - a$, which takes $\Theta(a + 1)$ time, and $f(n_i) = n_{i+1}$, where $n_i \triangleq f^{(i)}(n)$ has been  computed by the $i$th call, and is greater than $a$.  The total time is then
%	\begin{equation*}
%	\begin{aligned}
%	\runtime\big(\cdt{f}{a}\ , n\big)
%	& = \sum_{i=0}^{\cdt{f}{a}(n) - 1} \left[\runtime\left(f, f^{(i)}(n)\right) + \Theta(a + 1)\right] \\
%	& = \sum_{i=0}^{\cdt{f}{a}(n) - 1} \runtime\left(f, f^{(i)}(n)\right) + \Theta\big((a + 1)\cdt{f}{a}(n)\big)
%	\end{aligned}
%	\end{equation*}
\end{proof}

From Lemma~\ref{lem: compose-runtime} and Lemma~\ref{lem: cdt-runtime}, the following lemma follows easily:
\begin{lem} \label{lem: inv-ack-hier-runtime}
	Per Definition~\ref{defn: inv-ack-hier} and Definition~\ref{defn: countdown},
	\begin{equation*}
	\runtime_{\alpha_{i+1}}(n) = \sum_{k=0}^{\alpha_{i+1}(n)}\runtime_{\alpha_i}\left( \alpha_i^{(k)}(n)\right) + 3\alpha_{i+1}(n) + C_i(n)
	\end{equation*}
	where $C_i(n) \triangleq \alpha_i^{(\alpha_{i+1}(n) + 1)}(n) + 1 \in \{1, 2\}$ for all $i$ and $n$.
\end{lem}
This lemma implies $\runtime_{\alpha_{i+1}}(n) \ge \runtime_{\alpha_i}(n)$. If there is some $i$ such that $\runtime_{\alpha_i}(n) = \Theta\big(n^2\big)$, each function after $\alpha_i$ in the hierarchy will take at least $\Omega\big(n^2\big)$ to compute, thus making $\runtime(\alpha, n) = \Omega\big(n^2\big)$ per Definition~\ref{defn: inv-ack-worker}. The next lemma shows that $i = 2$ suffices.
\begin{lem}
	Per Definition~\ref{defn: inv-ack-hier}, $\runtime_{\alpha_2}(n) = \Theta\big(n^2\big)$
\end{lem}
\begin{proof}
	$\alpha_1 = \big(\cdt{\lambda m.(m-1)}{1}\big)\circ \big(\lambda m.(m-1)\big) = \lambda m.(m - 2)$. By Lemma~\ref{lem: inv-ack-hier-runtime},
	\begin{equation*}
	\runtime_{\alpha_1}(n) \ge \textstyle \sum_{i=0}^{n-1} \runtime\big(\lambda m.(m-1), n - i\big) + 3(n - 2) + 1 = 4n - 5\text{,}
	\end{equation*}
	since $\runtime\big(\lambda m.(m-1), k\big) = 1$. Because $\alpha_2 = \big(\cdt{\alpha_1}{0}\big)\circ \alpha_1 $, again by Lemma~\ref{lem: inv-ack-hier-runtime},
	\begin{equation*}
	\runtime_{\alpha_2}(n)
	\ge \textstyle \sum_{i=0}^{\left\lceil \frac{n-3}{2} \right\rceil} \big(4(n-2i) - 5\big) + 3\left\lceil \frac{n-3}{2} \right\rceil + 1
	= \Theta\big(n^2\big)
	\end{equation*}
\end{proof}
\begin{col}
	$\runtime(\alpha, n) = \Omega\big(n^2\big)$ per Definition~\ref{defn: inv-ack-worker}.
\end{col}

%This lemma implies $\runtime\big(\cdt{f}{a}\ , n\big) \ge \runtime(f, n)$. If for some $i$, $\runtime\big(2\angle{i}, n\big) = \Theta\big(n^2\big)$, the entire hierarchy will take at least $\Omega\big(n^2\big)$ from $2\angle{i}$, thus making $\runtime(\alpha, n) = \Omega\big(n^2\big)$ per \cref{defn: inv-ack-worker}. The next lemma shows $i = 2$ suffices.
%\begin{lem}
%	Per \cref{defn: inv-hyperop}, $\runtime\big(2\angle{2}, n\big) = \Theta\big(n^2\big) \ \forall n$.
%\end{lem}
%\begin{proof}
%	Since $2\angle{1} = \cdt{\lambda m.(m-1)}{2}\ \equiv \lambda m.(m - 2)$, by \cref{lem: cdt-runtime},
%	$$ \runtime\big(2\angle{1}, n\big) = \sum_{i=0}^{n-2} \runtime\big(\lambda m.(m-1), n - i\big) + \Theta\big(3(n-2)\big) = \Theta(n) $$
%	, since $\runtime\big(\lambda m.(m-1), k\big) = 1 \ \forall k$. Since $2\angle{2} = \cdt{2\angle{1}}{0}\ $, again by \cref{lem: cdt-runtime},
%	$$ \runtime\big(2\angle{2}, n\big)
%	= \sum_{i=0}^{\lceil n/2 \rceil} \runtime \big(2\angle{1}, n-2i\big) + \Theta\left(\frac{n}{2}\right)
%	= \Theta\left( \sum_{i=0}^{\lceil n/2 \rceil}(n - 2i) \right)
%	= \Theta\big(n^2\big) $$
%	The proof is complete.
%\end{proof}
%\begin{col}
%	$\runtime(\alpha, n) = \Omega\big(n^2\big)$ per \cref{defn: inv-ack-worker}.
%\end{col}
%Intuitively, the function $2\angle{1} \equiv \lambda n.(n-2)$ is responsible for dragging the whole hierarchy's performance due to one silly weakness: its does not know it will always output $n-2$ before beginning its computation, hence needs to tediously subtract $1$ until it goes below $2$. This observation leads to the next improvement.

\subsection{Hard-coding the second level: $O(n)$} \label{sect: hardcode-lvl2}

Intuitively, it is clear that the function $\alpha_2 \equiv \lambda n.(n-2)$
slows down the entire hierarchy's performance due to one silly weakness:
it does not know that it will always output $n-2$ before beginning its computation,
and so it tediously subtracts $1$ until \note{it goes below $2$.}
%This observation leads to the next improvement.

We can hardcode $\alpha_1$ as $\lambda n.(n-2)$ to reduce its runtime
from $\Theta(n)$ to $\Theta(1)$.
\begin{lstlisting}
Definition sub_2 (n : nat) : nat :=
  match n with | 0 => 0 | 1 => 1 | S (S n') => n' end.
\end{lstlisting}
%Without loss of generality, let us henceforth assume that the constant factors in $\runtime(\alpha_1, n)$ and Lemma~\ref{lem: inv-ack-hier-runtime} are both $1$.
The runtime for $\alpha_2$ can thus be bounded as follows:
\begin{equation*}
\runtime_{\alpha_2}(n)
 \le \textstyle \sum_{i=0}^{\left\lceil \frac{n-3}{2} \right\rceil} \runtime_{\alpha_1}\big(n-2i\big) + 3\left\lceil \frac{n-3}{2} \right\rceil + 2  =  4\left\lceil \frac{n-3}{2} \right\rceil + 2
 \le 2n - 2
\end{equation*}
Note that the above bound only applies for $n\ge 2$. When $n\le 1$, the LHS is 1. In fact, this allows every function in the hierarchy to be computed in linear time:
\begin{thm} \label{thm: inv-ack-hier-runtime-improved}
	$\runtime_{\alpha_i}(n) \le 4n + 19\cdot 2^{i-3}\log_2n + 2i$.
\end{thm}
We need three crucial technical lemmas to prove this theorem.
\begin{lem} \label{lem: inv-ack-3-runtime}
	If $n\ge 1$, $\runtime_{\alpha_3}(n) = 2$ $\runtime_{\alpha_3}(n) \le 4n + 4$.
\end{lem}
\begin{proof}
	It is easy to show that $\alpha_2^{(k)}(n) = \left\lfloor \frac{n+2}{2^k} \right\rfloor - 2$. Thus
	\begin{equation*}
	\begin{aligned}
		\runtime_{\alpha_3}(n)
		& \textstyle \le \ \sum_{k=0}^{\alpha_{3}(n)}\runtime_{\alpha_2}\left(\left\lfloor \frac{n+2}{2^k} \right\rfloor - 2\right) + 3\alpha_{3}(n) + 2 \\
		& \textstyle \le \ 2\sum_{k=0}^{\alpha_3(n)}\left(\frac{n+2}{2^k} - 3\right) + 3\alpha_3(n) + 2 \\
		& \le \ 4(n + 2) - 6(\alpha_3(n) + 1) + 3\alpha_3(n) + 2 \le 4n + 4.
	\end{aligned}
\end{equation*}
%	$\runtime_{\alpha_3}(n) \le $
%	\begin{equation*}
%\begin{array}{@{}l@{}}
%	\sum_{k=0}^{\alpha_{3}(n)}\runtime_{\alpha_2}\left(\left\lceil \frac{n+3}{2^k} \right\rceil - 3\right) + 3\alpha_{3}(n) + 2  \quad
%	\le \quad 2\sum_{k=0}^{\alpha_3(n)}\frac{n+3}{2^k} - 3\big(\alpha_3(n) + 1\big) + 3\alpha_3(n) + 2 \\
%	\le 4(n + 3) - 1 \le \quad 4n + 11 \\
%\end{array}
%	\end{equation*}
\end{proof}
\begin{lem}
	$\forall n$, $\displaystyle \sum_{k = 0}^{\log_2^*(n) - 1}\log_2^{(k)}n \le 2n$, where $\log_a^*(n) \triangleq \min\{k : \log_a^{(k)}n \le 1\}$. Note that the sum is $0$ when $n\le 1$.
\end{lem}
\begin{proof}
	Define $S: \mathbb{R}_{\ge 0}\to \mathbb{R}_{\ge 0}$ where $S(x) \triangleq \sum_{k = 0}^{\log_2^*(x) - 1}\log_2^{(k)}x$. Clearly $S$ is strictly increasing and $\forall x>1, S(x) = n + S(\log_2 x)$. Using the fact $\log_2^*(x)\in \mathbb{N}$, we prove by induction the statement $P(k) \triangleq$ $\forall x: \log_2^*(x) = k$, $S(x)\le 2x$.
	\begin{itemize}
		\item \emph{Base case.} $P(0)$, $P(1)$ hold trivially and $P(2) = \forall x: 2 < x \le 4$, $x + \log_2x \le 2x$, which is equivalent to $\log_2x \le x$, which holds for all $x > 2$.
		
		\item \emph{Inductive case.} Assume $P(k-1)$ holds where $k\ge 3$. Fix any $x$ such that $\log_2^*(x)=k$, then $x > 4$. The function $\lambda x.\frac{x}{\log_2x}$ is increasing on $[4, +\infty)$, so $\frac{4}{\log_24} \le \frac{x}{\log_2x}$ or $2\log_2x \le x$. Since $\log_2^*(x) = k$, we have $\log_2^*(\log_2x) = k-1$. By $P(k-1)$, $S(x) = x + S(\log_2x) \le x + 2\log_2x \le 2x$, which completes the proof.
	\end{itemize}
\end{proof}
\begin{lem} \label{lem: sum-alpha-repeat}
	$\forall l, \forall i \ge 3, \forall s\le \alpha_{i+1}(n)$.~$\displaystyle \sum_{k=s}^{\alpha_{i+1}(n)} \log_2^{(l)}\alpha_i^{(k)}(n) \le 2\log_2^{(l+s+1)}n$.
\end{lem}
\begin{proof}
%	Let the LHS be $S_i(n)$. Firstly, consider $i = 3$. Note that for $n\le 13$, $S_3(n) = 0$ and for $n\ge 14$, i.e. $\alpha_3(n)\ge 2$, $S_3(n) = \alpha_3(n) + S_3\big(\alpha_3(n)\big)$. The result thus holds for $n\le 13$. Suppose it holds for all $m < n$, where $n\ge 14$. Then
%	\begin{equation*}
%	S_3(n) \quad \le \quad \alpha_3(n) + 3\big\lceil \log_2(\alpha_3(n)) \big\rceil \quad \le \quad \big\lceil \log_2n \big\rceil + 3\big\lceil \log_2\log_2n \big\rceil
%	\end{equation*}
%	It is easy to prove \, $2\big\lceil \log_2\log_2n \big\rceil \le \big\lceil \log_2n \big\rceil$ by induction on $\big\lceil \log_2n \big\rceil$. Thus $S_3(n)~\le~3\big\lceil \log_2n \big\rceil$, as desired. Now for $i \ge 4$,
  $\forall i\ge 3, \forall n, \alpha_i(n)\le \log_2n$ and $\alpha_{i+1}(n)\le \log_2^*(n)-1$, therefore
	\begin{equation*}
%	S_i(n) \ = \
	\sum_{k=s}^{\alpha_{i+1}(n)} \log_2^{(l)}\alpha_i^{(k)}(n) \ \le \
%	\sum_{k=1}^{\log_2^*(n)-1} \log_2^{(l)}\alpha_i^{(k)}(n) \ \le \
	\sum_{k=s}^{\log_2^*(n)-1} \log_2^{(l+s+k)}(n) \ \le \
	2\log_2^{(l+s+1)} n
	\end{equation*}
%	Let $P(n) \triangleq 2\big\lceil \log_2\log_2n \big\rceil \le \big\lceil \log_2n \big\rceil$. It suffices to prove $P(n) \ \forall n$. Observe that $P(n)$ holds for $n\ge 4$.
\end{proof}
\begin{proof}[of Theorem~\ref{thm: inv-ack-hier-runtime-improved}]
	We have proved the result for $i = 0, 1, 2$. Let us proceed with $i\ge 3$ by induction. The case $i = 3$ has been covered by Lemma~\ref{lem: inv-ack-3-runtime}. Let $M_i \triangleq 2^{i-3}19 - 2i - 13$ for each $i$ and suppose the result holds for $i\ge 3$, we have
	\begin{equation*}
	\begin{array}{@{}l@{}}
	 \runtime_{\alpha_{i+1}}(n) \le \sum_{k=0}^{\alpha_{i+1}(n)} \runtime_{\alpha_i}\big(\alpha_i^{(k)}(n)\big) + 3\alpha_{i+1}(n) + 2 \\[3pt]
	\le \sum_{k=0}^{\alpha_{i+1}(n)}\big(4\alpha_i^{(k)}(n) + M_i\log_2\alpha_i^{(k)}(n) + 2i \big) + 3\alpha_{i+1}(n) + 2 \\[3pt]
%	\le 4n + 2(i+1) + (2i + 3)\alpha_{i+1}(n) + 4\sum_{k=1}^{\alpha_{i+1}(n)}\alpha_i^{(k)}(n) + M_i\sum_{k=0}^{\alpha_{i+1}(n)}\log_2\alpha_i^{(k)}(n) \\[3pt]
  \le 4n + 2(i+1) + (2i + 3)\underbrace{\alpha_{i+1}(n)}_{\le \log_2n} + 4\underbrace{\textstyle \sum_{k=1}^{\alpha_{i+1}(n)}\alpha_i^{(k)}(n)}_{\le 2\log_2n} + M_i\underbrace{\textstyle \sum_{k=0}^{\alpha_{i+1}(n)}\log_2\alpha_i^{(k)}(n)}_{\le 2\log_2n} \\[3pt]
	\le 4n + 2(i+1) + (2M_i + 2i + 3 + 8)\log_2n
	= 4n + M_{i+1}\log_2n + 2(i+1).
%	\le 4n + M_i\left\lceil\log_2n\right\rceil + 5 + (M_i+2)\sum_{k=1}^{\alpha_{i+1}(n)}\alpha_i^{(k)}(n) + 6\alpha_{i+1}(n) \\
%	\le 2n + M_i\left\lceil\log_2n\right\rceil + 5 +
%	3(M_i + 2)\left\lceil\log_2n\right\rceil + 6\left\lceil\log_2n\right\rceil ~~
%	= ~~ 2n + (4M_i + 12)\left\lceil\log_2n\right\rceil + 5 \\
%	= 2n + M_{i+1}\left\lceil\log_2n\right\rceil + 5\text{, since $4M_i + 12 = 4^{i+2} - 16 + 12 = M_{i+1}$}.
	\end{array}
	\end{equation*}
\end{proof}
By hardcoding $\alpha_1$ as $\lambda m.(m-2)$, the $i$th level of the inverse Ackermann hierarchy can be computed in $\Theta\big(n + 2^i\log n  + i\big)$ time, \emph{i.e.} linear time $\Theta(n)$ for fixed $i$.

%\subsection{An improved inverse Ackermann computation: $O(n)$}
%Building on the previous improvement, we can improve the running time of $\alpha(n)$ per \Cref{defn: inv-ack-hier} by hard-coding the output when $n\le 1 = \Ack(0, 0)$, and starting $\W\alpha$ with $f := \alpha_1$ and $n := \alpha_1(n)$. In other words,
This also allows us to improve the runtime of $\alpha(n)$ per Definition~\ref{defn: inv-ack-hier} by hardcoding the output when $n\le 1 = \Ack(0, 0)$, and starting $\alpha^{\mathcal{W}}$ with $f = \alpha_1$ and $n = \alpha_1(n)$:
\begin{equation*}
\tilde{\alpha}(n) = \begin{cases}
0 & \text{ if } n \le 1 \\ \alpha^{\mathcal{W}}\big(\alpha_1, \alpha_1(n), 1, n-1\big) & \text{ if } n \ge 2
\end{cases}
\end{equation*}
For $n > 1$, $1\le \min\big\{n-1, \alpha_1(n)\big\}$. So
$\alpha^{\mathcal{W}}\big(\alpha_0, \alpha_0(n), 0, n\big) =
\alpha^{\mathcal{W}}\big(\alpha_1, \alpha_1(n), 1, n-~1\big)$.
Thus $\tilde{\alpha}(n) = \alpha(n)$. At each recursive step, the transition from $\alpha^{\mathcal{W}}\big(\alpha_k, \alpha_k(n), k, n-~k\big)$ to $\alpha^{\mathcal{W}}\big(\alpha_{k+1}, \alpha_{k+1}(n), k+1, n-k-1\big)$ consists of the following computations:
 \begin{enumerate}[label={(\arabic*)}]
	\item $\cdt{\big(\alpha_k\big)}{1}(x)$ given $x\triangleq \alpha_k(n)$, taking time $\runtime_{\alpha_{k+1}}(n) - \runtime_{\alpha_k}(n)$ by Lemma~\ref{lem: compose-runtime}.
	\item $\alpha_k(n) - k$, taking time $\Theta(k)$.
\end{enumerate}
The computation will terminate at $k = \alpha(n)$. Thus $\forall n\ge 1$,
\begin{equation} \label{eq: inv-ack-runtime-improved}
\begin{aligned}
\runtime_{\tilde{\alpha}}(n)
& = \textstyle \runtime_{\alpha_1}(n) + \sum_{k=1}^{\alpha(n) - 1}
\left[ \runtime_{\alpha_{k+1}}(n) - \runtime_{\alpha_k}(n)
\right] + \sum_{k=1}^{\alpha(n)}\Theta(k) \\
& = \runtime_{\alpha_{\alpha(n)}}\left(n\right) + \Theta\left(\alpha(n)^2\right)
= \Theta\left(n + 2^{\alpha(n)}\log_2n + \alpha(n)^2\right) = \Theta(n).
\end{aligned}
\end{equation}
Therefore, $\tilde{\alpha}$ is able to compute $\alpha$ in linear time.
