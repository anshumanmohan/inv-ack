Let us formally define hyperoperations and clarify the intuition
given by Table~\ref{table: hyperop-ack-inv} by relating them to
the Ackermann function.
The first hyperoperation (level 0) is simply successor, and
every hyperoperation that follows is the repeated application of the previous.
Level 1 is thus addition, and $b$ repetitions of addition
gives level 2, multiplication. Next, $b$ repetitions of
multiplication gives level 3, exponentiation.
There is a subtlety here: in the former case, we add $a$
repeatedly to an \emph{initial value}, which should be the additive identity $0$.
In the latter case, we multiply $a$ repeatedly to an initial value, which
which should now be the multiplicative identity $1$. The formal definition of hyperoperation is:
%\marginpar{\tiny \color{blue} Should 2 just be under 1?}
\begin{equation}
\label{eq:hyper}
\begin{array}{lrcl}
\textit{1. 0$^{\textit{th}}$ level: } & a[0]b & ~ \triangleq ~ & b + 1 \\
\textit{2. Initial values: } & a[n+1]0 & ~ \triangleq ~ &
  \begin{cases}
    a & \text{when } n = 0 \\
    0 & \text{when } n = 1 \\
    1 & \text{otherwise}
  \end{cases} \\
\textit{3. Recursive rule: } \quad & a[n+1](b+1) & ~ \triangleq ~ & a[n]\big(a[n+1]b\big)
\end{array}
\end{equation}
The seemingly complicated recursive rule is actually just \emph{repeated application} in disguise. By fixing $a$ and treating $a[n]b$ as a function of $b$, we can write
% \begin{equation*}
% \begin{array}{lll}
% a[n+1]b & ~ = ~ a[n]\big(a[n+1](b-1)\big) & ~ = ~ a[n]\big(a[n](a[n+1](b-2))\big) \\
%  & ~ = ~ \underbrace{\big( a[n]\circ a[n]\circ \cdots \circ a[n] \big)}_{b \text{ times}} \big(a[n+1]0\big) & ~ = ~ \big(a[n]\big)^{(b)}\big(a[n+1]0\big)
% \end{array}
% \end{equation*}
\begin{equation*}
\begin{array}{lll}
a[n+1]b ~& = ~ a[n]\big(a[n+1](b-1)\big) ~ = ~ a[n]\big(a[n](a[n+1](b-2))\big) \\
& = ~ \underbrace{\big( a[n]\circ a[n]\circ \cdots \circ a[n] \big)}_{b \text{ times}} \big(a[n+1]0\big)  ~ = ~ \big(a[n]\big)^{(b)}\big(a[n+1]0\big)
\end{array}
\end{equation*}
%where $f^{(k)}(u) ~ \triangleq ~ \overbrace{(f\circ f\circ \cdots \circ f)}^{k \text{ times}} (u)$,
where $f^{(k)}(u) \triangleq ~ (f\circ f\circ \cdots \circ f)(u)$ denotes $k$ compositional applications of a function~$f$ to an
input~$u$; $f^{(0)}(u) = u$ (\emph{i.e.} applying $0$ times yields the identity).

This insight helps us encode hyperoperations~(\ref{eq:hyper}) and
Ackermann~(\ref{eq:ackermann}) in Coq.  Notice that the recursive case of hyperoperations --- and
indeed, the third case of Ackermann --- feature deep nested recursion, 
which makes our task tricky. 
In the outer recursive call, the first argument is shrinking
but the second is expanding explosively; in the inner recursive call, the first argument is
constant but the second is shrinking. The elegant solution uses double recursion~\cite{bertotcast} as follows:
\begin{lstlisting}
`\href{https://github.com/inv-ack/inv-ack/blob/6099297c6ab0e16d14b037fb5ed600c4d22818f6/repeater.v#L39-L40}{Definition hyperop\_init}` (a n : nat) : nat :=
  match n with 0 => a | 1 => 0 | _ => 1 end.

`\href{https://github.com/inv-ack/inv-ack/blob/6099297c6ab0e16d14b037fb5ed600c4d22818f6/repeater.v#L43-L52}{Fixpoint hyperop\_original}` (a n b : nat) : nat :=
  match n with
  | 0    => 1 + b
  | S n' => let fix hyperop' (b : nat) :=
              match b with
              | 0    => hyperop_init a n'
              | S b' => hyperop_original a n' (hyperop' b')
              end in hyperop' b
  end.

`\href{https://github.com/inv-ack/inv-ack/blob/6099297c6ab0e16d14b037fb5ed600c4d22818f6/repeater.v#L107-L116}{Fixpoint ackermann\_original}` (m n : nat) : nat :=
  match m with
  | 0    => 1 + n
  | S m' => let fix ackermann' (n : nat) : nat :=
              match n with
              | 0    => ackermann_original m' 1
              | S n' => ackermann_original m' (ackermann' n')
              end in ackermann' n
  end.
\end{lstlisting}
Coq is satisfied since both recursive calls are on structurally smaller arguments.
Moreover, our encoding makes the structural similarities
 readily apparent.  In fact, the only essential difference is the initial values
(\emph{i.e.} the second case of both definitions): the Ackermann function uses $\Ack(n-1,1)$, whereas
hyperoperations use the initial values given in Equation~\ref{eq:hyper}.

Having noticed that the deep recursion in both notions is expressing the same idea
of repeated application, we arrive at another useful idea. We can express the relationship
between the $(\text{n+1})^{\text{th}}$ and $\text{n}^{\text{th}}$ levels in
a \emph{functional} way via a higher-order function that transforms the latter level
to the former. For this we employ a version of the well-known function iterator
\li{iter}~\cite{bertotcast}:
\begin{defn}
$\forall a\in \mathbb{N}, f: \mathbb{N}\to \mathbb{N}$, the \emph{repeater from} $a$ of $f$, denoted by $\rf{f}{a}$ , is a function $\mathbb{N}\to \mathbb{N}$ such that $\rf{f}{a}(n) = f^{(n)}(a)$.
\begin{lstlisting}
`\href{https://github.com/inv-ack/inv-ack/blob/6099297c6ab0e16d14b037fb5ed600c4d22818f6/repeater.v#L21-L25}{Fixpoint repeater\_from}` (f : nat -> nat) (a n : nat) : nat :=
  match n with 0 => a | S n' => f (repeater_from f a n') end.
\end{lstlisting}
\end{defn}
This allows simple and function-oriented definitions of hyperoperations and the
Ackermann function that we give below. Note that the Curried $a[n-1]$ denotes
the single-variable function $\lambda b.a[n-1]b$.
\vspace{-0.5em}
\begin{equation*}
\vspace{-0.75em}
a[n]b ~ \triangleq ~ \begin{cases}
b + 1 & \text{when } n = 0 \\
\rf{a[n-1]}{a_{n-1}}(b) & \text{otherwise}
\end{cases}
\quad \text{ where } \ a_n ~ \triangleq ~ \begin{cases}
a & \text{when } n = 0 \\
0 & \text{when } n = 1 \\
1 & \text{otherwise}
\end{cases}
\end{equation*}
\begin{lstlisting}
`\href{https://github.com/inv-ack/inv-ack/blob/6099297c6ab0e16d14b037fb5ed600c4d22818f6/repeater.v#L55-L59}{Fixpoint hyperop}` (a n b : nat) : nat :=
  match n with
  | 0    => 1 + b
  | S n' => repeater_from (hyperop a n') (hyperop_init a n') b
  end.
\end{lstlisting}
\begin{equation*}
%\begin{array}{l}
A(n,m) ~ \triangleq ~ \begin{cases}
m + 1 & \text{when } n = 0 \\
\rf{A_{n-1}}{A(n-1,1)}(m) & \text{otherwise}
\end{cases} \qquad \qquad \qquad \qquad \qquad \qquad \qquad ~ 
%\end{array}
\end{equation*}
\begin{lstlisting}
`\href{https://github.com/inv-ack/inv-ack/blob/6099297c6ab0e16d14b037fb5ed600c4d22818f6/repeater.v#L119-L123}{Fixpoint ackermann}` (n m : nat) : nat :=
  match n with
  | 0    => S m
  | S n' => repeater_from (ackermann n') (ackermann n' 1) m
  end.
\end{lstlisting}
In the rest of this paper we construct efficient inverses to these
functions.  Our key idea is an inverse to the higher-order repeater function, which we call \emph{countdown}.

% Aquinas' promise:
% We explain our core techniques of repeaters and countdowns that allow us to define each level of the Ackermann hierarchy—and their upper inverses—in a straightforward and uniform manner. We show how countdowns, in particular, can be written structurally recursively.

% Anshuman proposes:
% We introduce our core techniques of repeaters and countdowns.
% We show how countdowns, in particular, can be written structurally recursively.







