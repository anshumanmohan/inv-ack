\section{Standalone Code for Linear-Time $\alpha$ in \li{nat}}
\label{apx:standalone_nat}

\begin{lstlisting}
Require Import Omega.
Require Import Program.Basics.

`\href{https://github.com/inv-ack/inv-ack/blob/6099297c6ab0e16d14b037fb5ed600c4d22818f6/countdown.v#L94-L100}{Fixpoint cdn\_wkr}` (a : nat) (f : nat -> nat) (n b : nat) : nat :=
match b with 0 => 0 | S b' => 
  if (n <=? a) then 0 else S (cdn_wkr f a (f n) k')
end.

`\href{https://github.com/inv-ack/inv-ack/blob/6099297c6ab0e16d14b037fb5ed600c4d22818f6/countdown.v#L104}{Definition countdown\_to}` a f n := cdn_wkr a f n n.

`\href{https://github.com/inv-ack/inv-ack/blob/6099297c6ab0e16d14b037fb5ed600c4d22818f6/inv_ack.v#L138-L148} {Fixpoint inv\_ack\_wkr}` (f : nat -> nat) (n k b : nat) : nat :=
match b with 0 => 0 | S b' => 
  if (n <=? k) then k else let g := (countdown_to f 1) in
                        inv_ack_wkr (compose g f) (g n) (S k) b
end.

Definition inv_ack_linear_time n :=
  match n with
  | 0 | 1 => 0
  | _     => let f := (fun x => x - 2) in
               inv_ack_wkr f (f n) 1 (n - 1)
  end.
\end{lstlisting}

\section{Proofs of Correctness for Worker Functions}
\label{apx:proof_correct_worker}

\subsection{Countdown Worker}
\label{apx:proof_correct_countdown_worker}
This subsection provides the fully detailed proof of Theorem~\ref{thm: cdt-repeat}, which states that the function in Definition~\ref{defn: countdown} correctly computes the countdown value as defined in Definition~\ref{defn: informal-countdown}.

We begin with a lemma demonstrating the internal working of \emph{countdown worker} at the $\text{i}^\text{th}$ recursive step, including the accumulated result $1+i$, the current input $f^{(1+i)}(n)$, and the current budget $b-i-1$.

\begin{lem} \label{lem: cdt-intermediate}
	$\forall a, n, b, i \in \mathbb{N}.~\forall f \in \contract$. such that $i < b$ and $a < f^{(i)}(n)$:~
	\begin{equation}  \label{eq: cdt-intermediate}
	\cdw{f}{a}(n, b) = 1 + i + \cdw{f}{a}\left(f^{(1+i)}(n), b - i - 1\right)
	\end{equation}
\end{lem}

\begin{proof}
	Fix $a$. We proceed by induction on $i$. Then define $P(i)$ as
	\begin{equation*}
	\begin{aligned}
	P(i) \triangleq \forall b, \forall n. & b\ge i+1 \Rightarrow f^{(i)}(n) > a\\
	 & \Rightarrow \cdw{f}{a}(n, b) = 1 + i + \cdw{f}{a}\left(f^{(1+i)}(n), b - i - 1\right)
	\end{aligned}
	\end{equation*}
	\begin{itemize}[leftmargin=*]
		\item \textit{Base case.} For $i = 0$, our goal $P(0)$ is:
		$\cdw{f}{a}(n, b) = 1 + \cdw{f}{a}\left(f(n), b - 1\right)$
		where $b \ge 1$ and $f(n)\ge a+1$. This is straightforward.
		\item \textit{Inductive step.} Assume $P(i)$ has been proved. Then $P(i+1)$ is
		\begin{equation*}
		\cdw{f}{a}(n, b) = 2 + i + \cdw{f}{a}\left( f^{(2+i)}(n), b - i - 2 \right)
		\end{equation*}
		where $b \ge i+2$ and $f^{(1+i)}(n) \ge a+1$. This also implies $b\ge i+1$ and $\displaystyle f^{(i)}(n) \ge f^{(1+i)}(n)\ge a+1$ by $f\in \contract$, and so $P(i)$ holds. It suffices to prove:
		\begin{equation*}
		\cdw{f}{a}\left(f^{(1+i)}(n), b-i-1\right) = 1 + \cdw{f}{a}\left( f^{(2+i)}(n), b-i-2 \right)
		\end{equation*}
		This is in fact $P(0)$ with $(b, n)$ substituted for $\left(b-i-1, f^{(1+i)}(n)\right)$. Since $f^{(1+i)}(n) \ge a+1$ and $b-i-1\ge 1$, the above holds and $P(i+1)$ follows.
	\end{itemize}
\end{proof}

Now, we are ready to prove the correctness of the \emph{countdown} computation. We reproduce Theorem~\ref{thm: cdt-repeat} to offer a convenient reference point for readers.
%\begin{repthm}{thm: cdt-repeat}
%	$	\forall n.~ \cdt{f}{a}(n) = \min\left\{ i : f^{(i)}(n) \le a \right\} $.
%\end{repthm}

\begin{usethmcounterof}{thm: cdt-repeat}
	$	\forall n.~ \cdt{f}{a}(n) = \min\left\{ i : f^{(i)}(n) \le a \right\} $.
\end{usethmcounterof}

\begin{proof}
	Since $f\in \contract_{a}$ and $\mathbb{N}$ is well-ordered, 
	let $m = \min\big\{i : f^{(i)}(n)\le a\big\}$.\footnote{We prove the existence 
		of the min in Coq’s intuitionistic logic \href{https://github.com/inv-ack/inv-ack/blob/6099297c6ab0e16d14b037fb5ed600c4d22818f6/countdown.v\#L125-L150}{here} in our codebase.} Our goal becomes 
	$\cdt{f}{a}(n) = m$. Note that this setup gives us: 
	
	\begin{equation} \label{eq: cdt-repeat-tmp}
	\left(f^{(m)}(n) \le a\right) \wedge
	\left(\forall k.~f^{(k)}(n)\le a \Rightarrow m \le k\right)
	\end{equation}
	
	If $m = 0$, then $n = f^{(0)}(n)\le a$. So $\cdt{f}{a}(n) = \cdw{f}{a}(n, n) = 0 = m$ by Definition~\ref{lem: cdt-init}.
	When $m > 0$, our plan is to apply Lemma~\ref{lem: cdt-intermediate} to get
	
	\begin{equation*}
	\cdt{f}{a}(n) = \cdw{f}{a}(n, n) = m + \cdw{f}{a}\left(f^{(m)}(n), n-m\right),
	\end{equation*}
	
	and then use Definition~\ref{lem: cdt-init} over (\ref{eq: cdt-repeat-tmp})'s first conjunct to conclude $\cdt{f}{a}(n) = m$. It suffices to prove the premises of Lemma~\ref{lem: cdt-intermediate}: $a < f^{(m-1)}(n)$ and $m-1 < n$.
	
	The former follows by contradiction: if $f^{(m-1)}(n) \le a$, (\ref{eq: cdt-repeat-tmp})'s second conjunct implies $m\le m-1$, which is impossible for $m > 0$. The latter then easily follows by $f\in \contract_{a}$, since
	$n \ge 1 + f(n) \ge 2 + f(f(n)) \ge \cdots \ge m + f^{(m)}(n)$.
	\linebreak Therefore, $\cdt{f}{a}(n) = m$ in all cases, which completes the proof.
\end{proof}

\subsection{Inverse Ackermann Worker}
\label{apx:proof_correct_inv_ack_worker}
This subsection provides the fully detailed proof of Theorem~\ref{thm: inv-ack-worker-correct}, which states that given the appropriate arguments, the function \emph{inverse Ackermann worker} in Definition~\ref{defn: inv-ack-worker} correctly computes the value of the inverse Ackermann function as defined in Definition~\ref{defn: inv-ack-new}.

The following lemma illustrates the working of \emph{inverse Ackermann worker} by examining each recursive call made during the execution of $\alpha^{\mathcal{W}}\big(\alpha_0, \alpha_0(n), 0, b\big)$. 

\begin{lem} \label{lem: inv-ack-worker-intermediate}
	$\forall n, b, k$ such that $k\le \min\big\{b, \alpha_{k-1}(n)\big\}$, 
	\begin{equation*}
	\alpha^{\W}\big(\alpha_0, \alpha_0(n), 0, b\big) = \alpha^{\W}\big(\alpha_{k+1}, \alpha_{k+1}(n), k+1, b-k-1\big)
	\end{equation*}
\end{lem}

\begin{proof}[Lemma~\ref{lem: inv-ack-worker-intermediate}]
	Fix $n$ and $b$. We prove $P(k)$ by induction, where
	
	\begin{equation*}
	\begin{aligned}
	P(k) \quad \triangleq \quad & (\alpha_k(n)\ge k+1) \wedge (b\ge k+1) \implies \null \\ & \alpha^{\W}\big(\alpha_0, \alpha_0(n), 0, b\big) = \alpha^{\W}\big(\alpha_{k+1}, \alpha_{k+1}(n), k+1, b-k-1\big)
	\end{aligned}
	\end{equation*}
	
	\begin{itemize}[leftmargin=*]
		\item \emph{Base case}. This case, where $k = 0$, is trivial since both sides of the desired equality are identical.
%		$\alpha^{\W}\big(\alpha_0, \alpha_0(n), 0, b\big) = \alpha^{\W}\big(\cdt{\big(\alpha_0}{1}\big)\circ \alpha_0, \cdt{\alpha_0}{1}\big(\alpha_0(n)\big), 1, b - 1\big) $ by
%		(\ref{eq: inv-ack-worker-recursion}) with $b\ge 1$ and $\alpha_0(n)\le 1$.
%		Definition~\ref{defn: inv-ack-hier} gives $\alpha_1 = \big(\cdt{\alpha_0}{1}\big) \circ \alpha_0$, so $P(0)$ holds.
		
		\item \emph{Inductive step.} Assume $P(k)$, we need to prove $P(k+1)$. Suppose $P(k+1)$'s premise, \emph{i.e.} $(\alpha_k(n)\ge k+1) \wedge (b\ge k+1)$. Then $\alpha_k(n) > k$ $\Rightarrow n > A(k, k)$ $\Rightarrow n > A(k-1, k-1)$ $\Rightarrow \alpha_{k-1}(n) > k - 1$ $\Rightarrow \alpha_{k-1}(n)\ge k$ and $b\ge k+1$ $\Rightarrow b\ge k$, so $P(k)$'s premise applies. Therefore we have $P(k)$'s conclusion. It thus suffices to show
		
		\begin{equation*}
		\alpha^{\W}\big(\alpha_{k+1}, \alpha_{k+1}(n), k+1, b-k-1\big)
		= \alpha^{\W}\big(\alpha_k, \alpha_k(n), k, b-k\big)
		\end{equation*}
		
		Definition~\ref{defn: inv-ack-hier} gives $\alpha_{k+1} = \big(\cdt{\alpha_k}{1}\big)\circ \alpha_k$, so this 
		is~(\ref{eq: inv-ack-worker-recursion}) when $b - k\ge 1$ and $\alpha_k(n)\ge k+1$. Thus $P(k+1)$ holds. 
		%\vspace*{-\baselineskip}
	\end{itemize}
\end{proof}

With this lemma, we can proceed to establish the correctness of \emph{inverse Ackermann worker}, \emph{i.e.} Theorem~\ref{thm: inv-ack-worker-correct}. We reproduce the statement here as a convenient point of reference for readers.

\begin{usethmcounterof}{thm: inv-ack-worker-correct}
	$\alpha^{\W}\big(\alpha_0, \alpha_0(n), 0, n\big) = \alpha(n)$.
\end{usethmcounterof}

\begin{proof}
	Since the sequence $\big\{\alpha_k(n)\big\}_{k=1}^{\infty}$ decreases while $\{k\}_{k=1}^{\infty}$ increases to infinity, there exists $m \triangleq \min\big\{k : \alpha_k(n) \le k \} = \alpha(n)$. Note that $m\le n$ since $\alpha_n(n)\le n$.
	If $m = 0$, $m - 1$ is also $0$, so $m = 0 \le \alpha_0(n)\le \alpha_{m-1}(n)$. If $m \ge 1$, $m - 1 < m$ $\implies m - 1 < \alpha_{m-1}(n)$ $\implies m \le \alpha_{m-1}(n)$. In both cases, Lemma~\ref{lem: inv-ack-worker-intermediate} implies:
	$$ \alpha^{\W}\big(\alpha_0, \alpha_0(n), 0, n\big) = \alpha^{\W}\big(\alpha_m, \alpha_m(n), m, n - m\big) = m $$
	where the last equality follows from \eqref{eq: inv-ack-worker-recursion} when $\alpha_m(n)\le m$.
\end{proof}

\section{Support for Binary Numbers in Coq}
\label{apx:bin_in_coq}
Coq has support for binary numbers with the type \li{N}, which consists 
of constructors \li{0} and \li{positive}:

\begin{lstlisting}
Inductive positive : Set := 
  | xI : positive -> positive | xO : positive -> positive
  | xH : positive.
\end{lstlisting}

Constructor \li{xH} represents $1$, and constructors \li{xO} and \li{xI} represent 
appending $0$ and $1$ respectively. 
By always starting with $1$, \li{positive} bypasses
the issue of disambiguating \emph{e.g.} the strings \li{011} and 
\li{00011}, which represent the same number but pose
a minor technical challenge. 
To represent $0$, the type \li{N} simply provides a separate constructor \li{0}. 

\hide{
\begin{lstlisting}
Definition exp (x y : N) : N := match y with
| 0 => 1 | Npos y'
  => match x with
     | 0 => 0 | _ => 
     let fix expPos (p : positive) :=
            match p with
            | xH => x
            | xI p' => let t := expPos p' in x * t * t
            | xO p' => let t := expPos p' in t * t
            end in expPos y'
     end end.
\end{lstlisting}
To analyse the performance of this definition, let us denote the computation time for \li{exp a b} and \li{a * b} respectively by $\Texp (a, b)$ and $ \Tmul (a, b)$.Now

\begin{equation*}
\begin{aligned}
\Texp(a, b)
& \le \Texp\left(a, \left\lfloor \frac{b}{2} \right\rfloor\right)
+ \Tmul\left(a^{\left\lfloor \frac{b}{2} \right\rfloor}, a^{\left\lfloor \frac{b}{2} \right\rfloor} \right) + \Tmul\left(a^{2\left\lfloor \frac{b}{2} \right\rfloor}, a\right) \\
& \le \Texp\left(a, \left\lfloor \frac{b}{2} \right\rfloor\right) + \left\lfloor \frac{b}{2} \right\rfloor^2 \big(\log_2 a + 1\big)^2 + 2\left\lfloor \frac{b}{2} \right\rfloor \big(\log_2 a + 1\big)^2 \\
& \le \Texp\left(a, \left\lfloor \frac{b}{2} \right\rfloor\right) + 3\left(\frac{b}{2}\right)^2 \big(\log_2 a + 1\big)^2
\end{aligned}
\end{equation*}

Thus, $\Texp(a, b) \le 3\big(\log_2 a + 1\big)^2 \sum_{k=1}^{\infty} \frac{b^{2k}}{4^k} = \bigO\left(b^2\big(\log_2 a + 1\big)^2\right)$. 
This a massive improvement over the bound $\Theta\left(a^b\right)$ of exponentiation in the unary system.

Apart from speeding up addition and multiplication, the binary system does the same for their successor, namely exponentiation. We achieve this by \emph{repeated squaring},
which by standard techniques allows us to calculate $a^b$ in $\bigO\left(b^2\big(\log_2 a + 1\big)^2\right)$.
}%end hide

\section{Proofs of Time Bound Lemmas on \li{nat}}
\label{apx:time_analysis}

This section provides fully detailed proofs for the time complexity bounds established in \S\ref{sec: inv-ack}, \emph{i.e.} Lemma~\ref{lem: inv-ack-hier-runtime} for the general running time formula of $\alpha_i$, Lemma~\ref{lem: alpha2_runtime_naive} for the lower bound on the na\"ive version and Theorem~\ref{thm: inv-ack-hier-runtime-improved} for the upper bound on the improved version. These results and their supporting lemmas are reproduced here for readers' easy reference.

We start by examining the running time of two operations used in the Coq definition of \emph{countdown worker} on \li{nat}, namely \li{leb} and \li{compose}.

\begin{lem} \label{lem: leb-runtime}
	The standard Coq library function $\li{leb}(x, y)$ returns \li{true} if $x\le y$ and \li{false} otherwise. $\runtime_{\li{leb}}(x, y) = \min\{x, y\} + 1$ when $n$ and $a$ are of type \li{nat}.
\end{lem}

\begin{proof}
	Per its Coq definition, $\li{leb}(x, y)$ returns \li{true} if $x = 0$. When $x > 0$, $\li{leb}(x, y)$ returns \li{false} if $y = 0$, or $\li{leb}(x-1, y-1)$ if $y > 0$. Therefore,
	\begin{equation*}
	\runtime_{\li{leb}}(x, y) = \begin{cases}
	1 & \text{ when } x = 0 \text{ or } y = 0 \\
	\runtime_{\li{leb}}(x - 1, y - 1) + 1 & \text{ when } x > 0, y > 0
	\end{cases}
	\end{equation*}
	Let $P(x) \triangleq \forall y. \runtime_{\li{leb}}(x, y) = \min\{x, y\} + 1$. $P(0)$ holds since both sides are $1$. Assume $P(x-1)$ holds for $x > 0$, consider any $y$
	\begin{itemize}
		\item If $y = 0$, $\runtime_{\li{leb}}(x, 0) = 1 = \min\{x-1, 0\} + 1$.
		\item If $y > 0$, $\runtime_{\li{leb}}(x, y)$ $= \runtime_{\li{leb}}(x - 1, y - 1) + 1$ 
		$ = \min\{x-1, y-1\} + 2$ (by $P(x-1)$) $= \min\{(x-1) + 1, (y-1) + 1\} + 1 = \min\{x, y\} + 1$.
	\end{itemize}
  In both cases, $P(x)$ holds. The proof is complete by induction on $x$.
\end{proof}

\begin{lem} \label{lem: compose-runtime}
	% In all encoding systems, 
	$\runtime_{f\circ g}(n) = \runtime_f(g(n)) + \runtime_g(n)$.
	% per Coq's definition of functional composition.
\end{lem}
\begin{proof}
	Per the Coq definition \li{compose}, computing $(f\circ g)(n)$, \emph{i.e.} $\li{(compose f g)}(n)$, requires computing firstly $m\triangleq g(n)$, then $f(m)$, thus the formula.
\end{proof}

\begin{lem} \label{lem: cdt-runtime-general}
	$\forall a\in \mathbb{N}, f\in \contract_{a}$, where $\li{succ}(x) \triangleq x+1$,
	\begin{equation} \label{eq: cdt-runtime-struct}
	\runtime_{\cdt{f}{a}}(n) =
	\sum_{i=0}^{\cdt{f}{a}(n) - 1} \runtime_f\left(f^{(i)}(n)\right)
	+ \sum_{i=0}^{\cdt{f}{a}(n)}\Tleb\left(f^{(i)}(n), a\right)
	+ \sum_{i=0}^{\cdt{f}{a}(n) - 1}\Tsucc(i)~~~
	\end{equation}
\end{lem}

\begin{proof}
	Since $f\in \contract_{a}$, $\cdt{f}{a}\ $ is the minimum $k$ such that $f^{(k)}(n) \le a$. The execution of $\cdw{f}{a}\big(n, n\big)$ then takes $k+1$ recursive calls, where the $i^{th}$ call for $0\le i \le k$ takes the arguments $i, a$ and $n_i \triangleq f^{(i)}(n)$ from the previous call (or the initial argument when $i = 0$), and performs the following computations:
	\begin{enumerate}
		\item Compute $\li{leb}\left(n_i, a\right)$ for $\Tleb\left(f^{(i)}(n), a\right)$ steps.
		\item If $\li{leb}\left(n_i, a\right) = \li{true}$, return $0$. Else proceed to the next step.
		\item Compute $n_{i+1} \triangleq f(n_i) = f^{(i+1)}(n)$ for $\runtime_f\left(f^{(i)}(n)\right)$ steps.
		\item Pass $n_{i+1}, i+1, a$ to the next recursive call and wait for it to return $k - i - 1$.
		\item Add $1$ to the result for $\Tsucc(k-i-1)$ steps and return $k - i$.
	\end{enumerate}
    Summing up the time of each call gives the desired expression for $\runtime_{\cdt{f}{a}}(n)$.
\end{proof}

\noindent Next, we restate and prove Lemma~\ref{lem: cdt-runtime}, which gives a formula for \emph{countdown}'s running time.

\begin{uselemcounterof}{lem: cdt-runtime}
	$\forall a\ge 1, \forall n \in \li{nat}, \forall f\in \contract_{a}$ we have:
	\begin{equation} \label{eq: cdt-runtime}
	\runtime_{\cdt{f}{a}}(n) =
	\sum_{i=0}^{\cdt{f}{a}(n) - 1} \runtime_f\left(f^{(i)}(n)\right)
	+ (a + 2)\cdt{f}{a}(n) + f^{\left(\cdt{f}{a}(n)\right)}(n) + 1
	\end{equation}
\end{uselemcounterof}

\begin{proof}
	Per Definition~\ref{defn: countdown} of countdown,
$\Tleb\big(f^{(i)}(n), a\big) = a + 1$ if $i < \cdt{f}{a}(n)$ and $f^{(i)}(n) + 1$ otherwise (\S\ref{apx:time_analysis}, Lemma~\ref{lem: leb-runtime}).  Thus,
the second summand in \eqref{eq: cdt-runtime-struct} is equal to $(a + 1)\cdt{f}{a}(n) + f^{\left(\cdt{f}{a}(n)\right)}(n) + 1$. Since $\Tsucc(i) = 1$ on \li{nat}, the third summand is equal to $\cdt{f}{a}(n)$, completing the desired formula.
\end{proof}
The above lemma is the last step for us to build a recursive formula for the running time of the inverse Ackermann hierarchy, with each level's runtime involving that of its previous one. We restate and prove Lemma~\ref{lem: inv-ack-hier-runtime} from \S\ref{sect: hardcode-lvl2}.

\begin{uselemcounterof}{lem: inv-ack-hier-runtime}
	When $\alpha_i$ is defined per Definitions~\ref{defn: inv-ack-hier} 
	and~\ref{defn: countdown},
	\begin{equation} \label{eq: inv-ack-hier-runtime}
	\runtime_{\alpha_{i+1}}(n) = \sum_{k=0}^{\alpha_{i+1}(n)}\runtime_{\alpha_i}\left( \alpha_i^{(k)}(n)\right) + 3\alpha_{i+1}(n) + C_i(n)
	\end{equation}
	\hspace{7em}where $\forall i, n.~C_i(n) \triangleq \alpha_i^{(\alpha_{i+1}(n) + 1)}(n) + 1 \in \{1, 2\}$
\end{uselemcounterof}

\begin{proof}
	 From Definition~\ref{defn: inv-ack-hier}, $\alpha_{i+1} = \big(\cdt{\alpha_i}{1}\big)\circ \alpha_i$. Since $\alpha_i\in \contract_{1}$, applying Lemma~\ref{lem: cdt-runtime} gives us:

	 \begin{equation*}
	 \begin{aligned}
	 \runtime_{\cdt{\alpha_i}{1}}\big(\alpha_i(n)\big)
	 & = \hspace{-5pt} \sum_{k=0}^{\cdt{\alpha_i}{1}(\alpha_i(n)) - 1} \hspace{-10pt} \runtime_{\alpha_i}\left(\alpha_i^{(k)}\big(\alpha_i(n)\big)\right)
	 + 3\cdt{\alpha_i}{1}\big(\alpha_i(n)\big)
	 + \alpha_i^{\big(\cdt{\alpha_i}{1}(\alpha_i(n))\big)}\big(\alpha_i(n)\big) + 1 \\
	 & = \sum_{k=0}^{\alpha_{i+1}(n) - 1} \runtime_{\alpha_i}\left(\alpha_i^{(k+1)}(n)\right)
	 + 3\alpha_{i+1}(n)
	 + \alpha_i^{\big(\alpha_{i+1}(n)\big)}\big(\alpha_i(n)\big) + 1 \\
	 & = \ \sum_{k=1}^{\alpha_{i+1}(n)} \runtime_{\alpha_i}\left(\alpha_i^{(k)}(n)\right)
	 + 3\alpha_{i+1}(n)
	 + \underbrace{\alpha_i^{\alpha_{i+1}(n) + 1}(n) + 1}_{C_i(n)}
	 \end{aligned}
	 \end{equation*}By Lemma~\ref{lem: compose-runtime}, $\runtime_{\alpha_{i+1}}(n) = \runtime_{\cdt{\alpha_i}{1}}\big(\alpha_i(n)\big) + \runtime_{\alpha_i}(n)$, which completes \eqref{eq: inv-ack-hier-runtime}. By Definition~\ref{defn: informal-countdown} of countdown, $\alpha_{i+1}(n) = \cdt{\alpha_i}{1}\big(\alpha_i(n)\big)$
	 $= \min\big\{l: \alpha_i^{(l+1)}(n)\le 1\big\}$. Thus $\alpha_i^{\alpha_{i+1}(n) + 1}(n) \in \{0, 1\}$, or $C_i(n)\in \{1, 2\}$. The proof is complete.
\end{proof}

\noindent We are finally ready to prove the time complexity bounds of $\alpha_i$. We start with proving the quadratic lower bound on the running time of the na\"ive version without any hardcoding, i.e. Lemma~\ref{lem: alpha2_runtime_naive}.

\begin{uselemcounterof}{lem: alpha2_runtime_naive}
	When $\alpha_i$ is defined per Definitions~\ref{defn: inv-ack-hier}, $\forall i\ge 2. \runtime_{\alpha_i}(n) = \Omega\big(n^2\big)$.
\end{uselemcounterof}

\begin{proof}
	$\alpha_0 = \lambda m.(m-1)$ so $\alpha_1 = \cdt{\big(\alpha_0\big)}{1}\circ \alpha_0 = \lambda m.(m - 2)$. By Lemma~\ref{lem: inv-ack-hier-runtime},
	\begin{equation*}
	\runtime_{\alpha_1}(n) \ge \sum_{i=0}^{n-1} \runtime\big(\lambda m.(m-1), n - i\big) + 3(n - 2) + 1 = 4n - 5\text{,}
	\end{equation*}
	since $\runtime_{\alpha_0}(k) = 1$. Because $\alpha_2 = \cdt{\big(\alpha_1\big)}{1}\circ \alpha_1 = \lambda m.\left\lceil \frac{m-3}{2} \right\rceil$, again by Lemma~\ref{lem: inv-ack-hier-runtime},
	\begin{equation*}
	\runtime_{\alpha_2}(n)
	\ge \sum_{i=0}^{\left\lceil \frac{n-3}{2} \right\rceil} \big(4(n-2i) - 5\big) + 3\left\lceil \frac{n-3}{2} \right\rceil + 1
	= \Omega\big(n^2\big)
	\end{equation*}
	By Lemma~\ref{lem: inv-ack-hier-runtime}, $\forall i. \runtime_{\alpha_{i+1}}(n)\ge \runtime_{\alpha_i}(n)$, hence $\forall i\ge 2. \runtime_{\alpha_i}(n) = \Omega\big(n^2\big)$.
\end{proof}

% TODO: find a good place to insert this
%\begin{col}
%	$\runtime(\alpha, n) = \Omega\big(n^2\big)$ per Definition~\ref{defn: inv-ack-worker}.
%\end{col}

With the hardcoding $\alpha_1 \triangleq \lambda n.(n - 2)$, we can improve the execution time of the whole hierarchy as noted in Remark~\ref{rem: inv-ack-hardcode}. We now show that this small improvement makes the inverse Ackermann hierarchy run in linear time at any level, \emph{i.e.} Theorem~\ref{thm: inv-ack-hier-runtime-improved}, via a few supporting lemmas.

\begin{lem} \label{lem: inv-ack-3-runtime}
	If $n\ge 1$, $\runtime_{\alpha_3}(n) = 2$ $\runtime_{\alpha_3}(n) \le 4n + 4$.
\end{lem}

\begin{proof}
	It is easy to show that $\alpha_2^{(k)}(n) = \left\lfloor \frac{n+2}{2^k} \right\rfloor - 2$. Thus
	
	\begin{equation*}
	\begin{aligned}
		\runtime_{\alpha_3}(n)
		&  \le \ \sum_{k=0}^{\alpha_{3}(n)}\runtime_{\alpha_2}\left(\left\lfloor \frac{n+2}{2^k} \right\rfloor - 2\right) + 3\alpha_{3}(n) + 2 \\
		& \le \ 2\sum_{k=0}^{\alpha_3(n)}\left(\frac{n+2}{2^k} - 3\right) + 3\alpha_3(n) + 2 \\
		& \le \ 4(n + 2) - 6(\alpha_3(n) + 1) + 3\alpha_3(n) + 2 \le 4n + 4.
	\end{aligned}
\end{equation*}
\end{proof}

\noindent The following two technical lemmas that bound a sum of repeated applications of $\log_2$ and $\alpha_i$ are also needed in time bound proofs for the binary version of the inverse Ackermann hierarchy in Appendix~\ref{apx:time_analysis_bin}, hence has been made as general as possible for maximal reusability.

\begin{lem} \label{lem: sum-log-repeat}
	$\forall n$, $\displaystyle \sum_{k = 0}^{\log_2^*(n) - 1}\log_2^{(k)}n \le 2n$, where $\log_a^*(n) \triangleq \min\{k : \log_a^{(k)}n \le 1\}$. Note that the sum is $0$ when $n\le 1$.
\end{lem}

\begin{proof}
	Define $S: \mathbb{R}_{\ge 0}\to \mathbb{R}_{\ge 0}$ where $S(x) \triangleq \sum_{k = 0}^{\log_2^*(x) - 1}\log_2^{(k)}x$. Clearly $S$ is strictly increasing and $\forall x>1, S(x) = n + S(\log_2 x)$. Using the fact $\log_2^*(x)\in \mathbb{N}$, we prove by induction on $k$ the statement $P(k) \triangleq$ $\forall x: \log_2^*(x) = k$, $S(x)\le 2x$.
	\begin{itemize}[leftmargin=*]
		\item \emph{Base case.} $P(0)$, $P(1)$ hold trivially and $P(2) = \forall x: 2 < x \le 4$, $x + \log_2x \le 2x$, which is equivalent to $\log_2x \le x$, which holds for all $x > 2$.
		\item \emph{Inductive case.} Assume $P(k-1)$ where $k\ge 3$. Fix any $x$ such that $\log_2^*(x)=~k$, then $x > 4$. The function $\lambda x.\frac{x}{\log_2x}$ is increasing on $[4, +\infty)$, so $\frac{4}{\log_24} \le \frac{x}{\log_2x}$ or $2\log_2x \le x$. Since $\log_2^*(x) = k$, we have $\log_2^*(\log_2x) = k-1$. By $P(k-1)$, $S(x) = x + S(\log_2x) \le x + 2\log_2x \le 2x$, which completes the proof.
	\end{itemize}
\end{proof}

\begin{lem} \label{lem: sum-alpha-repeat}
	$\forall l, \forall i \ge 3, \forall s\le \alpha_{i+1}(n)$.~$\displaystyle \sum_{k=s}^{\alpha_{i+1}(n)} \log_2^{(l)}\alpha_i^{(k)}(n) \le 2\log_2^{(l+s)}n$.
\end{lem}

\begin{proof}
  $\forall i\ge 3, \forall n, \alpha_i(n)\le \log_2n$ and $\alpha_{i+1}(n)\le \log_2^*(n)-1$, therefore
	\begin{equation*}
%	S_i(n) \ = \
	\sum_{k=s}^{\alpha_{i+1}(n)} \log_2^{(l)}\alpha_i^{(k)}(n) \ \le \
%	\sum_{k=1}^{\log_2^*(n)-1} \log_2^{(l)}\alpha_i^{(k)}(n) \ \le \
	\sum_{k=s}^{\log_2^*(n)-1} \log_2^{(l+k)}(n) \ \le \
	2\log_2^{(l+s)} n
	\end{equation*}
\end{proof}

With these lemma, we are ready to tackle Theorem~\ref{thm: inv-ack-hier-runtime-improved}.

\begin{usethmcounterof}{thm: inv-ack-hier-runtime-improved}
	When $\alpha_i$ is defined per Definition~\ref{defn: inv-ack-hier} with the added hardcoding of $\alpha_1$ to $\lambda n. (n - 2)$, $\forall i.~\runtime_{\alpha_i}(n) \le 4n + \left(19\cdot 2^{i-3} - 2i - 13\right)\log_2n + 2i = \bigO(n)$.
\end{usethmcounterof}

\begin{proof}
	We have proved the result for $i = 0, 1, 2$. Let us proceed with induction on $i\ge 3$. The case $i = 3$ has been covered by Lemma~\ref{lem: inv-ack-3-runtime}. Let $M_i \triangleq 2^{i-3}19 - 2i - 13$ for each $i$ and suppose the result holds for $i\ge 3$. We have
	\begin{equation*}
	\begin{aligned}
	%\begin{array}{@{}l@{}}
	& \runtime_{\alpha_{i+1}}(n) \le \sum_{k=0}^{\alpha_{i+1}(n)} \runtime_{\alpha_i}\big(\alpha_i^{(k)}(n)\big) + 3\alpha_{i+1}(n) + 2 \\
	& \le \sum_{k=0}^{\alpha_{i+1}(n)}\big(4\alpha_i^{(k)}(n) + M_i\log_2\alpha_i^{(k)}(n) + 2i \big) + 3\alpha_{i+1}(n) + 2 \\[-3pt]
  & = 4n + 2(i+1) + (2i + 3)\underbrace{\alpha_{i+1}(n)}_{\le \log_2n} + 4\underbrace{\sum_{k=1}^{\alpha_{i+1}(n)}\alpha_i^{(k)}(n)}_{\le 2\log_2n} + M_i\underbrace{\sum_{k=0}^{\alpha_{i+1}(n)}\log_2\alpha_i^{(k)}(n)}_{\le 2\log_2n} \\
	& \le 4n + 2(i+1) + (2M_i + 2i + 3 + 8)\log_2n
	= 4n + M_{i+1}\log_2n + 2(i+1)
	%\end{array}
  \end{aligned}
	\end{equation*}
\end{proof}

\section{Proofs of Time Bound Lemmas on \li{N}}
\label{apx:time_analysis_bin}

\renewcommand{\Tleb}{\runtime_{\li{N.leb}}}
\renewcommand{\Tsucc}{\runtime_{\li{N.succ}}}

This section provides fully detailed proofs for the time complexity bounds established in \S\ref{sec: binary},
\emph{i.e.} Lemma~\ref{lem: cdt-runtime-bin} for the upper bound on the time complexity of $\alpha_i$ on \li{N} (\li{bin_alpha}),
and Theorem~\ref{thm: inv-ack-runtime-bin} for the upper bound on the time complexity of the $\alpha$ on \li{N} (\li{bin_inv_ack}).
These results and their supporting lemmas are reproduced here for readers' easy reference.

We start by inspecting the execution time of crucial computations in each recursive call of \li{bin_countdown}, namely \li{N.leb} and \li{N.succ}. While \li{leb} is easy, \li{succ} requires more care as it is worst-case logarithmic time but amortized constant time.

\begin{lem} \label{lem: leb-runtime-bin}
	In binary encoding, $\Tleb(x, y) \le \lfloor \log_{2}\min\{x, y\} \rfloor + 1$.
\end{lem}
\begin{proof}
	\li{N.leb}'s Coq definition indirectly involves the function \li{Pos.compare_cont}, which maps from \li{Pos*Pos} to the set $\{Gt, Lt, Eq\}$ (respectively stand for ``Greater Than'', ``Less Than'' and ``Equal''). We henceforth use \li{Pc} as short form for \li{Pos.compare_cont}. Precisely we have:
	
	\begin{equation*}
	\li{N.leb}(x, y) = \begin{cases}
	\li{true} & \text{ when } x = 0 \\ \li{false} & \text{ when } x > 0, y = 0 \\
	\neg \big(\li{Pc}(Eq, x, y) == Gt \big) & \text{ when } x > 0, y > 0
	\end{cases}
	\end{equation*}

\hide{
	\li{N.leb}'s Coq definition involves a function \li{N.compare}, which takes input in \li{N} to values in the set $\{Gt, Lt, Eq\}$ (respectively stand for ``Greater Than'', ``Less Than'' and ``Equal'').
	$\li{N.leb}(x, y)$ returns \li{false} if $\li{N.compare}(x, y) =  Gt$ and \li{true} otherwise, thus $\runtime_{\li{N.leb}}(x, y) = \runtime_{\li{N.compare}}(x, y) + 1$.
	\li{N.compare} indirectly uses another function, which is \li{Pos.compare_cont}.
	
	\begin{equation*}
	\li{N.compare}(x, y) = \begin{cases}
	Eq & \text{ when } x = 0, y = 0 \\ Lt & \text{ when } x = 0, y > 0 \\
	Gt & \text{ when } x > 0, y = 0 \\ \li{Pos.compare_cont}(Eq, x, y) & \text{ when } x > 0, y > 0
	\end{cases}
	\end{equation*}
	where in the last case, $x$ and $y$ are treated as type \li{Pos}.
	Using \li{Nc} and \li{Pc} respectively as abbreviations for \li{N.compare} and \li{Pos.compare_cont}, we have $\runtime_{\li{Nc}}(x, y) = 1$ when $\min\{x, y\} = 0$ and $\runtime_{\li{Pc}}(x, y) + 1$ otherwise.
}%end hide	
	
		where in the last case, $x$ and $y$ are treated as type \li{Pos}. The value of $\li{Pc}(x, y)$ is given in the following table, where $x{\sim}0$ and $x{\sim}1$ respectively representing appending $0$ and $1$ to the end of $x$'s representation:
	
	\begin{equation*}
	\begin{array}{c|c|c|c}
	\hspace{10pt} x \backslash y \hspace{10pt} & \hspace{10pt} 1 \hspace{10pt} & y'{\sim}0 & y'{\sim}1 \\ \hline
	1 & R & Lt & Lt \\ \hline
	x'{\sim}0 & Gt & \ \li{Pc}(R, x', y') \ & \ \li{Pc}(Lt, x', y') \ \\ \hline
	x'{\sim}1 & Gt & \ \li{Pc}(Gt, x', y') \ & \ \li{Pc}(R, x', y') \
	\end{array}
	\end{equation*}
	
	\noindent It suffices to show $\runtime_{\li{Pc}}(x, y) \le \lfloor\log_2\min\{x, y\}\rfloor + 1$. The table above gives:
	
	\begin{equation*}
	\runtime_{\li{Pc}}(x, y) = \begin{cases}
	1 & \text{ if } x = 1 \vee y = 1 \\
	\runtime_{\li{Pc}}\left(\left\lfloor \frac{x}{2} \right\rfloor, \left\lfloor \frac{x}{2} \right\rfloor\right) + 1 & \text{ if } x > 1 \wedge y > 1
	\end{cases}
	\end{equation*}
	
	\noindent A simple binary induction on $\min\{x, y\}$ shows $\runtime_{\li{Pc}}(x, y) = \lfloor\log_2\min\{x, y\}\rfloor + 1$. The proof is complete.
\end{proof}

\begin{lem} \label{lem: succ-runtime-bin}
	$\forall n\ge 1.~S(n) = \displaystyle \sum_{i=0}^{n-1}\Tsucc(i) \le 2n + \log_2n$.
\end{lem}
\begin{proof}
	Per \li{N.succ}'s Coq definition, $\li{N.succ}(u) = 1$ if $u = 0$ and $\li{Pos.succ}(u)$ if $u > 0$, where when treating $u$ as \li{Pos} we have $\li{Pos.succ}(u)$ returns $2$ when $u = 1$, $u'{\sim}1$ when $u = u'{\sim}0$ and $(\li{Pos.succ}(u')){\sim}0$ when $u = u'{\sim}1$.
%	\begin{equation*}
%	\li{Pos.succ}(u) = \begin{cases}
%	2 & \text{ when } u = 1 \\
%	u'{\sim}1 & \text{ when } u = u'{\sim}0 \\ (\li{Pos.succ}(u')){\sim}0 & \text{ when } u = u'{\sim}1
%	\end{cases}
%	\end{equation*}
  A binary induction on $u$ confirms that for all $u$,
	
	\begin{equation*}
	\Tsucc(u) = \begin{cases}
	\Tsucc\left(\frac{u-1}{2}\right) + 1 & \text{ when } u \text{ is odd,} \\
	1 & \text{ when } u \text{ is even.}
	\end{cases}
	\end{equation*}
	
	\noindent We prove the lemma by induction on $n$.
	If $n = 1$, $S(n) = \Tsucc(0) = 1$, our goal is $1\le 2$, which is trivial. For $n > 1$, observe that . Thus for all $n$, $S(2n+1) = S(2n) + 1$ and
	
	\begin{equation*}
	\begin{aligned}
	S(2n) & = \ \sum_{i=0}^{n}\Tsucc(2i) + \sum_{i=0}^{n-1}\Tsucc(2i+1)
	= \ \sum_{i=0}^{n}1 + \sum_{i=0}^{n-1}\big(\Tsucc(i) + 1\big) \\
	& = n+1 + S(n) + n = S(n) + 2n + 1
	\end{aligned}
	\end{equation*}
	
	\noindent From the above, it is easy to complete the proof via induction on $n$.
\end{proof}
Now we are ready to prove Lemma~\ref{lem: cdt-runtime-bin}, reproduced here for readers' reference.

\begin{uselemcounterof}{lem: cdt-runtime-bin}
	$\forall n \in \li{N}$, if $f$ is a binary strict contraction above $a$,
	\begin{equation*}
	\runtime_{\cdt{f}{a}}(n) \le \sum_{i=0}^{\cdt{f}{a}(n) - 1} \hspace{-6pt}
	\runtime_f\big(f^{(i)}(n)\big) \ + \ (\log_2a + 3)\left(\cdt{f}{a}(n) + 1\right) \ + \ 2\log_2n \ + \ \log_2\cdt{f}{a}(n)
	\end{equation*}
\end{uselemcounterof}

\begin{proof}
	On strict binary contractions, \li{bin_countdown} does exactly the same computations as the \li{countdown} on \li{nat}, with the added step of computing the budget of \li{nat_size (n-a)}, thus its execution time structure is almost the same. The only difference is the running time of each component in the sum.
	\begin{equation} \label{eq: cdt-bin-runtime-struct}
	\runtime_{\cdt{f}{a}}(n) =
	\hspace{5pt}
	\begin{aligned} 
	& \sum_{i=0}^{\cdt{f}{a}(n) - 1} \runtime_f\left(f^{(i)}(n)\right)
	&\hspace{-4pt}+& ~ \sum_{i=0}^{\cdt{f}{a}(n)}\Tleb\left(f^{(i)}(n), a\right) \\
	& +~ \sum_{i=0}^{\cdt{f}{a}(n) - 1}\Tsucc(i)
	&\hspace{-4pt}+& ~ \runtime_{\li{sub}}(n, a) \ + \ \runtime_{\li{nat_size}}(n - a)
	\end{aligned},
	\end{equation}
	By Lemma~\ref{lem: leb-runtime-bin}, $\forall i, \Tleb\left(f^{(i)}(n), a\right) \le \log_2a + 1$, thus the second summand in \eqref{eq: cdt-bin-runtime-struct} is no more than $(\log_2a + 1)\big(\cdt{f}{a}(n) + 1\big)$. By Lemma~\ref{lem: succ-runtime-bin}, the third summand is no more than $2\cdt{f}{a}(n) + \log_2\cdt{f}{a}(n)$. Lastly, the last two summands are each no more than $\log_2n + 1$. Thus
	 \begin{equation*}
	 \begin{aligned}
	 & \runtime_{\cdt{f}{a}(n)}(n) \ \le \
	 \left(\begin{aligned}
	 & \sum_{i=0}^{\cdt{f}{a}(n) - 1} 
	 \runtime_f\big(f^{(i)}(n)\big)
	 \ + \ (\log_2a + 1)\big(f^{(i)}(n) + 1\big) \\
	 & \ + \ 2\cdt{f}{a}(n) \ + \ \log_2\cdt{f}{a}(n)
	 \ + \ 2\big(\!\log_2n + 1\big)
	 \end{aligned}\right)\\
	 & \le \ \sum_{i=0}^{\cdt{f}{a}(n) - 1}
	 \runtime_f\big(f^{(i)}(n)\big) \ + \ (\log_2a + 3)\left(\cdt{f}{a}(n) + 1\right) \ + \ 2\log_2n \ + \ \log_2\cdt{f}{a}(n)
	 \end{aligned}
	 \end{equation*}
\end{proof}

With all critical lemmas established and proven, including Lemma~\ref{lem: sum-alpha-repeat} in Appendix~\ref{apx:time_analysis}, we proceed to prove the time complexity bound of each level in \li{bin_alpha} in the form of Theorem~\ref{thm: inv-ack-runtime-bin}.

\begin{usethmcounterof}{thm: inv-ack-runtime-bin}
	When $\alpha_i$ on \li{N} is defined via \li{bin_alpha}, for all $i$ and $n$ we have $\runtime_{\alpha_i}(n) \le 2\log_2n + \left(3\cdot 2^i - 3i - 13\right)\log_2\log_2n + 3i$.
\end{usethmcounterof} 

\begin{proof}
	Firstly, we have:
	\begin{itemize}
		\item Clearly $\runtime_{\alpha_i}(n)\le \log_2n + 1$ for $i = 0, 1$.
		\item Computing $\alpha_2(n)$ consists of 2 steps: subtracting by 2 and dividing by 2 (shifting right by 1 bit), both of which takes no more than time $\log_2n + 1$ time, thus $\runtime_{\alpha_2}(n)\le 2\log_2n$.
		\item Computing $\alpha_3(n)$ consists of 3 steps: adding 2, then take base 2 logarithm and subtract 2 from the result, with the first two taking no longer than time $\log_2n + 1$ each and the last taking time at most $\log_2\log_2(n+2) + 1$ $\le \log_2\log_2 n + 2$. Thus $\runtime_{\alpha_3}(n)\le 2\log_2n + \log_2\log_2n + 3$.
	\end{itemize}
  Let $M_i = 3\cdot 2^i - 3i - 13$, then $\runtime_{\alpha_i}(n)\le 2\log_2n + M_i\log_2^{(2)}n + 3i$ for $i = 0, 1, 2, 3$.
	Suppose the same inequality holds for $i\ge 3$, by Lemmas~\ref{lem: compose-runtime} and \ref{lem: cdt-runtime-bin}, note that Lemma~\ref{lem: compose-runtime} works for all encoding systems, we have
	\begin{equation*}
	\begin{aligned}
	& \runtime_{\alpha_{i+1}}(n) \le
	\sum_{k=0}^{\alpha_{i+1}(n)} \hspace{-5pt} \runtime_{\alpha_i}\big(\alpha_i^{(k)}(n)\big)
	\ + \ 3\hspace{-4pt}\underbrace{\alpha_{i+1}(n)}_{\le \log_2\log_2n}
	+ \ 2\underbrace{\log_2\alpha_i(n)}_{\le \log_2\log_2n}
	+ \ \underbrace{\log_2\alpha_{i+1}(n)}_{\le \log_2\log_2n} + \ 3 \\
	& \le 2\log_2n
	+ 2\underbrace{\hspace{-5pt}\sum_{k=1}^{\alpha_{i+1}(n)}\hspace{-5pt} \log_2\alpha_i^{(k)}(n)}_{\le 2\log_2\log_2n}
	+ M_i\underbrace{\hspace{-5pt}\sum_{k=0}^{\alpha_{i+1}(n)}\hspace{-5pt} \log_2^{(2)}\alpha_i^{(k)}(n)}_{\le 2\log_2\log_2n}
	+ 3i\big(\hspace{-5.5pt}\underbrace{\alpha_{i+1}(n)}_{\le \log_2\log_2n}\hspace{-4pt} + 1\big) + 6\log_2^{(2)}n + 3 \\
	& \le 2\log_2n + (2M_i + 3i + 10)\log_2^{(2)}n + 3(i + 1)
	= 2\log_2 + M_{i+1}\log_2^{(2)}n + 3(i + 1)
	\end{aligned}
	\end{equation*}
	where $\forall i\ge 3, \alpha_i(n)\le \log_2n$ and $\alpha_{i+1}(n)\le \log_2^{(2)}n$, while the $2\log_2^{(2)}n$ upper bounds for the sums come from Lemma~\ref{lem: sum-alpha-repeat}.
	By induction on $i$, proof is complete.
\end{proof}
