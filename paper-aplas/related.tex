\label{sec:related}

%\newcommand{\ackt}{\ensuremath{\hat{\alpha}}}

\subsection{The value of a linear-time solution to the hierarchy}

Our functions' linear runtimes can be understood in two distinct but
complementary ways.  A runtime less than the bitlength is impossible
without prior knowledge of the size of the input.  Accordingly, in
an information-theory or pure-mathematical sense, our definitions are
optimal up to constant factors.  And of course in practice, linear-time
solutions are highly usable in real computations.

Sublinear solutions are possible with \emph{a priori} knowledge about
the function and bounds on the inputs one will receive.
An extreme case is $\alpha(n)$, which has value $4$ for all practical
inputs greater than $61$. Accordingly,
this function can be inverted in $O(1)$ in practice.  That said, 
such solutions require external knowledge of the problems and
lookup tables within the algorithm to store precomputed
values, and thus fall more into the realm of engineering than mathematics. 

\subsection{The two-parameter inverse Ackermann function}
Some authors~\cite{chazelle,tarjan} prefer a two-parameter inverse Ackermann function.
\begin{defn} \label{defn: 2para-alpha}
	The two-parameter inverse Ackermann function is defined as:
	\begin{equation} \label{eq: tmp-2para-alpha}
	\ackt (m, n) \triangleq \min\left\{i \ge 1 : \Ack\left(i, \left\lfloor \frac{m}{n} \right\rfloor \right)\ge \log_2n \right\}
	\end{equation}
\end{defn}
Note that $\ackt(n, n)$ and the single-parameter $\alpha(n)$
are neither equal nor directly related, but
it is straightforward to modify our techniques to compute $\ackt(m, n)$.
\hide{This function arises from deep runtime analysis of the disjoint-set data structure. Tarjan \cite{tarjan} showed that, in the disjoint-set data structure, the time required $t(m,n)$ for a sequence of $m$ \textsc{\color{magenta}FIND}s intermixed with $n-1$ \textsc{\color{magenta}UNION}s (such that $m \geq n$) is bounded as: $k_{1}m\cdot\alpha(m,n) \leq t(m,n) \leq k_{2}m\cdot\alpha(m,n)$. In graph theory, Chazelle \cite{chazelle} showed that the minimum spanning tree of a connected graph with $n$ vertices and $m$ edges can be found in time $O(m\cdot\alpha(m,n))$. Computing this function is in fact easier than $\alpha(n)$, as when $m$ and $n$ are given, we are reduced to finding the minimum $i\ge 1$ such that $\Ack_i(s)\ge t$ for $s, t$ fixed, which can be done with the following variant of the \emph{inverse Ackermann worker}.
}
\begin{defn} \label{defn: 2para-inv-ack-worker}
	The {two-parameter inverse Ackermann worker}
	,written $\ackt^{\W}$, is a function $\mathbb{N}^4\to \mathbb{N}$, defined by:
	\hide{$(\mathbb{N}\to \mathbb{N}) \times \mathbb{N}^3\to \mathbb{N}$ such that for all $n, k, b\in \mathbb{N}$ and $f:\mathbb{N}\to \mathbb{N}$:}
%	\begin{equation} \label{eq: 2para-inv-ack-worker-recursion}
%	\ackt^{\W}(f, n, k, b) = \begin{cases}
%	0 & \text{if } b = 0 \vee n\le k \\ 1 + \ackt^{\W}\big(\cdt{f}{1}\circ f , \cdt{f}{1}(n), k, b-1\big) & \text{if } b \ge 1 \wedge n \ge k+1
%	\end{cases}
%	\end{equation}
  \begin{equation} \label{eq: 2para-inv-ack-worker-recursion}
  \begin{aligned}
  & \ackt^{\W}(f, n, k, b) \\
  & \triangleq \begin{cases}
  0 & \text{if } b = 0 \vee n\le k \\ 1 + \ackt^{\W}\big(\cdt{f}{1}\circ f , \cdt{f}{1}(n), k, b-1\big) & \text{if } b \ge 1 \wedge n \ge k+1
  \end{cases}
  \end{aligned}
  \end{equation}
\end{defn}
%Similar to the one-parameter version, the following theorem establishes the correct setting for $\W\alpha_2$ to compute $\alpha(m, n)$.
%\begin{thm}
%	$\displaystyle \ackt(m, n) = 1 + \ackt^{\W}\left(\alpha_1, \alpha_1\big(\lceil\log_2n \rceil\big), \left\lfloor \frac{m}{n} \right\rfloor, \lceil\log_2n \rceil \right)$.
%\end{thm}
% Edited by Linh
\begin{thm} For all $m$ and $n$,
	\begin{equation*}
	\displaystyle \ackt(m, n) = 1 + \ackt^{\W}\left(\alpha_1, \alpha_1\big(\lceil\log_2n \rceil\big), \left\lfloor \frac{m}{n} \right\rfloor, \lceil\log_2n \rceil \right).
	\end{equation*}
\end{thm}
We mechanize the above for both \href{https://github.com/inv-ack/inv-ack/blob/7270e64a2600b771f2b1b1b151f7d13fb2ae6c97/inv_ack.v#L245-L248}{\color{blue}unary} and \href{https://github.com/inv-ack/inv-ack/blob/7270e64a2600b771f2b1b1b151f7d13fb2ae6c97/bin_inv_ack.v#L222-L228}{\color{blue}binary} inputs in our codebase.

\hide{
	\begin{proof}[Proof Sketch]
		It is easy to prove in a similar fashion to \cref{lem: inv-ack-worker-intermediate} that for all $n, b, k$ and $i$, if $\alpha_i(n) > k$ and $b > i$, then
		\begin{equation*}
		\W\alpha_2\big(\alpha_1, \alpha_1(b), k, b\big) = i + \W\alpha_2\big(\alpha_{i+1}, \alpha_{i+1}(n), k, b - i\big)
		\end{equation*}
		Now let $k \triangleq \lfloor m/n \rfloor$, $b \triangleq \lceil \log_2n \rceil$ and $l \triangleq \min\big\{i : \alpha_i(b)\le k\big\}$, which exists because $\Ack(i, \cdot)$ increases strictly with $i$. Then $\alpha(m, n) = \max{1, l}$. If $l = 0$, $\alpha_1(b) \le \alpha_0(b) \le k$, so $\W\alpha_2\big(\alpha_1, \alpha_1(b), k, b\big) = 0$, as desired. If $l \ge 1$,
		\begin{equation*}
		1 + \W\alpha_2\big(\alpha_1, \alpha_1(b), k, b\big)
		= 1 + l - 1 + \W\alpha_2\big(\alpha_l, \alpha_l(b), k, b-l\big) = l
		\end{equation*}
		Here $b\ge l$ due to the fact that $\Ack(b, k)\ge b$, so $\alpha_b(b)\le k$. This completes the proof.
\end{proof}}%end hide




\subsection{Historical notes}
% This creates an unnumbered paragraph. ie a smaller, less flashy, header

%\marginpar{\tiny \color{blue} Multiplication, Division, Algorisms. Representations of numbers (Egyption fractions/Roman numerals/Decimal/Zero). Exponentiation, Logarithm, Tetration, Log*, ...   Hyperoperations, Knuth Arrows.  Inverses as a separate notation? Mechanizations of the above?}
The operations successor, predecessor, addition, and subtraction have
been integral to counting forever. The ancient Egyptian
number system used glyphs denoting $1$, $10$, $100$, \emph{etc.},
and expressed numbers using additive combinations of these.
The Roman system, which is still in use, is similar, but
it combines glyphs using both addition and subtraction. This buys brevity,
since \emph{e.g.} $9_{\text{roman}}$ is two characters, ``one less than ten'',
and not a series of nine $1$s.
The ancient Babylonian system was, like the modern Hindu-Arabic decimal system,
an \emph{algorism}: the place value of a glyph determined how many times it 
counted towards the number being represented.
The Babylonians operated in
base $60$, and so \emph{e.g.} a three-gylph number $abc_{\text{babylonian}}$ could
be parsed as $a \times 60^2 + b \times 60 + c$. Sadly they lacked
a radix point, and so
$a \times 60^3 + b \times 60^2 + c \times 60$, $a \times 60 + b + c \div 60$, 
\emph{etc.} were also reasonable interpretations. 
Incorporating multiplication and division bought great brevity: a number $n$ was 
represented in $\lfloor \log_{60}n \rfloor + 1$ glyphs.

%\subsection*{The Ackermann function and its inverse.}
% This creates an unnumbered subsection

\subsection{The Ackermann function and its inverse}
%\marginpar{\tiny \color{blue} Several variations. Original. Peter.
%Primitive recursive. Hilbert? Ackermann is used in CS. Formalizations that use or define it. The grit of sand. The bug.}
% https://projecteuclid.org/download/pdf_1/euclid.bams/1183512393
% https://www.cs.princeton.edu/~chazelle/pubs/mst.pdf
%{\color{magenta}A brief sentence explaining what a primitive recursive function is and
%why total computable functions tend to be primitive recursive.}
The three-variable Ackermann function was presented by
Wilhelm Ackermann as an example of a total computable function that
is not primitive recursive~\cite{ackermann}.
It does not have the higher-order
relation to repeated application and hyperoperation that we have been studying in
this paper. Those properties emerged thanks to refinements by Rózsa Péter~\cite{peter},
and it is her variant, usually called the Ackermann-Péter function, 
that computer scientists commonly care about.

%\marginpar{\tiny \color{blue}implemented with }
The inverse Ackermann
features in the time bound analyses of several algorithms.
Tarjan \cite{tarjan} showed that the union-find data structure
takes time $O(m\cdot\alpha(m,n))$ for a sequence of $m$ operations
involving no more than $n$ elements. 
%Tarjan \cite{tarjan} showed that the union-find data structure
%with the optimisations of \emph{path compression} and \emph{weighted union}
%takes time $O(m\cdot\alpha(n))$ for a sequence of $m$ operations
%involving no more than $n$ elements.
Chazelle \cite{chazelle} showed that the minimum spanning tree
of a connected graph with $n$ vertices and $m$ edges
can be found in time $O(m\cdot\alpha(m,n))$.

% http://gallium.inria.fr/~fpottier/publis/chargueraud-pottier-uf-sltc.pdf
% http://gallium.inria.fr/~agueneau/publis/gueneau-chargueraud-pottier-coq-bigO.pdf
% https://scholar.google.com.sg/scholar?start=0&hl=en&as_sdt=2005&sciodt=0,5&cites=6488308509111085774&scipsc=
Charguéraud and Pottier, later joined by Guéneau \cite{charpott,gueneauetal},
extended Separation Logic with ``time credits'',
formalized the~$O$ notation in Coq,
and verified the correctness
and the time complexity of the union-find data structure in Coq.
They formalized a version of the Ackermann function, but not its inverse. 
Others \cite{others2,others4,others3,others1}
have also checked bounds on the resources
used by programs formally in proof assistants such as Coq, Isabelle/HOL, and Why3.

{\color{red}The Coq standard library has linear-time definitions 
of division and base-$2$ discrete logarithm on \li{nat} and \li{N}.
The Mathematical Components library~\cite{MathComp} 
has \li{log} on \li{nat} with prime bases. 
To our knowledge, we are the first to generalize this 
problem, extend it both 
upwards and downwards in the natural hierarchy of functions, and
provide linear computations up to representation size.}

Every pearl starts with a grain of sand.  We had the benefit of two: 
Nivasch~\cite{nivasch} and Seidel~\cite{seidel}.
They proposed a definition of the inverse Ackermann essentially in terms of
the inverse hyperoperations.  Unfortunately, their technique is unsound, since it diverges from
the true Ackermann inverse when the inputs grow sufficiently large.  Our technique is verified in Coq.

\subsection{Conclusion}
We have implemented a hierarchy of functions that calculate the upper inverses
to the Ackermann/hyperoperation hierarchy and used these inverses
to compute the inverse of the diagonal Ackermann function $\Ack(n)$.
Our functions are structurally recursive, and are thus immediately accepted by Coq,
and we have shown that they run in linear time.
%, and that it is consistent with
%the usual definition of the inverse Ackermann function $\alpha(n)$.


%{\color{magenta}This is a direction
%we intend to explore to formally check the $O(n)$ bound
%of the inverse Ackermann function.}

%Cite: Ackermann, Peter, Tarjan, Chazelle, Pottier? Anything in HOL? Anything in SSReflect?

%\paragraph*{Alternative strategies}


% Other ways to skin the cat.
% - You can define division via mutual recursion (subtraction and division simultaenously).
% - The inverse ackerman-lite by Anshuman.
% - The automata technique.
% - Binary representations
% - Division by constant, etc. is simpler.
% - Custom termination metrics.  Gas.
% - Space, tail recursion, time?

%\paragraph*{Other?}