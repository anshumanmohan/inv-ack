%%
%% Copyright 2007-2018 Elsevier Ltd
%%
%% This file is part of the 'Elsarticle Bundle'.
%% ---------------------------------------------
%%
%% It may be distributed under the conditions of the LaTeX Project Public
%% License, either version 1.2 of this license or (at your option) any
%% later version.  The latest version of this license is in
%%    http://www.latex-project.org/lppl.txt
%% and version 1.2 or later is part of all distributions of LaTeX
%% version 1999/12/01 or later.
%%
%% The list of all files belonging to the 'Elsarticle Bundle' is
%% given in the file `manifest.txt'.
%%

%% Template article for Elsevier's document class `elsarticle'
%% with numbered style bibliographic references
%% SP 2008/03/01
%%
%%
%%
%% $Id: elsarticle-template-num.tex 64 2013-05-15 12:23:51Z rishi $
%%
%%
\documentclass[preprint,12pt]{elsarticle}

%% Use the option review to obtain double line spacing
%% \documentclass[authoryear,preprint,review,12pt]{elsarticle}

%% Use the options 1p,twocolumn; 3p; 3p,twocolumn; 5p; or 5p,twocolumn
%% for a journal layout:
%% \documentclass[final,1p,times]{elsarticle}
%% \documentclass[final,1p,times,twocolumn]{elsarticle}
%% \documentclass[final,3p,times]{elsarticle}
%% \documentclass[final,3p,times,twocolumn]{elsarticle}
%% \documentclass[final,5p,times]{elsarticle}
%% \documentclass[final,5p,times,twocolumn]{elsarticle}

%% For including figures, graphicx.sty has been loaded in
%% elsarticle.cls. If you prefer to use the old commands
%% please give \usepackage{epsfig}

%% The amssymb package provides various useful mathematical symbols
\usepackage{amssymb}
\usepackage{amsmath}
%% The amsthm package provides extended theorem environments
\usepackage{amsthm}

\usepackage[colorlinks=True, citecolor=red, linkcolor=blue]{hyperref}
\usepackage{cleveref}

%% The lineno packages adds line numbers. Start line numbering with
%% \begin{linenumbers}, end it with \end{linenumbers}. Or switch it on
%% for the whole article with \linenumbers.
\usepackage{lineno}

%% Use package enumitem to align enumeration and itemization
\usepackage{enumitem}

\usepackage{listings}

\journal{Information Processing Letters}


%% NEW COMMANDS =============================================

\lstset{basicstyle=\tt,}
\makeatletter
\newlength{\@mli}
\newcommand{\mli}[1]{%
  \settowidth{\@mli}{\lstinline/#1/}
  \hspace{-.5ex}\begin{minipage}[t]{\@mli}\lstinline/#1/\end{minipage}}
\makeatother
\newcommand{\li}[1]{\ifmmode\mbox{\mli{#1}}\else\mbox{\lstinline/#1/}\fi}

\newcommand{\cdt}[2]{#1\mathrel{\mathchoice{\mkern-5mu}{\mkern-5mu}{\mkern-2mu}{\mkern-2mu}} \downarrow_{#2} \mathrel{\mathchoice{\mkern-3.5mu}{\mkern-3.5mu}{\mkern-2mu}{\mkern-2mu}}}

\newcommand{\rf}[2]{#1\mathrel{\mathchoice{\mkern-5mu}{\mkern-5mu}{\mkern-2mu}{\mkern-2mu}} \uparrow_{#2} \mathrel{\mathchoice{\mkern-3.5mu}{\mkern-3.5mu}{\mkern-2mu}{\mkern-2mu}}}

\newcommand{\invhyper}{I}
\renewcommand{\angle}[1]{\langle #1\rangle}

\newcommand{\W}{\text{W}}
\newcommand{\contract}{\text{\scshape{Contr}}}

\newcommand{\Ack}{\ensuremath{\text{A}}}
\newcommand{\CA}{\ensuremath{\text{C}}}
\renewcommand{\S}{\ensuremath{\text{S}}}
\newcommand{\CR}{\ensuremath{\text{CR}}}
\newcommand{\CRH}{\ensuremath{\text{CRH}}}
\newcommand{\wt}[1]{\ensuremath{\widetilde{#1}}}
\newcommand{\IAR}{\ensuremath{\text{IAR}}}
\newcommand{\IARH}{\ensuremath{\text{IARH}}}


\theoremstyle{plain}
\newtheorem{thm}{Theorem}[section]
\newtheorem{ex}[thm]{Example}
\newtheorem{prop}[thm]{Proposition}
\newtheorem{col}[thm]{Corollary}
\newtheorem{lem}[thm]{Lemma}
\newtheorem*{hypo}{Hypothesis}
\theoremstyle{definition}
\newtheorem{defn}{Definition}[section]
\newtheorem*{rem}{Remarks}

%% DOCUMENT =================================================

\begin{document}

\begin{frontmatter}

%% Title, authors and addresses

%% use the tnoteref command within \title for footnotes;
%% use the tnotetext command for theassociated footnote;
%% use the fnref command within \author or \address for footnotes;
%% use the fntext command for theassociated footnote;
%% use the corref command within \author for corresponding author footnotes;
%% use the cortext command for theassociated footnote;
%% use the ead command for the email address,
%% and the form \ead[url] for the home page:
\title{Inverting the Ackermann Hierarchy}

%% \tnotetext[label1]{}
%% \author{Name\corref{cor1}\fnref{label2}}
%% \ead{email address}
%% \ead[url]{home page}
%% \fntext[label2]{}
%% \cortext[cor1]{}
%% \address{Address\fnref{label3}}
%% \fntext[label3]{}

%% use optional labels to link authors explicitly to addresses:
%% \author[label1,label2]{}
%% \address[label1]{}
%% \address[label2]{}

\author{}

\address{}

\begin{abstract}


We build a hierarchy of functions that are an upper inverse
of the usual Ackermann hierarchy, and then use this inverse
hierarchy to compute the inverse of the diagonal Ackermann
function $\Ack(n, n)$. We show that this computation is
consistent with the usual definition of the inverse Ackermann
function $\alpha(n)$. We implement this computation in Gallina,
where we show that it runs in linear time.
\end{abstract}

%An alternative definition for the Inverse Ackermann function, and a linear time computation in Gallina



\begin{keyword}
Inverse Ackermann \sep Automata \sep Union-Find \sep Division
%% keywords here, in the form: keyword \sep keyword

%% PACS codes here, in the form: \PACS code \sep code

%% MSC codes here, in the form: \MSC code \sep code
%% or \MSC[2008] code \sep code (2000 is the default)
\end{keyword}

\end{frontmatter}

%% \linenumbers

%% main text
\section{Overview}
\label{sec:overview}
\begin{frame}
\frametitle{The Ackermann Function}
	
	The Ackermann-P\'eter function is defined as:
	\begin{equation*}
	A(n, m) \triangleq \begin{cases}
	m + 1 & \text{ when } n = 0 \\
	A(n-1, 1) & \text{ when } n > 0, m = 0 \\
	A\big(n-1, A(n, m-1)\big) & \text{ otherwise}
	\end{cases}
	\end{equation*}
	
	\pause 
	The \emph{diagonal} Ackermann function is $\Ack(n)~\triangleq~A(n, n)$.
	
	\bigskip
	
	\pause 
	First few values of $\Ack(n)$:
	
  \begin{minipage}{0.5\linewidth}
		\begin{equation*}
	\begin{array}{r|ccccc}
	 n & 0 & 1 & 2 & 3 & 4 \\ \hline
	 \Ack(n) & 1 & 3 & 7 & 61 & 2^{2^{2^{65536}}} - 3 \topspace{3pt}
	\end{array}
	\end{equation*}
  \end{minipage}
  \quad \pause 
  \begin{minipage}{0.4\linewidth}
  	\textcolor{red}{Explosive growth!}
  \end{minipage}

\end{frame}


%\begin{frame}
%\frametitle{Initial values for $\Ack(n)$ and $\alpha(n)$}
%\begin{columns}[T]
%	\begin{column}{0.4\textwidth}
%		
%		\begin{equation*}
%		\begin{array}{|ll|}
%		n & \Ack(n) \\
%		0 & 1 \\
%		1 & 3 \\
%		2 & 7 \\
%		3 & 61 \\
%		4 & 2^{2^{2^{65536}}} - 3
%		\end{array}
%		\end{equation*}
%		
%		Grows astronomically fast!
%	\end{column}
%
%  \begin{column}{0.5\textwidth}
%  	
%  	\begin{equation*}
%  	\begin{array}{|ll|}
%  	n & \alpha(n) \\
%  	0, 1 & 0 \\
%  	2, 3 & 1 \\
%  	4, 5, 6, 7 & 2 \\
%  	8, 9, \ldots, 61 & 3 \\
%  	62, \ldots, 2^{2^{2^{65536}}} - 3 & 4
%  	\end{array}
%  	\end{equation*}
%  	
%  	Grows astronomically slowly!
%  \end{column}
%\end{columns}
%\end{frame}


% \begin{frame}
% \frametitle{Introduction: Growth Patterns}

%  First few values of $\Ack(n)$ and $\alpha(n)$:
		
% %		\begin{equation*}
% %		\begin{array}{|ll|c|ll|}
% %		n & \Ack(n) & \hspace{3em} & n & \alpha(n) \\[3pt]
% %		0 & 1 & & 0, 1 & 0 \\[3pt]
% %		1 & 3 & & 2, 3 & 1 \\[3pt]
% %		2 & 7 & & 4, 5, 6, 7 & 2 \\[3pt]
% %		3 & 61 & & 8, 9, \ldots, 61 & 3 \\[3pt]
% %		4 & 2^{2^{2^{65536}}} - 3 & & 62, 63 \ldots, 2^{2^{2^{65536}}} - 3 & 4 \\[5pt]
% %		\multicolumn{2}{|l|}{\text{Grows astronomically fast!}} & & \multicolumn{2}{|l|}{\text{Grows astronomically slow!}}
% %		\end{array}
% %		\end{equation*}
		
% 		\begin{tikzpicture}
% 		\begin{axis}[
% %		symbolic x coords={a small bar,a medium bar,a large bar},
%     yticklabels={,,},
% 		xtick=data]
% 		\addplot coordinates {
% 			(0, 1)
% 			(1, 3)
% 			(2, 7)
% 			(3, 61)
% 			(4, 9999999)
			
% 		};
% 		\end{axis}
% 		\end{tikzpicture}

% %TODO Fix this graph? Graph for alpha? Figure with caption?

% \end{frame}


\begin{frame}
\frametitle{The Inverse Ackermann Function}

The \emph{inverse Ackermann function}, $\alpha$, maps $n$ to the smallest~$k$ for
which~$n \le \Ack(k)$, \emph{i.e} \impinline{$\alpha(n) \triangleq \min\left\{k\in \mathbb{N} : n \le \Ack(k)\right\}$}.

\smallskip

\pause 
$\alpha(n)$ grows slowly but is hard to \emph{compute} for large $n$
\\ because it is entangled with the explosively-growing $\Ack(k)$.

\bigskip


%\textbf{Naive Approach:} starting at $k=0$, calculate $\Ack(k)$,
%compare it to $n$, \\ and increment $k$ until $n \le \Ack(k)$.
\pause 
\textbf{Naive Approach:} Compute $\Ack(0), \Ack(1), \ldots$ until $n\le \Ack(k)$. Return $k$.

\bigskip

\pause 
\textbf{Time complexity:} $\Omega(\Ack(\alpha(n)))$.
\\ Computing $\alpha(100) \mapsto^{*} 4$ requires at least
%$\Ack(4) = 2^{2^{2^{65536}}}$ steps!
$\Ack(4) = 2^{2^{2^{65536}}} - 3$ steps!

\bigskip

\pause 
\textbf{Engineering hack:} Hardcode with lookup tables. $n > 61 \implies \text{ans} = 4$.

\bigskip

\pause 
\imppar{\text{\textbf{Our Goal.} Compute $\alpha$ for \emph{all inputs} without computing $\Ack$.}}
\end{frame}

%\begin{frame}
%\frametitle{Introduction: Ackermann \emph{vs} Hyperoperations}
%The Ackermann function is easy to define, but hard to
%understand.
%
%We see it as
%a sequence of $n$-indexed functions $\Ack_n \triangleq \lambda b.A(n,b)$, where for each $n>0$, $\Ack_n$ is the result of applying the previous $\Ack_{n-1}$ $b$ times.
%
%%with a
%%\href{https://github.com/inv-ack/inv-ack/blob/7270e64a2600b771f2b1b1b151f7d13fb2ae6c97/repeater.v\#L161-L177}{\emph{kludge}}. %Linked by Linh
%
%\bigskip
%
%The hierarchical structure resembles that of \emph{hyperoperations}.
%
%%To better understand the Ackermann function as a hierarchy and this kludge, we explore the closely-related hyperoperations.
%
%\end{frame}


\begin{frame}[fragile]
\frametitle{Our Solution}

\vspace{-1em}
\lstset{style=myTinyStyle}
% Linked by A
\begin{mdframed}[backgroundcolor=lightgray, roundcorner=10pt,leftmargin=0, rightmargin=0, innerleftmargin=0, innertopmargin=-5,innerbottommargin=-5, outerlinewidth=0, linecolor=lightgray]
\begin{lstlisting}
Require Import Omega Program.Basics.

`\href{https://github.com/inv-ack/inv-ack/blob/7270e64a2600b771f2b1b1b151f7d13fb2ae6c97/inv_ack_standalone.v#L6-L11}{Fixpoint cdn\_wkr}` (a : nat) (f : nat -> nat) (n b : nat) :=
 match b with 0 => 0 | S b' =>
  if (n <=? a) then 0 else S (cdn_wkr f a (f n) k')
 end.

`\href{https://github.com/inv-ack/inv-ack/blob/7270e64a2600b771f2b1b1b151f7d13fb2ae6c97/inv_ack_standalone.v#L14}{Definition countdown\_to}` a f n := cdn_wkr a f n n.

`\href{https://github.com/inv-ack/inv-ack/blob/7270e64a2600b771f2b1b1b151f7d13fb2ae6c97/inv_ack_standalone.v#L32-L38} {Fixpoint inv\_ack\_wkr}` (f : nat -> nat) (n k b : nat) :=
 match b with 0 => 0 | S b' =>
  if (n <=? k) then k else let g := (countdown_to f 1) in
                      inv_ack_wkr (compose g f) (g n) (S k) b
 end.

`\href{https://github.com/inv-ack/inv-ack/blob/7270e64a2600b771f2b1b1b151f7d13fb2ae6c97/inv_ack_standalone.v#L42-L46}{Definition inv\_ack\_linear}` (n : nat) : nat :=
 match n with 0 | 1 => 0 | _ => 
  let f := (fun x => x - 2) in inv_ack_wkr f (f n) 1 (n - 1)
 end.
\end{lstlisting}
\end{mdframed} 
\end{frame}


\begin{frame}
\frametitle{Ackermann \emph{vs} Hyperoperation}

The Ackermann function is easy to define, but hard to
understand.

\bigskip

\pause 	
Let's index by the first argument. \\ \smallskip
Define $\Ack_n \triangleq \lambda b.A(n,b)$. \\ \smallskip
Then, for $n>0$, $\Ack_n$ is the result of applying the previous $\Ack_{n-1}$ $b$ times.

%with a
%\href{https://github.com/inv-ack/inv-ack/blob/7270e64a2600b771f2b1b1b151f7d13fb2ae6c97/repeater.v\#L161-L177}{\emph{kludge}}. %Linked by Linh

\bigskip

\pause 
The hierarchical structure resembles that of \textcolor{red}{\emph{hyperoperations}}.

\smallskip
Studying hyperoperations helps us understand the Ackermann hierarchy.

%To better understand the Ackermann function as a hierarchy and this kludge, we explore the closely-related hyperoperations.
\end{frame}


\begin{frame}
\frametitle{Introduction: Ackermann \emph{vs} Hyperoperation}

Treating $b$ as the main argument, we can build their \emph{upper inverses}:

\begin{table}[t]
	\begin{centermath}
		\begin{array}{c@{\hskip 0.5em}|@{\hskip 1em}c@{\hskip 1em}c@{\hskip 1em}|@{\hskip 1em}c@{\hskip 1em}c}
%			  & \multicolumn{2}{|@{\hskip 0.5em}c@{\hskip 0.5em}|}{\text{Main hierarchies}} & \multicolumn{2}{|@{\hskip 0.5em}c@{\hskip 0.5em}|}{\text{Inverses hierarchies}} \\
			n & a [n] b & \Ack_n(b) & a \angle{n} b & \alpha_n(b)\\
			\hline
			0 & 1 + b & 1 + b & b - 1 & b - 1 \\
			1 & a + b & 2 + b & b - a & b - 2 \\
			2 & a \cdot b & 2b + 3 & \left\lceil \frac{b}{a} \right\rceil & \left\lceil \frac{b-3}{2} \right\rceil \\
			3 & a^b & 2^{b + 3} - 3 & \left\lceil \log_a ~ b \right\rceil & \left\lceil \log_2 ~ (b + 3)\right\rceil - 3 \\
			[1pt]
			4 & \underbrace{a^{.^{.^{.^a}}}}_b & \underbrace{2^{.^{.^{.^2}}}}_{b+3} - 3 & \log^*_a ~ b & \log^*_2 ~ (b + 3) - 3
		\end{array}
	\end{centermath}
	\label{table: hyperop-ack-inv}
\end{table}

\pause
Connection?
\pause
\\ \href{https://github.com/inv-ack/inv-ack/blob/7270e64a2600b771f2b1b1b151f7d13fb2ae6c97/repeater.v\#L161-L177}{Aha!} 
$\Ack_n(b) = {\color{red}2}[n](b{\color{red}+3}) {\color{red}- 3}$
\pause
\quad and \quad $\alpha_n(b) = {\color{red}2}\angle{n}(b{\color{red}+3}) {\color{red}- 3}$.

\end{frame}


\begin{frame}
\frametitle{Introduction: Inverse Hierarchies to Inverse Ackermann}

We explore the upper inverse relation:
\begin{equation*}
\begin{cases}
\forall b. \forall c.\quad b \le \Ack_n(c) & \!\! \iff \ \ \alpha_n(b)\le c \\
\forall b. \forall c. \quad b \le a[n]c & \!\! \iff \ \ a\angle{n}b \le c
\end{cases}
\end{equation*}

\textbf{Redefine $\bm{\alpha}$:}
$\alpha(n) = \min\{k: n\le \Ack_k(k) \} = \min\{k: \alpha_k(n)\le k \}$.

\bigskip

\pause
\textbf{Computing $\bm{\alpha} $ through $\bm{\alpha_i}$!} No need to go through $\Ack$.

\bigskip

\pause
\imppar{\textbf{Goal.} Build the inverse towers independent from the original towers.}

\end{frame}






%\begin{frame}
%\frametitle{}
%\end{frame}

\section{Repeater and Countdown}
\label{sec:hierarchy}
Let us formally define hyperoperations and clarify the intuition
given by Table~\ref{table: hyperop-ack-inv} by relating hyperoperations to
the Ackermann function.
The first hyperoperation (level 0) is simply successor, and
every hyperoperation that follows is the repeated application of the previous.
Level 1 is thus addition, and $b$ repetitions of addition
give level 2, multiplication. Next, $b$ repetitions of
multiplication give level 3, exponentiation.
There is a subtlety here: in the former case, we add $a$
repeatedly to the additive identity $0$, but in the
latter case, we multiply $a$ repeatedly to the multiplicative identity $1$. 
The formal definition of hyperoperation is:
%\marginpar{\tiny \color{blue} Should 2 just be under 1?}
%\begin{equation}
%\label{eq:hyper}
%\begin{array}{lrcl}
%\textit{1. 0$^{\textit{th}}$ level: } & a[0]b & ~ \triangleq ~ & b + 1 \\
%\textit{2. Initial values: } & a[n+1]0 & ~ \triangleq ~ &
%\begin{cases}
%a & \text{when } n = 0 \\
%0 & \text{when } n = 1 \\
%1 & \text{otherwise}
%\end{cases} \\
%\textit{3. Recursive rule: } \quad & a[n+1](b+1) & ~ \triangleq ~ & a[n]\big(a[n+1]b\big)
%\end{array}
%\end{equation}
% Edited by Linh
\begin{equation}
\label{eq:hyper}
\begin{array}{lrcl}
\textit{1. 0$^{\textit{th}}$ level:} & a[0]b & \triangleq & b + 1 \\
\textit{2. Initial values:} & a[n+1]0 & \triangleq &
  \begin{cases}
    a & \text{when } n = 0 \\
    0 & \text{when } n = 1 \\
    1 & \text{otherwise}
  \end{cases} \\
\textit{3. Recursive rule:} & a[n+1](b+1) & \triangleq & a[n]\big(a[n+1]b\big)
\end{array}
\end{equation}
The recursive rule looks complicated, but is actually just \emph{repeated application} in disguise. By fixing~$a$ and treating~$a[n]b$ as a function of~$b$, we can write
% \begin{equation*}
% \begin{array}{lll}
% a[n+1]b & ~ = ~ a[n]\big(a[n+1](b-1)\big) & ~ = ~ a[n]\big(a[n](a[n+1](b-2))\big) \\
%  & ~ = ~ \underbrace{\big( a[n]\circ a[n]\circ \cdots \circ a[n] \big)}_{b \text{ times}} \big(a[n+1]0\big) & ~ = ~ \big(a[n]\big)^{(b)}\big(a[n+1]0\big)
% \end{array}
% \end{equation*}
\begin{equation*}
\begin{array}{lll}
a[n+1]b ~& = ~ a[n]\big(a[n+1](b-1)\big) \\
         & = ~ a[n]\big(a[n](a[n+1](b-2))\big) \\
         & = ~ \underbrace{\big( a[n]\circ a[n]\circ \cdots \circ a[n] \big)}_{b \text{ times}} \big(a[n+1]0\big) \\
         & = ~ \big(a[n]\big)^{(b)}\big(a[n+1]0\big)
\end{array}
\end{equation*}
%where $f^{(k)}(u) ~ \triangleq ~ \overbrace{(f\circ f\circ \cdots \circ f)}^{k \text{ times}} (u)$,
where $f^{(k)}(u) \triangleq ~ (f\circ f\circ \cdots \circ f)(u)$ denotes $k$ compositional applications of a function~$f$ to an
input~$u$; $f^{(0)}(u) = u$ \linebreak (\emph{i.e.} applying $0$ times yields the identity).

This insight helps us encode hyperoperations~(\ref{eq:hyper}) and
Ackermann~(\ref{eq:ackermann}) in Coq.  
Notice that the recursive case of hyperoperations and
the third case of Ackermann both feature deep nested recursion, 
which makes our task tricky. 
In the outer recursive call, the first argument is shrinking
but the second is expanding explosively; in the inner recursive call, the first argument is
constant but the second is shrinking. The elegant solution uses double recursion~\cite{bertotcast} as follows:
% Linked by A
\begin{lstlisting}
`\href{https://github.com/inv-ack/inv-ack/blob/7270e64a2600b771f2b1b1b151f7d13fb2ae6c97/repeater.v#L51-L52}{\color{blue}Definition hyperop\_init}` (a n : nat) : nat :=
  match n with 0 => a | 1 => 0 | _ => 1 end.

`\href{https://github.com/inv-ack/inv-ack/blob/7270e64a2600b771f2b1b1b151f7d13fb2ae6c97/repeater.v#L55-L64}{\color{blue}Fixpoint hyperop\_original}` (a n b : nat) : nat :=
  match n with
  | 0    => 1 + b
  | S n' => let fix hyperop' (b : nat) :=
             match b with
             | 0    => hyperop_init a n'
             | S b' => hyperop_original a n' 
                                  (hyperop' b')
              end in hyperop' b
  end.

`\href{https://github.com/inv-ack/inv-ack/blob/7270e64a2600b771f2b1b1b151f7d13fb2ae6c97/repeater.v#L123-L132}{\color{blue}Fixpoint ackermann\_original}` (m n : nat) : nat :=
  match m with
  | 0    => 1 + n
  | S m' => let fix ackermann' (n : nat) : nat :=
             match n with
             | 0    => ackermann_original m' 1
             | S n' => ackermann_original m' 
                                 (ackermann' n')
             end in ackermann' n
  end.
\end{lstlisting}
Coq is satisfied since both recursive calls are on structurally smaller arguments.
Moreover, our encoding makes the structural similarities
 readily apparent.  In fact, the only essential difference is the initial values
(\emph{i.e.} the second case of both definitions): the Ackermann function uses $\Ack(n-1,1)$, whereas
hyperoperations use the initial values given in~\eqref{eq:hyper}.

We notice that the deep recursion in both cases is expressing the same notion
of repeated application, and this leads us to another useful idea. We can elegantly express the relationship
between the $(n+1)^{\text{th}}$ and $n^{\text{th}}$ levels via a higher-order function that transforms the latter level
to the former using a version of the well-known function iterator
\li{iter}~\cite{bertotcast}:
\begin{defn}
$\forall a\in \mathbb{N}, f: \mathbb{N}\to \mathbb{N}$, the
\href{https://github.com/inv-ack/inv-ack/blob/7270e64a2600b771f2b1b1b151f7d13fb2ae6c97/repeater.v#L32-L36}{\color{blue}\emph{repeater from}}
$a$ of $f$, denoted by $\rf{f}{a}$ , is a function $\mathbb{N}\to \mathbb{N}$ such that $\rf{f}{a}(n) = f^{(n)}(a)$.
% Linked by A
%\begin{lstlisting}
%`\href{https://github.com/inv-ack/inv-ack/blob/7270e64a2600b771f2b1b1b151f7d13fb2ae6c97/repeater.v#L32-L36}{Fixpoint repeater\_from}` (f : nat -> nat) (a n : nat) : nat :=
%  match n with 0 => a | S n' => f (repeater_from f a n') end.
%\end{lstlisting}
\begin{lstlisting}
`\href{https://github.com/inv-ack/inv-ack/blob/7270e64a2600b771f2b1b1b151f7d13fb2ae6c97/repeater.v#L32-L36}{\color{blue}Fixpoint repeater\_from}` f a n :=
  match n with
  | 0 => a
  | S n' => f (repeater_from f a n') 
  end.
\end{lstlisting}
\end{defn}
\noindent This allows simple and function-oriented definitions of hyperoperations and the
Ackermann function that we give below. Note that the Curried $a[n-1]$ denotes
the single-variable function $\lambda b.a[n-1]b$.
%\begin{equation*}
%\vspace{-0.75em}
%a[n]b ~ \triangleq ~ \begin{cases}
%b + 1 & \text{when } n = 0 \\
%\rf{a[n-1]}{a_{n-1}}(b) & \text{otherwise}
%\end{cases}
%\quad \text{ where } \ a_n ~ \triangleq ~ \begin{cases}
%a & \text{when } n = 0 \\
%0 & \text{when } n = 1 \\
%1 & \text{otherwise}
%\end{cases}
%\end{equation*}
% Edited by Linh
\begin{equation*}
a[n]b ~ \triangleq ~ \begin{cases}
b + 1 & \text{when } n = 0 \\
\rf{a[n-1]}{a_{n-1}}(b) & \text{otherwise}
\end{cases}
\end{equation*}
\begin{equation*}
\hspace{8em}\text{where  } a_n ~ \triangleq ~ \begin{cases}
a & \text{when } n = 0 \\
0 & \text{when } n = 1 \\
1 & \text{otherwise}
\end{cases}
\end{equation*}
% Linked by Anshuman
%\begin{lstlisting} 
%`\href{https://github.com/inv-ack/inv-ack/blob/7270e64a2600b771f2b1b1b151f7d13fb2ae6c97/repeater.v#L67-L71}{Fixpoint hyperop}` (a n b : nat) : nat :=
%  match n with
%  | 0    => 1 + b
%  | S n' => repeater_from (hyperop a n') (hyperop_init a n') b
%  end.
%\end{lstlisting}
% Edited by Linh
\begin{lstlisting} 
`\href{https://github.com/inv-ack/inv-ack/blob/7270e64a2600b771f2b1b1b151f7d13fb2ae6c97/repeater.v#L67-L71}{\color{blue}Fixpoint hyperop}` (a n b : nat) : nat :=
  match n with
  | 0    => 1 + b
  | S n' => repeater_from
              (hyperop a n') (hyperop_init a n') b
  end.
\end{lstlisting}

\pagebreak
\begin{equation*}
%\begin{array}{l}
A(n,m) ~ \triangleq ~ \begin{cases}
m + 1 & \text{when } n = 0 \\
\rf{\mathcal{A}_{n-1}}{A(n-1,1)\ }(m) & \text{otherwise}
\end{cases} \qquad \qquad \qquad \qquad \qquad \qquad \qquad ~ 
%\end{array}
\end{equation*}
% Linked by A
%\begin{lstlisting}
%`\href{https://github.com/inv-ack/inv-ack/blob/7270e64a2600b771f2b1b1b151f7d13fb2ae6c97/repeater.v#L135-L139}{Fixpoint ackermann}` (n m : nat) : nat :=
%  match n with
%  | 0    => S m
%  | S n' => repeater_from (ackermann n') (ackermann n' 1) m
%  end.
%\end{lstlisting}
% Edited by Linh
\begin{lstlisting}
`\href{https://github.com/inv-ack/inv-ack/blob/7270e64a2600b771f2b1b1b151f7d13fb2ae6c97/repeater.v#L135-L139}{\color{blue}Fixpoint ackermann}` (n m : nat) : nat :=
  match n with
  | 0    => S m
  | S n' => repeater_from
              (ackermann n') (ackermann n' 1) m
  end.
\end{lstlisting}
In the rest of this paper we construct efficient inverses to these
functions.  Our key idea is an inverse to the higher-order \emph{repeater} function, which we call \emph{countdown}.

% Aquinas' promise:
% We explain our core techniques of repeaters and countdowns that allow us to define each level of the Ackermann hierarchy—and their upper inverses—in a straightforward and uniform manner. We show how countdowns, in particular, can be written structurally recursively.

% Anshuman proposes:
% We introduce our core techniques of repeaters and countdowns.
% We show how countdowns, in particular, can be written structurally recursively.









\section{The Inverse Ackermann Hierarchy}
\label{sec:incr}
\subsection{Ackermann function and its inverse}

\marginpar{the old intro material is here} The time complexity of the

has traditionally
been hard to estimate, especially when it is implemented with the
heuristic rules of \emph{path compression} and \emph{weighted union}.
Tarjan \cite{tarjan} showed that for a sequence of $m$ FINDs
intermixed with $n-1$ UNIONs
such that $m \geq n$, the time required $t(m,n)$ is bounded
as: $k_{1}m\alpha(m,n) \leq t(m,n) \leq k_{2}m\alpha(m,n)$.
Here $k_{1}$ and $k_{2}$ are positive constants and $\alpha(m,n)$ is
the inverse of the Ackermann function.

The Ackermann function, commonly denoted $\Ack(m, n)$,
was first defined by Ackermann \cite{ackermann}, but this
definition was not as widely used as the following variant,
given by Peter and Robinson \cite{peter-ackermann}:




$\Ack(m, n)$ increases extremely fast on both inputs,
and, consequently, so does $\Ack(n, n)$.
This implies $\alpha(n)$ increases extremely slowly,
although it still tends to infinity. However, this does
not mean that computing $\alpha(n)$ for each $n$ is an easy
task. In fact, the naive method would iteratively check
$\Ack(k, k)$ for $k = 0, 1, \ldots, $ until $n \le \Ack(k, k)$.
The computation time would be staggering.
For instance, suppose $n > 1$, and $\alpha(n) = k+1$.
This is equivalent to

\begin{equation}
\Ack(k, k) < n \le \Ack(k+1, k+1)
\end{equation}

The naive algorithm would need to compute
$\Ack(t, t)$ for $t = 0, 1, \ldots, k, k+1$ before terminating.
{\color{magenta} Although one could argue that the total time to
compute $\Ack(t, t)$ for $t\le k$ is still $O(n)$, as they are all
less than $n$, the time to compute $\Ack(k+1, k+1)$ could be
astronomically larger than $n$.} This situation motivates the
need for an alternative, more efficient approach for computing
the inverse Ackermann function.

\subsection{The hierarchy of Ackermann functions}

If one denotes {\color{magenta}$\text{A}_m(n) = \Ack(m, n)$}, one can think
of the Ackermann function as a hierarchy of functions, each
level $A_m$ is a recursive function built with the previous
level $A_{m-1}$:

\begin{defn} \label{defn: ack_hier}
	The Ackermann hierarchy is a sequence of functions
	$\text{A}_0, \text{A}_1, \ldots $ defined as:
	\begin{enumerate}
		\item $A_0(n) = n + 1 \ \ \ \forall n\in \mathbb{N}$.
		\item $A_m(0) = A_{m-1}(1) \ \ \ \forall m\in \mathbb{N}_{>0}$.
		\item $A_{m}(n) = A_{m-1}^{(n)}(0) \ \ \ \forall n, m\in \mathbb{N}_{>0}$,
	\end{enumerate}
	\noindent where $f^{(n)}(x)$ {\color{magenta}denotes
	the result of applying $n$ times the function $f$ to
	the input $x$.} This hierarchy satisfies
	$\text{A}_m(n) = \Ack(m, n) \ \ \forall m, n\in \mathbb{N}$.
\end{defn}

This hierarchical perspective can be reversed, as shown in the
next section, to form an inverse Ackermann hierarchy of functions,
upon which we can compute the inverse Ackermann function as
defined in \Cref{defn: ack}. 

By extend, we mean to other members of the Ackermann/Knuth/hyperoperation hierarchy.  By this we
mean to the sequence of function families $f^a_0, f^a_1, \ldots, f^a_n, \ldots$, where $f^a_0(x) \triangleq x + 1$ and, for each $i > 0$, $f^a_i(x)$ is the result of iterating $f^a_{i-1}$ $x$ times
on $a$,
\[
\begin{array}{cclcl|c}
f^a_0(b) & \triangleq & 1 + \underbrace{1 + \ldots + 1}_b & = & b + 1 & b - 1 \\
f^a_1(b) & \triangleq & (\underbrace{f^a_0 \circ \ldots \circ f^a_0}_b)(a) & = & a + b & a - b \\
f^a_2(b) & \triangleq & (\underbrace{f^a_1 \circ \ldots \circ f^a_1}_b)(a) & = & a * b & \frac{b}{a} \\
f^a_3(b) & \triangleq & (\underbrace{f^a_2 \circ \ldots \circ f^a_2}_b)(a) & = & a^b & \mathsf{log}_a(b) \\
f^a_4(b) & \triangleq & (\underbrace{f^a_3 \circ \ldots \circ f^a_3}_b)(a) & = & \underbrace{a^{.^{.^{.^a}}}}_b & \mathsf{log}^*_a(b)  \\
\end{array}
\]

\footnote{The initial
value of $f_i(0)$ is a little idiosyncratic: for $i \in \{1,2\} it is $0$; for $i > 2 it is $1$.}, yielding the sequence of functions
  Indeed, troubles increase 
as one goes up (down?) the
inverse Ackermann hierarchy: although the standard library provides a $\mathsf{log}_2$ 
function, it does not provide a $\mathsf{log}_b$ function

, nor 
any flavor of the iterated logarithm $\mathsf{log}_b^{*}$ or functions further in the hierarchy.
And, of course, efficiently computing the inverse Ackermann function is harder
than computing the inverse of any particular level of the hierarchy.



\section{Some crappy theorems to put in place}
Before we begin, let us clarify that the definition of $\mathbb{N}$ and operations on $\mathbb{N}$ in Gallina follow the Presburger Arithmetic \cite{presburger}, in which the most operations agree with usual operations on $\mathbb{Z}$, except subtraction, which is defined as:
\begin{lstlisting}
Fixpoint sub (n m : nat) : nat :=
match n with
| 0 => n
| S k => match m with
         | 0 => n
         | S l => sub k l
         end
end.
\end{lstlisting}
Essentially $\li{sub} \ n \ m = \max\{n - m, 0\}$. We will use this subtraction for the rest of the paper.

The Presburger Arithmetic, despite being weaker than Peano Arithmetic, is a decidable theory. The Coq standard library includes \li{Omega}, an extensive listing of provable facts about $\mathbb{N}$ in Presburger Arithmetic, among which the most useful for our paper includes common laws of commutativity and associativity for addition, transitivity of $\le$ and $<$, and the law of excluded middle in comparisons on $\mathbb{N}$:
\begin{equation}
(n \le m) \vee (m + 1 \le n) \ \ \forall n, m
\end{equation}
, which is provable without the actual law of excluded middle in classical logic. This enables us to prove all results in this paper with Gallina's baseline intuitionistic logic. Readers can refer to the \Cref{appendix} for Coq versions of the proofs.

\begin{lem} \label{lem: contract-repeat-threshold}
For all $a, n\in\mathbb{N}$ and $f\in \contract_{1+a}$,
\begin{equation}
\exists m : \left(f^{(m)}(n) \le a \right) \wedge \left(f^{(l)}(n)\le a \implies m \le l \ \forall l \right)
\end{equation}
\end{lem}
\begin{proof}
Although the existence of $m$ is guaranteed by the well-ordering principle of $\mathbb{N}$, provided there is at least one $l$ so that $f^{(l)}(n)\le a$, we will achieve better by proving it in Gallina's baseline intuitionistic logic (without importing the law of excluded middle). Firstly we fix $n$ and observe that if $n\le a$, $m = 0$ is the desired choice since:
$$ f^{(0)}(n) = n \le a \ \text{ and } \ 0 \le l \ \forall l $$
Thus we can restrict ourselves to when $a\le n$, which then allows us to define $c$ such that $n = a + c$. We prove the following statement by induction:
$$ P(c) \triangleq \exists m : \left(f^{(m)}(n) \le n - c \right) \wedge \left(f^{(l)}(n)\le n - c \implies m \le l \ \forall l \right) $$
under assumptions $f\in \contract_{n-c+1}$, $c\le n$.
\begin{enumerate}[leftmargin=*]
	\item \textit{Base case.} For $c = 0$, this corresponds to when $n = a$, which has been proven above.
	\item \textit{Inductive step.} Suppose $P(c)$ is proved with witness $m_c$. Note that the assumptions are now $f\in \contract_{n-c}$ and $c+1\le n$, there are two cases:
	\begin{itemize}[leftmargin=*, label={-}]
		\item $f^{\left(m_c\right)}(n) = n - c$. Then $f^{\left(m_c+1\right)}(n) \le n - c - 1$ by the first assumption. Let $m_{c+1} = m_c + 1$, for all $l$:
		$$ f^{(l)}(n)\le n - c - 1 < f^{\left(m_c\right)}(n) \implies l > m_c \implies l \ge m_{c+1} $$
		\item $f^{\left(m_c\right)}(n) \le n - c - 1$. Let $m_{c+1} = m_c$, for all $l$:
		$$ f^{(l)}(n)\le n - c - 1\le n - c \overset{P(c)}{\implies} l\ge m_{c} = m_{c+1} $$
	\end{itemize}
	In any cases, we can find a witness $m_{c+1}$ for $P(c+1)$. Thus the proof is complete by induction.
\end{enumerate}
\end{proof}


\begin{lem} \label{lem: cdt-init}
For all $a, b\in \mathbb{N}$, $f : \mathbb{N}\to \mathbb{N}$ we have $\W\cdt{f}{a} (n, b) = 0 \ \forall n\le a$.
\end{lem}
\begin{proof}
Trivial.
\end{proof}

\begin{lem} \label{lem: cdt-intermediate}
For all $a, n, b, i\in \mathbb{N}$ and $f \in \contract$ such that $i < b$ and $a < f^{(i)}(n)$:
\begin{equation}  \label{eq: cdt-intermediate}
\W\cdt{f}{a}(n, b) = 1 + i + \W\cdt{f}{a}\left(f^{1+i}(n), b - i - 1\right)
\end{equation}
\end{lem}
\begin{proof}
We proceed by induction on $i$. Let
$$\begin{aligned}
 P(i) \triangleq \W\cdt{f}{a}(n, b) = 1 + i + \W\cdt{f}{a}\left(f^{1+i}(n), b - i - 1\right) \\ \forall b, n : b\ge i+1, f^{(i)}(n) > a
\end{aligned}$$
\begin{enumerate}[leftmargin=*]
	\item \textit{Base case.} For $i = 0$, our goal $P(0)$ is:
	$$ \W\cdt{f}{a}(n, b) = 1 + \W\cdt{f}{a}\left(f(n), b - 1\right) $$
	where $b \ge 1, f(n)\ge a+1$, which is trivial.
	\item \textit{Inductive case.} Suppose $P(i)$ has been proven. Then
	$$ P(i+1) \triangleq \W\cdt{f}{a}(n, b) = 2 + i + \W\cdt{f}{a}\left( f^{2+i}(n), b - i - 2 \right)$$
	for $b \ge i+2, f^{1+i}(n) \ge a+1$. This also implies $b\ge i+1$ and $\displaystyle f^{(i)}(n) \ge f^{1+i}(n)\ge a+1$ by $f\in \contract$, thus $P(i)$ holds. It suffices to prove:
	$$ \W\cdt{f}{a}\left(f^{1+i}(n), b-i-1\right) = 1 + \W\cdt{f}{a}\left( f^{2+i}(n), b-i-2 \right)$$
	Note that this is in fact $P(0)$ with $(b, n) \leftarrow \left(b-i-1, f^{(1+i)}(n)\right)$. Since $f^{(1+i)}(n) \ge a+1$ and $b-i-1\ge 1$, the above holds and proof is complete. 
\end{enumerate}
\end{proof}

\begin{thm} \label{thm: cdt-repeat}
For all $a\in \mathbb{N}$ and $f\in \contract_{1+a}$, we have
\begin{equation} \label{eq: cdt-minimum}
\cdt{f}{a}(n) = \min\left\{ i : f^{(i)}(n) \le a \right\} \ \ \forall n
\end{equation}
Or equivalently,
\begin{equation} \label{eq: cdt-repeat}
\cdt{f}{a}(n) \le k \iff f^{(k)}(n) \le a \ \ \forall n, k
\end{equation}
\end{thm}
\begin{proof}
First, to see why \eqref{eq: cdt-minimum} and \eqref{eq: cdt-repeat} are equivalent, we rewrite \eqref{eq: cdt-minimum} in the following way:
$$ \left(f^{(\cdt{f}{a}(n))}(n) \le a\right) \wedge \left(f^{(l)}(n) \le a \implies \cdt{f}{a}(n) \le l \ \ \forall l\right) \ \ \forall n$$
To prove $\eqref{eq: cdt-minimum} \implies \eqref{eq: cdt-repeat}$, it suffices to show
$$ \cdt{f}{a}(n) \le l \implies f^{(l)}(n) \le a \ \ \forall l $$
, which holds due to the fact $\displaystyle f^{(l)}(n) \le f^{(\cdt{f}{a}(n))}(n) \le a$ by $f\in \contract$. To prove $\eqref{eq: cdt-repeat}\implies \eqref{eq: cdt-minimum}$, it suffices to show 
$$\displaystyle f^{(\cdt{f}{a}(n))}(n) \le a$$
, which in turn holds by substituting $k$ by $\cdt{f}{a}(n)$ in \eqref{eq: cdt-repeat}. Thus $\eqref{eq: cdt-minimum}\iff \eqref{eq: cdt-repeat}$ and we need only to prove \eqref{eq: cdt-minimum}. Using \cref{lem: contract-repeat-threshold}, there exists an $m$ such that
\begin{align}
& f^{(m)}(n) \le a \label{eq: cdt-repeat-tmp-1} \\
& f^{(k)}(n)\le a \implies m \le k \ \ \forall k \label{eq: cdt-repeat-tmp-2}
\end{align}
It then suffices to prove $m = \cdt{f}{a}(n)$. Suppose firstly that $m = 0$. Then $n = f^{(0)}(n)\le a$, thus $\cdt{f}{a}(n) = \W\cdt{f}{a}(n, n) = 0 = m$ by \cref{lem: cdt-init}.

Now consider $m > 0$. We would like to apply \cref{lem: cdt-intermediate} to get
$$ \cdt{f}{a}(n) = \W\cdt{f}{a}(n, n) = m + \W\cdt{f}{a}\left(f^{(m)}(n), n-m\right) $$
, then use \cref{lem: cdt-init} over \eqref{eq: cdt-repeat-tmp-1} to conclude that $\cdt{f}{a}(n) = m$. It then suffices to prove the premises of \cref{lem: cdt-intermediate}, namely $a < f^{(m-1)}(n)$ and $m-1 < n$.

The former follows by contradiction (LEM on $\mathbb{N}$ is implicit here): if $f^{(m-1)}(n) \le a$, \eqref{eq: cdt-repeat-tmp-2} implies $m\le m-1$, which is wrong for $m > 0$. The latter then easily follows by $f\in \contract_{1+a}$:
$$ n \ge 1 + f(n) \ge 2 + f(f(n)) \ge \cdots \ge m + f^{(m)}(n) $$
Therefore, $\cdt{f}{a}(n) = m$ in all cases, which completes the proof.
\end{proof}

\begin{rem}
We will mainly be using \eqref{eq: cdt-repeat} in our subsequent proofs due to its power. As already shown, \eqref{eq: cdt-repeat} implies \eqref{eq: cdt-minimum} without the need for $f\in \contract_{1+a}$. It is also easier to manipulate algebraically due to its neat form as a logical sentence.
\end{rem}

\begin{thm} \label{thm: cdt-recursion}
For all $a\in \mathbb{N}$ and $f\in \contract_{1+a}$, $\cdt{f}{a}$ satisfies:
$$ \cdt{f}{a}(n) = \begin{cases}
0 & \text{ if } n \le a \\ 1 + \cdt{f}{a}(f(n)) & \text{ if } n \ge a + 1
\end{cases} $$
\end{thm}
\begin{proof}
The case $n\le a$ is trivial by \cref{lem: cdt-init}. Suppose $n\ge a+1$,
\begin{enumerate}[leftmargin=*]
	\item Let $\cdt{f}{a}(f(n)) = m$. We prove $\cdt{f}{a}(n)\le 1+m$, or $f^{1+m}(n) \le a$. But this is equivalent to $f^{(m)}(f(n)) \le a$, which holds by $m$'s definition.
	\item Note that $n\ge a+1$ implies $\cdt{f}{a}(n) = 1 + \W\cdt{f}{a}(f(n), n - 1)$ by \cref{lem: cdt-intermediate}. Let $\cdt{f}{a}(n) = p + 1$. We prove $\cdt{f}{a}(f(n))\le p$, or $f^{(p)}(f(n))\le a$. But this is equivalent to $f^{p+1}(n)\le a$, which holds by $p$'s definition.
\end{enumerate}
The proof is then complete.
\end{proof}

\begin{thm} \label{thm: cdt-contr-0}
For all $a\ge 1$, if $f\in \contract_{1+a}$, then $\cdt{f}{a}\ \in \contract_{1+c} \ \forall c$.
\end{thm}
\begin{proof}
Firstly we show that $\cdt{f}{a}$ is a contraction, namely showing $\cdt{f}{a}(n)\le n \ \forall n$, which has already been proved in the proof of \cref{thm: cdt-repeat}. To show $\cdt{f}{a}$ is strict from $1+c$, it suffices to show it is strict from $1$, equivalently $\cdt{f}{a}(n) \le n - 1 \ \forall n\ge 1$. By \cref{thm: cdt-repeat}, we need to show $f^{(n-1)}(n)\le a$. Assume the converse, then:
$$ n \ge 1 + f(n)\ge 2 + f(f(n)) \ge \cdots \ge (n-1) + f^{(n-1)}(n) $$
Thus $f^{(n-1)}(n)\le 1 \le a$, a contradiction. The theorem then follows.
\end{proof}

\begin{thm} \label{thm: upp-inv-cdt-rf}
For all $a\in \mathbb{N}$, if $f\in \contract_{1+a}$ is the upper inverse of $F: \mathbb{N}\to \mathbb{N}$, then $\cdt{f}{a}$ is the upper inverse of $\rf{F}{a}$.
\end{thm}
\begin{proof}
Assuming $f\in \contract_{1+a}$ and is the upper inverse of $F$, we need to show:
$$ \cdt{f}{a}(N) \le n \iff N \le \rf{F}{a}(n) \ \ \forall n, N $$
By \cref{thm: cdt-repeat}, the LHS is equivalent to $f^{(n)}(N)\le a$. Rewrite $\rf{F}{a}(n) = F^{(n)}(a)$, we have, by the upper inverse relation of $f$ to $F$:
$$ \begin{aligned}
f^{(n)}(N) \le a & \iff f^{(n-1)}(N) \le F(a) \iff f^{(n-2)}(N) \le F^{(2)}(a) \\
 & \iff \cdots \iff N \le F^{(n)}(a) 
\end{aligned}$$
The proof is complete.
\end{proof}

\begin{lem}  \label{lem: inv-hyperop-0-contr1}
For all $a\in \mathbb{N}$, $a\angle{0}\in \contract_1$.
\end{lem}
\begin{proof}
Trivial.
\end{proof}

\begin{thm}
For all $a, b\in \mathbb{N}$, $a\angle{1}b = b - a$.
\end{thm}
\begin{proof}
Rewriting $a\angle{1} = \cdt{\left(a\angle{0}\right)}{a}$. By \cref{lem: inv-hyperop-0-contr1}, $a\angle{0} \in \contract_1 \subset \contract_{1+a}$, \cref{thm: cdt-recursion} applies.
$$ a\angle{1}b = \begin{cases}
0 & \text{ if } b\le a \\ 1 + a\angle{1}(b - 1) & \text{ if } b\ge a+1
\end{cases} $$
A simple induction will then confirms $a\angle{1}b = b - a \ \forall b$.
\end{proof}

\begin{col} \label{col: inv-hyperop-1-contr1}
For all $a\ge 1$, $a\angle{1} \in \contract_1$.
\end{col}

\begin{col}
For all $a, b\in \mathbb{N}$, $a\ge 1$, $a\angle{2}b = \displaystyle \left\lceil \frac{b}{a} \right\rceil$.
\end{col}

\begin{thm}
For all $a\ge 2$, let $a_0 = a$, $a_1 = 0$, $a_n = 1 \ \forall n\ge 2$. Then for all $n$, $a\angle{n} \in \contract_{1+a_n}$ and is the upper inverse of $a[n]$.
\end{thm}
\begin{proof}
Let
$$\begin{aligned}
P(n) \triangleq & \ \left(a\angle{n} \in \contract_{1+a_n}\right)\\
Q(n) \triangleq & \ \left(a\angle{n}b \le c \iff b\le a[n]c \ \forall b, c\right)
\end{aligned}$$
Note that for all $n$, $a\angle{n+1} = \cdt{a\angle{n}}{a_n}$ and $a[n+1] = \rf{a[n]}{a_n}$. Thus $P(n)$ implies $Q(n+1)$ by \cref{thm: upp-inv-cdt-rf}. Hence it suffices to prove $Q(0)$ and $P(n)\ \forall n$. Now
$$ Q(0) \equiv \left(b - 1 \le c \iff b\le  c + 1 \ \forall b, c\right) $$
, which is trivial. Since $P(0)$ and $P(1)$ have been covered by \cref{lem: inv-hyperop-0-contr1} and \cref{col: inv-hyperop-1-contr1}, we prove $P(n)\ \forall n\ge 2$ by induction.
\begin{enumerate}[leftmargin=*]
	\item \textit{Base case.} By $P(1)$ and \cref{thm: cdt-repeat},
	$$ a\angle{2}b \le k \iff \left(a\angle{1}\right)^{(k)}(b) \le 0 \iff b \le ka $$
	For all $b$, $b\le ba$ and if $b\ge 2$, $b \le 2(b-1)\le a(b-1)$ since $a\ge 2$, so $a\angle{2} \in \contract_2$, as desired.
	\item \textit{Inductive step.} We need to prove $a\angle{n+1}\in \contract_2$, or $\cdt{a\angle{n}}{1} \ \in \contract_2$, given $a\angle{n}\in \contract_2$. But this follows directly from \cref{thm: cdt-contr-0} with $a = c = 1$.
\end{enumerate}
By induction, the proof is complete.
\end{proof}

\begin{col}
For all $a, b\in \mathbb{N}$, $a\ge 2$, $a\angle{3}b = \displaystyle \left\lceil \log_a(b) \right\rceil$.
\end{col}

\begin{col}
For all $a, b\in \mathbb{N}$, $a\ge 2$, $a\angle{4}b = \log^*_a(b) $.
\end{col}

\section{A linear time computation in Gallina}
\label{sec:computation}
The main idea to compute the inverse Ackermann function using what we have established so far is to use \Cref{thm: inv_ack_ack} to iteratively compute each level in the hierarchy starting from input $n+2$, then stop when the condition $\alpha_m(n+2) \le m + 3$ is met.

In order for this computation to run in linear time, we need to first make sure \textit{each} level $\alpha_m(n+2)$ is computed in linear time. We will then use a trick to achieve the linear time bound in the total computation using the relation
\begin{equation}
\alpha_m(n+2) = 1 + \alpha_m(\alpha_{m-1}(n+2)) \ \ \ \forall m\in \mathbb{N}_{>0}
\end{equation}

\subsection{Inverse Ackermann hierarchy in linear time}  \label{subsec: inv_ack_hier linear}

First note that, in the Gallina specification, all natural numbers are represented with a string of $\S$, the successor notation. All recursive functions in Gallina are required to decrease on one of its inputs, one or a few successors per recursive step. Thus all recursive functions, or Fixpoints in Gallina, must run in at least linear time over one of their inputs.

To compute the inverse Ackermann hierarchy for some input $n$ with a Gallina-complied function, we look at its generalization: Given a contraction $f$ over $\mathbb{N}$ and a Gallina function $\widetilde{f}$ computing $f$, we find a Gallina function $\widetilde{f^*}$ to compute its countdown, $f^*$.

\begin{defn} \label{defn: countdown rec helper}
Let $\mathcal{F}_k$ be the set of all Gallina functions $g: \mathbb{N}^k \to \mathbb{N}$. The \textit{countdown recursor helper} is an operator $\CRH : \mathcal{F}_1 \to \mathcal{F}_3$ such that for all $g\in \mathcal{F}_1$ and $n_0, n_1, c\in \mathbb{N}$:
\begin{equation}
\begin{aligned}
& \CRH\left(g\right)\left(n_0, n_1, c\right) = \\
& \begin{cases}
0 & \text{if } n_0 \le 1. \\
1 & \text{if } n_0 \ge 2, n_1 \le 1. \\
1 + \CRH \left(g\right)
		\left(n_0-1, g(n_1), n_1 - g(n_1) - 1\right)
		& \text{if } n_0\ge 2, n_1 \ge 2, c = 0. \\
\CRH \left(g\right)
			\left(n_0-1, n_1, c-1\right)
		& \text{if } n_0\ge 2, n_1\ge 2, c\ge 1.
\end{cases}
\end{aligned}
\end{equation}
\end{defn}

It is trivial to see that for all $g\in \mathcal{F}$, $\CRH(g)$ is a Gallina Fixpoint (a Gallina-complied recursive function) in $\mathcal{F}_3$ since its first input $n_0$ decreases by $1$ at every recursive step.

\begin{defn} \label{defn: countdown rec}
The \textit{countdown recursor} is an operator $\CR : \mathcal{F}_1 \to \mathcal{F}_1$ such that for all $g\in \mathcal{F}_1$ and $n\in \mathbb{N}$:
\begin{equation}
\CR(g)(n) = \CRH(g)(n, g(n), n - g(n) - 1)
\end{equation}
\end{defn}

Since it is built by a composition of a Gallina Fixpoint $\CRH$ and a Gallina function $g$, $\CR(g)$ is indeed a Gallina function in $\mathcal{F}_1$. We prove that $\CR$ is the equivalence of the countdown operation in Gallina for contractions:

\begin{lem} \label{lem: CRH_countdown}
Let $f: \mathbb{N}\to \mathbb{N}$ be a contraction. Suppose a function $\widetilde{f}\in \mathcal{F}_1$ computes $f$, then $\CR\left(\widetilde{f}\right)$ is a function in $\mathcal{F}_1$ computing $f^*$.
\end{lem}

\begin{proof}[Proof Sketch]
Let $g := \wt{f}$.
Firstly, from \Cref{defn: countdown rec helper}, we can prove that for all $n_0, n1, c, k\in \mathbb{N}$ such that $n_0\ge 2$ and $k\le \min\{n_0, c\}$:
\begin{equation*}
\CRH(g)\left(n_0, n_1, c\right) = \CRH(g)\left(n_0 - k, n_1, c - k\right)
\end{equation*}
This implies that if $n_0\ge c + 2$, then
\begin{equation*}
\CRH(g)\left(n_0, n_1, c\right) = 1 + \CRH(g)\left(n_0 - c - 1, g(n_1), n_1 - g(n_1) - 1\right)
\end{equation*}
If $n - 1\ge g(n) \ge 1$, let $n_0 := n$, $n_1 := g(n)$, $c := n - g(n) - 1$ gives:
\begin{equation*}
\begin{aligned}
& \CRH(g)\left(n, g(n), n - g(n) - 1\right) \\
& = 1 + \CRH(g)\left(g(n), g(g(n)), g(n) - g(g(n)) - 1\right)
\end{aligned}
\end{equation*}
Or
\begin{equation*}
\CR(g)(n) = 1 + \CR(g)(g(n))
\end{equation*}
Together with the initial values of $\CRH(g)$, we conclude that $\CRH(g)$, or $\CRH\left(\wt{f}\right)$ indeed computes $f^*$.

\end{proof}

With this lemma, we can define the equivalence of the inverse Ackermann hierarchy in Gallina:

\begin{defn}
The Gallina inverse Ackermann hierarchy is a sequence of functions $\wt{\alpha}_0, \wt{\alpha}_1, \ldots $ such that for all $m, n\in \mathbb{N}$:
\begin{equation}
\wt{\alpha}_m(n) = \begin{cases}
n - 2 & \text{if } m = 0 \\ \CR\left(\wt{\alpha}_{m-1}\right)(n) & \text{if } m \ge 1
\end{cases}
\end{equation}
Note that $x-y$ in Gallina is equivalent to $\max\{x-y, 0\}$ in practice.
\end{defn}

\Cref{lem: CRH_countdown} and \Cref{thm: inv_ack_countdown} trivially implies that $\wt{\alpha}_m$ is a Gallina computation of $\alpha_m$ for all $m\in \mathbb{N}$.

The important thing to come up with Gallina computations for the hierarchy is we want to compute them in linear time. We assert the hierarchy $\left\{\wt{\alpha}_m\right\}$ succeeds in doing so:

\begin{thm}  \label{thm: inv_ack_hier linear}
For each $m \in \mathbb{N}$, computing $\wt{\alpha}_m(n)$ takes at most $(m + o(1))n$ steps, where $n$ tends to infinity.
\end{thm}

\begin{proof}
TODO TODO TODO
\end{proof}

\subsection{Inverse Ackermann in linear time}  \label{subsec: inv_ack linear}

As mentioned, the task is to find the minimum $x$ for which $\alpha_x(n) \le x + 3$ for $n\ge 4$. It is tempting to go for a naive approach, after all the efficiencies we have developed so far: Starting from $x=0$, we iteratively compute $\alpha_x(n)$ and compare it with $x+3$. However, by our earlier analysis in \Cref{thm: inv_ack_hier linear}, the total amount of time needed is

$$ T(n) = \sum_{m = 0}^{\alpha(n)-1}(m + o(1))n = O\left(n\alpha(n)^2\right) $$

which is pretty efficient, given how slow-growing $\alpha(n)$ is. However, our ultimate goal is $O(n)$ time, thus we came up with a better approach. Interestingly, it is also based on \Cref{thm: inv_ack_hier linear}. Specifically, using \eqref{eq: countdown}, \Cref{thm: inv_ack_hier linear} implies we can actually compute $\alpha_m(n)$ in $O(\alpha_{m-1}(n))$ time, \textit{if} $\alpha_{m-1}(n)$ is given. It is thus beneficial if we retain the value of $\alpha_{m-1}(n)$ when computing $\alpha_m(n)$. The definition below is our Gallina recursor for this.

\begin{defn}  \label{defn: inv_ack_recursor_helper}
The inverse Ackermann recursor helper is a function $\IARH : \mathcal{F}_1\times \mathbb{N}^3 \to \mathbb{N}$ such that for all $g\in \mathcal{F}_1, n, c_0, c\in \mathbb{N}$, we have:
\begin{equation}
\begin{aligned}
& \IARH(g, n, c_0, c) =  \\
& \begin{cases}
c & \text{if } c = 1 \\
1 + \IARH\left(\CR(g), n, n - \CR(g)(n) - 1, c - 1\right) & \text{if } c \ge 1, c_0 = 0 \\
\IARH\left(g, n - 1, c_0 - 1, c - 1\right) & \text{if } c \ge 1, c_0 \ge 1
\end{cases}
\end{aligned}
\end{equation}
Note that $n - 1$ in Gallina means $\max\{n-1, 0\}$ here.
\end{defn}

\begin{defn}  \label{defn: inv_ack_recursor}
The inverse Ackermann recursor is a function $\IAR \in \mathcal{F}_1$ such that for all $n\in \mathbb{N}$, we have
\begin{equation}
\IAR(n) = \begin{cases}
0 & \text{if } n \le 1 \\ \IARH\left(\alpha_0, n + 2, 2, n\right) & \text{if } n\ge 2
\end{cases}
\end{equation}
\end{defn}

The following theorem is a central result in this paper, which asserts the correctness of $\IAR$.

\begin{thm} \label{thm: inv_ack_correct}
$\IAR$ is a Gallina function computing $\alpha(n)$.
\end{thm}

It is trivial to see that both $\IARH$ and $\IAR$ are Gallina functions. It suffices to show that $\IAR(n) = \alpha(n)$ for $n\ge 2$, since their values already match for $n\le 1$. Furthermore, if we trace the first two recursive steps in $\IARH$, we obtain
\begin{equation*}
\begin{aligned}
& \IARH(n+2, \alpha_0, 2, n) \\
& = 1 + \IARH(\alpha_1, n-2, n-\alpha_1(n)-1, n-3) \ \ \forall n\ge 2
\end{aligned}
\end{equation*}
It then suffices to show the following:

\begin{lem}  \label{lem: inv_ack_rec_helper}
For all $n\ge 2$,
$$\IARH(\alpha_1, n, n - \alpha_1(n) - 1, n - 3) = \min\left\{m : \alpha_m(n+2)\le m+3 \right\}$$.
\end{lem}

Before proving \Cref{lem: inv_ack_rec_helper}, we need an intermediate lemma.

\begin{lem}
For all $n\ge 
\end{lem}

%% The Appendices part is started with the command \appendix;
%% appendix sections are then done as normal sections
%% \appendix

%% \section{}
%% \label{}

%% If you have bibdatabase file and want bibtex to generate the
%% bibitems, please use
%%
\bibliographystyle{elsarticle-num}
%\bibliography{references.bib}

%% else use the following coding to input the bibitems directly in the
%% TeX file.

%%\begin{thebibliography}{00}

%% \bibitem{label}
%% Text of bibliographic item

%%\end{thebibliography}
\end{document}
\endinput
%%
%% End of file `elsarticle-template-num.tex'.
