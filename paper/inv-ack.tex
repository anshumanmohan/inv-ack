%%
%% Copyright 2007-2018 Elsevier Ltd
%%
%% This file is part of the 'Elsarticle Bundle'.
%% ---------------------------------------------
%%
%% It may be distributed under the conditions of the LaTeX Project Public
%% License, either version 1.2 of this license or (at your option) any
%% later version.  The latest version of this license is in
%%    http://www.latex-project.org/lppl.txt
%% and version 1.2 or later is part of all distributions of LaTeX
%% version 1999/12/01 or later.
%%
%% The list of all files belonging to the 'Elsarticle Bundle' is
%% given in the file `manifest.txt'.
%%

%% Template article for Elsevier's document class `elsarticle'
%% with numbered style bibliographic references
%% SP 2008/03/01
%%
%%
%%
%% $Id: elsarticle-template-num.tex 64 2013-05-15 12:23:51Z rishi $
%%
%%
\documentclass[a4paper,USenglish,cleveref,autoref]{lipics-v2019}
% \documentclass[preprint,12pt]{elsarticle}

%% Use the option review to obtain double line spacing
%% \documentclass[authoryear,preprint,review,12pt]{elsarticle}

%% Use the options 1p,twocolumn; 3p; 3p,twocolumn; 5p; or 5p,twocolumn
%% for a journal layout:
%% \documentclass[final,1p,times]{elsarticle}
%% \documentclass[final,1p,times,twocolumn]{elsarticle}
%% \documentclass[final,3p,times]{elsarticle}
%% \documentclass[final,3p,times,twocolumn]{elsarticle}
%% \documentclass[final,5p,times]{elsarticle}
%% \documentclass[final,5p,times,twocolumn]{elsarticle}

%% For including figures, graphicx.sty has been loaded in
%% elsarticle.cls. If you prefer to use the old commands
%% please give \usepackage{epsfig}

%% The amssymb package provides various useful mathematical symbols
\usepackage{amssymb}
\usepackage{amsmath}
%% The amsthm package provides extended theorem environments
\usepackage{amsthm}

%\usepackage[colorlinks=True, citecolor=red, linkcolor=blue]{hyperref}
%\usepackage{cleveref}

%% The lineno packages adds line numbers. Start line numbering with
%% \begin{linenumbers}, end it with \end{linenumbers}. Or switch it on
%% for the whole article with \linenumbers.
\usepackage{lineno}

%% Use package enumitem to align enumeration and itemization
\usepackage{enumitem}
\usepackage{listings}
\usepackage{courier}           % for the courier font (optional)
\usepackage{multicol}          % for two equations side by side
\usepackage[justification=centering]{caption} 
\usepackage{xcolor}

\hypersetup{ % play with these to change the look of hyperlinks
    colorlinks=true,
    linkcolor=black,
    filecolor=magenta,
    urlcolor=cyan,
    citecolor=black
}

\colorlet{red}{red!80!black}

%% NEW COMMANDS =============================================

\lstdefinestyle{myStyle}{
%	language=Coq,
    keywords={Definition,Fixpoint,match,with,end,let,in,fix},
	basicstyle=\normalfont\footnotesize\tt,
    keywordstyle=\color{red}, % Blue clashes with the cyan links. Change if you want.
	stepnumber=1,
	tabsize=2,
    numbers=none,
    numberstyle=\tiny,
    numbersep=5pt,
	showspaces=false,
    escapechar=`,
	showstringspaces=false
}
%basicstyle=\fontsize{10}{11}\selectfont\ttfamily,

\lstset{style=myStyle}
\makeatletter
\newlength{\@mli}
\newcommand{\mli}[1]{%
  \settowidth{\@mli}{\lstinline/#1/}
  \hspace{-.5ex}\begin{minipage}[t]{\@mli}\lstinline/#1/\end{minipage}}
\makeatother
\newcommand{\li}[1]{\ifmmode\mbox{\mli{#1}}\else\mbox{\lstinline/#1/}\fi}

\newcommand{\cdt}[2]{#1\mathrel{\mathchoice{\mkern-5mu}{\mkern-5mu}{\mkern-2mu}{\mkern-2mu}} ^{\mathcal{C}}_{#2} \mathrel{\mathchoice{\mkern-3.5mu}{\mkern-3.5mu}{\mkern-2mu}{\mkern-2mu}}}

\newcommand{\cdw}[2]{#1\mathrel{\mathchoice{\mkern-5mu}{\mkern-5mu}{\mkern-2mu}{\mkern-2mu}} ^{\textit{CW}}_{#2} \mathrel{\mathchoice{\mkern-3.5mu}{\mkern-3.5mu}{\mkern-2mu}{\mkern-2mu}}}

\newcommand{\rf}[2]{{#1\mathrel{\mathchoice{\mkern-5mu}{\mkern-5mu}{\mkern-2mu}{\mkern-2mu}} ^{\mathcal{R}}_{#2} \mathrel{\mathchoice{\mkern-3.5mu}{\mkern-3.5mu}{\mkern-2mu}{\mkern-2mu}}}}

\newcommand{\invhyper}{I}
\renewcommand{\angle}[1]{\langle #1\rangle}

\newcommand{\W}{\mathcal{W}}
\newcommand{\contract}{\text{\scshape{Contr}}}
\newcommand{\repeatable}{\text{\scshape{Rept}}}

\newcommand{\divc}{\text{divc}}
\newcommand{\logc}{\text{logc}}
\newcommand{\logstarc}{\text{logc*}}

\newcommand{\runtime}{\mathcal{T}}
\newcommand{\bigO}{\text{O}}

\newcommand{\Ack}{\ensuremath{\text{A}}}
\newcommand{\CA}{\ensuremath{\text{C}}}
%\renewcommand{\S}{\ensuremath{\text{S}}}
\newcommand{\CR}{\ensuremath{\text{CR}}}
\newcommand{\CRH}{\ensuremath{\text{CRH}}}
\newcommand{\wt}[1]{\ensuremath{\widetilde{#1}}}
\newcommand{\IAR}{\ensuremath{\text{IAR}}}
\newcommand{\IARH}{\ensuremath{\text{IARH}}}
\newcommand\hide[1]{}
\newenvironment{centermath}
 {\begin{center}$\displaystyle}
 {$\end{center}}
\renewcommand{\note}[2][polish]{{\color{red} #2}{\marginpar{\tiny \color{blue} #1}}}
\renewcommand{\implies}{\Rightarrow}
\renewcommand{\iff}{\Leftrightarrow}

\theoremstyle{plain}
\newtheorem{thm}{Theorem}[section]
\newtheorem{ex}[thm]{Example}
\newtheorem{prop}[thm]{Proposition}
\newtheorem{col}[thm]{Corollary}
\newtheorem{lem}[thm]{Lemma}
\newtheorem*{hypo}{Hypothesis}
\theoremstyle{definition}
\newtheorem{defn}[thm]{Definition}
\newtheorem{rem}[thm]{Remark}

\crefname{thm}{theorem}{theorems}
\crefname{defn}{definition}{definitions}
\crefname{lem}{lemma}{lemmas}
\crefname{col}{corollary}{corollaries}
\crefname{ex}{example}{examples}
\crefname{prop}{proposition}{propositions}

\Crefname{thm}{Theorem}{Theorems}
\Crefname{defn}{Definition}{Definitions}
\Crefname{lem}{Lemma}{Lemmas}
\Crefname{col}{Corollary}{Corollaries}
\Crefname{ex}{Example}{Examples}
\Crefname{prop}{Proposition}{Propositions}

%% DOCUMENT =================================================

\bibliographystyle{plainurl}% the mandatory bibstyle

\title{A Functional Proof Pearl: \protect\\ Inverting the Ackermann Hierarchy}

\titlerunning{Inverting the Ackermann Hierarchy}%optional, please use if title is longer than one line

\author{Linh Tran}{National University of Singapore}{}{}{}%TODO mandatory, please use full name;
\author{Anshuman Mohan}{National University of Singapore}{}{}{}%TODO mandatory, please use full name;
\author{Aquinas Hobor}{National University of Singapore}{}{}{}
%TODO mandatory, please use full name; only 1 author per \author macro; first two parameters are mandatory, other parameters can be empty. Please provide at least the name of the affiliation and the country. The full address is optional

\begin{CCSXML}
<ccs2012>
<concept>
<concept_id>10003752.10003777</concept_id>
<concept_desc>Theory of computation~Computational complexity and cryptography</concept_desc>
<concept_significance>300</concept_significance>
</concept>
<concept>
<concept_id>10003752.10003790.10002990</concept_id>
<concept_desc>Theory of computation~Logic and verification</concept_desc>
<concept_significance>300</concept_significance>
</concept>
<concept>
<concept_id>10003752.10003790.10003796</concept_id>
<concept_desc>Theory of computation~Constructive mathematics</concept_desc>
<concept_significance>300</concept_significance>
</concept>
</ccs2012>
\end{CCSXML}

\ccsdesc[300]{Theory of computation~Computational complexity and cryptography}
\ccsdesc[300]{Theory of computation~Logic and verification}
\ccsdesc[300]{Theory of computation~Constructive mathematics}

\authorrunning{L. Tran, A. Mohan, A. Hobor}

\Copyright{Linh Tran, Anshuman Mohan, and Aquinas Hobor}%TODO mandatory, please use full first names. LIPIcs license is "CC-BY";  http://creativecommons.org/licenses/by/3.0/


\keywords{Ackermann, hyperoperations, Coq}%TODO mandatory; please add comma-separated list of keywords


\begin{document}

\maketitle
%\begin{frontmatter}

%% Title, authors and addresses

%% use the tnoteref command within \title for footnotes;
%% use the tnotetext command for theassociated footnote;
%% use the fnref command within \author or \address for footnotes;
%% use the fntext command for theassociated footnote;
%% use the corref command within \author for corresponding author footnotes;
%% use the cortext command for theassociated footnote;
%% use the ead command for the email address,
%% and the form \ead[url] for the home page:

%% \tnotetext[label1]{}
%% \author{Name\corref{cor1}\fnref{label2}}
%% \ead{email address}
%% \ead[url]{home page}
%% \fntext[label2]{}
%% \cortext[cor1]{}
%% \address{Address\fnref{label3}}
%% \fntext[label3]{}

%% use optional labels to link authors explicitly to addresses:
%% \author[label1,label2]{}
%% \address[label1]{}
%% \address[label2]{}

%\author{}

%\address{}

\begin{abstract}
We implement in Gallina a hierarchy of functions that calculate the upper inverses
to the Ackermann/hyperoperation hierarchy, and then use our inverses
to compute the inverse of the diagonal Ackermann function $\Ack(n)$. 
We show that our computation runs in linear time, and that it is consistent with 
the usual definition of the inverse Ackermann function $\alpha(n)$.
\end{abstract}

%\end{frontmatter}

%% \linenumbers

%% main text
\section{Overview}
\label{sec: overview}
\begin{frame}
\frametitle{The Ackermann Function}
	
	The Ackermann-P\'eter function is defined as:
	\begin{equation*}
	A(n, m) \triangleq \begin{cases}
	m + 1 & \text{ when } n = 0 \\
	A(n-1, 1) & \text{ when } n > 0, m = 0 \\
	A\big(n-1, A(n, m-1)\big) & \text{ otherwise}
	\end{cases}
	\end{equation*}
	
	\pause 
	The \emph{diagonal} Ackermann function is $\Ack(n)~\triangleq~A(n, n)$.
	
	\bigskip
	
	\pause 
	First few values of $\Ack(n)$:
	
  \begin{minipage}{0.5\linewidth}
		\begin{equation*}
	\begin{array}{r|ccccc}
	 n & 0 & 1 & 2 & 3 & 4 \\ \hline
	 \Ack(n) & 1 & 3 & 7 & 61 & 2^{2^{2^{65536}}} - 3 \topspace{3pt}
	\end{array}
	\end{equation*}
  \end{minipage}
  \quad \pause 
  \begin{minipage}{0.4\linewidth}
  	\textcolor{red}{Explosive growth!}
  \end{minipage}

\end{frame}


%\begin{frame}
%\frametitle{Initial values for $\Ack(n)$ and $\alpha(n)$}
%\begin{columns}[T]
%	\begin{column}{0.4\textwidth}
%		
%		\begin{equation*}
%		\begin{array}{|ll|}
%		n & \Ack(n) \\
%		0 & 1 \\
%		1 & 3 \\
%		2 & 7 \\
%		3 & 61 \\
%		4 & 2^{2^{2^{65536}}} - 3
%		\end{array}
%		\end{equation*}
%		
%		Grows astronomically fast!
%	\end{column}
%
%  \begin{column}{0.5\textwidth}
%  	
%  	\begin{equation*}
%  	\begin{array}{|ll|}
%  	n & \alpha(n) \\
%  	0, 1 & 0 \\
%  	2, 3 & 1 \\
%  	4, 5, 6, 7 & 2 \\
%  	8, 9, \ldots, 61 & 3 \\
%  	62, \ldots, 2^{2^{2^{65536}}} - 3 & 4
%  	\end{array}
%  	\end{equation*}
%  	
%  	Grows astronomically slowly!
%  \end{column}
%\end{columns}
%\end{frame}


% \begin{frame}
% \frametitle{Introduction: Growth Patterns}

%  First few values of $\Ack(n)$ and $\alpha(n)$:
		
% %		\begin{equation*}
% %		\begin{array}{|ll|c|ll|}
% %		n & \Ack(n) & \hspace{3em} & n & \alpha(n) \\[3pt]
% %		0 & 1 & & 0, 1 & 0 \\[3pt]
% %		1 & 3 & & 2, 3 & 1 \\[3pt]
% %		2 & 7 & & 4, 5, 6, 7 & 2 \\[3pt]
% %		3 & 61 & & 8, 9, \ldots, 61 & 3 \\[3pt]
% %		4 & 2^{2^{2^{65536}}} - 3 & & 62, 63 \ldots, 2^{2^{2^{65536}}} - 3 & 4 \\[5pt]
% %		\multicolumn{2}{|l|}{\text{Grows astronomically fast!}} & & \multicolumn{2}{|l|}{\text{Grows astronomically slow!}}
% %		\end{array}
% %		\end{equation*}
		
% 		\begin{tikzpicture}
% 		\begin{axis}[
% %		symbolic x coords={a small bar,a medium bar,a large bar},
%     yticklabels={,,},
% 		xtick=data]
% 		\addplot coordinates {
% 			(0, 1)
% 			(1, 3)
% 			(2, 7)
% 			(3, 61)
% 			(4, 9999999)
			
% 		};
% 		\end{axis}
% 		\end{tikzpicture}

% %TODO Fix this graph? Graph for alpha? Figure with caption?

% \end{frame}


\begin{frame}
\frametitle{The Inverse Ackermann Function}

The \emph{inverse Ackermann function}, $\alpha$, maps $n$ to the smallest~$k$ for
which~$n \le \Ack(k)$, \emph{i.e} \impinline{$\alpha(n) \triangleq \min\left\{k\in \mathbb{N} : n \le \Ack(k)\right\}$}.

\smallskip

\pause 
$\alpha(n)$ grows slowly but is hard to \emph{compute} for large $n$
\\ because it is entangled with the explosively-growing $\Ack(k)$.

\bigskip


%\textbf{Naive Approach:} starting at $k=0$, calculate $\Ack(k)$,
%compare it to $n$, \\ and increment $k$ until $n \le \Ack(k)$.
\pause 
\textbf{Naive Approach:} Compute $\Ack(0), \Ack(1), \ldots$ until $n\le \Ack(k)$. Return $k$.

\bigskip

\pause 
\textbf{Time complexity:} $\Omega(\Ack(\alpha(n)))$.
\\ Computing $\alpha(100) \mapsto^{*} 4$ requires at least
%$\Ack(4) = 2^{2^{2^{65536}}}$ steps!
$\Ack(4) = 2^{2^{2^{65536}}} - 3$ steps!

\bigskip

\pause 
\textbf{Engineering hack:} Hardcode with lookup tables. $n > 61 \implies \text{ans} = 4$.

\bigskip

\pause 
\imppar{\text{\textbf{Our Goal.} Compute $\alpha$ for \emph{all inputs} without computing $\Ack$.}}
\end{frame}

%\begin{frame}
%\frametitle{Introduction: Ackermann \emph{vs} Hyperoperations}
%The Ackermann function is easy to define, but hard to
%understand.
%
%We see it as
%a sequence of $n$-indexed functions $\Ack_n \triangleq \lambda b.A(n,b)$, where for each $n>0$, $\Ack_n$ is the result of applying the previous $\Ack_{n-1}$ $b$ times.
%
%%with a
%%\href{https://github.com/inv-ack/inv-ack/blob/7270e64a2600b771f2b1b1b151f7d13fb2ae6c97/repeater.v\#L161-L177}{\emph{kludge}}. %Linked by Linh
%
%\bigskip
%
%The hierarchical structure resembles that of \emph{hyperoperations}.
%
%%To better understand the Ackermann function as a hierarchy and this kludge, we explore the closely-related hyperoperations.
%
%\end{frame}


\begin{frame}[fragile]
\frametitle{Our Solution}

\vspace{-1em}
\lstset{style=myTinyStyle}
% Linked by A
\begin{mdframed}[backgroundcolor=lightgray, roundcorner=10pt,leftmargin=0, rightmargin=0, innerleftmargin=0, innertopmargin=-5,innerbottommargin=-5, outerlinewidth=0, linecolor=lightgray]
\begin{lstlisting}
Require Import Omega Program.Basics.

`\href{https://github.com/inv-ack/inv-ack/blob/7270e64a2600b771f2b1b1b151f7d13fb2ae6c97/inv_ack_standalone.v#L6-L11}{Fixpoint cdn\_wkr}` (a : nat) (f : nat -> nat) (n b : nat) :=
 match b with 0 => 0 | S b' =>
  if (n <=? a) then 0 else S (cdn_wkr f a (f n) k')
 end.

`\href{https://github.com/inv-ack/inv-ack/blob/7270e64a2600b771f2b1b1b151f7d13fb2ae6c97/inv_ack_standalone.v#L14}{Definition countdown\_to}` a f n := cdn_wkr a f n n.

`\href{https://github.com/inv-ack/inv-ack/blob/7270e64a2600b771f2b1b1b151f7d13fb2ae6c97/inv_ack_standalone.v#L32-L38} {Fixpoint inv\_ack\_wkr}` (f : nat -> nat) (n k b : nat) :=
 match b with 0 => 0 | S b' =>
  if (n <=? k) then k else let g := (countdown_to f 1) in
                      inv_ack_wkr (compose g f) (g n) (S k) b
 end.

`\href{https://github.com/inv-ack/inv-ack/blob/7270e64a2600b771f2b1b1b151f7d13fb2ae6c97/inv_ack_standalone.v#L42-L46}{Definition inv\_ack\_linear}` (n : nat) : nat :=
 match n with 0 | 1 => 0 | _ => 
  let f := (fun x => x - 2) in inv_ack_wkr f (f n) 1 (n - 1)
 end.
\end{lstlisting}
\end{mdframed} 
\end{frame}


\begin{frame}
\frametitle{Ackermann \emph{vs} Hyperoperation}

The Ackermann function is easy to define, but hard to
understand.

\bigskip

\pause 	
Let's index by the first argument. \\ \smallskip
Define $\Ack_n \triangleq \lambda b.A(n,b)$. \\ \smallskip
Then, for $n>0$, $\Ack_n$ is the result of applying the previous $\Ack_{n-1}$ $b$ times.

%with a
%\href{https://github.com/inv-ack/inv-ack/blob/7270e64a2600b771f2b1b1b151f7d13fb2ae6c97/repeater.v\#L161-L177}{\emph{kludge}}. %Linked by Linh

\bigskip

\pause 
The hierarchical structure resembles that of \textcolor{red}{\emph{hyperoperations}}.

\smallskip
Studying hyperoperations helps us understand the Ackermann hierarchy.

%To better understand the Ackermann function as a hierarchy and this kludge, we explore the closely-related hyperoperations.
\end{frame}


\begin{frame}
\frametitle{Introduction: Ackermann \emph{vs} Hyperoperation}

Treating $b$ as the main argument, we can build their \emph{upper inverses}:

\begin{table}[t]
	\begin{centermath}
		\begin{array}{c@{\hskip 0.5em}|@{\hskip 1em}c@{\hskip 1em}c@{\hskip 1em}|@{\hskip 1em}c@{\hskip 1em}c}
%			  & \multicolumn{2}{|@{\hskip 0.5em}c@{\hskip 0.5em}|}{\text{Main hierarchies}} & \multicolumn{2}{|@{\hskip 0.5em}c@{\hskip 0.5em}|}{\text{Inverses hierarchies}} \\
			n & a [n] b & \Ack_n(b) & a \angle{n} b & \alpha_n(b)\\
			\hline
			0 & 1 + b & 1 + b & b - 1 & b - 1 \\
			1 & a + b & 2 + b & b - a & b - 2 \\
			2 & a \cdot b & 2b + 3 & \left\lceil \frac{b}{a} \right\rceil & \left\lceil \frac{b-3}{2} \right\rceil \\
			3 & a^b & 2^{b + 3} - 3 & \left\lceil \log_a ~ b \right\rceil & \left\lceil \log_2 ~ (b + 3)\right\rceil - 3 \\
			[1pt]
			4 & \underbrace{a^{.^{.^{.^a}}}}_b & \underbrace{2^{.^{.^{.^2}}}}_{b+3} - 3 & \log^*_a ~ b & \log^*_2 ~ (b + 3) - 3
		\end{array}
	\end{centermath}
	\label{table: hyperop-ack-inv}
\end{table}

\pause
Connection?
\pause
\\ \href{https://github.com/inv-ack/inv-ack/blob/7270e64a2600b771f2b1b1b151f7d13fb2ae6c97/repeater.v\#L161-L177}{Aha!} 
$\Ack_n(b) = {\color{red}2}[n](b{\color{red}+3}) {\color{red}- 3}$
\pause
\quad and \quad $\alpha_n(b) = {\color{red}2}\angle{n}(b{\color{red}+3}) {\color{red}- 3}$.

\end{frame}


\begin{frame}
\frametitle{Introduction: Inverse Hierarchies to Inverse Ackermann}

We explore the upper inverse relation:
\begin{equation*}
\begin{cases}
\forall b. \forall c.\quad b \le \Ack_n(c) & \!\! \iff \ \ \alpha_n(b)\le c \\
\forall b. \forall c. \quad b \le a[n]c & \!\! \iff \ \ a\angle{n}b \le c
\end{cases}
\end{equation*}

\textbf{Redefine $\bm{\alpha}$:}
$\alpha(n) = \min\{k: n\le \Ack_k(k) \} = \min\{k: \alpha_k(n)\le k \}$.

\bigskip

\pause
\textbf{Computing $\bm{\alpha} $ through $\bm{\alpha_i}$!} No need to go through $\Ack$.

\bigskip

\pause
\imppar{\textbf{Goal.} Build the inverse towers independent from the original towers.}

\end{frame}






%\begin{frame}
%\frametitle{}
%\end{frame}

\section{Hyperoperations, Ackermann, and Repeater}
\label{sec: countdown-repeater}
Let us formally define hyperoperations and clarify the intuition
given by Table~\ref{table: hyperop-ack-inv} by relating hyperoperations to
the Ackermann function.
The first hyperoperation (level 0) is simply successor, and
every hyperoperation that follows is the repeated application of the previous.
Level 1 is thus addition, and $b$ repetitions of addition
give level 2, multiplication. Next, $b$ repetitions of
multiplication give level 3, exponentiation.
There is a subtlety here: in the former case, we add $a$
repeatedly to the additive identity $0$, but in the
latter case, we multiply $a$ repeatedly to the multiplicative identity $1$. 
The formal definition of hyperoperation is:
%\marginpar{\tiny \color{blue} Should 2 just be under 1?}
%\begin{equation}
%\label{eq:hyper}
%\begin{array}{lrcl}
%\textit{1. 0$^{\textit{th}}$ level: } & a[0]b & ~ \triangleq ~ & b + 1 \\
%\textit{2. Initial values: } & a[n+1]0 & ~ \triangleq ~ &
%\begin{cases}
%a & \text{when } n = 0 \\
%0 & \text{when } n = 1 \\
%1 & \text{otherwise}
%\end{cases} \\
%\textit{3. Recursive rule: } \quad & a[n+1](b+1) & ~ \triangleq ~ & a[n]\big(a[n+1]b\big)
%\end{array}
%\end{equation}
% Edited by Linh
\begin{equation}
\label{eq:hyper}
\begin{array}{lrcl}
\textit{1. 0$^{\textit{th}}$ level:} & a[0]b & \triangleq & b + 1 \\
\textit{2. Initial values:} & a[n+1]0 & \triangleq &
  \begin{cases}
    a & \text{when } n = 0 \\
    0 & \text{when } n = 1 \\
    1 & \text{otherwise}
  \end{cases} \\
\textit{3. Recursive rule:} & a[n+1](b+1) & \triangleq & a[n]\big(a[n+1]b\big)
\end{array}
\end{equation}
The recursive rule looks complicated, but is actually just \emph{repeated application} in disguise. By fixing~$a$ and treating~$a[n]b$ as a function of~$b$, we can write
% \begin{equation*}
% \begin{array}{lll}
% a[n+1]b & ~ = ~ a[n]\big(a[n+1](b-1)\big) & ~ = ~ a[n]\big(a[n](a[n+1](b-2))\big) \\
%  & ~ = ~ \underbrace{\big( a[n]\circ a[n]\circ \cdots \circ a[n] \big)}_{b \text{ times}} \big(a[n+1]0\big) & ~ = ~ \big(a[n]\big)^{(b)}\big(a[n+1]0\big)
% \end{array}
% \end{equation*}
\begin{equation*}
\begin{array}{lll}
a[n+1]b ~& = ~ a[n]\big(a[n+1](b-1)\big) \\
         & = ~ a[n]\big(a[n](a[n+1](b-2))\big) \\
         & = ~ \underbrace{\big( a[n]\circ a[n]\circ \cdots \circ a[n] \big)}_{b \text{ times}} \big(a[n+1]0\big) \\
         & = ~ \big(a[n]\big)^{(b)}\big(a[n+1]0\big)
\end{array}
\end{equation*}
%where $f^{(k)}(u) ~ \triangleq ~ \overbrace{(f\circ f\circ \cdots \circ f)}^{k \text{ times}} (u)$,
where $f^{(k)}(u) \triangleq ~ (f\circ f\circ \cdots \circ f)(u)$ denotes $k$ compositional applications of a function~$f$ to an
input~$u$; $f^{(0)}(u) = u$ \linebreak (\emph{i.e.} applying $0$ times yields the identity).

This insight helps us encode hyperoperations~(\ref{eq:hyper}) and
Ackermann~(\ref{eq:ackermann}) in Coq.  
Notice that the recursive case of hyperoperations and
the third case of Ackermann both feature deep nested recursion, 
which makes our task tricky. 
In the outer recursive call, the first argument is shrinking
but the second is expanding explosively; in the inner recursive call, the first argument is
constant but the second is shrinking. The elegant solution uses double recursion~\cite{bertotcast} as follows:
% Linked by A
\begin{lstlisting}
`\href{https://github.com/inv-ack/inv-ack/blob/7270e64a2600b771f2b1b1b151f7d13fb2ae6c97/repeater.v#L51-L52}{\color{blue}Definition hyperop\_init}` (a n : nat) : nat :=
  match n with 0 => a | 1 => 0 | _ => 1 end.

`\href{https://github.com/inv-ack/inv-ack/blob/7270e64a2600b771f2b1b1b151f7d13fb2ae6c97/repeater.v#L55-L64}{\color{blue}Fixpoint hyperop\_original}` (a n b : nat) : nat :=
  match n with
  | 0    => 1 + b
  | S n' => let fix hyperop' (b : nat) :=
             match b with
             | 0    => hyperop_init a n'
             | S b' => hyperop_original a n' 
                                  (hyperop' b')
              end in hyperop' b
  end.

`\href{https://github.com/inv-ack/inv-ack/blob/7270e64a2600b771f2b1b1b151f7d13fb2ae6c97/repeater.v#L123-L132}{\color{blue}Fixpoint ackermann\_original}` (m n : nat) : nat :=
  match m with
  | 0    => 1 + n
  | S m' => let fix ackermann' (n : nat) : nat :=
             match n with
             | 0    => ackermann_original m' 1
             | S n' => ackermann_original m' 
                                 (ackermann' n')
             end in ackermann' n
  end.
\end{lstlisting}
Coq is satisfied since both recursive calls are on structurally smaller arguments.
Moreover, our encoding makes the structural similarities
 readily apparent.  In fact, the only essential difference is the initial values
(\emph{i.e.} the second case of both definitions): the Ackermann function uses $\Ack(n-1,1)$, whereas
hyperoperations use the initial values given in~\eqref{eq:hyper}.

We notice that the deep recursion in both cases is expressing the same notion
of repeated application, and this leads us to another useful idea. We can elegantly express the relationship
between the $(n+1)^{\text{th}}$ and $n^{\text{th}}$ levels via a higher-order function that transforms the latter level
to the former using a version of the well-known function iterator
\li{iter}~\cite{bertotcast}:
\begin{defn}
$\forall a\in \mathbb{N}, f: \mathbb{N}\to \mathbb{N}$, the
\href{https://github.com/inv-ack/inv-ack/blob/7270e64a2600b771f2b1b1b151f7d13fb2ae6c97/repeater.v#L32-L36}{\color{blue}\emph{repeater from}}
$a$ of $f$, denoted by $\rf{f}{a}$ , is a function $\mathbb{N}\to \mathbb{N}$ such that $\rf{f}{a}(n) = f^{(n)}(a)$.
% Linked by A
%\begin{lstlisting}
%`\href{https://github.com/inv-ack/inv-ack/blob/7270e64a2600b771f2b1b1b151f7d13fb2ae6c97/repeater.v#L32-L36}{Fixpoint repeater\_from}` (f : nat -> nat) (a n : nat) : nat :=
%  match n with 0 => a | S n' => f (repeater_from f a n') end.
%\end{lstlisting}
\begin{lstlisting}
`\href{https://github.com/inv-ack/inv-ack/blob/7270e64a2600b771f2b1b1b151f7d13fb2ae6c97/repeater.v#L32-L36}{\color{blue}Fixpoint repeater\_from}` f a n :=
  match n with
  | 0 => a
  | S n' => f (repeater_from f a n') 
  end.
\end{lstlisting}
\end{defn}
\noindent This allows simple and function-oriented definitions of hyperoperations and the
Ackermann function that we give below. Note that the Curried $a[n-1]$ denotes
the single-variable function $\lambda b.a[n-1]b$.
%\begin{equation*}
%\vspace{-0.75em}
%a[n]b ~ \triangleq ~ \begin{cases}
%b + 1 & \text{when } n = 0 \\
%\rf{a[n-1]}{a_{n-1}}(b) & \text{otherwise}
%\end{cases}
%\quad \text{ where } \ a_n ~ \triangleq ~ \begin{cases}
%a & \text{when } n = 0 \\
%0 & \text{when } n = 1 \\
%1 & \text{otherwise}
%\end{cases}
%\end{equation*}
% Edited by Linh
\begin{equation*}
a[n]b ~ \triangleq ~ \begin{cases}
b + 1 & \text{when } n = 0 \\
\rf{a[n-1]}{a_{n-1}}(b) & \text{otherwise}
\end{cases}
\end{equation*}
\begin{equation*}
\hspace{8em}\text{where  } a_n ~ \triangleq ~ \begin{cases}
a & \text{when } n = 0 \\
0 & \text{when } n = 1 \\
1 & \text{otherwise}
\end{cases}
\end{equation*}
% Linked by Anshuman
%\begin{lstlisting} 
%`\href{https://github.com/inv-ack/inv-ack/blob/7270e64a2600b771f2b1b1b151f7d13fb2ae6c97/repeater.v#L67-L71}{Fixpoint hyperop}` (a n b : nat) : nat :=
%  match n with
%  | 0    => 1 + b
%  | S n' => repeater_from (hyperop a n') (hyperop_init a n') b
%  end.
%\end{lstlisting}
% Edited by Linh
\begin{lstlisting} 
`\href{https://github.com/inv-ack/inv-ack/blob/7270e64a2600b771f2b1b1b151f7d13fb2ae6c97/repeater.v#L67-L71}{\color{blue}Fixpoint hyperop}` (a n b : nat) : nat :=
  match n with
  | 0    => 1 + b
  | S n' => repeater_from
              (hyperop a n') (hyperop_init a n') b
  end.
\end{lstlisting}

\pagebreak
\begin{equation*}
%\begin{array}{l}
A(n,m) ~ \triangleq ~ \begin{cases}
m + 1 & \text{when } n = 0 \\
\rf{\mathcal{A}_{n-1}}{A(n-1,1)\ }(m) & \text{otherwise}
\end{cases} \qquad \qquad \qquad \qquad \qquad \qquad \qquad ~ 
%\end{array}
\end{equation*}
% Linked by A
%\begin{lstlisting}
%`\href{https://github.com/inv-ack/inv-ack/blob/7270e64a2600b771f2b1b1b151f7d13fb2ae6c97/repeater.v#L135-L139}{Fixpoint ackermann}` (n m : nat) : nat :=
%  match n with
%  | 0    => S m
%  | S n' => repeater_from (ackermann n') (ackermann n' 1) m
%  end.
%\end{lstlisting}
% Edited by Linh
\begin{lstlisting}
`\href{https://github.com/inv-ack/inv-ack/blob/7270e64a2600b771f2b1b1b151f7d13fb2ae6c97/repeater.v#L135-L139}{\color{blue}Fixpoint ackermann}` (n m : nat) : nat :=
  match n with
  | 0    => S m
  | S n' => repeater_from
              (ackermann n') (ackermann n' 1) m
  end.
\end{lstlisting}
In the rest of this paper we construct efficient inverses to these
functions.  Our key idea is an inverse to the higher-order \emph{repeater} function, which we call \emph{countdown}.

% Aquinas' promise:
% We explain our core techniques of repeaters and countdowns that allow us to define each level of the Ackermann hierarchy—and their upper inverses—in a straightforward and uniform manner. We show how countdowns, in particular, can be written structurally recursively.

% Anshuman proposes:
% We introduce our core techniques of repeaters and countdowns.
% We show how countdowns, in particular, can be written structurally recursively.









\section{Inverses and Countdown}
\label{sec:countdown}
Many functions on $\mathbb{R}$ are bijections and thus have an intuituve inverse.  
Functions on~$\mathbb{N}$ are often non-bijections and so their inverses
do not come as naturally.

\subsection{Upper Inverses, Expansions, and Repeatability}
%; indeed, simply defining an inverse can be a little subtle.
\begin{defn} \label{defn: inverse}
The \emph{upper inverse} of $F$, written $F^{-1}_{+}$, 
% {\color{red} $F^{-1}$, $F^{-1}_{\mathit{\shortuparrow}}$, $F^{-1}_{\upharpoonleft}$} 
is $\min\{m : F(m)\ge n\}$.\end{defn}
Notice that this is well-defined as long as $F$ is unbounded, \emph{i.e.} $\forall b.~\exists a.~ b \leq F(a)$.  However, as a notion of ``inverse,'' it really only makes sense if $F$ is strictly
increasing, \emph{i.e.} $\forall n,m.~n < m \Rightarrow F(n) < F(m)$, which is, in some sense, the analogue of injectivity in the discrete domain.

We call this function the ``upper inverse'' because, for strictly increasing functions like
addition, multiplication, and exponentiation, the upper inverse is the ceiling of the 
corresponding inverse functions on $\mathbb{R}$. We can characterize inverses more meaningfully as follows: 
\hide{
\color{red} It is reasonable to wonder about the floor.
\begin{defn} \label{defn: lower_inverse}
The converse \emph{lower inverse}, written $F^{-1}_{-}$,
is defined as $\max\{m : F(m)\le n\}$.
\end{defn}
Even if $F$ strictly increases, as $a[n]$ does for every $a\ge 2$, notice that the lower inverse will be undefined for $n < F(0)$, \emph{e.g.} $\{m : a[n]m \le 0 \} = \varnothing$ for $n\ge 3$.
Thus we focus on upper inverses (hereafter just ``inverses''), and discuss lower inverses in \cref{sec: discussion}. 
}%end hide 
\begin{thm} \label{thm: upp-inverse-rel}
	If $F:\mathbb{N}\to \mathbb{N}$ is increasing, then $f$ is the upper inverse of $F$ if and only if $\ \forall n, m.~ f(n)\le m \iff n \le F(m)$.
\end{thm}
\begin{proof}
Fix $n$. The sentence $\forall m.~ f(n)\le m \iff n\le F(m)$ implies: (1) $f(n)$ is a lower bound to $\{m: n \le F(m)\}$ and (2) $f(n)$ is itself in the set, since plugging in $m \triangleq f(n)$ yields $n\le F(f(n))$, which makes $f$ the upper inverse of $F$. Conversely, if $f$ is the upper inverse of $F$, we know $\forall m.~n\le F(m)\Rightarrow f(n)\le m$. Now, $\forall m \ge f(n)$.~$F(m)\ge F(f(n)) \ge n$ by increasing-ness, thus completing the proof.
\end{proof}
\begin{col}
If $F$ is strictly increasing, then $F^{-1}_{+} \circ F$ is
the identity function.
\end{col}
\begin{proof}
By ($\Leftarrow$) of Theorem~\ref{thm: upp-inverse-rel}, $F(n) \le F(n)$ implies 
$(F^{-1}_{+} \circ F)(n) \le n$.  By ($\Rightarrow$) of the same theorem, $(F^{-1}_{+} \circ F)(n) \le (F^{-1}_{+} \circ F)(n)$ implies $F(n) \le F \big((F^{-1}_{+} \circ F)(n)\big)$. $F$ is strictly increasing, so $n \le (F^{-1}_{+} \circ F)(n)$.
\end{proof}

Our setup for inverse requires increasing functions, and our definitions for 
hyperoperations/Ackermann use repeater.  Suppose $F$ is a strictly increasing function.
For a given $a$, is $\rf{F}{a}$ strictly increasing?  No!  For example, the identity function
$\li{id}$ is strictly increasing, but $\rf{\li{id}}{a}(n) = (\li{id} \circ \ldots \circ \li{id}) (a) = a$ is a constant function.  We need a little more.
\begin{defn}
A function $F:\mathbb{N}\to\mathbb{N}$ is an \emph{expansion} if $\forall n.~ F(n)\ge n$. Further, given $a\in \mathbb{N}$, an expansion $F$ is \emph{strict from} $a$ if \note{$\forall n \ge a.~ F(n)\ge n+1$}.
\end{defn}
If $a\ge 1$ and $F$ is an expansion \emph{strict from} $a$, we quickly get: 
$\forall n.~ \rf{F}{a}(n) = F^{(n)}(a) \ge a + n \ge 1 + n$. That is, $\rf{F}{a} \ $ is itself an expansion strict from $0$. 

\begin{defn} \label{rem: repeatable-subset}
$\forall a\ge 1$, if~$F$ is a strictly increasing function 
that is also a strict expansion from~$a$, then $\rf{F}{a}$ preserves 
repeatability.
We thus say that such a function $F$ is \emph{repeatable from} $a$, and denote the 
set of functions repeatable from $a$ as $\repeatable_a$. 
It is straightforward to see that $\forall s,t.~ s \le t \Rightarrow \repeatable_s \subseteq \repeatable_t $.
\end{defn}

\subsection{Contractions and the countdown operation}

Suppose $F \in \repeatable_a$ for some $a \ge 1$, and 
let $f \triangleq F^{-1}_{+}$, \emph{i.e.} the inverse of $F$.
Our goal is to use $f$ to compute an inverse to $\rf{F}{a}$.  
From the preceeding discussion we know that this inverse must exist, 
since $F \in \repeatable_a$ implies $\rf{F}{a} \in \repeatable_0$.  
For reasons that will be clear momentarily, we write this inverse as $\cdt{f}{a}$.  Now
fix $n$ and observe that $\forall m.~f^{(m)}(n)\le a \iff m \ge \cdt{f}{a}(n)$ since
\begin{equation} \label{eq: rf-upp-inv}
\begin{aligned}
\cdt{f}{a}(n)\le m & \iff n\le \rf{F}{a}(m) = F^{(m)}(a) \iff f(n)\le F^{(m-1)}(a) \\
& \iff f^{(2)}(n)\le F^{(m-2)}(a) \iff \ldots \iff f^{(m)}(n)\le a
\end{aligned}
\end{equation}
Moreover, setting $m = \cdt{f}{a}(n)$, we realize that 
$f^{(\cdt{f}{a}(n))}(n) \le a$.  
\textbf{Together these imply that $\cdt{f}{a}(n)$ is the minimum number of 
times~$f$ needs to be compositionally applied to $n$ before equalling or 
passing $a$.} 
In other words, count the length of the chain $\{n, f(n), f^{(2)}(n), \ldots\}$ that 
terminates as soon as we reach/pass $a$.  For this process to work, 
we need each chain link
to be strictly less than the previous, \emph{i.e.} $f$ must be a \emph{contraction}.
\begin{defn} \label{defn: contracting}
	A function $f : \mathbb{N} \to \mathbb{N}$ is a \emph{contraction} if $\forall n.~ f(n) \le n$. Given an $a \ge 1$, a contraction $f$ is 
	\emph{strict above} $a$ if $\forall n > a.~n > f(n)$. We denote the set of contractions by $\contract$ and the set of contractions strict above $a$ by $\contract_a$. Analogously to our observation in 
	Definition \ref{rem: repeatable-subset}, $\forall s\le t.~ \contract_s \subseteq \contract_t$.
\end{defn}
What kinds of functions have contractive inverses? Expansions, naturally:
\begin{thm} \label{thm: expansion-inv-contraction}
$\forall a\in \mathbb{N}.~F\in \repeatable_a \Rightarrow F^{-1}_+ \in \contract_a$.
\end{thm}
\begin{proof}
$\forall n.~F(n)\ge n \Rightarrow n \ge F^{-1}_+(n)$, so $F^{-1}_+ \in
\contract$. Note, $n > a \iff n-1\ge a$.
Since $F\in \repeatable_a$, $F(n-1)\ge n$ holds. 
Next, $n-1\ge F^{-1}_+(n) \Rightarrow n > F^{-1}_+(n)$. 
\end{proof}
This clarifies the inverse relationship between expansions strict from some $a$ and contractions strict above that same $a$. The inverse of an expansion's 
repeater exists and can be built \note{from its own inverse}, in a method formalized as \emph{countdown}.
\begin{defn} \label{defn: informal-countdown} \label{eq: countdown}
Let $f\in \contract_a$. The \textit{countdown to} $a$ of $f$, written 
$\cdt{f}{a}(n)$, is the smallest number of times $f$ needs to be applied to 
$n$ for the answer to equal or go below $a$. \emph{i.e.}, 
$\cdt{f}{a}(n) \triangleq \min\{m: f^{(m)}(n)\le a \}$.
\end{defn}
Inspired by Equation~\ref{eq: rf-upp-inv}, we provide a neat, algebraically manipulable logical sentence equivalent to Definition~\ref{eq: countdown}, which is more useful later in our paper:
\begin{col} \label{col: cdt-repeat}
If $a \in \mathbb{N}$ and $f\in \contract_{a}$, then $\forall n, m.~ \cdt{f}{a}(n)\le m \iff f^{(m)}(n)\le a$.
\end{col}
\begin{proof}
	Fix $a$ and $n$. The interesting direction is $(\Rightarrow)$. Suppose $\cdt{f}{a}(n)\le m$, we get $f^{(m)}(n)\le f^{(\cdt{f}{a}(n))}(n)$ due to $f\in \contract$, and $f^{(\cdt{f}{a}(n))}(n)\le a$ due to Definition~\ref{eq: countdown}.
\end{proof}
%\begin{thm} \label{thm: upp-inv-cdt-rf}
%	For all $a\in \mathbb{N}$, if $f\in \contract_{a}$ is the upper inverse of $F: \mathbb{N}\to \mathbb{N}$, then $\cdt{f}{a}$ is the upper inverse of $\rf{F}{a}$.
%\end{thm}
Another useful result is the recursive formula for \emph{countdown}:
\begin{thm} \label{thm: cdt-recursion}
	For all $a\in \mathbb{N}$ and $f\in \contract_{a}$, $\cdt{f}{a}$ satisfies:
	\begin{equation*}
	\cdt{f}{a}(n) = \begin{cases}
	0 & \text{ if } n \le a \\ 1 + \cdt{f}{a}(f(n)) & \text{ if } n \ge a + 1
	\end{cases}
	\end{equation*}
\end{thm}
\begin{proof}
When $n \le a$, use Corollary~\ref{col: cdt-repeat} as follows: 
$n = f^{(0)}(n)\le a \iff \cdt{f}{a}(n)\le 0$. 
When $n\ge a+1$, proceed by antisymmetry. 
Define $m \triangleq \cdt{f}{a}(f(n))$, and note that 
Corollary~\ref{col: cdt-repeat} gives 
$\cdt{f}{a}(n)\le 1+m \iff f^{(1+m)}(n) \le a$.
Next, a simple expansion of the last clause gives $f^{(m)}(f(n)) \le a$,
and unfolding the definition of~$m$ shows that this last clause is true.
Now since $n\ge a+1$, we have $\cdt{f}{a}(n)\ge 1$ by the above. 
Define $p \triangleq \cdt{f}{a}(n) - 1$. 
It remains to show $\cdt{f}{a}(f(n))\le p \Rightarrow f^{(p)}(f(n))\le a \Rightarrow f^{(p+1)}(n)\le a$, which holds by unfolding $p$'s definition.
\end{proof}

%\begin{thm} \label{thm: cdt-contr-0}
%	For all $a\ge 1$, if $f\in \contract_{a}$, then $\cdt{f}{a}\ \in \contract_{c} \ \forall c$.
%\end{thm}
%\begin{proof}
%	Firstly we show that $\cdt{f}{a}$ is a contraction, namely showing $\cdt{f}{a}(n)\le n \ \forall n$, which has already been proved in the proof of \cref{thm: cdt-repeat}. To show $\cdt{f}{a}$ is strict from $1+c$, it suffices to show it is strict from $1$, equivalently $\cdt{f}{a}(n) \le n - 1 \ \forall n\ge 1$. By \cref{thm: cdt-repeat}, we need to show $f^{(n-1)}(n)\le a$. Assume the converse, then:
%	$$ n \ge 1 + f(n)\ge 2 + f(f(n)) \ge \cdots \ge (n-1) + f^{(n-1)}(n) $$
%	Thus $f^{(n-1)}(n)\le 1 \le a$, a contradiction. The theorem then follows.
%\end{proof}

\subsection{A Structurally Recursive Computation for Countdown}

The higher-order repeater function is well-defined for any input functions, 
even those not in $\repeatable_a$ (although for such functions it may not 
be useful), and so is easy to define
in Coq as shown in \S\ref{sec: countdown-repeater}. In contrast, a 
\emph{countdown} only exists for certain functions, most conveniently 
contractions, which makes it a little harder to encode into Coq. 
Our strategy is to define a \emph{countdown worker} which is written with only
structural recursion and is thus more palatable to Coq, and then 
prove that this worker computes the countdown when passed a contraction.

\begin{defn} \label{defn: countdown-worker}
For any $a\in \mathbb{N}$ and $f: \mathbb{N}\to \mathbb{N}$, the \emph{countdown worker} to $a$ of $f$ is a function $\cdw{f}{a}\ : \mathbb{N}^2\to \mathbb{N}$ such that:
\begin{equation*}
\cdw{f}{a}(n, b) = \begin{cases}
0 & \text{if } b = 0 \vee n\le a \\ 1 +\cdw{f}{a}(f(n), b-1) & \text{if } b \ge 1 \wedge n > a
\end{cases}
\end{equation*}
\end{defn}
The \emph{countdown worker} operates on two arguments: 
the \emph{true argument} $n$, for which we want to simulate 
\emph{countdown to} $a$, 
and the \emph{budget} $b$, the maximum number of times we shall attempt 
to compositionally apply $f$ on the input before giving up. 
If the input goes below or equal $a$ after $k$ applications, \emph{i.e.} $f^{(k)}(n) \le a$, we return the count $k$. If the budget is exhausted (\emph{i.e.} $b = 0$) while the result is still above $a$, we fail by returning the original budget. This definition is admittedly inelegant, but it can clearly be written as a Coq \li{Fixpoint}:
\begin{lstlisting}
`\href{https://github.com/inv-ack/inv-ack/blob/6099297c6ab0e16d14b037fb5ed600c4d22818f6/countdown.v#L94-L100}{Fixpoint countdown\_worker}` (a : nat) (f : nat -> nat) (n k : nat) : nat :=
match k with
 | 0    => 0
 | S k' => match (n - a) with
           | 0 => 0
           | _ => S (countdown_worker a f (f n) k') 
           end
 end.
\end{lstlisting}
Given an $f$ that is a contraction strict from $a$, 
and given a sufficient budget, \emph{countdown worker} 
will compute the correct \emph{countdown} value.  
Careful readers may have noticed that \emph{budget} is similar to 
\emph{gas}, which we discussed in ~\S\ref{sec:incfuncinv} 
and dismissed for potentially 
being too computationally expensive to calculate. 
Budget is actually a refinement of gas because 
we support its use with a method to calculate it in constant time.
In fact, we will soon show that a budget of $n$ is sufficient. 
This lets us define \emph{countdown} in Coq for the first time:
\begin{defn} \label{defn: countdown}
Redefine $\cdt{f}{a}(n) \triangleq \cdw{f}{a}(n, n)$.
\begin{lstlisting}
`\href{https://github.com/inv-ack/inv-ack/blob/6099297c6ab0e16d14b037fb5ed600c4d22818f6/countdown.v#L104}{Definition countdown\_to}` a f n := countdown_worker a f n n.
\end{lstlisting}
\end{defn}
%Before beginning, let us clarify that the definition of $\mathbb{N}$ and operations on $\mathbb{N}$ in Gallina follow the Presburger Arithmetic \cite{presburger}, which despite being weaker than Peano Arithmetic, is a decidable theory. The Coq standard library includes \texttt{Omega}, an extensive listing of provable facts about $\mathbb{N}$ in Presburger Arithmetic, including everything used in this paper, most notably the law of excluded middle for comparisons on $\mathbb{N}$:
%\begin{equation*}
%(n \le m) \vee (m + 1 \le n) \ \ \forall n, m
%\end{equation*}
%, which is provable without the actual law of excluded middle in classical logic. This enables us to prove all results in this paper with Coq's baseline intuitionistic logic. Readers can refer to the \Cref{appendix} for Coq versions of the proofs.
%in which the most operations agree with usual operations on $\mathbb{Z}$, except subtraction, which is defined as:
%\begin{lstlisting}
%Fixpoint sub (n m : nat) : nat :=
%match n with
%| 0 => n
%| S k => match m with
%| 0 => n
%| S l => sub k l end
%end.
%\end{lstlisting}
%Essentially $\li{sub} \ n \ m = \max\{n - m, 0\}$. We will use this subtraction for the rest of the paper.

%Firstly, we begin with a lemma asserting the existence of the countdown value itself. Although its existence is guaranteed by the well-ordering principle of $\mathbb{N}$, we will achieve better by proving it in intuitionistic logic.
%
%\begin{lem} \label{lem: contract-repeat-threshold}
%	For all $a, n\in\mathbb{N}$ and $f\in \contract_{a}$,
%	\begin{equation}
%	\exists m : \left(f^{(m)}(n) \le a \right) \wedge \left(f^{(l)}(n)\le a \Rightarrow m \le l \ \forall l \right)
%	\end{equation}
%\end{lem}
%\begin{proof}
%  Fix $n$ and observe that if $n\le a$, $m = 0$ is the desired choice since $ f^{(0)}(n) = n \le a \ \text{ and } \ 0 \le l \ \forall l $.
%	Consider only when $a\le n$, we can define $c$ such that $n = a + c$. We prove the following statement by induction:
%	\begin{equation*}
%	P(c) \triangleq \exists m : \left(f^{(m)}(n) \le n - c \right) \wedge \left(f^{(l)}(n)\le n - c \Rightarrow m \le l \ \forall l \right)
%	\end{equation*}
%	under assumptions $f\in \contract_{n-c}$ and $c\le n$.
%	\begin{enumerate}[leftmargin=*]
%		\item \textit{Base case.} The case $c = 0$ implies $n = a$, which has been proven above.
%		\item \textit{Inductive step.} Suppose $P(c)$ is proved with witness $m_c$. Note that the assumptions are now $f\in \contract_{n-c}$ and $c+1\le n$, there are two cases:
%		\begin{itemize}[leftmargin=*, label={--}]
%			\item $f^{\left(m_c\right)}(n) = n - c$. Then $f^{\left(m_c+1\right)}(n) \le n - c - 1$. Let $m_{c+1} = m_c + 1$, for all $l$:
%			\begin{equation*}
%			f^{(l)}(n)\le n - c - 1 < f^{\left(m_c\right)}(n) \Rightarrow l > m_c \Rightarrow l \ge m_{c+1}
%			\end{equation*}
%			\item $f^{\left(m_c\right)}(n) \le n - c - 1$. Let $m_{c+1} = m_c$, for all $l$:
%			\begin{equation*}
%			f^{(l)}(n)\le n - c - 1\le n - c \overset{P(c)}{\Rightarrow} l\ge m_{c} = m_{c+1}
%			\end{equation*}
%		\end{itemize}
%		In any cases, we can find a witness $m_{c+1}$ for $P(c+1)$. Thus the proof is complete by induction.\vspace*{-\baselineskip}
%	\end{enumerate}
%\end{proof}

\begin{lem} \label{lem: cdt-init}
	$\forall n\le a.~\cdw{f}{a} (n, b) = 0$.
\end{lem}
\begin{proof}
This simple fact follows directly from Definition \ref{defn: countdown-worker}.
\end{proof}
Next, we show the internal working of \emph{countdown worker} at the $\text{i}^\text{th}$ recursive step, including the accumulated result $1+i$, the current input $f^{(1+i)}(n)$, and the current budget $b-i-1$.
\begin{lem} \label{lem: cdt-intermediate}
	$\forall a, n, b, i \in \mathbb{N}.~\forall f \in \contract$. such that $i < b$ and $a < f^{(i)}(n)$:~
	\begin{equation}  \label{eq: cdt-intermediate}
	\cdw{f}{a}(n, b) = 1 + i + \cdw{f}{a}\left(f^{1+i}(n), b - i - 1\right)
	\end{equation}
\end{lem}
\begin{proof}
	Fix $a$. We proceed by induction on $i$. Then define $P(i)$ as
	\begin{equation*}
	P(i) \triangleq \forall b, n.~ b\ge i+1 \Rightarrow f^{(i)}(n) > a \Rightarrow \cdw{f}{a}(n, b) = 1 + i + \cdw{f}{a}\left(f^{1+i}(n), b - i - 1\right)
	\end{equation*}
	\begin{enumerate}[leftmargin=*]
		\item \textit{Base case.} For $i = 0$, our goal $P(0)$ is:
		$\cdw{f}{a}(n, b) = 1 + \cdw{f}{a}\left(f(n), b - 1\right)$
		where $b \ge 1$ and $f(n)\ge a+1$. This is straightforward.
		\item \textit{Inductive step.} Assume $P(i)$ has been proved. Then $P(i+1)$ is
		\begin{equation*}
		\cdw{f}{a}(n, b) = 2 + i + \cdw{f}{a}\left( f^{2+i}(n), b - i - 2 \right)
		\end{equation*}
		where $b \ge i+2$ and $f^{1+i}(n) \ge a+1$. This also implies $b\ge i+1$ and $\displaystyle f^{(i)}(n) \ge f^{1+i}(n)\ge a+1$ by $f\in \contract$, and so $P(i)$ holds. It suffices to prove:
		\begin{equation*}
		\cdw{f}{a}\left(f^{1+i}(n), b-i-1\right) = 1 + \cdw{f}{a}\left( f^{2+i}(n), b-i-2 \right)
		\end{equation*}
		This is in fact $P(0)$ with $(b, n)$ substituted for $\left(b-i-1, f^{(1+i)}(n)\right)$. Since $f^{(1+i)}(n) \ge a+1$ and $b-i-1\ge 1$, the above holds and $P(i+1)$ follows.
	\end{enumerate}
\end{proof}
Now it is time to prove the correctness of \emph{countdown}, \emph{i.e.} that our computational Definition~\ref{defn: countdown} computes the value originally specified in Definition~\ref{defn: informal-countdown}.
\begin{thm} \label{thm: cdt-repeat}
	$\forall a \in \mathbb{N}.~ \forall f\in \contract_{a}$, we have 
%	\begin{equation} \label{eq: cdt-minimum}
$	\forall n.~ \cdt{f}{a}(n) = \min\left\{ i : f^{(i)}(n) \le a \right\} $.
%	\end{equation}
%	Or equivalently,
%	\begin{equation} \label{eq: cdt-repeat}
%	\cdt{f}{a}(n) \le k \iff f^{(k)}(n) \le a \ \ \forall n, k
%	\end{equation}
\end{thm}
\begin{proof}
%	First, to see why \eqref{eq: cdt-minimum} and \eqref{eq: cdt-repeat} are equivalent, we rewrite \eqref{eq: cdt-minimum} in the following way:
%	$$ \left(f^{(\cdt{f}{a}(n))}(n) \le a\right) \wedge \left(f^{(l)}(n) \le a \Rightarrow \cdt{f}{a}(n) \le l \ \ \forall l\right) \ \ \forall n$$
%	To prove $\eqref{eq: cdt-minimum} \Rightarrow \eqref{eq: cdt-repeat}$, it suffices to show
%	$$ \cdt{f}{a}(n) \le l \Rightarrow f^{(l)}(n) \le a \ \ \forall l $$
%	, which holds due to the fact $\displaystyle f^{(l)}(n) \le f^{(\cdt{f}{a}(n))}(n) \le a$ by $f\in \contract$. To prove $\eqref{eq: cdt-repeat}\Rightarrow \eqref{eq: cdt-minimum}$, it suffices to show $\displaystyle f^{(\cdt{f}{a}(n))}(n) \le a$
%	, which in turn holds by substituting $k$ by $\cdt{f}{a}(n)$ in \eqref{eq: cdt-repeat}. Thus $\eqref{eq: cdt-minimum}\iff \eqref{eq: cdt-repeat}$ and we need only to prove \eqref{eq: cdt-minimum}.
Since $f\in \contract_{a}$ and $\mathbb{N}$ is well-ordered, 
let $m = \min\big\{i : f^{(i)}(n)\le a\big\}$.\footnote{We prove the existence 
of the min in Coq’s intuitionistic logic \href{https://github.com/inv-ack/inv-ack/blob/6099297c6ab0e16d14b037fb5ed600c4d22818f6/countdown.v\#L125-L150}{here} in our codebase.} Our goal becomes 
$\cdt{f}{a}(n) = m$. Note that this setup gives us: 
	\begin{equation}
	\left(f^{(m)}(n) \le a\right) \label{eq: cdt-repeat-tmp} \wedge
	 \left(\forall k.~f^{(k)}(n)\le a \Rightarrow m \le k\right)
	\end{equation}
If $m = 0$, then $n = f^{(0)}(n)\le a$. So $\cdt{f}{a}(n) = \cdw{f}{a}(n, n) = 0 = m$ by Lemma~\ref{lem: cdt-init}.
When $m > 0$, our plan is to apply Lemma~\ref{lem: cdt-intermediate} to get
	\begin{equation*}
	\cdt{f}{a}(n) = \cdw{f}{a}(n, n) = m + \cdw{f}{a}\left(f^{(m)}(n), n-m\right),
	\end{equation*}
and then use Lemma~\ref{lem: cdt-init} over (\ref{eq: cdt-repeat-tmp})'s first conjunct to conclude $\cdt{f}{a}(n) = m$. It suffices to prove the premises of Lemma~\ref{lem: cdt-intermediate}: $a < f^{(m-1)}(n)$ and $m-1 < n$.
	
	The former follows by contradiction: if $f^{(m-1)}(n) \le a$, (\ref{eq: cdt-repeat-tmp})'s second conjunct implies $m\le m-1$, which is impossible for $m > 0$. The latter then easily follows by $f\in \contract_{a}$, since
	$n \ge 1 + f(n) \ge 2 + f(f(n)) \ge \cdots \ge m + f^{(m)}(n)$.
	\linebreak Therefore, $\cdt{f}{a}(n) = m$ in all cases, which completes the proof.
\end{proof}
Theorem~\ref{thm: cdt-repeat} and (\ref{eq: rf-upp-inv}) establish the correctness of the Coq definitions of \emph{countdown worker} and \emph{countdown}, thereby justifying our budget of $n$ and our unification of 
Definitions \ref{defn: informal-countdown} and \ref{defn: countdown}. We wrap everything together with the following theorem:
\begin{thm} \label{thm: cdt-inv-rf}
	$\forall F\in \repeatable_a.~f\triangleq F^{-1}_+$ satisfies $f \in \contract_{a}$ and $\displaystyle \cdt{f}{a} \ = \left(\rf{F}{a}\ \right)^{-1}_+$. Furthermore, if $a\ge 1$, then $\rf{F}{a}\ \in \repeatable_0$ and $\cdt{f}{a}\ \in \contract_0$.
\end{thm}
\begin{proof}
	By Theorem~\ref{thm: expansion-inv-contraction}, $f\triangleq F^{-1}_+ \in \contract_a$. 
	Equation~\ref{eq: rf-upp-inv} and Corollary~\ref{col: cdt-repeat}
	then show $\cdt{f}{a}\ = \left(\rf{F}{a}\ \right)^{-1}_+$.
	Now if $a\ge 1$, a simple induction shows that $F^{(n)}(a)\ge a + n\ge 1 + n$, so $\rf{F}{a}\ \in \repeatable_0$, hence $\cdt{f}{a} \ = \left(\rf{F}{a}\ \right)^{-1}_+ \in \contract_0$ by Theorem~\ref{thm: expansion-inv-contraction}.
\end{proof}



\section{Inverting the Hyperoperations and the Ackermann Function}
\label{sec: inv-hyperop}
We now use \emph{countdown} to define the inverse
hyperoperation hierarchy, which features elegant new definitions of
division, $\log$, and $\log^{*}$.
We then modify this technique to arrive at the inverse
Ackermann hierarchy.

\subsection{Inverse hyperoperations, including \li{div}, \li{log}, \li{log*}}

\begin{defn} \label{defn: inv-hyperop}
	The inverse hyperoperations, written $a\angle{n}b$, are defined as:
%	\begin{equation}
%	a\angle{n}b \triangleq \begin{cases}
%	b - 1 & \text{if } n = 0 \\
%	\cdt{a\angle{n-1}}{a_n}(b) & \text{if } n \ge 1
%	\end{cases}
%	\ \ \text{ where } \ a_n = \begin{cases}
%	a & \hspace{-10pt}\text{ if } n = 1 \\
%	0 & \hspace{-10pt}\text{ if } n = 2 \\
%	1 & \hspace{-10pt}\text{ if } n \ge 3
%	\end{cases}
%	\end{equation}
% Edited by Linh
	\begin{equation}
  a\angle{n}b \triangleq \begin{cases}
  b - 1 & \hspace{-5pt}\text{if } n = 0 \\
  \cdt{a\angle{n-1}}{a_n}(b) & \hspace{-5pt}\text{if } n \ge 1
  \end{cases}
   \ \text{ where } \ a_n = \begin{cases}
   a & \hspace{-10pt}\text{ if } n = 1 \\
   0 & \hspace{-10pt}\text{ if } n = 2 \\
  1 & \hspace{-10pt}\text{ if } n \ge 3
  \end{cases}
\end{equation}
% linked by A
%\begin{lstlisting}
%`\href{https://github.com/inv-ack/inv-ack/blob/7270e64a2600b771f2b1b1b151f7d13fb2ae6c97/applications.v#L28-L33}{Fixpoint inv\_hyperop}` (a n b : nat) : nat :=
%  match n with 0 => b - 1 | S n' =>
%    countdown_to (hyperop_init a n') (inv_hyperop a n') b
%  end.
%\end{lstlisting}
% Edited by Linh
\begin{lstlisting}
`\href{https://github.com/inv-ack/inv-ack/blob/7270e64a2600b771f2b1b1b151f7d13fb2ae6c97/applications.v#L28-L33}{\color{blue}Fixpoint inv\_hyperop}` (a n b : nat) : nat :=
  match n with 0 => b - 1 | S n' =>
    countdown_to
      (hyperop_init a n') (inv_hyperop a n') b
  end.
\end{lstlisting}
\end{defn}
\hide{
Continuing with the notion of upper-inverses we have used thus far,
we aim for the \emph{ceiling} of division and $\log$ on real numbers, $\left\lceil b/a \right\rceil$ and $\left\lceil \log_ab \right\rceil$. The iterated logarithm is formally defined as the minimum number of times $\log$ needs to be iteratively
applied to the input for the result to become less than or equal to $1$.
Thus it is natural to define it using \emph{countdown}.

\begin{defn} \label{defn: divc} \label{defn: logc} \label{defn: log*}
\[
\divc(a, b) \! \triangleq \! \cdt{\left(\lambda b. (b{-}a)\right)}{0}(b) \quad
\logc(a, b) \! \triangleq \! \cdt{\left(\lambda b. \divc(a, b)\right)}{1}(b) \quad
\log^*(a, b) \! \triangleq \! \cdt{\left(\lambda b. \logc(a, b)\right)}{1}(b)
%\begin{array}{c}
%\divc(a, b) \triangleq \cdt{\left(\lambda b. (b-a)\right)}{0}(b) \qquad
%\logc(a, b) \triangleq \cdt{\left(\lambda b. \divc(a, b)\right)}{1}(b) \\
%\log^*(a, b) \triangleq \cdt{\left(\lambda b. \logc(a, b)\right)}{1}(b)
%\end{array}
\]
\begin{lstlisting}
Definition divc a b := countdown_to 0 (fun n => n - a) b.
Definition logc a b := countdown_to 1 (divc a) b.
Definition logstar a b := countdown_to 1 (logc a) b.
\end{lstlisting}
\end{defn}
}%end hide
\hide{Proving $\forall a, b \ge 1,~\divc(a, b) = \left\lceil b/a \right\rceil$ serves
to prove that all three definitions are correct, since that statement implies
the correctness of  $\logc$, which in turn implies the correctness of $\log^*$. We will prove this in \ref{blah},
together with the correctness of the entire hyperoperation hierarchy.
}
\noindent where the Curried $a\angle{n{-}1}$ denotes the single-variable function
$\lambda b.a\angle{n{-}1}b$.
We now show that $a\angle{n}$ is the inverse \lb to $a[n]$.
First note that $a\angle{0}\in \contract_0$.~Then:
% Linked by Linh
\begin{lem}
	\href{https://github.com/inv-ack/inv-ack/blob/7270e64a2600b771f2b1b1b151f7d13fb2ae6c97/applications.v#L48-L62}{\color{blue}\coq}
$\forall a.~\forall b.~a\angle{1}b = b - a$.
\end{lem}
\begin{proof}
Theorem~\ref{thm: cdt-recursion} applies because $a\angle{0} \in \contract_0 \subseteq \contract_{a}$, giving the intermediate step shown below.
Thereafter we have $a\angle{1}b = b - a$ by induction on $b$.
%\vspace{-0.8em}
%\begin{equation*}
%a\angle{1}b \quad = \quad \cdt{\left(a\angle{0}\right)}{a}(b) \quad = \quad \begin{cases}
%0 & \text{ if } b\le a \\ 1 + a\angle{1}(b - 1) & \text{ if } b\ge a+1 \end{cases} \\
%\end{equation*}
% Edited by Linh
\begin{equation*}
a\angle{1}b \ = \ \cdt{\left(a\angle{0}\right)}{a}(b) \ = \ \begin{cases}
0 & \text{ if } b\le a \\ 1 + a\angle{1}(b - 1) & \text{ if } b\ge a+1 \end{cases} \\
\end{equation*}
\end{proof}
\begin{col} \label{col: inv-hyperop-1-contr1}
$\forall a \ge 1, a\angle{1} \in \contract_1$.
\end{col}
\hide{\begin{col} \label{col: inv-hyperop-234}
{\color{magenta}$\forall a \in \mathbb{N}$}:
 {\color{blue} For all $a \in \mathbb{N}$},
\begin{enumerate}
	\item If $a\ge 1$, $a\angle{2}b = \divc(a, b) \ \forall b$.
	\item If $a\ge 2$, $a\angle{3}b = \logc(a, b) \ \forall b$ and $a\angle{4}b = \logstarc(a, b) \ \forall b$.
\end{enumerate}
\end{col}}%end hide
%Now we are ready to prove the correctness of the inverse hyperoperations.
\noindent \textbf{N.B.} $a\angle{n}b$ is a total function, but it is never actually used for $a = 0$ when $n \ge 2$ or for $a=1$ when $n \ge 3$. For the values
we do care about, we have our inverse:
% Linked by Linh
\begin{thm} \label{thm: inv-hyperop-correct}
	\href{https://github.com/inv-ack/inv-ack/blob/7270e64a2600b771f2b1b1b151f7d13fb2ae6c97/applications.v#L72-L98}{\color{blue}\coq}
When $n\le 1$, or $n \le 2 \wedge a\ge 1$, or $a\ge 2$, then
$a\angle{n} = \left(a[n]\right)^{-1}$.
\end{thm}
\begin{proof}
$\forall n \ge 2,~$ let $a_0 = a, a_1 = 0, a_n = 1$. Define $P$ and $Q$ as:
\begin{equation*}
P(n) \triangleq  \Big(a[n] \in \repeatable_{a_n}\Big)\ \text{ and } \
Q(n) \triangleq  \Big((a\angle{n} = \left(a[n]\right)^{-1} \Big).
\end{equation*}
We have three goals:
\begin{enumerate}
	\item $\forall a.~ Q(0) \wedge Q(1)$ \\ \vspace{-1.2em}
	\item $\forall a \ge 1.~ Q(2)$ \\ \vspace{-1.2em}
	\item $\forall a \ge 2.~ Q(n)$ \\ \vspace{-1.2em}
\end{enumerate}
	Note that \ \mbox{$\forall n.~ a\angle{n+1}$} $= \cdt{a\angle{n}}{a_n}$ ~and \ $a[n+1] = \rf{a[n]}{a_n}$. By Theorem~\ref{thm: cdt-inv-rf},
% Edited by Linh
%\vspace*{-1em}
%\noindent\begin{minipage}{.45\linewidth}
%\begin{equation}
%P(n) \Rightarrow Q(n) \Rightarrow Q(n+1) \label{eq: tmp-induction-1}
%\end{equation}
%\end{minipage}
%\begin{minipage}{.55\linewidth}
%\begin{equation}
%\qquad a_{n} \ge 1 \Rightarrow P(n) \Rightarrow P(n+1) \label{eq: tmp-induction-2}
%\end{equation}
%\end{minipage}
% Edited by Linh
	\begin{align}
	P(n) & \ \Rightarrow \ Q(n) \ \Rightarrow \ Q(n+1) \label{eq: tmp-induction-1} \\
	\qquad a_{n} \ge 1 & \ \Rightarrow \ P(n) \ \Rightarrow \ P(n+1) \label{eq: tmp-induction-2}
	\end{align}
Goal 1: $P(0) \iff \lambda b.(b+1)\in \repeatable_a$ and
$Q(0)\hspace{-0.3em}\iff\hspace{-0.3em}a\angle{0}\hspace{-0.3em}=\hspace{-0.3em}\left(a[0]\right)^{-1} \lb \iff (b-~1\le c \iff b\le c+1)$.
These are both straightforward, and $Q(1)$ holds by~(\ref{eq: tmp-induction-1}). 
Goal 2: we have $a \ge 1$, so
$P(1)$ holds by $P(0)$ and~(\ref{eq: tmp-induction-2}), and
$Q(2)$ holds by $Q(1)$ and~(\ref{eq: tmp-induction-1}).
Goal 3: we have $a\ge 2$, and using~(\ref{eq: tmp-induction-1}) and $Q(0)$
reduces the goal to $P(n)$. Using~(\ref{eq: tmp-induction-2}) and the fact that $\forall n \neq 1.~a_n\ge 1$, the goal reduces to $P(2)$. This unfolds to:
\begin{equation*}
a[2]\in \repeatable_0 \iff \forall b < c.~ab < ac \quad \wedge \quad \forall b \ge 1.~ab \ge b+1\text{,}
\end{equation*}
which is straightforward for $a\ge 2$. Induction on $n$ gives us the third goal.
\end{proof}
%With this theorem, we have proved the correctness of $\divc$, $\logc$ and $\log^*$, as well as the whole inverse hyperoperation hierarchy.
\begin{rem}
Three early hyperoperations are $a[2]b = ab$, $a[3]b = a^b$ and
$a[4]b = \! ^ba$, so, by Theorem~\ref{thm: inv-hyperop-correct}, we can define their inverses $\left\lceil b/a \right\rceil$, $\left\lceil \log_a b \right\rceil$, and $\log^*_a b$ as	
\href{https://github.com/inv-ack/inv-ack/blob/7270e64a2600b771f2b1b1b151f7d13fb2ae6c97/applications.v#L102-L113}{\color{blue}$a\angle{2}b$},	
\href{https://github.com/inv-ack/inv-ack/blob/7270e64a2600b771f2b1b1b151f7d13fb2ae6c97/applications.v#L115-L124}{\color{blue}$a\angle{3}b$}, and	
\href{https://github.com/inv-ack/inv-ack/blob/7270e64a2600b771f2b1b1b151f7d13fb2ae6c97/applications.v#L126-L128}{\color{blue}$a\angle{4}b$}.
Note that the functions $\log_a b$ and $\log^*_a b$ are not in the Coq Standard Library but are one-liners for us:

\begin{lstlisting}
Definition divc a b := inv_hyperop a 2 b.
Definition logc a b := inv_hyperop a 3 b.
Definition logstar a b := inv_hyperop a 4 b.
\end{lstlisting}
\end{rem}



\hide{
\begin{col}
	$\forall a \in \mathbb{N}$:
	\begin{enumerate}
		\item If $a\ge 1$, $\divc(a, b) = \left\lceil \frac{b}{a} \right\rceil$.
		\item If $a\ge 2$, $\logc(a, b) = \left\lceil \log_ab \right\rceil $ and $\logstarc(a, b) = \log^*_ab$.
	\end{enumerate}
\end{col}
\begin{proof}
	The result follows from \cref{col: inv-hyperop-234}, \cref{thm: inv-hyperop-correct} and the first few expansions of the hyperoperation hierarchy: $a[2]b = ab$, $a[3]b = a^b$ and $a[4]b = \ ^ba$.
\end{proof}

\begin{rem}
	\Cref{thm: cdt-recursion} leads to useful recursive formulas for $\divc$, $\logc$ and $\logstarc$:
	\begin{align*}
	\divc(a, b) & = 1 + \divc(a, b - a) & \text{ if } b > 0 &\text{, and } 0 \text{ otherwise}. \\
	\logc(a, b) & = 1 + \logc\left(a, \left\lceil b/a \right\rceil\right) & \text{ if } b > 1 &\text{, and } 0 \text{ otherwise}. \\
	\logstarc(a, b) & = 1 + \divc(a, \left\lceil \log_ab \right\rceil) & \text{ if } b > 1 &\text{, and } 0 \text{ otherwise}.
	\end{align*}
\end{rem}
}

\subsection{The inverse Ackermann hierarchy}
\label{subsec: inv_ack_hier}

% NEW METHOD FOR INVERSE ACKERMANN
Next, we want to use \emph{countdown} to build the \emph{inverse Ackermann hierarchy}, where each
level $\alpha_i$ inverts the level $\Ack_i$.
We know $\Ack_{i+1} = \rf{\Ack_i}{\Ack_i(1)}$\hspace{0.2em}, so the recursive
rule $\alpha_{i+1} \triangleq \cdt{\alpha_i}{\Ack_i(1)}$ is tempting.
But this approach is flawed because it still depends on $\Ack_i$.
Instead, we reexamine the inverse relationship: suppose $\alpha_i = \big(\Ack_i\big)^{-1}$ and $\alpha_{i+1} = \big(\Ack_{i+1}\big)^{-1}$. Then $\Ack_{i+1}(m) = \big(\Ack_{i}\big)^{(m)} \big(\Ack_{i}(1)\big)$. We then have:
\begin{equation} \label{eq: inv-ack-hier-derive}
\alpha_{i+1}(n)\le m \iff n\le \big(\mathcal{A}_i\big)^{(m+1)}(1) \iff \big(\alpha_i\big)^{(m+1)}(n) \le 1
\end{equation}
Equivalently, $\alpha_{i+1}(n) = \min\big\{m : \big( \alpha_i \big)^{(m+1)}(n)\le 1\big\}$, or $\alpha_{i+1}(n) = \cdt{\alpha_i}{1}\big(\alpha_i(n)\big)$. From \eqref{eq: inv-ack-hier-derive} we can thus define the inverse Ackermann hierarchy:
\begin{defn} \label{defn: inv-ack-hier}
	$ \alpha_i \triangleq \begin{cases}
	\lambda n.(n - 1) & \text{if } i = 0
	\\ \big(\cdt{\alpha_{i-1}}{1}\big)\circ \alpha_{i-1} & \text{if } i\ge 1 \end{cases}
$
\end{defn}
Abracadabra! We can express $\alpha$ using the hierarchy \emph{without referring to $\Ack$ itself}:
\begin{thm} \label{thm: inv-ack-new}
	%For all $n, k$, $n\le \Ack(k, k) \!\!\iff \!\! \alpha_k(n)\le k$. Thus
	$\forall n.~ \alpha(n) = \min\big\{k : \alpha_k(n)\le k \big\}$.
\end{thm}


\begin{table}[t]
	\begin{centermath}
		\begin{array}{c | c@{\hspace{1.5em}} c@{\hspace{1.5em}} c@{\hspace{1.5em}} c@{\hspace{1.5em}} c@{\hspace{1.5em}} c@{\hspace{1.5em}} c@{\hspace{1.5em}} c@{\hspace{1.5em}} c@{\hspace{1.5em}} c@{\hspace{1.5em}}}
			          $\diagbox[height=\line]{$\alpha_k$}{$n$}$ & 0\tikzmark{zero} & 1\tikzmark{one} & \tikzmark{twoa}2\tikzmark{two} & \tikzmark{threea}3\tikzmark{three} & \tikzmark{foura}4\tikzmark{four} & \tikzmark{fivea}5\tikzmark{five} & \tikzmark{sixa}6\tikzmark{six} & \tikzmark{sevena}7\tikzmark{seven} & \tikzmark{eighta}8\tikzmark{eight} & \tikzmark{ninea}9\tikzmark{nine} \\
			\hline
			\alpha_0 & 0\tikzmark{zero_succ} & 0\tikzmark{one_succ} & \tikzmark{two_fail}1 & \tikzmark{three_fail}2 & \tikzmark{four_fail1}3 & \tikzmark{five_fail1}4 & \tikzmark{six_fail1}5 & \tikzmark{seven_fail1}6 & \tikzmark{eight_fail1}7 & \tikzmark{nine_fail1}8 \\
			\alpha_1 & 0 & 0 & 0\tikzmark{two_succ} & 1\tikzmark{three_succ} & \tikzmark{four_fail2}2 & \tikzmark{five_fail2}3 & \tikzmark{six_fail2}4 & \tikzmark{seven_fail2}5 & \tikzmark{eight_fail2}6 & \tikzmark{nine_fail2}7 \\
			\alpha_2 & 0 & 0 & 0 & 0 & 1\tikzmark{four_succ} & 1\tikzmark{five_succ} & 2\tikzmark{six_succ} & 2\tikzmark{seven_succ} & \tikzmark{eight_fail3}3 & \tikzmark{nine_fail3}3 \\
			\alpha_3 & 0 & 0 & 0 & 0 & 0 & 0 & 1 & 1 & 1\tikzmark{eight_succ} & 1\tikzmark{nine_succ} \\
		\end{array}
	\end{centermath}
	\caption{Intuition for $\alpha(n)$ defined without $\Ack(n)$.}
	\label{table:inv_intuition}
\end{table}
  \begin{tikzpicture}[overlay, remember picture, yshift=.25\baselineskip, shorten >=.5pt, shorten <=.5pt]
    \draw [->, green] ({pic cs:zero}) to [bend left]({pic cs:zero_succ});
    \draw [->, green] ({pic cs:one}) to [bend left]({pic cs:one_succ});
   	\draw [->, red] ({pic cs:twoa}) to [bend right]({pic cs:two_fail});
   	\draw [->, green] ({pic cs:two}) to [bend left]({pic cs:two_succ});
   	\draw [->, red] ({pic cs:threea}) to [bend right]({pic cs:three_fail});
   	\draw [->, green] ({pic cs:three}) to [bend left]({pic cs:three_succ});
   	\draw [->, red] ({pic cs:foura}) to [bend right]({pic cs:four_fail1});
   	\draw [->, red] ({pic cs:foura}) to [bend right]({pic cs:four_fail2});
   	\draw [->, green] ({pic cs:four}) to [bend left]({pic cs:four_succ});
   	\draw [->, red] ({pic cs:fivea}) to [bend right]({pic cs:five_fail1});
   	\draw [->, red] ({pic cs:fivea}) to [bend right]({pic cs:five_fail2});
   	\draw [->, green] ({pic cs:five}) to [bend left]({pic cs:five_succ});
   	\draw [->, red] ({pic cs:sixa}) to [bend right]({pic cs:six_fail1});
   	\draw [->, red] ({pic cs:sixa}) to [bend right]({pic cs:six_fail2});
   	\draw [->, green] ({pic cs:six}) to [bend left]({pic cs:six_succ});
   	\draw [->, red] ({pic cs:sevena}) to [bend right]({pic cs:seven_fail1});
   	\draw [->, red] ({pic cs:sevena}) to [bend right]({pic cs:seven_fail2});
   	\draw [->, green] ({pic cs:seven}) to [bend left]({pic cs:seven_succ});
   	\draw [->, red] ({pic cs:eighta}) to [bend right]({pic cs:eight_fail1});
   	\draw [->, red] ({pic cs:eighta}) to [bend right]({pic cs:eight_fail2});
   	\draw [->, red] ({pic cs:eighta}) to [bend right]({pic cs:eight_fail3});
   	\draw [->, green] ({pic cs:eight}) to [bend left]({pic cs:eight_succ});
   	\draw [->, red] ({pic cs:ninea}) to [bend right]({pic cs:nine_fail1});
   	\draw [->, red] ({pic cs:ninea}) to [bend right]({pic cs:nine_fail2});
   	\draw [->, red] ({pic cs:ninea}) to [bend right]({pic cs:nine_fail3});
   	\draw [->, green] ({pic cs:nine}) to [bend left]({pic cs:nine_succ});
  \end{tikzpicture}

% \vspace{-3em}
Table~\ref{table:inv_intuition} gives intuition for our new definition of $\alpha$.
This table is populated exactly as suggested by 
Table~\ref{table: hyperop-ack-inv}:
$\alpha_{0}(n)$ is $n-1$, 
$\alpha_{1}(n)$ is $n-2$,
$\alpha_{2}(n)$ is $\left\lceil \frac{n-3}{2} \right\rceil$, and
$\alpha_{3}(n)$ is $\left\lceil \log_2 ~ (b + 3)\right\rceil - 3$.
Now for any $n$, we search down its column, which is indexed by $k$, 
looking for the smallest $k$ such that $\alpha_k(n) \le k$.

Consider $n = 8$. $\alpha_{0}(8) = 7$. Because $7 \not\le 0$, $\alpha_0$ is rejected.
Similarly $\alpha_{1}(8) = 6 \not\le 1$ and $\alpha_{2}(8) = 3 \not\le 2$
are rejected. Finally $\alpha_{3}(8) = 1$ is accepted because $1 \le 3$. 
Indeed, $\alpha(8) = 3$.
Similar reasoning applies to the other values for $n$. To aid intuition in this table,
we show show unsuccessful searches in {\color{red}red} and successful searches in 
{\color{green}green}.

Now all that remains is to provide a structurally-recursive function that computes $\alpha$.
%\begin{defn} \label{defn: inv-ack-worker}
%	The inverse Ackermann worker is a function $\alpha^{\W}$: %(\mathbb{N}\to \mathbb{N}) \times \mathbb{N}^3\to \mathbb{N}$ such that for all $n, k, b\in \mathbb{N}$ and $f:\mathbb{N}\to \mathbb{N}$:
%	\begin{equation} \label{eq: inv-ack-worker-recursion}
%	\alpha^{\W}(f, n, k, b) \triangleq \begin{cases}
%	k & \text{if } b = 0 \vee n\le k \\ \alpha^{\W}(\cdt{f}{1}\circ f , \cdt{f}{1}(n), k+1, b-1) & \text{if } b \ge 1 \wedge n \ge k+1
%	\end{cases}
%	\end{equation}
%\end{defn}
% Edited by Linh
\begin{defn} \label{defn: inv-ack-worker}
	The inverse Ackermann worker, written $\alpha^{\W}$, is a function from $\mathbb{N}^4$ to $\mathbb{N}$ defined as: %(\mathbb{N}\to \mathbb{N}) \times \mathbb{N}^3\to \mathbb{N}$ such that for all $n, k, b\in \mathbb{N}$ and $f:\mathbb{N}\to \mathbb{N}$:
	\begin{equation} \label{eq: inv-ack-worker-recursion}
	\begin{aligned}
	& \alpha^{\W}(f, n, k, b) \\ 
	& \triangleq \begin{cases}
	k & \text{if } b = 0 \vee n\le k \\ \alpha^{\W}(\cdt{f}{1}\circ f , \cdt{f}{1}(n), k+1, b-1) & \text{if } b \ge 1 \wedge n \ge k+1
	\end{cases}
		\end{aligned}
	\end{equation}
\end{defn}
\noindent Next, we show that this function computes the inverse Ackermann function when passed appropriate arguments.
\hide{
Given the arguments $\big(\alpha_i, \alpha_i(n), i, b - i\big)$ such that $\alpha_i(n) > i$ and $b > i$, $\alpha^{\W}$ takes on arguments $\big(\alpha_{i+1}, \alpha_{i+1}(n), i+1, b - (i+1)\big)$ at the next recursive call. Thus if $\alpha^{\W}$ is given a sufficient budget $b$, it will recursively transform the tuple $(\alpha_k, \alpha_k(n), k, b - k)$ until a point $k$ where $\alpha_k(n)\le k$, and will then return $k$. We now need to show that $\alpha^{\W}$ correctly computes $\alpha(n)$ given a reasonable budget.
The following theorem demonstrates that a budget of $n$ suffices.
}%end hide
\hide{
\noindent We are finally ready for a strategy to compute the inverse Ackermann function:
}%end hide
%formalizes a setting for $\W\alpha$ to work.
% Linked by Linh
\begin{thm} \label{thm: inv-ack-worker-correct}
	\href{https://github.com/inv-ack/inv-ack/blob/7270e64a2600b771f2b1b1b151f7d13fb2ae6c97/inv_ack.v#L199-L231}{\color{blue}\coq}
	$\forall n.~\alpha^{\W}\big(\alpha_0, \alpha_0(n), 0, n\big) = \alpha(n)$.
\end{thm}
\begin{proof}[Proof outline]
	When given the arguments $\big(\alpha_i, \alpha_i(n), i, b - i\big)$ such that $\alpha_i(n) > i$ and $b > i$, $\alpha^{\W}$ takes on the arguments $\big(\alpha_{i+1}, \alpha_{i+1}(n), i+1, b - (i+1)\big)$ at the next recursive call. A simple induction on $k$ then shows that if $k\le \min\big\{b, \alpha_k(n)\big\}$,
	\begin{equation} \label{eq: inv-ack-worker-intermediate}
	\alpha^{\W}\big(\alpha_0, \alpha_0(n), 0, b\big) = \alpha^{\W}\big(\alpha_k, \alpha_k(n), k, b-k\big)
	\end{equation}
	Let $m \triangleq \min\big\{k : \alpha_k(n) \le k \}$. Then $m\le n$ since $\alpha_n(n)\le n$. \eqref{eq: inv-ack-worker-intermediate} then implies:
	$$ \alpha^{\W}\big(\alpha_0, \alpha_0(n), 0, n\big) = \alpha^{\W}\big(\alpha_m, \alpha_m(n), m, n - m\big) = m = \alpha(n) $$
\end{proof}
\noindent We put a more involved mathematical proof of correctness in Appendix~\ref{apx:proof_correct_inv_ack_worker}, and a mechanized proof
% Linked by Linh
	\href{https://github.com/inv-ack/inv-ack/blob/7270e64a2600b771f2b1b1b151f7d13fb2ae6c97/inv_ack.v#L163-L231}{\color{blue}here}.
We thus have a (re-)definition of inverse Ackermann that is %computationally
definable via a Coq-accepted worker, \emph{i.e.} $\alpha(n) \triangleq \alpha^{\W}\big(\alpha_0, \alpha_0(n), 0, n\big)$. Below, we present the inverse
Ackermann function in Gallina.

% Linked by A
%\begin{lstlisting}
%`\href{https://github.com/inv-ack/inv-ack/blob/7270e64a2600b771f2b1b1b151f7d13fb2ae6c97/inv_ack.v#L155-L161} {Fixpoint inv\_ack\_wkr}` (f : nat -> nat) (n k b : nat) : nat :=
%  match b with 0 => 0 | S b' =>
%    if (n <=? k) then k else let g := (countdown_to f 1) in
%      inv_ack_wkr (compose g f) (g n) (S k) b
%  end.
%
%`\href{https://github.com/inv-ack/inv-ack/blob/7270e64a2600b771f2b1b1b151f7d13fb2ae6c97/inv_ack.v#L37-L41}{Fixpoint alpha}` (m x : nat) : nat :=
%  match m with 0 => x - 1 | S m' =>
%    countdown_to 1 (alpha m') (alpha m' x)
%  end.
%
%`\href{https://github.com/inv-ack/inv-ack/blob/7270e64a2600b771f2b1b1b151f7d13fb2ae6c97/inv_ack.v#L167}{Definition inv\_ack}` := inv_ack_wkr (alpha 0) (alpha 0 n) 0 n.
%\end{lstlisting}
% Edited by Linh
\begin{lstlisting}
`\href{https://github.com/inv-ack/inv-ack/blob/7270e64a2600b771f2b1b1b151f7d13fb2ae6c97/inv_ack.v#L155-L161} {\color{blue}Fixpoint inv\_ack\_wkr}` f n k b :=
  match b with 0 => 0 | S b' =>
    if (n <=? k) then k
      else let g := (countdown_to f 1) in
        inv_ack_wkr (compose g f) (g n) (S k) b
  end.

`\href{https://github.com/inv-ack/inv-ack/blob/7270e64a2600b771f2b1b1b151f7d13fb2ae6c97/inv_ack.v#L37-L41}{\color{blue}Fixpoint alpha}` m x :=
  match m with 0 => x - 1 | S m' =>
    countdown_to 1 (alpha m') (alpha m' x)
  end.

`\href{https://github.com/inv-ack/inv-ack/blob/7270e64a2600b771f2b1b1b151f7d13fb2ae6c97/inv_ack.v#L167}{\color{blue}Definition inv\_ack}` :=
  inv_ack_wkr (alpha 0) (alpha 0 n) 0 n.
\end{lstlisting}

\noindent Note that this is not the linear-time computation we presented in
Figure~\ref{fig:standalone}. We will arrive at that code via an improvement discussed in the next section.
% OLD METHOD FOR INVERSE ACKERMANN

%Using the Ackermann kludge from \cref{sec: overview}, we sketch a method to find the inverse Ackermann function from the inverse hyperoperations.
%\begin{thm} \label{thm: inv-hyperop-inv-ack}
%	For all $n, k$, $n\le \Ack(k, k) \iff 2\angle{k}(n+3)\le k+3$.
%\end{thm}
%\begin{proof}
%	$n\le \Ack(k, k) \iff n\le 2[k](k+3) - 3$. Now $2[k]$ is an expansion by \cref{thm: inv-hyperop-correct}'s proof, so $2[k](k+3) \ge k+3\ge 3$, so  $n\le 2[k](k+3) - 3 \iff n+3 \le 2[k](k+3)\iff 2\angle{k}(n+3)\le k+3$, again by \cref{thm: inv-hyperop-correct}.
%\end{proof}
%\Cref{thm: inv-hyperop-inv-ack} allows a simple method to compute the Inverse Ackermann function, based on the \emph{budget} idea in \emph{countdown worker}.
%\begin{defn} \label{defn: inv-ack-worker}
%	The \emph{inverse Ackermann worker} of $f$ is a function $\W\alpha\ : \mathbb{N}^4\to \mathbb{N}$ such that for all $n, k, b\in \mathbb{N}$ and $f:\mathbb{N}\to \mathbb{N}$:
%	$$ \W\alpha(f, n, k, b) = \begin{cases}
%	k - 3 & \text{if } b = 0 \vee f(n)\le k+3 \\ \W\alpha(\cdt{f}{a_k}\ , n, k+1, b-1) & \text{if } b \ge 1 \wedge f(n) > k+3
%	\end{cases} $$
%	where $a_0 = 2$, $a_1 = 0$ and $a_k = 1 \ \forall k\ge 2$.
%\end{defn}
%The function $\W\alpha$ takes a function $f$ and pretends it was the first level of the inverse hyperoperation hierarchy. It then keeps applying countdown to $f$, supposedly arrives at the $\text{m}^{\text{th}}$-level at the $\text{m}^{\text{th}}$-recursive step, until the budget $b$ is exhausted or $f(n)\le k+3$, i.e. \emph{early stopping}, and will output $k$. If at the beginning $k=3$ and $f = 2\angle{0}$, the early stopping condition becomes $2\angle{k}n \le k+3$, which when replace $n$ by $n+3$ gives what we need in \cref{thm: inv-hyperop-inv-ack}.
%The inverse Ackermann function can be defined as follows.
%\begin{thm} For all $n$, we have $\alpha(n) = \W\alpha(\lambda m.(m - 1), n+3, 3, n)$.
%\end{thm} 

\section{Time Bound of Our Inverses}
\label{sec: inv-ack}
In this section, we provide a time analysis for the previously defined computation of the inverse Ackermann function. In particular, we prove that its running time is $O(n^2)$. We then provide a simple improvement that can cut the running time to $O(n\cdot\alpha(n))$, and a subsequent improvement that ultimately reduces it to $O(n)$.
\subsection{Time Analysis}
In this section, let us forget about the axiom of extensionality and identify each function on $\mathbb{N}$ with its \emph{computation}, i.e. the program that computes it in Coq. We will be careful not to conclude $f = g$ when they agree on all inputs but are computed with different pieces of code. It allows us to formalize a definition of running time of functions.
\begin{defn}
	Given a function $f:\mathbb{N}\to\mathbb{N}$ in Coq, the \emph{running time} of $f$ on input $n\in \mathbb{N}$, denoted by $\runtime(f, n)$ is the total number of computational steps it takes to compute $f(n)$.
\end{defn}
Below are several useful lemmas for our analysis.
\begin{lem} \label{lem: sub-runtime}
	For all $a, n$, $\runtime(\lambda n.(n - a), n) = \Theta(\min\{a, n\})$ per subtraction's Coq definition.
\end{lem}
\begin{lem} \label{lem: compose-runtime}
	Per composition's Coq definition, for all $f$ and $g: \mathbb{N}\to \mathbb{N}$, $\runtime(f\circ g, n) = \runtime(f, g(n)) + \runtime(g, n)$.
\end{lem}
\begin{lem} \label{lem: cdt-runtime}
	Per our definition of \emph{countdown} (\Cref{defn: countdown}), for all $a\ge 1$ and $f\in \contract_{a}$, $\runtime\big(\cdt{f}{a}\ , n\big) = 1 \ \forall n\le 1$ and
	\begin{equation*}
	\runtime\big(\cdt{f}{a}\ , n\big) = \sum_{i=0}^{\cdt{f}{a}(n) - 1} \runtime\left(f, f^{(i)}(n)\right) + \Theta\big(a\cdot \cdt{f}{a}(n)\big) \ \ \forall n\ge 2
	\end{equation*}
\end{lem}
From \cref{lem: compose-runtime} and \cref{lem: cdt-runtime}, the following lemma easily follows.
\begin{lem} \label{lem: inv-ack-hier-runtime}
	Per \cref{defn: inv-ack-hier} and \cref{defn: countdown}, for all $i$, 
	\begin{equation*}
	\runtime\big(\alpha_{i+1}, n\big) = \sum_{k=0}^{\alpha_{i+1}(n)}\runtime\left(\alpha_i, \alpha_i^{(k)}(n)\right) + \Theta\big(\alpha_{i+1}(n)\big)
	\end{equation*}
\end{lem}
This lemma implies $\runtime\big(\alpha_{i+1}, n\big) \ge \runtime\big(\alpha_i, n\big)$. If for some $i$, $\runtime\big(\alpha_i, n\big) = \Theta\big(n^2\big)$, each function in the hierarchy will take at least $\Omega\big(n^2\big)$ to compute from $\alpha_i$, thus making $\runtime(\alpha, n) = \Omega\big(n^2\big)$ per \cref{defn: inv-ack-worker}. The next lemma shows $i = 2$ suffices.
\begin{lem}
	Per \cref{defn: inv-ack-hier}, $\runtime\big(\alpha_2, n\big) = \Theta\big(n^2\big) \ \forall n$. 
\end{lem}
\begin{proof}
	$\alpha_1 = \big(\cdt{\lambda m.(m-1)}{1}\big)\circ \big(\lambda m.(m-1)\big) \equiv \lambda m.(m - 2)$, by \cref{lem: inv-ack-hier-runtime},
	\begin{equation*}
	\runtime\big(\alpha_1, n\big) = \sum_{i=0}^{n-1} \runtime\big(\lambda m.(m-1), n - i\big) + \Theta\big((n-2)\big) = \Theta(n)
	\end{equation*}
	, since $\runtime\big(\lambda m.(m-1), k\big) = 1 \ \forall k$. Since $\alpha_2 = \big(\cdt{\alpha_1}{0}\big)\circ \alpha_1 $, again by \cref{lem: inv-ack-hier-runtime},
	\begin{equation*}
	\runtime\big(\alpha_2, n\big)
	= \sum_{i=0}^{\lceil (n-3)/2 \rceil} \runtime \big(\alpha_1, n-2i\big) + \Theta\left(\frac{n}{2}\right)
	= \Theta\left( \sum_{i=0}^{\lceil (n-3)/2 \rceil}(n - 2i) \right)
	= \Theta\big(n^2\big)
	\end{equation*}
	The proof is complete.
\end{proof}
\begin{col}
	$\runtime(\alpha, n) = \Omega\big(n^2\big)$ per \cref{defn: inv-ack-worker}.
\end{col}
Intuitively, the function $\alpha_2 \equiv \lambda n.(n-2)$ is responsible for dragging the whole hierarchy's performance due to one silly weakness: its does not know it will always output $n-2$ before beginning its computation, hence needs to tediously subtract $1$ until it goes below $2$. This observation leads to the next improvement.

%This lemma implies $\runtime\big(\cdt{f}{a}\ , n\big) \ge \runtime(f, n)$. If for some $i$, $\runtime\big(2\angle{i}, n\big) = \Theta\big(n^2\big)$, the entire hierarchy will take at least $\Omega\big(n^2\big)$ from $2\angle{i}$, thus making $\runtime(\alpha, n) = \Omega\big(n^2\big)$ per \cref{defn: inv-ack-worker}. The next lemma shows $i = 2$ suffices.
%\begin{lem}
%	Per \cref{defn: inv-hyperop}, $\runtime\big(2\angle{2}, n\big) = \Theta\big(n^2\big) \ \forall n$. 
%\end{lem}
%\begin{proof}
%	Since $2\angle{1} = \cdt{\lambda m.(m-1)}{2}\ \equiv \lambda m.(m - 2)$, by \cref{lem: cdt-runtime},
%	$$ \runtime\big(2\angle{1}, n\big) = \sum_{i=0}^{n-2} \runtime\big(\lambda m.(m-1), n - i\big) + \Theta\big(3(n-2)\big) = \Theta(n) $$
%	, since $\runtime\big(\lambda m.(m-1), k\big) = 1 \ \forall k$. Since $2\angle{2} = \cdt{2\angle{1}}{0}\ $, again by \cref{lem: cdt-runtime},
%	$$ \runtime\big(2\angle{2}, n\big)
%	= \sum_{i=0}^{\lceil n/2 \rceil} \runtime \big(2\angle{1}, n-2i\big) + \Theta\left(\frac{n}{2}\right)
%	= \Theta\left( \sum_{i=0}^{\lceil n/2 \rceil}(n - 2i) \right)
%	= \Theta\big(n^2\big) $$
%	The proof is complete.
%\end{proof}
%\begin{col}
%	$\runtime(\alpha, n) = \Omega\big(n^2\big)$ per \cref{defn: inv-ack-worker}.
%\end{col}
%Intuitively, the function $2\angle{1} \equiv \lambda n.(n-2)$ is responsible for dragging the whole hierarchy's performance due to one silly weakness: its does not know it will always output $n-2$ before beginning its computation, hence needs to tediously subtract $1$ until it goes below $2$. This observation leads to the next improvement.

\subsection{Hard-coding the second level}
Per our observation, one obvious improvement would be hard-coding $\alpha_1$ as $\lambda n.(n-2)$. This way its running time will be reduced to $\Theta(1)$, as opposed to the previous $\Theta(n)$.
\begin{lstlisting}
Definition sub_2 (n : nat) : nat
:= match n with | 0 => 0 | 1 => 1 | S (S n') => n' end.
\end{lstlisting}
Without loss of generality, let us assume from now on the constant factors in $\runtime(\alpha_1, n)$ and \cref{lem: inv-ack-hier-runtime} are both $1$. The running time for $\alpha_2$ is then
\begin{equation*}
\runtime\big(\alpha_2, n\big)
 = \sum_{i=0}^{\lceil (n-3)/2 \rceil} \runtime \big(\alpha_1, n-2i\big) + \left\lceil \frac{n-3}{2} \right\rceil =  2\left\lceil \frac{n-3}{2} \right\rceil
 < n 
\end{equation*}
In fact with this improvement, every function in the hierarchy can be computed in linear time, as shown by the next theorem.
\begin{thm} \label{thm: inv-ack-hier-runtime-improved}
	For all $i, n$, $\runtime\big(\alpha_i, n\big) \le 2n + (4^{i+1} - 4)\lceil \log_2n\rceil + 5$.
\end{thm}
Before proving this lemma, we need two crucial technical lemmas.
\begin{lem} \label{lem: inv-ack-3-runtime}
	For all $n$, $\runtime\big(\alpha_3, n\big) \le 2n + 4$.
\end{lem}
\begin{proof}
	It is easy to show that for all $k$, $\alpha_2^{(k)}(n) = \left\lceil \frac{n+3}{2^k} \right\rceil - 3$. Thus
	\begin{equation*}
	\begin{aligned}
	\runtime\big(\alpha_3, n\big) & =
	\sum_{k=0}^{\alpha_{3}(n)}\runtime\left(\alpha_2, \left\lceil \frac{n+3}{2^k} \right\rceil - 3\right) + \alpha_{3}(n) \\
	& \le \sum_{k=0}^{\alpha_3(n)}\left(\frac{n+3}{2^k} + 1\right) - 3\big(\alpha_3(n) + 1\big) + \alpha_3(n) \\
	& \le \sum_{k=0}^{\alpha_3(n)}\frac{n+3}{2^k} - \alpha_3(n) - 2 \le 2(n+3) - 2 = 2n + 4
	\end{aligned}
	\end{equation*}
\end{proof}
\begin{lem}
	For all $i\ge 3$, $\displaystyle \sum_{k=1}^{\alpha_{i+1}(n)} \alpha_i^{(k)}(n) \le 3\big\lceil \log_2n \big\rceil \ \forall n$.
\end{lem}
\begin{proof}
	Let the LHS be $S_i(n)$. Firstly, consider $i = 3$. Note that for $n\le 13$, $S_3(n) = 0$ and for $n\ge 14$, i.e. $\alpha_3(n)\ge 2$, $S_3(n) = \alpha_3(n) + S_3\big(\alpha_3(n)\big)$. The result thus holds for $n\le 13$. Suppose it holds for all $m < n$, where $n\ge 14$. Then
	\begin{equation*}
	S_3(n) \le \alpha_3(n) + 3\big\lceil \log_2(\alpha_3(n)) \big\rceil \le \big\lceil \log_2n \big\rceil + 3\big\lceil \log_2\log_2n \big\rceil
	\end{equation*}
	It is easy to prove $2\big\lceil \log_2\log_2n \big\rceil \le \big\lceil \log_2n \big\rceil$ via induction on $\big\lceil \log_2n \big\rceil$. Thus $S_3(n)\le 3\big\lceil \log_2n \big\rceil$, as desired. Now for $i \ge 4$,
	\begin{equation*}
	S_i(n) = \sum_{k=1}^{\alpha_{i+1}(n)} \alpha_i^{(k)}(n) \le
	\sum_{k=1}^{\alpha_{i+1}(n)} \alpha_3^{(k)}(n) \le
	\sum_{k=1}^{\alpha_{4}(n)} \alpha_3^{(k)}(n) \le 3\big\lceil \log_2n \big\rceil
	\end{equation*}
	The proof is complete.	
%	Let $P(n) \triangleq 2\big\lceil \log_2\log_2n \big\rceil \le \big\lceil \log_2n \big\rceil$. It suffices to prove $P(n) \ \forall n$. Observe that $P(n)$ holds for $n\ge 4$.
\end{proof}
We are ready to present the proof for \Cref{thm: inv-ack-hier-runtime-improved}.
\begin{proof}[Proof of \Cref{thm: inv-ack-hier-runtime-improved}]
	We have proved the result for $i = 0, 1, 2$. Let us proceed with $i\ge 3$ by induction. The base case $i = 3$ has been covered by \cref{lem: inv-ack-3-runtime}. Now suppose the result holds for $i\ge 3$, let $M_i\triangleq 4^{i+1}-4$ for each $i$, we have
	\begin{equation*}
	\begin{aligned}
	& \runtime\big(\alpha_{i+1}, n\big) = \sum_{k=0}^{\alpha_{i+1}(n)} \runtime\left(\alpha_i, \alpha_i^{(k)}(n)\right) + \alpha_{i+1}(n) \\
	& \le \sum_{k=0}^{\alpha_{i+1}(n)}\left(2\alpha_i^{(k)}(n) + M_i\left\lceil\log_2\left(\alpha_i^{(k)}(n)\right)\right\rceil + 5 \right) + \alpha_{i+1}(n) \\
	& \le 2n + M_i\left\lceil\log_2n\right\rceil + 5 + (M_i+2)\sum_{k=1}^{\alpha_{i+1}(n)}\alpha_i^{(k)}(n) + 6\alpha_{i+1}(n) \\
	& \le 2n + M_i\left\lceil\log_2n\right\rceil + 5 +
	3(M_i + 2)\left\lceil\log_2n\right\rceil + 6\left\lceil\log_2n\right\rceil \\
	& = 2n + (4M_i + 12)\left\lceil\log_2n\right\rceil + 5 = 2n + M_{i+1}\left\lceil\log_2n\right\rceil + 5
	\end{aligned}
	\end{equation*}
	, since $4M_i + 12 = 4^{i+2} - 16 + 12 = M_{i+1}$. The proof is complete.
\end{proof}
Thus by hard-coding $\alpha_1$ as $\lambda m.(m-2)$, each level of the inverse Ackermann hierarchy can be computed in $\Theta\big(n + 4^i\log n\big)$ time.

\subsection{An improved inverse Ackermann computation}
Using the previous improvement, we can improve the running time of $\alpha(n)$ per \Cref{defn: inv-ack-hier} by hard-coding the output when $n\le 1 = \Ack(0, 0)$, and starting $\W\alpha$ with $f := \alpha_1$ and $n := \alpha_1(n)$. In other words,
\begin{equation*}
\tilde{\alpha}(n) = \begin{cases}
0 & \text{ if } n \le 1 \\ \W\alpha\big(\alpha_1, \alpha_1(n), 1, n-1\big) & \text{ if } n \ge 2
\end{cases}
\end{equation*}
For $n > 1$, $1\le \min\big\{n-1, \alpha_1(n)\big\}$, so
\begin{equation*}
\W\alpha\big(\alpha_0, \alpha_0(n), 0, n\big) =
\W\alpha\big(\alpha_1, \alpha_1(n), 1, n-1\big)
\end{equation*}
Thus $\tilde{\alpha}(n) = \alpha(n) \ \forall n$. Now, to transform from $\W\alpha\big(\alpha_k, \alpha_k(n), k, n-k\big)$ to $\W\alpha\big(\alpha_{k+1}, \alpha_{k+1}(n), k+1, n-k-1\big)$, $\W\alpha$ needs to compute $\big(\cdt{\alpha_k}{1}\big)\big(\alpha_k(n)\big)$ given $\alpha_k(n)$, which takes time $\runtime\big(\alpha_{k+1}, n\big) - \runtime\big(\alpha_k, n\big)$ by \cref{lem: compose-runtime}; as well as $\alpha_k(n) - k$ for $\Theta(k)$ time. The computation will terminate at $k = \alpha(n)$. Thus for all $n\ge 1$,
\begin{equation*}
\begin{aligned}
& \runtime\big(\tilde{\alpha}, n\big) = \runtime\big(\alpha_1, n\big) + \sum_{k=1}^{\alpha(n) - 1}
\left[ \runtime\big(\alpha_{k+1}, n\big) - \runtime\big(\alpha_k, n\big)
\right] + \sum_{k=1}^{\alpha(n)}\Theta(k) \\
& = \runtime\left(\alpha_{\alpha(n)}, n\right) + \Theta\left(\alpha(n)^2\right)
= \Theta\left(2n + 4^{\alpha(n)}\log_2n + \alpha(n)^2\right) = \Theta(n)
\end{aligned}
\end{equation*}
Therefore, $\tilde{\alpha}$ is able to compute $\alpha$ in linear time.


\section{Further Discussion}
\label{sec: discussion}
\newcommand{\ackt}{\ensuremath{\hat{\alpha}}}

\subsection{The value of a linear-time solution to the hierarchy}

Our functions' linear runtimes can be understood in two distinct but
complementary ways.  A runtime less than the bitlength is impossible
without prior knowledge of the size of the input.  Accordingly, in
an information-theory or pure-mathematical sense, our definitions are
optimal up to constant factors.  And of course in practice, linear-time
solutions are highly usable in real computations.

Sublinear solutions are possible with \emph{a priori} knowledge about
the function and bounds on the inputs one will receive.
An extreme case is $\alpha(n)$, which has value $4$ for all practical
inputs greater than $61$. Accordingly,
this function can be inverted in $O(1)$ in practice.  That said, 
such solutions require external knowledge of the problems and
lookup tables within the algorithm to store precomputed
values, and thus fall more into the realm of engineering than mathematics. 

\subsection{The two-parameter inverse Ackermann function}
Some authors~\cite{chazelle,tarjan} prefer a two-parameter inverse Ackermann function.
\begin{defn} \label{defn: 2para-alpha}
	The two-parameter inverse Ackermann function is defined as:
	\begin{equation} \label{eq: tmp-2para-alpha}
	\ackt (m, n) \triangleq \min\left\{i \ge 1 : \Ack\left(i, \left\lfloor \frac{m}{n} \right\rfloor \right)\ge \log_2n \right\}
	\end{equation}
\end{defn}
Note that $\ackt(n, n)$ and the single-parameter $\alpha(n)$
are neither equal nor directly related, but
it is straightforward to modify our techniques to compute $\ackt(m, n)$.
\hide{This function arises from deep runtime analysis of the disjoint-set data structure. Tarjan \cite{tarjan} showed that, in the disjoint-set data structure, the time required $t(m,n)$ for a sequence of $m$ \textsc{\color{magenta}FIND}s intermixed with $n-1$ \textsc{\color{magenta}UNION}s (such that $m \geq n$) is bounded as: $k_{1}m\cdot\alpha(m,n) \leq t(m,n) \leq k_{2}m\cdot\alpha(m,n)$. In graph theory, Chazelle \cite{chazelle} showed that the minimum spanning tree of a connected graph with $n$ vertices and $m$ edges can be found in time $O(m\cdot\alpha(m,n))$. Computing this function is in fact easier than $\alpha(n)$, as when $m$ and $n$ are given, we are reduced to finding the minimum $i\ge 1$ such that $\Ack_i(s)\ge t$ for $s, t$ fixed, which can be done with the following variant of the \emph{inverse Ackermann worker}.
}
\begin{defn} \label{defn: 2para-inv-ack-worker}
	The {two-parameter inverse Ackermann worker}
	,written $\ackt^{\W}$, is a function $\mathbb{N}^4\to \mathbb{N}$, defined by:
	\hide{$(\mathbb{N}\to \mathbb{N}) \times \mathbb{N}^3\to \mathbb{N}$ such that for all $n, k, b\in \mathbb{N}$ and $f:\mathbb{N}\to \mathbb{N}$:}
%	\begin{equation} \label{eq: 2para-inv-ack-worker-recursion}
%	\ackt^{\W}(f, n, k, b) = \begin{cases}
%	0 & \text{if } b = 0 \vee n\le k \\ 1 + \ackt^{\W}\big(\cdt{f}{1}\circ f , \cdt{f}{1}(n), k, b-1\big) & \text{if } b \ge 1 \wedge n \ge k+1
%	\end{cases}
%	\end{equation}
  \begin{equation} \label{eq: 2para-inv-ack-worker-recursion}
  \begin{aligned}
  & \ackt^{\W}(f, n, k, b) \\
  & \triangleq \begin{cases}
  0 & \text{if } b = 0 \vee n\le k \\ 1 + \ackt^{\W}\big(\cdt{f}{1}\circ f , \cdt{f}{1}(n), k, b-1\big) & \text{if } b \ge 1 \wedge n \ge k+1
  \end{cases}
  \end{aligned}
  \end{equation}
\end{defn}
%Similar to the one-parameter version, the following theorem establishes the correct setting for $\W\alpha_2$ to compute $\alpha(m, n)$.
%\begin{thm}
%	$\displaystyle \ackt(m, n) = 1 + \ackt^{\W}\left(\alpha_1, \alpha_1\big(\lceil\log_2n \rceil\big), \left\lfloor \frac{m}{n} \right\rfloor, \lceil\log_2n \rceil \right)$.
%\end{thm}
% Edited by Linh
\begin{thm} For all $m$ and $n$,
	\begin{equation*}
	\displaystyle \ackt(m, n) = 1 + \ackt^{\W}\left(\alpha_1, \alpha_1\big(\lceil\log_2n \rceil\big), \left\lfloor \frac{m}{n} \right\rfloor, \lceil\log_2n \rceil \right).
	\end{equation*}
\end{thm}
We mechanize the above for both \href{https://github.com/inv-ack/inv-ack/blob/7270e64a2600b771f2b1b1b151f7d13fb2ae6c97/inv_ack.v#L245-L248}{\color{blue}unary} and \href{https://github.com/inv-ack/inv-ack/blob/7270e64a2600b771f2b1b1b151f7d13fb2ae6c97/bin_inv_ack.v#L222-L228}{\color{blue}binary} inputs in our codebase.

\hide{
	\begin{proof}[Proof Sketch]
		It is easy to prove in a similar fashion to \cref{lem: inv-ack-worker-intermediate} that for all $n, b, k$ and $i$, if $\alpha_i(n) > k$ and $b > i$, then
		\begin{equation*}
		\W\alpha_2\big(\alpha_1, \alpha_1(b), k, b\big) = i + \W\alpha_2\big(\alpha_{i+1}, \alpha_{i+1}(n), k, b - i\big)
		\end{equation*}
		Now let $k \triangleq \lfloor m/n \rfloor$, $b \triangleq \lceil \log_2n \rceil$ and $l \triangleq \min\big\{i : \alpha_i(b)\le k\big\}$, which exists because $\Ack(i, \cdot)$ increases strictly with $i$. Then $\alpha(m, n) = \max{1, l}$. If $l = 0$, $\alpha_1(b) \le \alpha_0(b) \le k$, so $\W\alpha_2\big(\alpha_1, \alpha_1(b), k, b\big) = 0$, as desired. If $l \ge 1$,
		\begin{equation*}
		1 + \W\alpha_2\big(\alpha_1, \alpha_1(b), k, b\big)
		= 1 + l - 1 + \W\alpha_2\big(\alpha_l, \alpha_l(b), k, b-l\big) = l
		\end{equation*}
		Here $b\ge l$ due to the fact that $\Ack(b, k)\ge b$, so $\alpha_b(b)\le k$. This completes the proof.
\end{proof}}%end hide



\subsection{Related Work and Conclusion}

%\newcommand{\ackt}{\ensuremath{\hat{\alpha}}}

\subsection{The value of a linear-time solution to the hierarchy}

Our functions' linear runtimes can be understood in two distinct but
complementary ways.  A runtime less than the bitlength is impossible
without prior knowledge of the size of the input.  Accordingly, in
an information-theory or pure-mathematical sense, our definitions are
optimal up to constant factors.  And of course in practice, linear-time
solutions are highly usable in real computations.

Sublinear solutions are possible with \emph{a priori} knowledge about
the function and bounds on the inputs one will receive.
An extreme case is $\alpha(n)$, which has value $4$ for all practical
inputs greater than $61$. Accordingly,
this function can be inverted in $O(1)$ in practice.  That said, 
such solutions require external knowledge of the problems and
lookup tables within the algorithm to store precomputed
values, and thus fall more into the realm of engineering than mathematics. 

\subsection{The two-parameter inverse Ackermann function}
Some authors~\cite{chazelle,tarjan} prefer a two-parameter inverse Ackermann function.
\begin{defn} \label{defn: 2para-alpha}
	The two-parameter inverse Ackermann function is defined as:
	\begin{equation} \label{eq: tmp-2para-alpha}
	\ackt (m, n) \triangleq \min\left\{i \ge 1 : \Ack\left(i, \left\lfloor \frac{m}{n} \right\rfloor \right)\ge \log_2n \right\}
	\end{equation}
\end{defn}
Note that $\ackt(n, n)$ and the single-parameter $\alpha(n)$
are neither equal nor directly related, but
it is straightforward to modify our techniques to compute $\ackt(m, n)$.
\hide{This function arises from deep runtime analysis of the disjoint-set data structure. Tarjan \cite{tarjan} showed that, in the disjoint-set data structure, the time required $t(m,n)$ for a sequence of $m$ \textsc{\color{magenta}FIND}s intermixed with $n-1$ \textsc{\color{magenta}UNION}s (such that $m \geq n$) is bounded as: $k_{1}m\cdot\alpha(m,n) \leq t(m,n) \leq k_{2}m\cdot\alpha(m,n)$. In graph theory, Chazelle \cite{chazelle} showed that the minimum spanning tree of a connected graph with $n$ vertices and $m$ edges can be found in time $O(m\cdot\alpha(m,n))$. Computing this function is in fact easier than $\alpha(n)$, as when $m$ and $n$ are given, we are reduced to finding the minimum $i\ge 1$ such that $\Ack_i(s)\ge t$ for $s, t$ fixed, which can be done with the following variant of the \emph{inverse Ackermann worker}.
}
\begin{defn} \label{defn: 2para-inv-ack-worker}
	The {two-parameter inverse Ackermann worker}
	,written $\ackt^{\W}$, is a function $\mathbb{N}^4\to \mathbb{N}$, defined by:
	\hide{$(\mathbb{N}\to \mathbb{N}) \times \mathbb{N}^3\to \mathbb{N}$ such that for all $n, k, b\in \mathbb{N}$ and $f:\mathbb{N}\to \mathbb{N}$:}
%	\begin{equation} \label{eq: 2para-inv-ack-worker-recursion}
%	\ackt^{\W}(f, n, k, b) = \begin{cases}
%	0 & \text{if } b = 0 \vee n\le k \\ 1 + \ackt^{\W}\big(\cdt{f}{1}\circ f , \cdt{f}{1}(n), k, b-1\big) & \text{if } b \ge 1 \wedge n \ge k+1
%	\end{cases}
%	\end{equation}
  \begin{equation} \label{eq: 2para-inv-ack-worker-recursion}
  \begin{aligned}
  & \ackt^{\W}(f, n, k, b) \\
  & \triangleq \begin{cases}
  0 & \text{if } b = 0 \vee n\le k \\ 1 + \ackt^{\W}\big(\cdt{f}{1}\circ f , \cdt{f}{1}(n), k, b-1\big) & \text{if } b \ge 1 \wedge n \ge k+1
  \end{cases}
  \end{aligned}
  \end{equation}
\end{defn}
%Similar to the one-parameter version, the following theorem establishes the correct setting for $\W\alpha_2$ to compute $\alpha(m, n)$.
%\begin{thm}
%	$\displaystyle \ackt(m, n) = 1 + \ackt^{\W}\left(\alpha_1, \alpha_1\big(\lceil\log_2n \rceil\big), \left\lfloor \frac{m}{n} \right\rfloor, \lceil\log_2n \rceil \right)$.
%\end{thm}
% Edited by Linh
\begin{thm} For all $m$ and $n$,
	\begin{equation*}
	\displaystyle \ackt(m, n) = 1 + \ackt^{\W}\left(\alpha_1, \alpha_1\big(\lceil\log_2n \rceil\big), \left\lfloor \frac{m}{n} \right\rfloor, \lceil\log_2n \rceil \right).
	\end{equation*}
\end{thm}
We mechanize the above for both \href{https://github.com/inv-ack/inv-ack/blob/7270e64a2600b771f2b1b1b151f7d13fb2ae6c97/inv_ack.v#L245-L248}{\color{blue}unary} and \href{https://github.com/inv-ack/inv-ack/blob/7270e64a2600b771f2b1b1b151f7d13fb2ae6c97/bin_inv_ack.v#L222-L228}{\color{blue}binary} inputs in our codebase.

\hide{
	\begin{proof}[Proof Sketch]
		It is easy to prove in a similar fashion to \cref{lem: inv-ack-worker-intermediate} that for all $n, b, k$ and $i$, if $\alpha_i(n) > k$ and $b > i$, then
		\begin{equation*}
		\W\alpha_2\big(\alpha_1, \alpha_1(b), k, b\big) = i + \W\alpha_2\big(\alpha_{i+1}, \alpha_{i+1}(n), k, b - i\big)
		\end{equation*}
		Now let $k \triangleq \lfloor m/n \rfloor$, $b \triangleq \lceil \log_2n \rceil$ and $l \triangleq \min\big\{i : \alpha_i(b)\le k\big\}$, which exists because $\Ack(i, \cdot)$ increases strictly with $i$. Then $\alpha(m, n) = \max{1, l}$. If $l = 0$, $\alpha_1(b) \le \alpha_0(b) \le k$, so $\W\alpha_2\big(\alpha_1, \alpha_1(b), k, b\big) = 0$, as desired. If $l \ge 1$,
		\begin{equation*}
		1 + \W\alpha_2\big(\alpha_1, \alpha_1(b), k, b\big)
		= 1 + l - 1 + \W\alpha_2\big(\alpha_l, \alpha_l(b), k, b-l\big) = l
		\end{equation*}
		Here $b\ge l$ due to the fact that $\Ack(b, k)\ge b$, so $\alpha_b(b)\le k$. This completes the proof.
\end{proof}}%end hide



The Coq standard library has linear-time definitions
of division and base-$2$ discrete logarithm on \li{nat} and \li{N}.
The Mathematical Components library~\cite{MathComp}
has a discrete logarithm with arbitrary base, with inputs encoded in \li{nat}.
The Isabelle/HOL Archive of Formal Proofs~\cite{isastan2019} 
provides a definition of discrete logarithm
with arbitrary base along with a separate computation strategy.
None of these libraries provides a definition of iterated logarithm or
further members of the hierarchy.
Indeed, to our knowledge, we are the first to generalize this
problem in a proof assistant, extend it both
upwards and downwards in the natural hierarchy of functions, and
provide linear-time computations.

\subsection{Historical notes}
% This creates an unnumbered paragraph. ie a smaller, less flashy, header

%\marginpar{\tiny \color{blue} Multiplication, Division, Algorisms. Representations of numbers (Egyption fractions/Roman numerals/Decimal/Zero). Exponentiation, Logarithm, Tetration, Log*, ...   Hyperoperations, Knuth Arrows.  Inverses as a separate notation? Mechanizations of the above?}
The operations successor, predecessor, addition, and subtraction have
been integral to counting forever. The ancient Egyptian
number system used glyphs denoting $1$, $10$, $100$, \emph{etc.},
and expressed numbers using additive combinations of these.
The Roman system is similar, but
it combines glyphs using both addition and subtraction.
This buys brevity and readability,
since \emph{e.g.} $9_{\text{roman}}$ is two characters, ``one less than ten'',
and not a series of nine $1$s.
The ancient Babylonian system, like the modern Hindu-Arabic decimal system,
enabled \emph{algorisms}: the place value of a glyph determined how many times it
counted towards the number being represented.
The Babylonians operated in
base $60$, and so \emph{e.g.} a three-gylph number $abc_{\text{babylonian}}$ could
be parsed as $a \times 60^2 + b \times 60 + c$. Sadly they lacked
a radix point, and so
$a \times 60^3 + b \times 60^2 + c \times 60$, $a \times 60 + b + c \div 60$,
\emph{etc.} were also reasonable interpretations.
Including multiplication and division bought great concision: a number $n$ was
represented in $\lfloor \log_{60}n \rfloor + 1$ glyphs.

exp, log, the fact that the reverse came later, tetration, log*...

%\subsection*{The Ackermann function and its inverse.}
% This creates an unnumbered subsection

\subsection{The Ackermann function and its inverse}
%\marginpar{\tiny \color{blue} Several variations. Original. Peter.
%Primitive recursive. Hilbert? Ackermann is used in CS. Formalizations that use or define it. The grit of sand. The bug.}
% https://projecteuclid.org/download/pdf_1/euclid.bams/1183512393
% https://www.cs.princeton.edu/~chazelle/pubs/mst.pdf
%{\color{magenta}A brief sentence explaining what a primitive recursive function is and
%why total computable functions tend to be primitive recursive.}
The three-variable Ackermann function was presented by
Wilhelm Ackermann as an example of a total computable function that
is not primitive recursive~\cite{ackermann}.
It does not have the higher-order
relation to repeated application and hyperoperation that we have been studying in
this paper. Those properties emerged thanks to refinements by Rózsa Péter~\cite{peter},
and it is her variant, usually called the Ackermann-Péter function,
that computer scientists commonly care about.

The inverse Ackermann
features in the time bound analyses of several algorithms.
Tarjan~\cite{tarjan} showed that the union-find data structure
takes time $O(m\cdot\alpha(m,n))$ for a sequence of $m$ operations
involving no more than $n$ elements.
Tarjan and van Leeuwen~\cite{tarjan2} later refined this to $O(m\cdot\alpha(n))$.
Chazelle~\cite{chazelle} showed that the minimum spanning tree
of a connected graph with $n$ vertices and $m$ edges
can be found in time $O(m\cdot\alpha(m,n))$.

% http://gallium.inria.fr/~fpottier/publis/chargueraud-pottier-uf-sltc.pdf
% http://gallium.inria.fr/~agueneau/publis/gueneau-chargueraud-pottier-coq-bigO.pdf
% https://scholar.google.com.sg/scholar?start=0&hl=en&as_sdt=2005&sciodt=0,5&cites=6488308509111085774&scipsc=
Chargu\'eraud and Pottier~\cite{charpott}
verified the time complexity of union-find in Coq.
They use another variant
of the Ackermann function (\emph{i.e.} not the Ackermann-Péter function we have studied), for which they provide a computational definition.  On the other hand, they do not provide any function that computes the inverse of the Ackermann variant they use.
%Others \cite{others2,others4,others3,others1}
%have also checked bounds on the resources
%used by programs formally in proof assistants such as Coq, Isabelle/HOL, and Why3.

%Every pearl starts with a grain of sand.  We had the benefit of two:
%Nivasch~\cite{nivasch} and Seidel~\cite{seidel}.
%They proposed a definition of the inverse Ackermann essentially in terms of
%the inverse hyperoperations, \emph{i.e.} along the lines of our Theorem~\ref{thm: inv-ack-new}: $\forall n.~ \alpha(n) \stackrel{?}{=} \Theta(\min\big\{k : 2 \langle k \rangle n\le 3 \big\})$.  Unfortunately, their technique is unsound (it should be $\le k$), since it diverges from
%the true Ackermann inverse when the inputs grow sufficiently large; even adjusting for this, their technique is off by an additive constant.  Our technique is exact and verified in Coq.

\section{Conclusion}
We have implemented a hierarchy of functions that calculate the upper inverses
to the hyperoperation/Ackermann hierarchy and used these inverses
to compute the inverse of the diagonal Ackermann function~$\Ack(n)$.
Our functions run in~$\Theta(b)$ time on unary-represented inputs.  On
binary-represented inputs, our \mbox{base-2} hyperoperations and inverse Ackermann
run in~$\Theta(b)$ time as well, where~$b$ is bitlength; our general binary 
hyperoperations run in~$O(b^2)$.
Our functions are structurally recursive,
and thus easily satisfy Coq's termination checker.

Every pearl starts with a grain of sand.  We had the benefit of two:
Nivasch~\cite{nivasch} and Seidel~\cite{seidel}.
They proposed a definition of the inverse Ackermann essentially in terms of
the inverse hyperoperations.  Unfortunately, their technique is unsound, since it diverges from
the true Ackermann inverse when the inputs grow sufficiently large.  Our technique is verified in Coq.

%, and that it is consistent with
%the usual definition of the inverse Ackermann function $\alpha(n)$.


%{\color{magenta}This is a direction
%we intend to explore to formally check the $O(n)$ bound
%of the inverse Ackermann function.}

%Cite: Ackermann, Peter, Tarjan, Chazelle, Pottier? Anything in HOL? Anything in SSReflect?

%\paragraph*{Alternative strategies}


% Other ways to skin the cat.
% - You can define division via mutual recursion (subtraction and division simultaenously).
% - The inverse ackerman-lite by Anshuman.
% - The automata technique.
% - Binary representations
% - Division by constant, etc. is simpler.
% - Custom termination metrics.  Gas.
% - Space, tail recursion, time?

%\paragraph*{Other?} 

%% The Appendices part is started with the command \appendix;
%% appendix sections are then done as normal sections
%% \appendix

%% \section{}
%% \label{}

%% If you have bibdatabase file and want bibtex to generate the
%% bibitems, please use
%%
%\bibliographystyle{plainurl}% the mandatory bibstyle
\bibliography{inv-ack}

%% else use the following coding to input the bibitems directly in the
%% TeX file.

%%\begin{thebibliography}{00}

%% \bibitem{label}
%% Text of bibliographic item

%%\end{thebibliography}
\end{document}
\endinput
%%
%% End of file `elsarticle-template-num.tex'.
