\subsection{Ackermann function and its inverse}

The time complexity of the union-find data structure has traditionally
been hard to estimate, especially when it is implemented with the 
heuristic rules of \emph{path compression} and \emph{weighted union}. 
Tarjan \cite{tarjan} showed that for a sequence of $m$ FINDs intermixed with $n-1$ UNIONs 
such that $m \geq n$, the time required $t(m,n)$ is bounded
as: $k_{1}m\alpha(m,n) \leq t(m,n) \leq k_{2}m\alpha(m,n)$.

Here $k_{1}$ and $k_{2}$ are positive constants and $\alpha(m,n)$ is 
the inverse of the Ackermann function.

The Ackermann function, commonly denoted $\Ack(m, n)$, was first defined by Ackermann \cite{ackermann}, but this definition was not as widely used as the below variant, given by Peter and Robinson \cite{peter-ackermann}:

\begin{defn} \label{defn: ack}
The Peter-Ackermann function is a recursive two-variable function $\text{A} : \mathbb{N}^2 \to \mathbb{N}$ such that:
\begin{equation}
A(m, n) = \begin{cases}
n + 1 & \text{ if } m = 0 \\
A(m-1, 1) & \text{ if } m > 0, n = 0 \\
A(m-1, A(m, n-1)) & \text{ if } m > 0, n > 0
\end{cases}
\end{equation}
The diagonal Ackermann function is then denoted simply as:
\begin{equation}
\Ack(n) = \Ack(n, n)
\end{equation}
\end{defn}


\begin{defn} \label{defn: inv_ack}
The inverse Ackermann function $\alpha(n)$, as defined by many authors, is the minimum $k$ for which $n \le \Ack(k, k)$:
\begin{equation}
\alpha(n) = \min\left\{k\in \mathbb{N} : n \le \Ack(k, k)\right\}
\end{equation}
\end{defn}

As many texts have suggested, $\Ack(m, n)$ increases extremely fast on both inputs, hence does $\Ack(n, n)$. This implies $\alpha(n)$ increases extremely slow, although it still tends to infinity. However, it does not mean computing $\alpha(n)$ for each $n$ is an easy task. In fact, the naive method would iteratively check $\Ack(k, k)$ for $k = 0, 1, \ldots, $ until $n \le \Ack(k, k)$, which could lead to unimaginably large computation time. For instance, suppose $n > 1$, and $\alpha(n) = k+1$. This is equivalent to

\begin{equation}
\Ack(k, k) < n \le \Ack(k+1, k+1)
\end{equation}

The naive algorithm would need to compute $\Ack{t, t}$ for $t = 0, 1, \ldots, k, k+1$ before terminating. Although one could argue that the total time to compute $\Ack(t, t)$ for $t\le k$ is still $O(n)$, as they are all less than $n$, the time to compute $\Ack(k+1, k+1)$ could be astronomically larger than $n$. This situation motivates the need for an alternative, more efficient approach to compute the inverse Ackermann function.

\subsection{The hierarchy of Ackermann functions}

If one denotes $\text{A}_m(n) = \Ack(m, n)$, one can think of the Ackermann function as a hierarchy of functions, each level $A_m$ is a recursive function built with the previous level $A_{m-1}$:

\begin{defn} \label{defn: ack_hier}
	The Ackermann hierarchy is a sequence of functions $\text{A}_0, \text{A}_1, \ldots $ defined as:
	\begin{enumerate}
		\item $A_0(n) = n + 1 \ \ \ \forall n\in \mathbb{N}$.
		\item $A_m(0) = A_{m-1}(1) \ \ \ \forall m\in \mathbb{N}_{>0}$.
		\item $A_{m}(n) = A_{m-1}^{(n)}(0) \ \ \ \forall m\in \mathbb{N}_{>0}$,
	\end{enumerate}
	\noindent where $f^{(n)}(x)$ denotes the result of applying $n$ times the function $f$ to the input $x$. This hierarchy satisfies $\text{A}_m(n) = \Ack(m, n) \ \ \forall m, n\in \mathbb{N}$.
\end{defn}

This hierarchical perspective can be reversed, as shown in the next section, to form an inverse Ackermann hierarchy of functions, upon which we can compute the inverse Ackermann function as defined in \Cref{defn: ack}.