On $\mathbb{R}$, many functions are bijections and thus have an inverse in a normal sense.  Functions on $\mathbb{N}$ are often non-bijections and thus should be treated differently.

\subsection{Upper Inverses and Expansions}
%; indeed, simply defining an inverse can be a little subtle.
\begin{defn} \label{defn: inverse}
Define the \emph{upper inverse} of $F$, written $F^{-1}_{+}$ as $\min\{m : F(m)\ge n\}$.
Notice that this is well-defined as long as $F$ is unbounded, \emph{i.e.} $\forall b.~\exists a.~ b \leq F(a)$.  However, as a notion of ``inverse,'' it really only makes sense if $F$ is strictly
increasing, \emph{i.e.} $\forall n,m.~ n < m \Rightarrow F(n) < F(m)$, which is in some sense the analogue of injectivity in the discrete domain.
\end{defn}
We call this function the ``upper inverse'' because for strictly increasing functions like
addition, multiplication, and exponentiation, the upper inverse is the ceiling of the 
corresponding inverse functions on $\mathbb{R}$.  {\color{red} It is reasonable to wonder about the floor.
\begin{defn} \label{defn: lower_inverse}
The converse \emph{lower inverse}, written $F^{-1}_{-}$,
is defined as $\max\{m : F(m)\le n\}$.
\end{defn}
Even if $F$ strictly increases, as $a[n]$ does for every $a\ge 2$, notice that the lower inverse will be undefined for $n < F(0)$, \emph{e.g.} $\{m : a[n]m \le 0 \} = \varnothing$ for $n\ge 3$.
Thus we focus on upper inverses (hereafter just ``inverses''), and discuss lower inverses in \cref{sec: discussion}. } Since we focus on strictly increasing functions, we can characterize inverses more meaningfully as follows:
\begin{thm} \label{thm: upp-inverse-rel}
	If $F:\mathbb{N}\to \mathbb{N}$ is {\color{red} strictly} increasing, then $f$ is the upper inverse of $F$ if and only if $\ \forall n, m.~ f(n)\le m \iff n \le F(m)$.
\end{thm}
\begin{proof}
Fix $n$, the sentence $\forall m.~ n\le F(m) \iff f(n)\le m$ implies: (1) $f(n)$ is a lower bound to $\{m: F(m)\ge n \}$ and (2) $f(n)$ is in the set itself since plugging in $m := f(n)$ will yield $n\le F(f(n))$, which makes $f$ the upper inverse of $F$. Conversely, if $f$ is the upper inverse of $F$, we immediately have $\forall m.~n\le F(m)\implies f(n)\le m$. Now for all $m \ge f(n)$, $F(m)\ge F(f(n)) \ge n$ by increasing-ness, thus complete the proof.
\end{proof}
\begin{col}
If $F:\mathbb{N}\to \mathbb{N}$ is strictly increasing, then $F^{-1}_{+} \circ F$ is
the identity function.
\end{col}
\begin{proof}
By ($\Leftarrow$) of \ref{thm: upp-inverse-rel}, $F(n) \le F(n)$ implies 
$(F^{-1}_{+} \circ F)(n) \le n$.  By ($\Rightarrow$), $(F^{-1}_{+} \circ F)(n) \le (F^{-1}_{+} \circ F)(n)$ implies $F(n) \le F \big((F^{-1}_{+} \circ F)(n)\big)$; $F$ is strictly increasing, so $n \le (F^{-1}_{+} \circ F)(n)$.
\end{proof}

Our setup for inverse requires increasing functions, and our definitions for 
hyperoperations/Ackermann use repeater.  Suppose $F$ is a strictly increasing function.
For a given $a$, is $\rf{F}{a}$ strictly increasing?  No!  For example, the identity function
$\mathsf{id}$ is strictly increasing, but $\rf{\mathsf{id}}{a}(n) = (\mathsf{id} \circ \ldots \circ \mathsf{id}) (a) = a$ is a constant function.  We need a little more.
\begin{defn}
Given $a\in \mathbb{N}$, a function $F:\mathbb{N}\to\mathbb{N}$ is an \emph{expansion} if $\forall n.~ F(n)\ge n$. An expansion $F$ is \emph{strict from} $a$ if $\forall n \ge a.~ F(n)\ge n+1$.
\end{defn}
If $a\ge 1$ and $F$ is an expansion strict from $a$, $\forall n.~ \rf{F}{a}(n) = F^{(n)}(a) \ge a + n \ge 1 + n$, so $\rf{F}{a} \ $ is itself an expansion strict from $0$. We refer to strictly increasing $f$ as \emph{repeatable} from $a\ge 1$ if they are also strict expansions from $a$, so that repeatability is preserved through $\rf{F}{a}$.
\begin{defn}
We denote the set of functions repeatable from $a$ as $\repeatable_a$.
\end{defn}
\begin{rem} \label{rem: repeatable-subset}
	It is trivial to see that $\forall s, t.~ s \le t \Rightarrow \repeatable_s \subseteq \repeatable_t $.
\end{rem}
{\color{red}
\begin{rem}
Which hypers are in Rept?
\end{rem}
}

\subsection{Contractions and the countdown operation}

Suppose that $F \in \repeatable_a$ for any $a \ge 1$ and let $f$ be $F$'s inverse, \emph{i.e.} $F^{-1}_{+}$.  Our goal is to use $f$ to compute an inverse to $F$'s repeater $\rf{F}{a}$.  Notice that this inverse must exist since $F \in \repeatable_a$ implies $\rf{F}{a} \in \repeatable_0$.  
For reasons that will be clear momentarily, we write this inverse as $\cdt{f}{a}$.  Now
fix $n$ and observe that for all $m$, $f^{(m)}(n)\le a \iff m \ge \cdt{f}{a}(n)$ since
\begin{equation} \label{eq: rf-upp-inv}
\begin{aligned}
\cdt{f}{a}(n)\le m & \iff n\le \rf{F}{a}(m) = F^{(m)}(a) \iff f(n)\le F^{(m-1)}(a) \\
& \iff f^{(2)}(n)\le F^{(m-2)}(a) \iff \ldots \iff f^{(m)}(n)\le a
\end{aligned}
\end{equation}
Moreover, setting $m = \cdt{f}{a}(n)$, we realize that $f^{(\cdt{f}{a}(n))}\le a$.  
\textbf{Together these imply that $\cdt{f}{a}(n)$ is the minimum number of times $f$ 
needs to be compositionally applied to $n$ before equalling or passing $a$.} 
In other words, count the length of the chain $\{n, f(n), f^{(2)}(n), \ldots\}$ that 
terminates as soon as we reach/pass $a$.  For this process to work we need each chain link
to be strictly less than the previous, \emph{i.e.} $f$ must be a \emph{contraction}.
\begin{defn} \label{defn: contracting}
	A function $f : \mathbb{N} \to \mathbb{N}$ is a \textit{contraction} if $\forall n.~ f(n) \le n$. Given an $a \ge 1$, a contraction $f$ is \textit{strict above} $a$ if $\forall n > a.~ n\ge f(n)+1$. We denote the set of contractions by $\contract$ and the set of contractions strict above $a$ by $\contract_a$.
\end{defn}
\begin{rem}
	Similar to \cref{rem: repeatable-subset}, $\forall s\le t.~ \contract_s \subseteq \contract_t$.
\end{rem}
What kinds of functions have contractive inverses? Expansions, naturally:
\begin{thm} \label{thm: expansion-inv-contraction}
For all $a\in \mathbb{N}$, $F\in \repeatable_a \implies F^{-1}_+ \in \contract_a$.
\end{thm}
\begin{proof}
For all $n$, $F(n)\ge n \implies n \ge F^{-1}_+(n)$, so $F^{-1}_+$ is a contraction. If $n\ge a+1$, $n-1\ge a$, so $F(n-1)\ge n \implies n-1\ge F^{-1}_+(n)$, so $F^{-1}_+$ is strict above $a$.
\end{proof}
\Cref{thm: expansion-inv-contraction} shows a clear inverse relationship between expansions strict from some $a$ and contractions strict above that same $a$. It ensures that the inverse of an expansion's repeater not only exists but can be built from its own inverse, in a method formalized as \emph{countdown}.
\begin{defn} \label{defn: informal-countdown} \label{eq: countdown}
Let $f\in \contract_a$, the \textit{countdown to} $a$ of $f$, denoted by $\cdt{f}{a}(n)$, is the minimum number of times $f$ needs to be applied to $n$ to reach/pass $a$: 
$\min\{m: f^{(m)}(n)\le a \}$.
\end{defn}
Inspired by \eqref{eq: rf-upp-inv}, we provide a neat, algebraically manipulable logical sentence equivalent to \eqref{eq: countdown}, which is more useful later in our paper.
\begin{col} \label{col: cdt-repeat}
Let $a \in \mathbb{N}$ and $f\in \contract_{a}$.  Then $\forall n, m.~ \cdt{f}{a}(n)\le m \iff f^{(m)}(n)\le a$.
\end{col}
\begin{proof}
	Fix $a$ and $n$. The interesting direction is $(\!\!\implies\!\!)$. Suppose $\cdt{f}{a}(n)\le m$, we get $f^{(m)}(n)\le f^{(\cdt{f}{a}(n))}(n)$ due to $f\in \contract$, and $f^{(\cdt{f}{a}(n))}(n)\le a$ due to \cref{eq: countdown}.
\end{proof}
%\begin{thm} \label{thm: upp-inv-cdt-rf}
%	For all $a\in \mathbb{N}$, if $f\in \contract_{a}$ is the upper inverse of $F: \mathbb{N}\to \mathbb{N}$, then $\cdt{f}{a}$ is the upper inverse of $\rf{F}{a}$.
%\end{thm}
Another useful result is the recursive formula for \emph{countdown}.
\begin{thm} \label{thm: cdt-recursion}
	For all $a\in \mathbb{N}$ and $f\in \contract_{a}$, $\cdt{f}{a}$ satisfies:
	\begin{equation*}
	\cdt{f}{a}(n) = \begin{cases}
	0 & \text{ if } n \le a \\ 1 + \cdt{f}{a}(f(n)) & \text{ if } n \ge a + 1
	\end{cases}
	\end{equation*}
\end{thm}
\begin{proof}
By \cref{col: cdt-repeat}, $n\le a \iff f^{(0)}(n)\le a \iff \cdt{f}{a}(n)\le 0$, thus the case $n\le a$ is resolved. Suppose $n\ge a+1$ and let $\cdt{f}{a}(f(n)) = m$. We have $\cdt{f}{a}(n)\le 1+m \!\! \iff \!\! f^{(1+m)}(n) \le a$, which is equivalent to $f^{(m)}(f(n)) \le a$, which holds by $m$'s definition.

Now since $n\ge a+1$, $\cdt{f}{a}(n)\ge 1$ by the above. Let $p$ be \emph{s.t.} $\cdt{f}{a}(n) = p+1$. It remains to prove $\cdt{f}{a}(f(n))\le p$, or $f^{(p)}(f(n))\le a$, or $f^{(p+1)}(n)\le a$, which holds by $p$'s definition.
\end{proof}

%\begin{thm} \label{thm: cdt-contr-0}
%	For all $a\ge 1$, if $f\in \contract_{a}$, then $\cdt{f}{a}\ \in \contract_{c} \ \forall c$.
%\end{thm}
%\begin{proof}
%	Firstly we show that $\cdt{f}{a}$ is a contraction, namely showing $\cdt{f}{a}(n)\le n \ \forall n$, which has already been proved in the proof of \cref{thm: cdt-repeat}. To show $\cdt{f}{a}$ is strict from $1+c$, it suffices to show it is strict from $1$, equivalently $\cdt{f}{a}(n) \le n - 1 \ \forall n\ge 1$. By \cref{thm: cdt-repeat}, we need to show $f^{(n-1)}(n)\le a$. Assume the converse, then:
%	$$ n \ge 1 + f(n)\ge 2 + f(f(n)) \ge \cdots \ge (n-1) + f^{(n-1)}(n) $$
%	Thus $f^{(n-1)}(n)\le 1 \le a$, a contradiction. The theorem then follows.
%\end{proof}

\subsection{A Structurally Recursive Computation for Countdown}

While a repeater exists for all functions not necessarily repeatable, and hence can be written in Coq without much trouble, a \emph{countdown} only exists for certain functions, most conveniently contractions, hence proves a little more challenging for Coq. In the next section we give a formal computation using a \emph{countdown worker}, which terminates by Coq standards for all functions, while successfully produces the countdown when given a contraction.


\begin{defn} \label{defn: countdown-worker}
For any $a\in \mathbb{N}$ and $f: \mathbb{N}\to \mathbb{N}$, the \emph{countdown worker} to $a$ of $f$ is a function $\W\cdt{f}{a}\ : \mathbb{N}^2\to \mathbb{N}$ such that:
\begin{equation*}
\W\cdt{f}{a}(n, b) = \begin{cases}
0 & \text{if } b = 0 \vee n\le a \\ 1 +\W\cdt{f}{a}(f(n), b-1) & \text{if } b \ge 1 \wedge n > a
\end{cases}
\end{equation*}
\end{defn}

Essentially, \emph{countdown worker} operates on two arguments, the \emph{true argument} $n$, which we wish to count down to $a$, and the \emph{budget} $b$,
the maximum number of times we attempt to compositionally
apply $f$ on the input before giving up. If the input goes below or equal $a$ after $k$ applications, \emph{i.e.} $f^{(k)}(n) \le a$, we return $k$ as the true countdown value. If the budget is exhausted, i.e. $b = 0$, while the result is still above $a$,
we fail by returning the original budget. This defnition is somewhat clunky, but it can clearly be written as a Coq fixpoint, with the budget as the decreasing
argument.
\begin{lstlisting}
Fixpoint countdown_worker a (f: nat->nat) n k :=
match k with
| 0    => 0
| S k' => match (n - a) with
          | 0 => 0
          | _ => S (countdown_worker a f (f n) k') end
end.
\end{lstlisting}
It is clear that with a sufficient budget, \emph{countdown worker} should compute the correct \emph{countdown} value when $f$ is a contraction strict from $a$. We will show that a budget of $n$ is sufficient, using several lemmas about \emph{countdown worker}. Let us use following Coq-compatible definition of \emph{countdown}, which is also used in our code base.
\begin{defn} \label{defn: countdown}
In this section, define  $\cdt{f}{a}(n) = \W\cdt{f}{a}(n, n) \ \forall n$.
\begin{lstlisting}
Definition countdown_to a f n := countdown_worker a f n n.
\end{lstlisting}
\end{defn}
%Before beginning, let us clarify that the definition of $\mathbb{N}$ and operations on $\mathbb{N}$ in Gallina follow the Presburger Arithmetic \cite{presburger}, which despite being weaker than Peano Arithmetic, is a decidable theory. The Coq standard library includes \texttt{Omega}, an extensive listing of provable facts about $\mathbb{N}$ in Presburger Arithmetic, including everything used in this paper, most notably the law of excluded middle for comparisons on $\mathbb{N}$:
%\begin{equation*}
%(n \le m) \vee (m + 1 \le n) \ \ \forall n, m
%\end{equation*}
%, which is provable without the actual law of excluded middle in classical logic. This enables us to prove all results in this paper with Coq's baseline intuitionistic logic. Readers can refer to the \Cref{appendix} for Coq versions of the proofs.
%in which the most operations agree with usual operations on $\mathbb{Z}$, except subtraction, which is defined as:
%\begin{lstlisting}
%Fixpoint sub (n m : nat) : nat :=
%match n with
%| 0 => n
%| S k => match m with
%| 0 => n
%| S l => sub k l end
%end.
%\end{lstlisting}
%Essentially $\li{sub} \ n \ m = \max\{n - m, 0\}$. We will use this subtraction for the rest of the paper.

%Firstly, we begin with a lemma asserting the existence of the countdown value itself. Although its existence is guaranteed by the well-ordering principle of $\mathbb{N}$, we will achieve better by proving it in intuitionistic logic.
%
%\begin{lem} \label{lem: contract-repeat-threshold}
%	For all $a, n\in\mathbb{N}$ and $f\in \contract_{a}$,
%	\begin{equation}
%	\exists m : \left(f^{(m)}(n) \le a \right) \wedge \left(f^{(l)}(n)\le a \implies m \le l \ \forall l \right)
%	\end{equation}
%\end{lem}
%\begin{proof}
%  Fix $n$ and observe that if $n\le a$, $m = 0$ is the desired choice since $ f^{(0)}(n) = n \le a \ \text{ and } \ 0 \le l \ \forall l $.
%	Consider only when $a\le n$, we can define $c$ such that $n = a + c$. We prove the following statement by induction:
%	\begin{equation*}
%	P(c) \triangleq \exists m : \left(f^{(m)}(n) \le n - c \right) \wedge \left(f^{(l)}(n)\le n - c \implies m \le l \ \forall l \right)
%	\end{equation*}
%	under assumptions $f\in \contract_{n-c}$ and $c\le n$.
%	\begin{enumerate}[leftmargin=*]
%		\item \textit{Base case.} The case $c = 0$ implies $n = a$, which has been proven above.
%		\item \textit{Inductive step.} Suppose $P(c)$ is proved with witness $m_c$. Note that the assumptions are now $f\in \contract_{n-c}$ and $c+1\le n$, there are two cases:
%		\begin{itemize}[leftmargin=*, label={--}]
%			\item $f^{\left(m_c\right)}(n) = n - c$. Then $f^{\left(m_c+1\right)}(n) \le n - c - 1$. Let $m_{c+1} = m_c + 1$, for all $l$:
%			\begin{equation*}
%			f^{(l)}(n)\le n - c - 1 < f^{\left(m_c\right)}(n) \implies l > m_c \implies l \ge m_{c+1}
%			\end{equation*}
%			\item $f^{\left(m_c\right)}(n) \le n - c - 1$. Let $m_{c+1} = m_c$, for all $l$:
%			\begin{equation*}
%			f^{(l)}(n)\le n - c - 1\le n - c \overset{P(c)}{\implies} l\ge m_{c} = m_{c+1}
%			\end{equation*}
%		\end{itemize}
%		In any cases, we can find a witness $m_{c+1}$ for $P(c+1)$. Thus the proof is complete by induction.\vspace*{-\baselineskip}
%	\end{enumerate}
%\end{proof}

We start with a simple fact that \emph{countdown worker} returns $0$ when $n\le a$, which is intended for \emph{countdown}. It follows trivially from \cref{defn: countdown-worker}.

\begin{lem} \label{lem: cdt-init}
	For all $a, b\in \mathbb{N}$, $f : \mathbb{N}\to \mathbb{N}$ we have $\W\cdt{f}{a} (n, b) = 0 \ \forall n\le a$.
\end{lem}

The next lemma shows the internal working of \emph{countdown worker} at the $\text{i}^\text{th}$ recursive step, including the accumulated result $1+i$, the current input $f^{(1+i)}(n)$, and the current budget $b-i-1$.

\begin{lem} \label{lem: cdt-intermediate}
	For all $a, n, b, i\in \mathbb{N}$ and $f \in \contract$ such that $i < b$ and $a < f^{(i)}(n)$:
	\begin{equation}  \label{eq: cdt-intermediate}
	\W\cdt{f}{a}(n, b) = 1 + i + \W\cdt{f}{a}\left(f^{1+i}(n), b - i - 1\right)
	\end{equation}
\end{lem}
\begin{proof}
	We proceed by induction on $i$. Define
	\begin{equation*}
	P(i) \triangleq \left(\W\cdt{f}{a}(n, b) = 1 + i + \W\cdt{f}{a}\left(f^{1+i}(n), b - i - 1\right) \ \forall b, n : b\ge i+1, f^{(i)}(n) > a\right)
	\end{equation*}
	\begin{enumerate}[leftmargin=*]
		\item \textit{Base case.} For $i = 0$, our goal $P(0)$ is:
		$\W\cdt{f}{a}(n, b) = 1 + \W\cdt{f}{a}\left(f(n), b - 1\right)$
		where $b \ge 1, f(n)\ge a+1$, which is trivial.
		\item \textit{Inductive step.} Suppose $P(i)$ has been proven. Then
		\begin{equation*}
		P(i+1) \triangleq \W\cdt{f}{a}(n, b) = 2 + i + \W\cdt{f}{a}\left( f^{2+i}(n), b - i - 2 \right)
		\end{equation*}
		for $b \ge i+2, f^{1+i}(n) \ge a+1$. This also implies $b\ge i+1$ and $\displaystyle f^{(i)}(n) \ge f^{1+i}(n)\ge a+1$ by $f\in \contract$, thus $P(i)$ holds. It suffices to prove:
		\begin{equation*}
		\W\cdt{f}{a}\left(f^{1+i}(n), b-i-1\right) = 1 + \W\cdt{f}{a}\left( f^{2+i}(n), b-i-2 \right)
		\end{equation*}
		This is in fact $P(0)$ with $(b, n)$ substituted for $\left(b-i-1, f^{(1+i)}(n)\right)$. Since $f^{(1+i)}(n) \ge a+1$ and $b-i-1\ge 1$, the above holds and $P(i+1)$ follows. The proof is complete.\vspace*{-\baselineskip}
	\end{enumerate}
\end{proof}
Now it is time to prove the correctness of \emph{countdown worker}.

\begin{thm} \label{thm: cdt-repeat}
	For all $a\in \mathbb{N}$ and $f\in \contract_{a}$, we have
	\begin{equation} \label{eq: cdt-minimum}
	\cdt{f}{a}(n) = \min\left\{ i : f^{(i)}(n) \le a \right\} \ \ \forall n
	\end{equation}
%	Or equivalently,
%	\begin{equation} \label{eq: cdt-repeat}
%	\cdt{f}{a}(n) \le k \iff f^{(k)}(n) \le a \ \ \forall n, k
%	\end{equation}
\end{thm}
\begin{proof}
%	First, to see why \eqref{eq: cdt-minimum} and \eqref{eq: cdt-repeat} are equivalent, we rewrite \eqref{eq: cdt-minimum} in the following way:
%	$$ \left(f^{(\cdt{f}{a}(n))}(n) \le a\right) \wedge \left(f^{(l)}(n) \le a \implies \cdt{f}{a}(n) \le l \ \ \forall l\right) \ \ \forall n$$
%	To prove $\eqref{eq: cdt-minimum} \implies \eqref{eq: cdt-repeat}$, it suffices to show
%	$$ \cdt{f}{a}(n) \le l \implies f^{(l)}(n) \le a \ \ \forall l $$
%	, which holds due to the fact $\displaystyle f^{(l)}(n) \le f^{(\cdt{f}{a}(n))}(n) \le a$ by $f\in \contract$. To prove $\eqref{eq: cdt-repeat}\implies \eqref{eq: cdt-minimum}$, it suffices to show $\displaystyle f^{(\cdt{f}{a}(n))}(n) \le a$
%	, which in turn holds by substituting $k$ by $\cdt{f}{a}(n)$ in \eqref{eq: cdt-repeat}. Thus $\eqref{eq: cdt-minimum}\iff \eqref{eq: cdt-repeat}$ and we need only to prove \eqref{eq: cdt-minimum}.
Since $f\in \contract_{a}$ and $\mathbb{N}$ is well-ordered, let $m = \min\big\{i : f^{(i)}(n)\le a\big\}$ (we are also able prove its existence in Coq's baseline intuitionistic logic in our code base), then
	\begin{equation}
	\left(f^{(m)}(n) \le a\right) \label{eq: cdt-repeat-tmp} \wedge
	 \left(f^{(k)}(n)\le a \implies m \le k \ \ \forall k\right)
	\end{equation}
	It then suffices to prove $m = \cdt{f}{a}(n)$. Suppose firstly that $m = 0$. Then $n = f^{(0)}(n)\le a$, thus $\cdt{f}{a}(n) = \W\cdt{f}{a}(n, n) = 0 = m$ by \cref{lem: cdt-init}.
	
	Now consider $m > 0$. We would like to apply \cref{lem: cdt-intermediate} to get
	\begin{equation*}
	\cdt{f}{a}(n) = \W\cdt{f}{a}(n, n) = m + \W\cdt{f}{a}\left(f^{(m)}(n), n-m\right)
	\end{equation*}
	, then use \cref{lem: cdt-init} over \eqref{eq: cdt-repeat-tmp}'s first conjunct to conclude that $\cdt{f}{a}(n) = m$. It then suffices to prove the premises of \cref{lem: cdt-intermediate}, namely $a < f^{(m-1)}(n)$ and $m-1 < n$.
	
	The former follows by contradiction: if $f^{(m-1)}(n) \le a$, \eqref{eq: cdt-repeat-tmp}'s second conjunct implies $m\le m-1$, which is wrong for $m > 0$. The latter then easily follows by $f\in \contract_{a}$:
	\begin{equation*}
	n \ge 1 + f(n) \ge 2 + f(f(n)) \ge \cdots \ge m + f^{(m)}(n)
	\end{equation*}
	Therefore, $\cdt{f}{a}(n) = m$ in all cases, which completes the proof.
\end{proof}
Now, \eqref{eq: rf-upp-inv} and \cref{thm: cdt-repeat} are already enough to establish the correctness of the Coq definitions \emph{countdown} and \emph{countdown worker}. We thus unite \cref{defn: informal-countdown,defn: countdown} as they are equivalent. We wrap everything up in the following two theorems, one for $a\ge 1$ and one for $a = 0$.
\begin{thm} \label{thm: cdt-inv-rf}
	For all $a$ and $F\in \repeatable_a$, $f\triangleq F^{-1}_+$ satisfies $f \in \contract_{a}$ and $\displaystyle \cdt{f}{a} \ = \left(\rf{F}{a}\ \right)^{-1}_+$. Furthermore, if $a\ge 1$, $\rf{F}{a}\ \in \repeatable_0$ and $\cdt{f}{a}\ \in \contract_0$.
\end{thm}
\begin{proof}
	By \cref{thm: expansion-inv-contraction}, $f\triangleq F^{-1}_+ \in \contract_a$, and $\cdt{f}{a}\ = \left(\rf{F}{a}\ \right)^{-1}_+$ follows from \eqref{eq: rf-upp-inv} and \cref{col: cdt-repeat}.
	Now if $a\ge 1$, a simple induction shows that $F^{(n)}(a)\ge a + n\ge 1 + n \ \forall n$, so $\rf{F}{a}\ \in \repeatable_0$, hence $\cdt{f}{a} \ = \left(\rf{F}{a}\ \right)^{-1}_+ \in \contract_0$ by \cref{thm: expansion-inv-contraction}.
\end{proof}

