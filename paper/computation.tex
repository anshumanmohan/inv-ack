The main idea to compute the inverse Ackermann function using what we have established so far is to use \Cref{thm: inv_ack_ack} to iteratively compute each level in the hierarchy starting from input $n+2$, then stop when the condition $\alpha_m(n+2) \le m + 3$ is met.

In order for this computation to run in linear time, we need to first make sure \textit{each} level $\alpha_m(n+2)$ is computed in linear time. We will then use a trick to achieve the linear time bound in the total computation using the relation
\begin{equation}
\alpha_m(n+2) = 1 + \alpha_m(\alpha_{m-1}(n+2)) \ \ \ \forall m\in \mathbb{N}_{>0}
\end{equation}

\subsection{Inverse Ackermann hierarchy in linear time}  \label{subsec: inv_ack_hier linear}

First note that, in the Gallina specification, all natural numbers are represented with a string of $\S$, the successor notation. All recursive functions in Gallina are required to decrease on one of its inputs, one or a few successors per recursive step. Thus all recursive functions, or Fixpoints in Gallina, must run in at least linear time over one of their inputs.

To compute the inverse Ackermann hierarchy for some input $n$ with a Gallina-complied function, we look at its generalization: Given a contraction $f$ over $\mathbb{N}$ and a Gallina function $\widetilde{f}$ computing $f$, we find a Gallina function $\widetilde{f^*}$ to compute its countdown, $f^*$.

\begin{defn} \label{defn: countdown rec helper}
Let $\mathcal{F}_k$ be the set of all Gallina functions $g: \mathbb{N}^k \to \mathbb{N}$. The \textit{countdown recursor helper} is an operator $\CRH : \mathcal{F}_1 \to \mathcal{F}_3$ such that for all $g\in \mathcal{F}_1$ and $n_0, n_1, c\in \mathbb{N}$:
\begin{equation}
\begin{aligned}
& \CRH\left(g\right)\left(n_0, n_1, c\right) = \\
& \begin{cases}
0 & \text{if } n_0 \le 1. \\
1 & \text{if } n_0 \ge 2, n_1 \le 1. \\
1 + \CRH \left(g\right)
		\left(n_0-1, g(n_1), n_1 - g(n_1) - 1\right)
		& \text{if } n_0\ge 2, n_1 \ge 2, c = 0. \\
\CRH \left(g\right)
			\left(n_0-1, n_1, c-1\right)
		& \text{if } n_0\ge 2, n_1\ge 2, c\ge 1.
\end{cases}
\end{aligned}
\end{equation}
\end{defn}

It is trivial to see that for all $g\in \mathcal{F}$, $\CRH(g)$ is a Gallina Fixpoint (a Gallina-complied recursive function) in $\mathcal{F}_3$ since its second input $n_0$ decreases by $1$ at every recursive step.

\begin{defn} \label{defn: countdown rec}
The \textit{countdown recursor} is an operator $\CR : \mathcal{F}_1 \to \mathcal{F}_1$ such that for all $g\in \mathcal{F}_1$ and $n\in \mathbb{N}$:
\begin{equation}
\CR(g)(n) = \CRH(g)(n, g(n), n - g(n) - 1)
\end{equation}
\end{defn}

Since it is built by a composition of a Gallina Fixpoint $\CRH$ and a Gallina function $g$, $\CR(g)$ is indeed a Gallina function in $\mathcal{F}_1$. We prove that $\CR$ is the equivalence of the countdown operation in Gallina for contractions:

\begin{lem} \label{lem: CRH_countdown}
Let $f: \mathbb{N}\to \mathbb{N}$ be a contraction. Suppose a function $\widetilde{f}\in \mathcal{F}_1$ computes $f$, then $\CR\left(\widetilde{f}\right)$ is a function in $\mathcal{F}_1$ computing $f^*$.
\end{lem}

\begin{proof}
TODO TODO TODO
\end{proof}

With this lemma, we can define the equivalence of the inverse Ackermann hierarchy in Gallina:

\begin{defn}
The Gallina inverse Ackermann hierarchy is a sequence of functions $\wt{\alpha}_0, \wt{\alpha}_1, \ldots $ such that for all $m, n\in \mathbb{N}$:
\begin{equation}
\wt{\alpha}_m(n) = \begin{cases}
n - 2 & \text{if } m = 0 \\ \CR\left(\wt{\alpha}_{m-1}\right)(n) & \text{if } m \ge 1
\end{cases}
\end{equation}
Note that $x-y$ in Gallina is equivalent to $\max\{x-y, 0\}$ in practice.
\end{defn}

\Cref{lem: CRH_countdown} and \Cref{thm: inv_ack_countdown} trivially implies that $\wt{\alpha}_m$ is a Gallina computation of $\alpha_m$ for all $m\in \mathbb{N}$.

The important thing to come up with Gallina computations for the hierarchy is we want to compute them in linear time. We assert the hierarchy $\left\{\wt{\alpha}_m\right\}$ succeeds in doing so:

\begin{thm}  \label{thm: inv_ack_hier linear}
For each $m \in \mathbb{N}$, computing $\wt{\alpha}_m(n)$ takes at most $(m + o(1))n$ steps, where $n$ tends to infinity.
\end{thm}

\begin{proof}
TODO TODO TODO
\end{proof}

\subsection{Inverse Ackermann in linear time}  \label{subsec: inv_ack linear}

As mentioned, the task is to find the minimum $x$ for which $\alpha_x(n) \le x + 3$ for $n\ge 4$. It is tempting to go for a naive approach, after all the efficiencies we have developed so far: Starting from $x=0$, we iteratively compute $\alpha_x(n)$ and compare it with $x+3$. However, by our earlier analysis in \Cref{thm: inv_ack_hier linear}, the total amount of time needed is

$$ T(n) = \sum_{m = 0}^{\alpha(n)-1}(m + o(1))n = O\left(n\alpha(n)^2\right) $$

which is pretty efficient, given how slow-growing $\alpha(n)$ is. However, our ultimate goal is $O(n)$ time, thus we came up with a better approach. Interestingly, it is also based on \Cref{thm: inv_ack_hier linear}. Specifically, using \eqref{eq: countdown}, \Cref{thm: inv_ack_hier linear} implies we can actually compute $\alpha_m(n)$ in $O(\alpha_{m-1}(n))$ time, \textit{if} $\alpha_{m-1}(n)$ is given. It is thus beneficial if we retain the value of $\alpha_{m-1}(n)$ when computing $\alpha_m(n)$. The definition below is our Gallina recursor for this.

\begin{defn}  \label{defn: inv_ack_recursor_helper}
The inverse Ackermann recursor helper is a function $\IARH : \mathcal{F}_1\times \mathbb{N}^3 \to \mathbb{N}$ such that for all $g\in \mathcal{F}_1, n, c_0, c\in \mathbb{N}$, we have:
\begin{equation}
\begin{aligned}
& \IARH(g, n, c_0, c) =  \\
& \begin{cases}
c & \text{if } c = 1 \\
1 + \IARH\left(\CR(g), n, n - \CR(g)(n) - 1, c - 1\right) & \text{if } c \ge 1, c_0 = 0 \\
\IARH\left(g, n - 1, c_0 - 1, c - 1\right) & \text{if } c \ge 1, c_0 \ge 1
\end{cases}
\end{aligned}
\end{equation}
Note that $n - 1$ in Gallina means $\max\{n-1, 0\}$ here.
\end{defn}

\begin{defn}  \label{defn: inv_ack_recursor}
The inverse Ackermann recursor is a function $\IAR \in \mathcal{F}_1$ such that for all $n\in \mathbb{N}$, we have
\begin{equation}
\IAR(n) = \begin{cases}
0 & \text{if } n \le 1 \\ \IARH\left(\alpha_0, n + 2, 2, n\right) & \text{if } n\ge 2
\end{cases}
\end{equation}
\end{defn}

The following theorem is a central result in this paper, which asserts the correctness of $\IAR$.

\begin{thm} \label{thm: inv_ack_correct}
$\IAR$ is a Gallina function computing $\alpha(n)$.
\end{thm}

It is trivial to see that both $\IARH$ and $\IAR$ are Gallina functions. It suffices to show that $\IAR(n) = \alpha(n)$ for $n\ge 2$, since their values already match for $n\le 1$. Furthermore, if we trace the first two recursive steps in $\IARH$, we obtain
\begin{equation}
\IARH(n+2, \alpha_0, 2, n) = 1 + \IARH(\alpha_1, n-2, n-\alpha_1(n)-1, n-3) \ \ \forall n\ge 2
\end{equation}
It then suffices to show the following:

\begin{lem}
For all $n\ge 2$, $\IARH(\alpha_1, n, n - \alpha_1(n) - 1, n - 3) = \min\left\{m : \alpha_m(n+2)\le m+3 \right\}$.
\end{lem}