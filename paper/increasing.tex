\subsection{Ackermann function and its inverse}

\marginpar{the old intro material is here} The time complexity of the

has traditionally
been hard to estimate, especially when it is implemented with the
heuristic rules of \emph{path compression} and \emph{weighted union}.
Tarjan \cite{tarjan} showed that for a sequence of $m$ FINDs
intermixed with $n-1$ UNIONs
such that $m \geq n$, the time required $t(m,n)$ is bounded
as: $k_{1}m\alpha(m,n) \leq t(m,n) \leq k_{2}m\alpha(m,n)$.
Here $k_{1}$ and $k_{2}$ are positive constants and $\alpha(m,n)$ is
the inverse of the Ackermann function.

The Ackermann function, commonly denoted $\Ack(m, n)$,
was first defined by Ackermann \cite{ackermann}, but this
definition was not as widely used as the following variant,
given by Peter and Robinson \cite{peter-ackermann}:




$\Ack(m, n)$ increases extremely fast on both inputs,
and, consequently, so does $\Ack(n, n)$.
This implies $\alpha(n)$ increases extremely slowly,
although it still tends to infinity. However, this does
not mean that computing $\alpha(n)$ for each $n$ is an easy
task. In fact, the naive method would iteratively check
$\Ack(k, k)$ for $k = 0, 1, \ldots, $ until $n \le \Ack(k, k)$.
The computation time would be staggering.
For instance, suppose $n > 1$, and $\alpha(n) = k+1$.
This is equivalent to

\begin{equation}
\Ack(k, k) < n \le \Ack(k+1, k+1)
\end{equation}

The naive algorithm would need to compute
$\Ack(t, t)$ for $t = 0, 1, \ldots, k, k+1$ before terminating.
{\color{magenta} Although one could argue that the total time to
compute $\Ack(t, t)$ for $t\le k$ is still $O(n)$, as they are all
less than $n$, the time to compute $\Ack(k+1, k+1)$ could be
astronomically larger than $n$.} This situation motivates the
need for an alternative, more efficient approach for computing
the inverse Ackermann function.

\subsection{The hierarchy of Ackermann functions}

If one denotes {\color{magenta}$\text{A}_m(n) = \Ack(m, n)$}, one can think
of the Ackermann function as a hierarchy of functions, each
level $A_m$ is a recursive function built with the previous
level $A_{m-1}$:

\begin{defn} \label{defn: ack_hier}
	The Ackermann hierarchy is a sequence of functions
	$\text{A}_0, \text{A}_1, \ldots $ defined as:
	\begin{enumerate}
		\item $A_0(n) = n + 1 \ \ \ \forall n\in \mathbb{N}$.
		\item $A_m(0) = A_{m-1}(1) \ \ \ \forall m\in \mathbb{N}_{>0}$.
		\item $A_{m}(n) = A_{m-1}^{(n)}(0) \ \ \ \forall n, m\in \mathbb{N}_{>0}$,
	\end{enumerate}
	\noindent where $f^{(n)}(x)$ {\color{magenta}denotes
	the result of applying $n$ times the function $f$ to
	the input $x$.} This hierarchy satisfies
	$\text{A}_m(n) = \Ack(m, n) \ \ \forall m, n\in \mathbb{N}$.
\end{defn}

This hierarchical perspective can be reversed, as shown in the
next section, to form an inverse Ackermann hierarchy of functions,
upon which we can compute the inverse Ackermann function as
defined in \Cref{defn: ack}. 

By extend, we mean to other members of the Ackermann/Knuth/hyperoperation hierarchy.  By this we
mean to the sequence of function families $f^a_0, f^a_1, \ldots, f^a_n, \ldots$, where $f^a_0(x) \triangleq x + 1$ and, for each $i > 0$, $f^a_i(x)$ is the result of iterating $f^a_{i-1}$ $x$ times
on $a$,
\[
\begin{array}{cclcl|c}
f^a_0(b) & \triangleq & 1 + \underbrace{1 + \ldots + 1}_b & = & b + 1 & b - 1 \\
f^a_1(b) & \triangleq & (\underbrace{f^a_0 \circ \ldots \circ f^a_0}_b)(a) & = & a + b & a - b \\
f^a_2(b) & \triangleq & (\underbrace{f^a_1 \circ \ldots \circ f^a_1}_b)(a) & = & a * b & \frac{b}{a} \\
f^a_3(b) & \triangleq & (\underbrace{f^a_2 \circ \ldots \circ f^a_2}_b)(a) & = & a^b & \mathsf{log}_a(b) \\
f^a_4(b) & \triangleq & (\underbrace{f^a_3 \circ \ldots \circ f^a_3}_b)(a) & = & \underbrace{a^{.^{.^{.^a}}}}_b & \mathsf{log}^*_a(b)  \\
\end{array}
\]

\footnote{The initial
value of $f_i(0)$ is a little idiosyncratic: for $i \in \{1,2\} it is $0$; for $i > 2 it is $1$.}, yielding the sequence of functions
  Indeed, troubles increase 
as one goes up (down?) the
inverse Ackermann hierarchy: although the standard library provides a $\mathsf{log}_2$ 
function, it does not provide a $\mathsf{log}_b$ function

, nor 
any flavor of the iterated logarithm $\mathsf{log}_b^{*}$ or functions further in the hierarchy.
And, of course, efficiently computing the inverse Ackermann function is harder
than computing the inverse of any particular level of the hierarchy.

