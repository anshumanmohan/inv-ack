\label{sec:related}

\paragraph*{Hyperoperations and their inverses}
% This creates an unnumbered paragraph. ie a smaller, less flashy, header
Successor, Predecessor, Addition, Subtraction, Multiplication, Division, Algorisms.  Representations of numbers (Egyption fractions/Roman numerals/Decimal/Zero).  Exponentiation, Logarithm, Tetration, Log*, ...   Hyperoperations, Knuth Arrows.  Inverses as a separate notation?

Mechanizations of the above?

%\subsection*{The Ackermann function and its inverse.}
% This creates an unnumbered subsection

\paragraph*{The Ackermann function and its inverse}
\marginpar{\tiny \color{blue} Several variations. Original. Peter. 
Primitive recursive. Hilbert? Ackermann is used in CS. Formalizations that use or define it. The grit of sand. The bug.}
% ref: https://projecteuclid.org/download/pdf_1/euclid.bams/1183512393
%ref: https://www.cs.princeton.edu/~chazelle/pubs/mst.pdf
{\color{magenta}A brief sentence explaining what a primitive recursive function is and 
why total computable functions tend to be primitive recursive.}
The orignal three-variable Ackermann function was discovered by 
Wilhelm Ackermann as an example of a total computable function that 
is not primitive recursive. It grows tremendously fast, but does not have the higher-order
relation to repeated application and hyperoperation that we have been studying in
this paper. Those properties emerged thanks to refinements by Rózsa Péter, and it is 
her variant that computer scientists commonly care about.

\marginpar{\tiny \color{blue}implemented with the heuristic 
rules of \emph{path compression} and \emph{weighted union}}
The Ackermann-Péter function is mostly a toy curiosity, but its inverse 
occasionally features in the time bound analyses of algorithms. 
Tarjan \cite{tarjan} showed that, in the disjoint-set data structure, 
the time required $t(m,n)$ for a sequence of $m$ \textsc{\color{magenta}FIND}s
intermixed with $n-1$ \textsc{\color{magenta}UNION}s (such that $m \geq n$) is bounded
as: $k_{1}m\cdot\alpha(m,n) \leq t(m,n) \leq k_{2}m\cdot\alpha(m,n)$. 
Chazelle \cite{chazelle} showed that the minimum spanning tree
of a connected graph with $n$ vertices and $m$ edges 
can be found in time $O(m\cdot\alpha(m,n))$.

Cite: Ackermann, Peter, Tarjan, Chazelle, Pottier, ?  Anything in HOL?  Anything in SSReflect?

\paragraph*{Alternative strategies}

Division by constant, etc. is simpler.

Custom termination metrics.  Gas.  Mutual recursion.  Automata.  Alternative implementation strategy?  (Space, tail recursion, time?)

\paragraph*{Other?}