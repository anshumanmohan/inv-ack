In this section we build the inverse Ackermann hierarchy, define and prove the inverse relationship between them and the Ackermann hierarchy.

\subsection{The countdown operation}

Firstly, to define some sort of ``inverse'' for the repeat application operation found in \Cref{defn: ack_hier}, we define the following \textit{countdown} operation:

\begin{defn} \label{defn: countdown}
Given a function $f: \mathbb{N} \to \mathbb{N}$. The \textit{countdown} of $f$, denoted by $f^*$ is given by:
\begin{equation} \label{eq: countdown}
f^*(n) = \begin{cases}
0 & \text{ if } n \le \max\{0, 1, f(n)\} \\ 1 + f^*(f(n)) & \text{ if } n > \max\{0, 1, f(n)\}
\end{cases}
\end{equation}
\end{defn}
The important observation is that, if the sequence $\{n, f(n), f(f(n)), \ldots\}$ strictly decreases to $0$, then $f^*$ counts the minimum index where it reaches $1$ or below. We give a formal definition for functions with such decreasing sequence:

\begin{defn} \label{defn: contraction}
A function $f: \mathbb{N} \to \mathbb{N}$ is a \textit{contraction} if $f(0) = 0$ and $f(n) \le n-1 \ \forall n > 0$.
\end{defn}

\begin{thm} \label{thm: countdown contraction}
If $f: \mathbb{N} \to \mathbb{N}$ is a contraction, then
\begin{equation}
A = \left\{k: f^{(k)}(n) \le 1\right\} \neq \varnothing \ \text{ and } \ f^*(n) = \min A
\end{equation}
\end{thm}

\begin{proof}[Proof sketch]
By \eqref{eq: countdown} and \cref{defn: contraction}, we have
\begin{equation*}
f^*(n) = \begin{cases}
0 & \text{ if } n \le 1 \\ 1 + f^*(f(n)) & \text{ if } n > 1
\end{cases}
\end{equation*}
Then by a strong induction on $n$, the theorem holds.
\end{proof}

\subsection{The inverse Ackermann hierarchy}

With the above countdown operation, we can build the inverse Ackermann hierarchy.

\begin{defn} \label{defn: inv_ack_hier}
The inverse Ackermann hierarchy is a sequence of functions $\alpha_0, \alpha_1, \ldots $ recursively defined as:
\begin{enumerate}
	\item $\alpha_0(n) = \max\{n-2, 0\} \ \ \ \forall n \in \mathbb{N}$.
	\item $\alpha_m = \alpha_{m-1}^*  \ \ \ \forall m\in \mathbb{N}_{>0}$.
\end{enumerate}
\end{defn}

We will prove that each function $\alpha_m$ is a contraction, thus the $*$ operations are truly counting down to $1$.

\begin{thm} \label{thm: inv_ack_countdown}
For all $m\in\mathbb{N}$, $\alpha_m$ is a contraction and
\begin{equation}
\alpha_{m+1}(n) = \min\left\{ k : \alpha_m^{(k)}(n) \le 1 \right\}
\end{equation}
\end{thm}

\begin{proof}
TODO TODO TODO
\end{proof}

Having sufficiently established the inverse Ackermann hierarchy, we now link it to the Ackermann hierarchy in \cref{defn: ack_hier}. However, the relationship is not trivially clear, as the Peter-Ackermann function is in fact not exactly their inverses. We first define a \textit{canonical} variant of the Ackermann hierarchy, then use it as an intermediate to link first two hierarchies.

\subsection{The canonical Ackermann hierarchy}

The canonical Ackermann hierarchy is in short a somewhat ``cleaner'' variant of the Ackermann hierarchy, with simpler initial values, but still built on the repeated application operation.

\begin{defn} \label{defn: can_ack_hier}
The canonical Ackermann hierarchy is a sequence of functions $\text{C}_0, \text{C}_1, \ldots $ defined as:
\begin{enumerate}
	\item $\CA_0(n) = n + 2 \ \ \ \forall n\in \mathbb{N}$.
	\item $\CA_m(0) = 1 \ \ \ \forall m\in\mathbb{N}_{>0}$.
	\item $\CA_m(n) = \CA_{m-1}\left(\CA_m(n-1)\right) = \CA_{m-1}^{(n)}(0) \ \ \forall m, n\in \mathbb{N}_{>0}$.
\end{enumerate}
\end{defn}

Firstly, we explore the relationship between $C_m$ and $\alpha_m$. We use the following theorem:

\begin{thm} \label{thm: inv_ack_can_ack}
For all $m, n \in \mathbb{N}$, we have:
\begin{equation} \label{eq: inv_ack_can_ack}
\forall N \in \mathbb{N}: \alpha_m(N) \le n \iff N \le \CA_m(n)
\end{equation}
In other words
\begin{equation} \label{eq: inv_ack_can_ack_0}
\CA_m(n) = \max\left\{N : \alpha_m(N) \le n\right\}
\end{equation}
\end{thm}

Before proving this theorem, we will need to prove each function in the inverse and canonical Ackermann hierarchy is increasing (non-strictly). Then \eqref{eq: inv_ack_can_ack} is sufficient to explain the ``inverse'' relationship between them.

\begin{lem} \label{lem: inv_ack_hier_incr}
For all $m\in \mathbb{N}$, $\alpha_m$ is increasing.
\end{lem}

\begin{proof}
TODO TODO TODO
\end{proof}

\begin{lem}
For all $m\in \mathbb{N}$, $\CA_m$ is increasing.
\end{lem}

\begin{proof}
TODO TODO TODO
\end{proof}

Now we can proceed with proof for \Cref{thm: inv_ack_can_ack}

\begin{proof}[Proof of \Cref{thm: inv_ack_can_ack}]
We prove by induction on $m$.

\noindent\textbf{Base case:} $m = 0$. The goal becomes $N - 2\le n \iff N\le n + 2$, which is trivial.

\noindent\textbf{Inductive step:} Let us choose any $m, n_0, N_0\in \mathbb{N}$. Suppose that we have:
\begin{equation}
\forall n, N\in \mathbb{N}, \alpha_{m-1}(N) \le n \iff N \le \CA_{m-1}(n)
\end{equation}
There are two directions:
\begin{enumerate}[leftmargin=*]
	\item $(\implies)$. Suppose $\alpha_m(N_0) \le n_0$. If $N_0\le 1$, we automatically have $N_0 = C_m(0)\le C_m(n)$. Otherwise we have
	\begin{equation*}
	\alpha_m(N) = 1 + \alpha_m(\alpha_{m-1}(N)) \le n
	\implies \alpha_m(\alpha_{m-1})
	\end{equation*}
	TODO TODO TODO TODO
	
	\item $N_0\le \CA_m(n_0) \implies \alpha_m(N_0) \le n_0$. TODO TODO TODO
\end{enumerate}

\end{proof}

Now that we have established our canonical Ackermann hierarchy and its relation with the inverse Ackermann hierarchy, let us connect it to the original Ackermann hierarchy to complete the link.

\begin{thm} \label{thm: can_ack_ack}
For all $m, n\in \mathbb{N}$, we have $\CA_m(n+2) = \Ack_{m+1}(n) + 2$.
\end{thm}

\begin{proof}
TODO TODO TODO
\end{proof}

\subsection{Linking it all together}

In this conclusive part of this section, we link everything together by stating and proving the main theorem that connects the inverse Ackermann hierarchy and the Ackermann hierarchy. We then use this theorem to state and prove a relation between the inverse Ackermann hierarchy and the inverse Ackermann function.

\begin{thm}  \label{thm: inv_ack_ack}
For all $m, n, k \in \mathbb{N}$, we have:
\begin{equation}
n \le \Ack(m, k) \iff \begin{cases}
n \le k + 1 & \text{ if } m = 0 \\ \alpha_{m-1}(n+2) \le k + 2 & \text{ if } m > 0
\end{cases}
\end{equation}
\end{thm}


\begin{proof}
TODO TODO TODO
\end{proof}

The next theorem is a corollary of the above, which lays the theoretical groundwork for us to compute the inverse Ackermann function in linear time.

\begin{thm}
For all $n\mathbb{N}$, we have:
\begin{equation}
\alpha(n) = \begin{cases}
0 & \text{ if } n \le 1 \\
1 + \min\left\{ m: \alpha_m(n+2) \le m + 3 \right\} & \text{ if } n \ge 2
\end{cases}
\end{equation}
\end{thm}

\begin{proof}
TODO TODO TODO
\end{proof}

In the next section, we will devise an algorithm to compute each of the functions in the inverse Ackermann hierarchy in linear time, and an algorithm to compute the inverse Ackermann function in linear time.