In this section we build the inverse Ackermann hierarchy, 
defining and proving the inverse relationship between 
this heirarchy and the Ackermann hierarchy.

\subsection{The countdown operation}

Firstly, to define some sort of ``inverse'' for the repeat 
application operation found in \Cref{defn: ack_hier}, we 
define the following \textit{countdown} operation:

\begin{defn} \label{defn: countdown}
Given a function $f: \mathbb{N} \to \mathbb{N}$. 
The \textit{countdown} of $f$, denoted by $f^*$, is given by:
\begin{equation} \label{eq: countdown}
f^*(n) = \begin{cases}
0 & \text{ if } n \le \max\{0, 1, f(n)\} \\ 1 + f^*(f(n)) & \text{ if } n > \max\{0, 1, f(n)\}
\end{cases}
\end{equation}
\end{defn}
The important observation is that, if the sequence 
$\{n, f(n), f(f(n)), \ldots\}$ strictly decreases to 
$0$, then $f^*$ counts the minimum index where it reaches 
$1$ or below. We give a formal definition for functions 
with such decreasing sequences:

\begin{defn} \label{defn: contraction}
A function $f: \mathbb{N} \to \mathbb{N}$ is a \textit{contraction} if $f(0) = 0$ and $f(n) \le n-1 \ \forall n > 0$.
\end{defn}

\begin{thm} \label{thm: countdown contraction}
If $f: \mathbb{N} \to \mathbb{N}$ is a contraction, then
\begin{equation*}
A = \left\{k: f^{(k)}(n) \le 1\right\} \neq \varnothing \ \text{ and } \ f^*(n) = \min A
\end{equation*}
In other words,
\begin{equation*}
\forall n, k\in \mathbb{N}, \ f^*(n) \le k \iff f^{(k)}(n) \le 1
\end{equation*}
\end{thm}

\begin{proof}[Proof sketch]
By \eqref{eq: countdown} and \cref{defn: contraction}, we have
\begin{equation*}
f^*(n) = \begin{cases}
0 & \text{ if } n \le 1 \\ 1 + f^*(f(n)) & \text{ if } n > 1
\end{cases}
\end{equation*}
Then by a strong induction on $n$, the theorem holds.
\end{proof}

\subsection{The inverse Ackermann hierarchy}

With the above countdown operation in hand, we can 
build the inverse Ackermann hierarchy.

\begin{defn} \label{defn: inv_ack_hier}
The inverse Ackermann hierarchy is a sequence of 
functions $\alpha_0, \alpha_1, \ldots $ recursively defined as:
\begin{enumerate}
	\item $\alpha_0(n) = \max\{n-2, 0\} \ \ \ \forall n \in \mathbb{N}$.
	\item $\alpha_m = \alpha_{m-1}^*  \ \ \ \forall m\in \mathbb{N}_{>0}$.
\end{enumerate}
\end{defn}

We will prove that each function $\alpha_m$ is a contraction, 
thus the $*$ operations are truly counting down to $1$.

\begin{thm} \label{thm: inv_ack_countdown}
For all $m\in\mathbb{N}$, $\alpha_m$ is a contraction and
\begin{equation}
\alpha_{m+1}(n) = \min\left\{ k : \alpha_m^{(k)}(n) \le 1 \right\}
\end{equation}
\end{thm}

\begin{proof}
TODO TODO TODO
\end{proof}

Having sufficiently established the inverse Ackermann 
hierarchy, we now link it to the Ackermann hierarchy 
in \cref{defn: ack_hier}. However, the relationship is 
not trivially clear, as the Peter-Ackermann function is 
in fact not exactly {\color{magenta}their inverses.} 
We first define a \textit{canonical} variant of the 
Ackermann hierarchy, then use it as an intermediate to 
link first two hierarchies.

\subsection{The canonical Ackermann hierarchy}

The canonical Ackermann hierarchy is a somewhat ``cleaner'' 
variant of the Ackermann hierarchy, with simpler initial 
values, but still built on the repeated application operation.

\begin{defn} \label{defn: rep_app}
For any function $F: \mathbb{N} \to \mathbb{N}$, the 
\textit{repeater} of $F$ is another function $F^R : 
\mathbb{N} \to \mathbb{N}$ defined by:
\begin{equation}
F^R(n) = \begin{cases}
1 & \text{ if } n = 0 \\ F\left(F^R(n - 1)\right) & \text{ if } n \ge 1
\end{cases}
\end{equation}
In other words, $\forall n, F^R(n) = F^{(n)}\left(F^R(0)\right) = F^{(n)}(1) \ \ $.
\end{defn}

Formally defining the repeated application operation 
helps us define the canonical Ackermann hierarchy in 
a neat way below.

\begin{defn} \label{defn: can_ack_hier}
The canonical Ackermann hierarchy is a sequence of 
functions $\text{C}_0, \text{C}_1, \ldots $ defined as:
\begin{enumerate}
	\item $\CA_0(n) = n + 2 \ \ \ \forall n\in \mathbb{N}$.
	\item $\CA_m = \CA_{m-1}^R \ \ \forall m\in \mathbb{N}_{>0}$.
\end{enumerate}
\end{defn}

Firstly, we explore the relationship between $C_m$ and 
$\alpha_m$. We use the following theorem:

\begin{thm} \label{thm: inv_ack_can_ack}
For all $m, n \in \mathbb{N}$, we have:
\begin{equation} \label{eq: inv_ack_can_ack}
\forall N \in \mathbb{N}: \alpha_m(N) \le n \iff N \le \CA_m(n)
\end{equation}
In other words
\begin{equation} \label{eq: inv_ack_can_ack_0}
\CA_m(n) = \max\left\{N : \alpha_m(N) \le n\right\}
\end{equation}
\end{thm}

Before proving this theorem, we will need to prove each 
function in the inverse and canonical Ackermann hierarchy 
is increasing (non-strictly). Then \eqref{eq: inv_ack_can_ack} 
is sufficient to explain the ``inverse'' relationship between them.

%\begin{lem} \label{lem: inv_ack_hier_incr}
%For all contraction $f$, if $f$ is increasing then so is $f^*$.
%\end{lem}
%
%\begin{proof}
%
%\end{proof}
%
%\begin{lem}
%For all function $F: \mathbb{N}\to \mathbb{N}$, if $F$ is increasing then so is $F^R$.
%\end{lem}
%
%\begin{proof}
%TODO TODO TODO
%\end{proof}

\begin{lem}  \label{lem: countdown_rep_app}
For any function $f, F: \mathbb{N} \to \mathbb{N}$, 
if $f$ is a contraction and $\forall n, N\in \mathbb{N}$, 
$f(N) \le n \iff N \le F(n)$, then
$$ \forall n, N\in \mathbb{N}, N\le f^*(n) \iff N \le F^R(n) $$
\end{lem}

\begin{proof}
By \Cref{thm: countdown contraction} and \Cref{defn: rep_app}, we have:
$\displaystyle f^*(N)\le n \iff f^{(n)}(N) \le 1 \iff N \le F^{(n)}(1) = F^R(n) $
\end{proof}

Now we can proceed with the proof for \Cref{thm: inv_ack_can_ack}

\begin{proof}[Proof of \Cref{thm: inv_ack_can_ack}]
We prove by induction on $m$. The base case is the following statement:
$$ \forall n, N\in \mathbb{N}, \ N - 2\le n \iff N \le n + 2 $$
, which is trivial. The inductive step follows directly from \Cref{lem: countdown_rep_app}.
\end{proof}

Now that we have established our canonical Ackermann hierarchy 
and its relation with the inverse Ackermann hierarchy, let us 
connect it to the original Ackermann hierarchy to complete the link.

\begin{thm} \label{thm: can_ack_ack}
For all $m, n\in \mathbb{N}$, we have $\CA_m(n+2) = \Ack_{m+1}(n) + 2$.
\end{thm}

\begin{proof} TODO TODO TODO.
%An induction on $m$ will suffice. The base case is the following:
%$$ \forall n\in \mathbb{N}, \ (n + 2) + 2 = \Ack_1(n) + 2 $$
%, which is trivial. For the inductive step, suppose $m\ge 1$ and
%$\displaystyle \forall n\in \mathbb{N}, \CA_{m-1}(n+2) = \Ack_m(n) + 2 $.
%
%We need to prove the statement $P(n) := \CA_m(n+2) = \Ack_{m+1}(n) + 2$ holds for all $n\in \mathbb{N}$. Again, we use induction on $n$.
%
%For $n = 0$, $P(0) := \CA_m(2) = \Ack_{m+1}(0) + 2$, or $\CA_{m-1}(\CA_{m-1}(1)) = \Ack_m(1) + 2$.
\end{proof}

\subsection{Linking it all together}

To conclude this section, we link everything together by 
stating and proving the main theorem that connects the 
inverse Ackermann hierarchy and the Ackermann hierarchy. 
We then use this theorem to state and prove a relation 
between the inverse Ackermann hierarchy and the inverse 
Ackermann function.

\begin{thm}  \label{thm: inv_ack_ack}
For all $m, n, k \in \mathbb{N}$, we have:
\begin{equation}
n \le \Ack(m, k) \iff \begin{cases}
n \le k + 1 & \text{ if } m = 0 \\ \alpha_{m-1}(n+2) \le k + 2 & \text{ if } m > 0
\end{cases}
\end{equation}
\end{thm}


\begin{proof}
TODO TODO TODO
\end{proof}

The next theorem is a corollary of the above, which lays the 
theoretical groundwork for us to compute the inverse Ackermann 
function in linear time.

\begin{thm}
For all $n\mathbb{N}$, we have:
\begin{equation}
\alpha(n) = \begin{cases}
0 & \text{ if } n \le 1 \\
1 + \min\left\{ m: \alpha_m(n+2) \le m + 3 \right\} & \text{ if } n \ge 2
\end{cases}
\end{equation}
\end{thm}

\begin{proof}
TODO TODO TODO
\end{proof}

In the next section, we will devise an algorithm to compute each 
of the functions in the inverse Ackermann hierarchy in linear time, 
and an algorithm to compute the inverse Ackermann function in linear time.