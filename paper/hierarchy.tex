Let us now consider hyperoperations more carefully
to clarify how they relate to the Ackermann function.
The first hyperoperation (level 0) is simply successor, and after that,
every hyperoperation is the repeated application of the previous.
Addition is level 1, and $b$ repetitions of addition
give multiplication, which is level 2. Next, $b$ repetitions of
multiplication give exponentiation, which is level 3. However,
there is a subtlety here: in the former case, we add $a$
repeatedly to an initial value, \emph{which should be $0$}.
In the latter case, we multiply $a$ repeatedly to an initial value,
\emph{which should be $1$}. The formal definition for hyperoperation is:
\marginpar{\tiny \color{blue} Should 2 just be under 1?}
\begin{equation}
\label{eq:hyper}
\begin{array}{@{}l@{\qquad\quad}l}
\textit{1. Initial level: } a[0]b ~ \triangleq ~ b + 1 &
\textit{2. Initial values: } a[n+1]0 ~ \triangleq ~
\begin{cases}
a & \text{when } n = 0 \\
0 & \text{when } n = 1 \\
1 & \text{otherwise}
\end{cases}
\\
\multicolumn{2}{@{}l}{\textit{3. Recursive rule: } a[n+1](b+1) ~ \triangleq ~ a[n]\big(a[n+1]b\big)}
\end{array}
\end{equation}
The seemingly complicated recursive rule is in fact just \emph{repeated application} in disguise. By fixing $a$ and treating $a[n]b$ as a function of $b$, we can write
\begin{equation*}
\begin{array}{lll}
a[n+1]b & ~ = ~ a[n]\big(a[n+1](b-1)\big) & ~ = ~ a[n]\big(a[n](a[n+1](b-2))\big) \\
 & ~ = ~ \underbrace{\big( a[n]\circ a[n]\circ \cdots \circ a[n] \big)}_{b \text{ times}} \big(a[n+1]0\big) & ~ = ~ \big(a[n]\big)^{(b)}\big(a[n+1]0\big)
\end{array}
\end{equation*}
where $f^{(k)}(u) ~ \triangleq ~ (f\circ f\circ \cdots \circ f)(u)$ denotes $k$ successive applications of a function $f$ to an input $u$, with $f^{(0)}(u) = u $ (applying $0$ times). {\color{blue} Would you consider $f^k(u)$, $\big(a[n]\big)^b\big(a[n+1]0\big)$, etc?}

This insight will help us encode both the Ackermann function~(\eqref{eq:ackermann}) and hyperoperations~(\eqref{eq:hyper}) in Coq.  Notice that the recursive case of hyperoperation---and
indeed, the third case of the Ackermann function---is troublesome to encode in Coq due to
the deep nested recursion.  In the outer recursive call, the first argument ($n$) is shrinking
but the second is expanding explosively; in the inner recursive call, the first argument is
constant but the second is shrinking. The elegant solution uses double recursion~\cite{someone?} as follows:
\begin{lstlisting}
Fixpoint ackermann_original (m n : nat) : nat :=
  match m with
  | 0    => 1 + n
  | S m' => let fix ackermann' (n : nat) : nat :=
              match n with
              | 0    => ackermann_original m' 1
              | S n' => ackermann_original m' (ackermann' n')
              end
            in ackermann' n
  end.

Definition hyperop_init (a n : nat) : nat :=
  match n with 0 => a | 1 => 0 | _ => 1 end.

Fixpoint hyperop_original (a n b : nat) : nat :=
  match n with
  | 0    => 1 + b
  | S n' => let fix hyperop' (b : nat) :=
              match b with
              | 0    => hyperop_init a n'
              | S b' => hyperop_original a n' (hyperop' b')
              end
            in hyperop' b
  end.
\end{lstlisting}
Coq is immediately satisfied since both recursive calls are structurally smaller.
Moreover, having encoded both notions in this style, the structural similarities
are readily apparent.  In fact, the only essential difference is the initial values
(the second case of both definitions): the Ackermann function uses $\Ack(n-1,1)$, whereas
hyperoperations use the initial values given in~\eqref{eq:hyper}.

Having noticed that the deep recursion in both notions is expressing the same idea
of repeated application, we arrive at another useful idea. We can express the relationship
between the $(\text{n+1})^{\text{th}}$- and $\text{n}^{\text{th}}$-levels in
a \emph{functional} way if we develop a higher-order function that transforms the latter level
to the former.  We call this idea a \emph{repeater}:
\begin{defn}
For all $a\in \mathbb{N}$ and $f: \mathbb{N}\to \mathbb{N}$, the \emph{repeater from} $a$ of $f$, denoted by $\rf{f}{a}$ , is a function $\mathbb{N}\to \mathbb{N}$ such that $\rf{f}{a}(n) = f^{(n)}(a)$.
\end{defn}
The Gallina definition is modified to structurally decrease on $n$:
\begin{lstlisting}
Fixpoint repeater_from (f : nat->nat) (a : nat) (n : nat) : nat :=
  match n with
  | 0    => a
  | S n' => f (repeater_from f a n')
  end.
\end{lstlisting}
The notation $\rf{f}{a}(b)$ does much better at separating the function, i.e. the repeater of $f$, and the variable $n$ than $f^{(n)}(a)$, while making clear that $a$ is a parameter of \emph{repeater} itself. It allows a simple and function-oriented definition of hyperoperations:
\begin{equation*}
a[n]b ~ \triangleq ~ \begin{cases}
b + 1 & \text{when } n = 0 \\
\rf{a[n-1]}{a_n}(b) & \text{otherwise}
\end{cases}
{\color{red}\qquad \qquad \text{ where } \ a_n ~ \triangleq ~ \begin{cases}
a & \text{when } n = 1 \\
0 & \text{when } n = 2 \\
1 & \text{otherwise}
\end{cases}}
\end{equation*}
Note we use $a[n-1]$ in a Curried style to denote the single-variable function $\lambda b.a[n-1]b$.
The Ackermann function is likewise expressed elegantly using repeater:
\begin{equation*}
\Ack(n,m) ~ \triangleq ~ \begin{cases}
m + 1 & \text{when } n = 0 \\
\rf{\Ack_{n-1}}{\color{magenta}\Ack(n-1,1)}(m) & \text{otherwise}
\end{cases}
\end{equation*}
In Coq these two definitions are written as follows:
\begin{lstlisting}
Fixpoint hyperop (a n b : nat) : nat :=
  match n with
  | 0    => 1 + b
  | S n' => repeater_from (hyperop_init a n') (hyperop a n') b
  end.

Fixpoint ackermann (n m : nat) : nat :=
  match n with
  | 0    => S m
  | S n' => repeater_from (ackermann n' 1) (ackermann n') m
  end.
\end{lstlisting}
In the remainder of this paper we construct efficient inverses to these
functions.  Our key idea is an inverse to the higher-order repeater function; we call this inverse \emph{countdown}.

% Aquinas' promise:
% We explain our core techniques of repeaters and countdowns that allow us to define each level of the Ackermann hierarchy—and their upper inverses—in a straightforward and uniform manner. We show how countdowns, in particular, can be written structurally recursively.

% Anshuman proposes:
% We introduce our core techniques of repeaters and countdowns.
% We show how countdowns, in particular, can be written structurally recursively.







