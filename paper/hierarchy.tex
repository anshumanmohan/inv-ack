% Aquinas' promise:
% We explain our core techniques of repeaters and countdowns that allow us to define each level of the Ackermann hierarchy—and their upper inverses—in a straightforward and uniform manner. We show how countdowns, in particular, can be written structurally recursively.

% Anshuman proposes:
% We introduce our core techniques of repeaters and countdowns.
% We show how countdowns, in particular, can be written structurally recursively.

\iffalse
In this section we build the inverse Ackermann hierarchy,
defining and proving the inverse relationship between
this hierarchy and the Ackermann hierarchy.
\fi
% Hiding because this will now be in the next section.

\subsection{The repeater operation}

The first hyperoperation is simply successor, and after that,
every hyperoperation is the repeated application of the previous.
Addition is level 1, and $b$ repetitions of addition
give multiplication, which is level 2. Next, $b$ repetitions of
multiplication give exponentiation, which is level 3. However,
there is a subtlety here: in the former case, we add $a$
repeatedly to an \textit{initial value}, which must clearly be $0$.
In the latter case, we multiply $a$ repeatedly to an initial value,
which is $1$. In general, the formal definition for hyperoperation is
\[
\begin{array}{@{}l@{\qquad\quad}l}
\textit{1. Initial level: } a[0]b ~ \triangleq ~ b + 1 & 
\textit{2. Initial values: } a[n+1]0 ~ \triangleq ~
\begin{cases}
a & \text{when } n = 0 \\
0 & \text{when } n = 1 \\
1 & \text{otherwise}
\end{cases}
\\
\multicolumn{2}{@{}l}{\textit{3. Recursive rule: } a[n+1](b+1) ~ \triangleq ~ a[n]\big(a[n+1]b\big)}
\end{array}
\]
The seemingly complicated recursive rule is in fact just \emph{repeated application} in disguise. By fixing $a$ and treating $a[n]b$ as a function of $b$, we can write
\begin{equation*}
\begin{array}{lll}
a[n+1]b & ~ = ~ a[n]\big(a[n+1](b-1)\big) & ~ = ~ a[n]\big(a[n](a[n+1](b-2))\big) \\
 & ~ = ~ \underbrace{\big( a[n]\circ a[n]\circ \cdots \circ a[n] \big)}_{b \text{ times}} \big(a[n+1]0\big) & ~ = ~ \big(a[n]\big)^{(b)}\big(a[n+1]0\big)
\end{array}
\end{equation*}
where $f^{(k)}(u) ~ \triangleq ~ (f\circ f\circ \cdots \circ f)(u)$ denotes $k$ successive applications of a function $f$ to an input $u$, with $f^{(0)}(u) = u $ (applying $0$ times). In order to view the relationship between the $(\text{n+1})^{\text{th}}$- and $\text{n}^{\text{th}}$-levels in a \emph{functional} way, we need an operation that transforms the latter to the former, which leads to the notion of \emph{repeater}.
\begin{defn}
For all $a\in \mathbb{N}$ and $f: \mathbb{N}\to \mathbb{N}$, the \emph{repeater from} $a$ of $f$, denoted by $\rf{f}{a}$ , is a function $\mathbb{N}\to \mathbb{N}$ such that $\rf{f}{a}(n) = f^{(n)}(a)$.
\end{defn}
The Gallina definition is modified to structurally decrease on $n$:
\marginpar{maybe f should be the first argument here?}
\begin{lstlisting}
Fixpoint repeater_from (a:nat) (f:nat->nat) (n:nat) : nat :=
  match n with
  | 0 => a
  | S n' => f (repeater_from a f n')
  end.
\end{lstlisting}
The notation $\rf{f}{a}(b)$ does much better at separating the function, i.e. the repeater of $f$, and the variable $n$ than $f^{(n)}(a)$, while making clear that $a$ is a parameter of \emph{repeater} itself. It allows a simple and function-oriented definition of hyperoperations:
\begin{equation*}
a[n]b ~ \triangleq ~ \begin{cases}
b + 1 & \text{when } n = 0 \\
\rf{a[n-1]}{a_n}(b) & \text{otherwise}
\end{cases}
\qquad \qquad \text{ where } \ a_n ~ \triangleq ~ \begin{cases}
a & \text{when } n = 1 \\
0 & \text{when } n = 2 \\
1 & \text{otherwise}
\end{cases}
\end{equation*}

\subsection{The Upper Inverse and Expansions}

Our goal is to find an inverse hierarchy to the hyperoperations, which should include division, logarithm and iterated logarithm as inverses to multiplication, exponentiation and tetration. The advent of repeater motivates an ``inverse repeater'' operation that takes us to the next level in the supposed ``inverse hyperoperation'' hierarchy. Before going into details, we clarify our setting, starting with the notion of ``inverse''.

\paragraph{Inverse of increasing functions} On $\mathbb{Q}$ or $\mathbb{R}$, many functions are bijections and thus have an inverse in a normal sense. Functions on $\mathbb{N}$ are often non-bijections and thus should be treated differently.
\begin{defn} \label{defn: inverse}
	For all $F:\mathbb{N}\to \mathbb{N}$, $f:\mathbb{N}\to \mathbb{N}$ is the \emph{upper inverse} of $F$ if $f(n) = \min\{m : F(m)\ge n\} \ \forall n$, denoted by $F^{-1}_+$, and is the \emph{lower inverse} of $F$ if $f(n) = \max\{m : F(m)\le n\} \ \forall n$, denoted by $F^{-1}_-$.
\end{defn}
To construct a useful notion of inverse for our purpose, observe that for $a\ge 2$, each $a[n]$ is strictly increasing and grows to infinity with its input, while not always reaches $0$, which ensures the existence of the upper but not the lower inverse, since $\{m : a[n]m \le 0 \} = \varnothing$ for $n\ge 3$. Thus we decide to go on with the upper inverse notion, and leave the other for discussion in \cref{sec: discussion}. Here after we shall refer to ``upper inverse'' as simply ``inverse'', unless otherwise stated. Furthermore, we can focus on strictly increasing functions $F$, which in fact gives a more meaningful, bidirectional relationship between $F$ and its inverse:
\begin{thm} \label{thm: upp-inverse-rel}
	If $F:\mathbb{N}\to \mathbb{N}$ is increasing, then $f$ is the inverse of $F$ if and only if $\ \forall n, m : f(n)\le m \iff n \le F(m)$.
\end{thm}
\begin{proof}
Fix $n$, the sentence $n\le F(m) \iff f(n)\le m \ \forall m$ implies: (1) $f(n)$ is a lower bound to $\{m: F(m)\ge n \}$ and (2) $f(n)$ is in the set itself since plugging in $m := f(n)$ will yield $n\le F(f(n))$, which makes $f$ the upper inverse of $F$. Conversely, if $f$ is the upper inverse of $F$, we immediately have $n\le F(m)\implies f(n)\le m \ \forall m$. Now for all $m \ge f(n)$, $F(m)\ge F(f(n)) \ge n$ by increasing-ness, thus complete the proof.
\end{proof}

\paragraph{Expansions} Another property $F$ needs to have is one that ensures strict increasing-ness is preserved through \emph{repeater}. Suppose for some $a$ and strictly increasing $F$, the function $\rf{F}{a}\ $ is strictly increasing, then the chain $a$, $F(a)$, $F^{(2)}(a)$, \ldots has to be strictly increasing, which leads to the notion of \emph{expansions} and \emph{strict expansions}.
\begin{defn}
Given $a\in \mathbb{N}$, a function $F:\mathbb{N}\to\mathbb{N}$ is an \emph{expansion} if $F(n)\ge n \ \forall n$. An expansion $F$ is \emph{strict from} $a$ if $F(n)\ge n+1 \ \forall n\ge a$.
\end{defn}
Note that a strictly increasing function on $\mathbb{N}$ is already an expansion, though not necessarily strict. Conveniently, if $a\ge 1$ and $F$ is an expansion strict from $a$, $\rf{F}{a}(n) = F^{(n)}(a) \ge a + n \ge 1 + n$ $\forall n$, so $\rf{F}{a}\ $ is itself an expansion strict from $0$. Collectively, we can refer to strictly increasing $f$ as \emph{repeatable} from $a\ge 1$ if they are also strict expansions from $a$, so that repeatability is preserved through \emph{repeater from} $a$. We denote the set of functions repeatable from $a$ as $\repeatable_a$.
\begin{rem} \label{rem: repeatable-subset}
	It is trivial to see that $\repeatable_s \subset \repeatable_t \ \forall s\le t $.
\end{rem}

\subsection{Contractions and the countdown operation}

This is the final step leading to the inverse hyperoperations, where we develop a way to compute the inverse of a function's repeater from its own inverse. Let $a\ge 1$ and $F\in \repeatable_a$ with inverse $f$, then $\rf{F}{a}\ \in \repeatable_0$ and has an inverse $f^*$. Fix $n$, for all $m$ we have
\begin{equation} \label{eq: rf-upp-inv}
\begin{aligned}
f^*(n)\le m & \iff n\le \rf{F}{a}(m) = F^{(m)}(a) \iff f(n)\le F^{(m-1)}(a) \\
& \iff f^{(2)}(n)\le F^{(m-2)}(a) \iff \ldots \iff f^{(m)}(n)\le a
\end{aligned}
\end{equation}
Letting $m = f^*(n)$ to get $f^{(f^*(n))}\le a$ and combined with the above:
\begin{equation*}
\left(f^{(f^*(n))} \le a \right) \wedge \left( \forall m, f^{(m)}(n)\le a \implies m \ge f^*(n) \right)
\end{equation*}
This implies $f^*(n)$ is the minimum number of times $f$ needs to be compositionally applied to $n$ to yield a result below or equal to $a$. Before we can define $f^*(n)$ to be the minimum of $\{m: f^{(m)}(n)\le a \}$ and claim success, we need to make sure: (1) that set is non-empty, so that it has a minimum, and (2) we can reliably find that minimum within finitely many steps. The most ideal situation would be that the chain $n$, $f(n)$, $f^{(2)}(n)$, \ldots decreases strictly until reaching $a$, which leads to the notion of \emph{contractions}.
\begin{defn} \label{defn: contracting}
	A function $f : \mathbb{N} \to \mathbb{N}$ is a \textit{contraction} if $f(n) \le n \ \forall n\in \mathbb{N}$. Given an $a \ge 1$, a contraction $f$ is \textit{strict above} $a$ if $n\ge f(n)+1 \ \forall n > a$. We further denote by $\contract$ the set of contractions and $\contract_a$ the set of contractions strict above $a$.
\end{defn}
\begin{rem}
	Similar to \cref{rem: repeatable-subset}, $\contract_s \subset \contract_t \ \forall s\le t$.
\end{rem}
That leaves the question: what type of functions $F$ whose inverses are contractions? Wonderfully, we already have an answer:
\begin{thm} \label{thm: expansion-inv-contraction}
For all $a\in \mathbb{N}$, $F\in \repeatable_a \implies f\triangleq F^{-1}_+\in \contract_a$.
\end{thm}
\begin{proof}
For all $n$, $F(n)\ge n \implies n \ge f(n)$, so $f$ is a contraction. If $n\ge a+1$, $n-1\ge a$, so $F(n-1)\ge n \implies n-1\ge f(n)$, so $f$ is strict above $a$.
\end{proof}
\Cref{thm: expansion-inv-contraction} shows a clear inverse relationship between expansions strict from some $a$ and contractions strict above that same $a$. It ensures that the inverse of an expansion's repeater not only exists but can be built from its own inverse, in a method formalized as \emph{countdown}.

\begin{defn} \label{defn: informal-countdown}
Let $f\in \contract_a$, the \textit{countdown to} $a$ of $f$, denoted by $\cdt{f}{a}(n)$, is the least number of times $f$ needs to be compositionally applied to $n$ to not stay above $a$.
\begin{equation} \label{eq: countdown}
\cdt{f}{a}( n ) \triangleq \min\{m: f^{(m)}(n)\le a \}
\end{equation}
\end{defn}

Inspired by \eqref{eq: rf-upp-inv}, we provide a neat, algebraically manipulable logical sentence equivalent to \eqref{eq: countdown}, which is more useful later in our paper.
\begin{col} \label{col: cdt-repeat}
	$\forall a, \forall f\in \contract_{a}$, $\cdt{f}{a}(n)\le m \iff f^{(m)}(n)\le a \ \forall n, m$.
\end{col}
\begin{proof}
	Fix $a$ and $n$. It suffices to prove the $(\!\!\implies\!\!)$ direction, since the other is trivial. Suppose $\cdt{f}{a}(n)\le m$, we get $f^{(m)}(n)\le f^{(\cdt{f}{a}(n))}(n)$ due to $f\in \contract$, and $f^{(\cdt{f}{a}(n))}(n)\le a$ due to \cref{thm: cdt-repeat}, which completes the proof.
\end{proof}
%\begin{thm} \label{thm: upp-inv-cdt-rf}
%	For all $a\in \mathbb{N}$, if $f\in \contract_{a}$ is the upper inverse of $F: \mathbb{N}\to \mathbb{N}$, then $\cdt{f}{a}$ is the upper inverse of $\rf{F}{a}$.
%\end{thm}
Another useful result it the recursive formula for \emph{countdown}.
\begin{thm} \label{thm: cdt-recursion}
	For all $a\in \mathbb{N}$ and $f\in \contract_{a}$, $\cdt{f}{a}$ satisfies:
	\begin{equation*}
	\cdt{f}{a}(n) = \begin{cases}
	0 & \text{ if } n \le a \\ 1 + \cdt{f}{a}(f(n)) & \text{ if } n \ge a + 1
	\end{cases}
	\end{equation*}
\end{thm}
\begin{proof}
By \cref{col: cdt-repeat}, $n\le a \iff f^{(0)}(n)\le a \iff \cdt{f}{a}(n)\le 0$, thus the case $n\le a$ is resolved. Suppose $n\ge a+1$ and let $\cdt{f}{a}(f(n)) = m$. We have $\cdt{f}{a}(n)\le 1+m \!\! \iff \!\! f^{(1+m)}(n) \le a$, which is equivalent to $f^{(m)}(f(n)) \le a$, which holds by $m$'s definition.

Now since $n\ge a+1$, $\cdt{f}{a}(n)\ge 1$ by the above. Let $\cdt{f}{a}(n) = p+1$. It remains to prove $\cdt{f}{a}(f(n))\le p$, or $f^{(p)}(f(n))\le a$, or $f^{p+1}(n)\le a$, which holds by $p$'s definition.
\end{proof}

While a repeater exists for all functions not necessarily repeatable, and hence can be written in Coq without much trouble, a \emph{countdown} only exists for certain functions, most conveniently contractions, hence proves a little more challenging for Coq. In the next section we give a formal computation using a \emph{countdown worker}, which terminates by Coq standards for all functions, while successfully produces the countdown when given a contraction.

%\begin{thm} \label{thm: cdt-contr-0}
%	For all $a\ge 1$, if $f\in \contract_{a}$, then $\cdt{f}{a}\ \in \contract_{c} \ \forall c$.
%\end{thm}
%\begin{proof}
%	Firstly we show that $\cdt{f}{a}$ is a contraction, namely showing $\cdt{f}{a}(n)\le n \ \forall n$, which has already been proved in the proof of \cref{thm: cdt-repeat}. To show $\cdt{f}{a}$ is strict from $1+c$, it suffices to show it is strict from $1$, equivalently $\cdt{f}{a}(n) \le n - 1 \ \forall n\ge 1$. By \cref{thm: cdt-repeat}, we need to show $f^{(n-1)}(n)\le a$. Assume the converse, then:
%	$$ n \ge 1 + f(n)\ge 2 + f(f(n)) \ge \cdots \ge (n-1) + f^{(n-1)}(n) $$
%	Thus $f^{(n-1)}(n)\le 1 \le a$, a contradiction. The theorem then follows.
%\end{proof}

\subsection{Countdown Computation}

\begin{defn} \label{defn: countdown-worker}
For any $a\in \mathbb{N}$ and $f: \mathbb{N}\to \mathbb{N}$, the \emph{countdown worker} to $a$ of $f$ is a function $\W\cdt{f}{a}\ : \mathbb{N}^2\to \mathbb{N}$ such that:
\begin{equation*}
\W\cdt{f}{a}(n, b) = \begin{cases}
0 & \text{if } b = 0 \vee n\le a \\ 1 +\W\cdt{f}{a}(f(n), b-1) & \text{if } b \ge 1 \wedge n > a
\end{cases}
\end{equation*}
\end{defn}

Essentially, \emph{countdown worker} operates on two arguments, the \emph{true argument} $n$, which we wish to count down to $a$, and the \emph{budget} $b$,
the maximum number of times we attempt to compositionally
apply $f$ on the input before giving up. If the input goes below or equal $a$ after $k$ applications, \emph{i.e.} $f^{(k)}(n) \le a$, we return $k$ as the true countdown value. If the budget is exhausted, i.e. $b = 0$, while the result is still above $a$,
we fail by returning the original budget. This defnition is somewhat clunky, but it can clearly be written as a Coq fixpoint, with the budget as the decreasing
argument.
\begin{lstlisting}
Fixpoint countdown_worker a (f: nat->nat) n k :=
match k with
| 0    => 0
| S k' => match (n - a) with
          | 0 => 0
          | _ => S (countdown_worker a f (f n) k') end
end.
\end{lstlisting}
It is clear that with a sufficient budget, \emph{countdown worker} should compute the correct \emph{countdown} value when $f$ is a contraction strict from $a$. We will show that a budget of $n$ is sufficient, using several lemmas about \emph{countdown worker}. Let us use following Coq-compatible definition of \emph{countdown}, which is also used in our code base.
\begin{defn} \label{defn: countdown}
In this section, define  $\cdt{f}{a}(n) = \W\cdt{f}{a}(n, n) \ \forall n$.
\begin{lstlisting}
Definition countdown_to a f n := countdown_worker a f n n.
\end{lstlisting}
\end{defn}
%Before beginning, let us clarify that the definition of $\mathbb{N}$ and operations on $\mathbb{N}$ in Gallina follow the Presburger Arithmetic \cite{presburger}, which despite being weaker than Peano Arithmetic, is a decidable theory. The Coq standard library includes \texttt{Omega}, an extensive listing of provable facts about $\mathbb{N}$ in Presburger Arithmetic, including everything used in this paper, most notably the law of excluded middle for comparisons on $\mathbb{N}$:
%\begin{equation*}
%(n \le m) \vee (m + 1 \le n) \ \ \forall n, m
%\end{equation*}
%, which is provable without the actual law of excluded middle in classical logic. This enables us to prove all results in this paper with Coq's baseline intuitionistic logic. Readers can refer to the \Cref{appendix} for Coq versions of the proofs.
%in which the most operations agree with usual operations on $\mathbb{Z}$, except subtraction, which is defined as:
%\begin{lstlisting}
%Fixpoint sub (n m : nat) : nat :=
%match n with
%| 0 => n
%| S k => match m with
%| 0 => n
%| S l => sub k l end
%end.
%\end{lstlisting}
%Essentially $\li{sub} \ n \ m = \max\{n - m, 0\}$. We will use this subtraction for the rest of the paper.

%Firstly, we begin with a lemma asserting the existence of the countdown value itself. Although its existence is guaranteed by the well-ordering principle of $\mathbb{N}$, we will achieve better by proving it in intuitionistic logic.
%
%\begin{lem} \label{lem: contract-repeat-threshold}
%	For all $a, n\in\mathbb{N}$ and $f\in \contract_{a}$,
%	\begin{equation}
%	\exists m : \left(f^{(m)}(n) \le a \right) \wedge \left(f^{(l)}(n)\le a \implies m \le l \ \forall l \right)
%	\end{equation}
%\end{lem}
%\begin{proof}
%  Fix $n$ and observe that if $n\le a$, $m = 0$ is the desired choice since $ f^{(0)}(n) = n \le a \ \text{ and } \ 0 \le l \ \forall l $.
%	Consider only when $a\le n$, we can define $c$ such that $n = a + c$. We prove the following statement by induction:
%	\begin{equation*}
%	P(c) \triangleq \exists m : \left(f^{(m)}(n) \le n - c \right) \wedge \left(f^{(l)}(n)\le n - c \implies m \le l \ \forall l \right)
%	\end{equation*}
%	under assumptions $f\in \contract_{n-c}$ and $c\le n$.
%	\begin{enumerate}[leftmargin=*]
%		\item \textit{Base case.} The case $c = 0$ implies $n = a$, which has been proven above.
%		\item \textit{Inductive step.} Suppose $P(c)$ is proved with witness $m_c$. Note that the assumptions are now $f\in \contract_{n-c}$ and $c+1\le n$, there are two cases:
%		\begin{itemize}[leftmargin=*, label={--}]
%			\item $f^{\left(m_c\right)}(n) = n - c$. Then $f^{\left(m_c+1\right)}(n) \le n - c - 1$. Let $m_{c+1} = m_c + 1$, for all $l$:
%			\begin{equation*}
%			f^{(l)}(n)\le n - c - 1 < f^{\left(m_c\right)}(n) \implies l > m_c \implies l \ge m_{c+1}
%			\end{equation*}
%			\item $f^{\left(m_c\right)}(n) \le n - c - 1$. Let $m_{c+1} = m_c$, for all $l$:
%			\begin{equation*}
%			f^{(l)}(n)\le n - c - 1\le n - c \overset{P(c)}{\implies} l\ge m_{c} = m_{c+1}
%			\end{equation*}
%		\end{itemize}
%		In any cases, we can find a witness $m_{c+1}$ for $P(c+1)$. Thus the proof is complete by induction.\vspace*{-\baselineskip}
%	\end{enumerate}
%\end{proof}

We start with a simple fact that \emph{countdown worker} returns $0$ when $n\le a$, which is intended for \emph{countdown}. It follows trivially from \cref{defn: countdown-worker}.

\begin{lem} \label{lem: cdt-init}
	For all $a, b\in \mathbb{N}$, $f : \mathbb{N}\to \mathbb{N}$ we have $\W\cdt{f}{a} (n, b) = 0 \ \forall n\le a$.
\end{lem}

The next lemma shows the internal working of \emph{countdown worker} at the $\text{i}^\text{th}$ recursive step, including the accumulated result $1+i$, the current input $f^{(1+i)}(n)$, and the current budget $b-i-1$.

\begin{lem} \label{lem: cdt-intermediate}
	For all $a, n, b, i\in \mathbb{N}$ and $f \in \contract$ such that $i < b$ and $a < f^{(i)}(n)$:
	\begin{equation}  \label{eq: cdt-intermediate}
	\W\cdt{f}{a}(n, b) = 1 + i + \W\cdt{f}{a}\left(f^{1+i}(n), b - i - 1\right)
	\end{equation}
\end{lem}
\begin{proof}
	We proceed by induction on $i$. Define
	\begin{equation*}
	P(i) \triangleq \left(\W\cdt{f}{a}(n, b) = 1 + i + \W\cdt{f}{a}\left(f^{1+i}(n), b - i - 1\right) \ \forall b, n : b\ge i+1, f^{(i)}(n) > a\right)
	\end{equation*}
	\begin{enumerate}[leftmargin=*]
		\item \textit{Base case.} For $i = 0$, our goal $P(0)$ is:
		$\W\cdt{f}{a}(n, b) = 1 + \W\cdt{f}{a}\left(f(n), b - 1\right)$
		where $b \ge 1, f(n)\ge a+1$, which is trivial.
		\item \textit{Inductive step.} Suppose $P(i)$ has been proven. Then
		\begin{equation*}
		P(i+1) \triangleq \W\cdt{f}{a}(n, b) = 2 + i + \W\cdt{f}{a}\left( f^{2+i}(n), b - i - 2 \right)
		\end{equation*}
		for $b \ge i+2, f^{1+i}(n) \ge a+1$. This also implies $b\ge i+1$ and $\displaystyle f^{(i)}(n) \ge f^{1+i}(n)\ge a+1$ by $f\in \contract$, thus $P(i)$ holds. It suffices to prove:
		\begin{equation*}
		\W\cdt{f}{a}\left(f^{1+i}(n), b-i-1\right) = 1 + \W\cdt{f}{a}\left( f^{2+i}(n), b-i-2 \right)
		\end{equation*}
		This is in fact $P(0)$ with $(b, n)$ substituted for $\left(b-i-1, f^{(1+i)}(n)\right)$. Since $f^{(1+i)}(n) \ge a+1$ and $b-i-1\ge 1$, the above holds and $P(i+1)$ follows. The proof is complete.\vspace*{-\baselineskip}
	\end{enumerate}
\end{proof}
Now it is time to prove the correctness of \emph{countdown worker}.

\begin{thm} \label{thm: cdt-repeat}
	For all $a\in \mathbb{N}$ and $f\in \contract_{a}$, we have
	\begin{equation} \label{eq: cdt-minimum}
	\cdt{f}{a}(n) = \min\left\{ i : f^{(i)}(n) \le a \right\} \ \ \forall n
	\end{equation}
%	Or equivalently,
%	\begin{equation} \label{eq: cdt-repeat}
%	\cdt{f}{a}(n) \le k \iff f^{(k)}(n) \le a \ \ \forall n, k
%	\end{equation}
\end{thm}
\begin{proof}
%	First, to see why \eqref{eq: cdt-minimum} and \eqref{eq: cdt-repeat} are equivalent, we rewrite \eqref{eq: cdt-minimum} in the following way:
%	$$ \left(f^{(\cdt{f}{a}(n))}(n) \le a\right) \wedge \left(f^{(l)}(n) \le a \implies \cdt{f}{a}(n) \le l \ \ \forall l\right) \ \ \forall n$$
%	To prove $\eqref{eq: cdt-minimum} \implies \eqref{eq: cdt-repeat}$, it suffices to show
%	$$ \cdt{f}{a}(n) \le l \implies f^{(l)}(n) \le a \ \ \forall l $$
%	, which holds due to the fact $\displaystyle f^{(l)}(n) \le f^{(\cdt{f}{a}(n))}(n) \le a$ by $f\in \contract$. To prove $\eqref{eq: cdt-repeat}\implies \eqref{eq: cdt-minimum}$, it suffices to show $\displaystyle f^{(\cdt{f}{a}(n))}(n) \le a$
%	, which in turn holds by substituting $k$ by $\cdt{f}{a}(n)$ in \eqref{eq: cdt-repeat}. Thus $\eqref{eq: cdt-minimum}\iff \eqref{eq: cdt-repeat}$ and we need only to prove \eqref{eq: cdt-minimum}.
Since $f\in \contract_{a}$ and $\mathbb{N}$ is well-ordered, let $m = \min\big\{i : f^{(i)}(n)\le a\big\}$ (we are also able prove its existence in Coq's baseline intuitionistic logic in our code base), then
	\begin{equation}
	\left(f^{(m)}(n) \le a\right) \label{eq: cdt-repeat-tmp} \wedge
	 \left(f^{(k)}(n)\le a \implies m \le k \ \ \forall k\right)
	\end{equation}
	It then suffices to prove $m = \cdt{f}{a}(n)$. Suppose firstly that $m = 0$. Then $n = f^{(0)}(n)\le a$, thus $\cdt{f}{a}(n) = \W\cdt{f}{a}(n, n) = 0 = m$ by \cref{lem: cdt-init}.
	
	Now consider $m > 0$. We would like to apply \cref{lem: cdt-intermediate} to get
	\begin{equation*}
	\cdt{f}{a}(n) = \W\cdt{f}{a}(n, n) = m + \W\cdt{f}{a}\left(f^{(m)}(n), n-m\right)
	\end{equation*}
	, then use \cref{lem: cdt-init} over \eqref{eq: cdt-repeat-tmp}'s first conjunct to conclude that $\cdt{f}{a}(n) = m$. It then suffices to prove the premises of \cref{lem: cdt-intermediate}, namely $a < f^{(m-1)}(n)$ and $m-1 < n$.
	
	The former follows by contradiction: if $f^{(m-1)}(n) \le a$, \eqref{eq: cdt-repeat-tmp}'s second conjunct implies $m\le m-1$, which is wrong for $m > 0$. The latter then easily follows by $f\in \contract_{a}$:
	\begin{equation*}
	n \ge 1 + f(n) \ge 2 + f(f(n)) \ge \cdots \ge m + f^{(m)}(n)
	\end{equation*}
	Therefore, $\cdt{f}{a}(n) = m$ in all cases, which completes the proof.
\end{proof}
Now, \eqref{eq: rf-upp-inv} and \cref{thm: cdt-repeat} are already enough to establish the correctness of the Coq definitions \emph{countdown} and \emph{countdown worker}. We thus unite \cref{defn: informal-countdown,defn: countdown} as they are equivalent. We wrap everything up in the following two theorems, one for $a\ge 1$ and one for $a = 0$.
\begin{thm} \label{thm: cdt-inv-rf}
	For all $a$ and $F\in \repeatable_a$, $f\triangleq F^{-1}_+$ satisfies $f \in \contract_{a}$ and $\displaystyle \cdt{f}{a} \ = \left(\rf{F}{a}\ \right)^{-1}_+$. Furthermore, if $a\ge 1$, $\rf{F}{a}\ \in \repeatable_0$ and $\cdt{f}{a}\ \in \contract_0$.
\end{thm}
\begin{proof}
	By \cref{thm: expansion-inv-contraction}, $f\triangleq F^{-1}_+ \in \contract_a$, and $\cdt{f}{a}\ = \left(\rf{F}{a}\ \right)^{-1}_+$ follows from \eqref{eq: rf-upp-inv} and \cref{col: cdt-repeat}.
	Now if $a\ge 1$, a simple induction shows that $F^{(n)}(a)\ge a + n\ge 1 + n \ \forall n$, so $\rf{F}{a}\ \in \repeatable_0$, hence $\cdt{f}{a} \ = \left(\rf{F}{a}\ \right)^{-1}_+ \in \contract_0$ by \cref{thm: expansion-inv-contraction}.
\end{proof}

