\subsection{The repeated application pattern}

\begin{frame}
\frametitle{Review: Repeated Application via \texttt{iter}}

\pause 
Let $X$ be any set and any function $f: X\to X$. Define the notation:
\begin{equation*}
f^{(k)}(u) \triangleq ~ \underbrace{(f\circ f\circ \cdots \circ f)}_{\text{$k$ times}}(u),
\end{equation*}

\smallskip

\pause 
Coq's standard library has \texttt{iter}. \quad $f^{(k)}(u) = \texttt{iter}(k, f, u).$

\bigskip

\pause 
The next level in the hyperoperations/Ackermann hierarchy is the result of $b$ compositional applications of the current level to some \emph{initial value}.


%The following recursive rule applies:
%\begin{enumerate}
%	\item $f^{(0)}(u) = u$ (\emph{i.e.} applying $0$ times yields the identity).
%	\item $f^{(k+1)}(u) = f \left(f^{(k)}(u)\right)$.
%\end{enumerate}
%
%\bigskip
%
%Repeated application plays a vital role in both hyperoperations and Ackermann hierarchy.

\end{frame}


\begin{frame}
\frametitle{Formal Definition of Hyperoperation Hierarchy}

\begin{equation*}
\begin{array}{lrcl}
\textit{1. 0$^{\textit{th}}$ level: } & \uncover<2->{a[0]b & ~ \triangleq ~ & b + 1} \\[3pt]
\textit{2. Initial values: } & \uncover<3->{a[n+1]0 & ~ \triangleq ~ &
\small \begin{cases}
a & \text{when } n = 0 \\
0 & \text{when } n = 1 \\
1 & \text{otherwise}
\end{cases}} \\[15pt]
\textit{3. Recursive rule: } \quad & \uncover<4->{a[n+1](b+1) & ~ \triangleq ~ & a[n]\big(a[n+1]b\big)}
\end{array}
\end{equation*}

\begin{equation*}
\begin{aligned}
& a[n+1]b \\
& \uncover<5->{= ~ a[n]\big(a[n+1](b-1)\big) ~ = ~ a[n]\big(a[n](a[n+1](b-2))\big)} \\
& \uncover<6->{= ~ \big(a[n]\circ a[n]\big)\big(a[n+1](b-2)\big) ~ = ~ \ldots} \\
& \uncover<7->{= ~ \underbrace{\big( a[n]\circ a[n]\circ \cdots \circ a[n] \big)}_{b \text{ times}} \big(a[n+1]0\big)  ~ = ~ \big(a[n]\big)^{(b)}\underbrace{\big(a[n+1]0\big)}_{\text{initial value}}}
\end{aligned}
\end{equation*}

\end{frame}


\begin{frame}
\frametitle{Formal Definition of Ackermann Hierarchy}
\begin{equation*}
\begin{array}{lrcl}
\textit{1. 0$^{\textit{th}}$ level: } & \uncover<2->{\Ack_0(b) & ~ \triangleq ~ & b + 1} \\[5pt]
\textit{2. Initial values: } & \uncover<2->{\Ack_{n+1}(0) & ~ \triangleq ~ & \Ack_n(1)} \\[5pt]
\textit{3. Recursive rule: } \quad & \uncover<2->{\Ack_{n+1}(b+1) & ~ \triangleq ~ & \Ack_n\big(\Ack_{n+1}(b)\big)}
\end{array}
\end{equation*}

\bigskip

\begin{equation*}
\begin{aligned}
& \Ack_{n+1}(b) \\
& \uncover<3->{= ~ \Ack_n\big(\Ack_{n+1}(b-1)\big) ~ = ~ \Ack_n\big(\Ack_n\big(\Ack_{n+1}(b-2)\big)\big) \\
& = ~ \big(\Ack_n\circ \Ack_n\big) \big(\Ack_{n+1}(b-2)\big) ~ = ~ \ldots \\
& = ~ \underbrace{\big( \Ack_n\circ \Ack_n\circ \cdots \circ \Ack_n \big)}_{b \text{ times}} \big(\Ack_{n+1}(0)\big)  ~ = ~ \big(\Ack_n\big)^{(b)}\underbrace{\big(\Ack_{n+1}(0)\big)}_{\text{initial value}}}
\end{aligned}
\end{equation*}

\end{frame}


\subsection{The repeater operation}


\begin{frame}[fragile]
\frametitle{Notation: From \texttt{iter} to Repeater}
%The next level in the hyperoperations/Ackermann hierarchy is the result of $b$ compositional applications of the current level to an initial value.
%
%\bigskip
%
%We can abstract the concept of repeated application in a higher-order function called \emph{repeater}:
%
%\smallskip

Writing $a[n+1]b = \big(a[n]\big)^{(b)}\big(a[n+1]0\big)$ or $\Ack_{n+1}(b) = \big(\Ack_n\big)^{(b)}\big(\Ack_{n+1}(0)\big)$ is clumsy since $b$ is no longer the main argument.

\bigskip

\pause 
\textbf{Repeater.} A rearranged form of \texttt{iter}, where $b$ becomes the main argument:
\impinline{$\rf{f}{u}(b) \triangleq \texttt{iter}(b, f, u) = f^{(b)}(u)$}.

\smallskip

Read $\rf{f}{a}$ \ as ``the \emph{repeater from} $a$ of $f$''.

\smallskip

%$\forall a\in \mathbb{N}, f: \mathbb{N}\to \mathbb{N}$, the
%\href{https://github.com/inv-ack/inv-ack/blob/7270e64a2600b771f2b1b1b151f7d13fb2ae6c97/repeater.v\#L32-L36}{\emph{repeater from}}
%$a$ of $f$, denoted by $\rf{f}{a}$ , is a function $\mathbb{N}\to \mathbb{N}$ such that $\rf{f}{a}(n) = f^{(n)}(a)$.

% Linked by A
\begin{lstlisting}
`\href{https://github.com/inv-ack/inv-ack/blob/7270e64a2600b771f2b1b1b151f7d13fb2ae6c97/repeater.v#L32-L36}{Fixpoint repeater\_from}` (f : nat -> nat) (a n : nat) : nat :=
match n with 0 => a | S n' => f (repeater_from f a n') end.
\end{lstlisting}

\smallskip

%\begin{minipage}{0.25\linewidth}
%	Drop $b$ to form \\ a recursive rule:
%\end{minipage}
\pause 
\begin{minipage}{0.4\linewidth}
	Drop $b$ to form higher-order \\ recursive rule between levels:
\end{minipage}
\pause \quad 
$\begin{cases}
\displaystyle a[n+1] & = \rf{\big(a[n]\big)}{a[n+1]0} \\[5pt]
\Ack_{n+1} & = \rf{\big(\Ack_n\big)}{\Ack_{n+1}(0)}
\end{cases}$.

\end{frame}


\begin{frame}[fragile]
\frametitle{Hyperoperations via Repeater}
Without Repeater (via double recursion):
\begin{lstlisting}
`\href{https://github.com/inv-ack/inv-ack/blob/7270e64a2600b771f2b1b1b151f7d13fb2ae6c97/repeater.v#L51-L52}{Definition hyperop\_init}` (a n : nat) : nat :=
  match n with 0 => a | 1 => 0 | _ => 1 end.

`\href{https://github.com/inv-ack/inv-ack/blob/7270e64a2600b771f2b1b1b151f7d13fb2ae6c97/repeater.v#L55-L64}{Fixpoint hyperop\_original}` (a n b : nat) : nat :=
  match n with
  | 0    => 1 + b
  | S n' => let fix hyperop' (b : nat) := match b with
              | 0    => hyperop_init a n'
              | S b' => hyperop_original a n' (hyperop' b')
              end in hyperop' b
  end.
\end{lstlisting}

\pause 
With Repeater:
% Linked by Anshuman
\begin{lstlisting} 
`\href{https://github.com/inv-ack/inv-ack/blob/7270e64a2600b771f2b1b1b151f7d13fb2ae6c97/repeater.v#L67-L71}{Fixpoint hyperop}` (a n b : nat) : nat :=
  match n with
  | 0    => 1 + b
  | S n' => repeater_from (hyperop a n') (hyperop_init a n') b
  end.
\end{lstlisting}
\end{frame}


\begin{frame}[fragile]
\frametitle{Ackermann via Repeater}
Without Repeater (via double recursion):
\begin{lstlisting}
`\href{https://github.com/inv-ack/inv-ack/blob/7270e64a2600b771f2b1b1b151f7d13fb2ae6c97/repeater.v#L123-L132}{Fixpoint ackermann\_original}` (m n : nat) : nat :=
  match m with
  | 0    => 1 + n
  | S m' => let fix ackermann' (n : nat) : nat := match n with
              | 0    => ackermann_original m' 1
              | S n' => ackermann_original m' (ackermann' n')
              end in ackermann' n
  end.
\end{lstlisting}

\pause 
With Repeater:
% Linked by A
\begin{lstlisting}
`\href{https://github.com/inv-ack/inv-ack/blob/7270e64a2600b771f2b1b1b151f7d13fb2ae6c97/repeater.v#L135-L139}{Fixpoint ackermann}` (n m : nat) : nat :=
  match n with
  | 0    => S m
  | S n' => repeater_from (ackermann n') (ackermann n' 1) m
  end.
\end{lstlisting}
\end{frame}

%
%\begin{frame}
%\frametitle{}
%\end{frame}