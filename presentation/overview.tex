\subsection*{The Ackermann and Inverse Ackermann functions}
\begin{frame}
\frametitle{Introduction: $A(n, m)$, $\Ack(n)$, and $\alpha(n)$}
%\frametitle{The Ackermann and Inverse Ackermann functions}
\label{defn: ack}
	\uncover<1->{
		The Ackermann-P\'eter function is defined as:
	\begin{equation*}
	A(n, m) \triangleq \begin{cases}
	m + 1 & \text{ when } n = 0 \\
	A(n-1, 1) & \text{ when } n > 0, m = 0 \\
	A\big(n-1, A(n, m-1)\big) & \text{ otherwise}
	\end{cases}
	\end{equation*}
  }
	
	\uncover<2->{The one-variable \emph{diagonal} Ackermann function is $\Ack(n)~\triangleq~A(n, n)$.}
	
	\bigskip
	
	\uncover<3->{
	Its \emph{inverse}, $\alpha(n)$, is the smallest~$k$ for
	which~$n \le \Ack(k)$, \emph{i.e}:
	\begin{equation*}
	\alpha(n) \triangleq \min\left\{k\in \mathbb{N} : n \le \Ack(k)\right\}
	\end{equation*}
}
\end{frame}


%\begin{frame}
%\frametitle{Initial values for $\Ack(n)$ and $\alpha(n)$}
%\begin{columns}[T]
%	\begin{column}{0.4\textwidth}
%		
%		\begin{equation*}
%		\begin{array}{|ll|}
%		n & \Ack(n) \\
%		0 & 1 \\
%		1 & 3 \\
%		2 & 7 \\
%		3 & 61 \\
%		4 & 2^{2^{2^{65536}}} - 3
%		\end{array}
%		\end{equation*}
%		
%		Grows astronomically fast!
%	\end{column}
%
%  \begin{column}{0.5\textwidth}
%  	
%  	\begin{equation*}
%  	\begin{array}{|ll|}
%  	n & \alpha(n) \\
%  	0, 1 & 0 \\
%  	2, 3 & 1 \\
%  	4, 5, 6, 7 & 2 \\
%  	8, 9, \ldots, 61 & 3 \\
%  	62, \ldots, 2^{2^{2^{65536}}} - 3 & 4
%  	\end{array}
%  	\end{equation*}
%  	
%  	Grows astronomically slowly!
%  \end{column}
%\end{columns}
%\end{frame}


\begin{frame}
\frametitle{Introduction: Growth Patterns}

 First few values of $\Ack(n)$ and $\alpha(n)$:
		
%		\begin{equation*}
%		\begin{array}{|ll|c|ll|}
%		n & \Ack(n) & \hspace{3em} & n & \alpha(n) \\[3pt]
%		0 & 1 & & 0, 1 & 0 \\[3pt]
%		1 & 3 & & 2, 3 & 1 \\[3pt]
%		2 & 7 & & 4, 5, 6, 7 & 2 \\[3pt]
%		3 & 61 & & 8, 9, \ldots, 61 & 3 \\[3pt]
%		4 & 2^{2^{2^{65536}}} - 3 & & 62, 63 \ldots, 2^{2^{2^{65536}}} - 3 & 4 \\[5pt]
%		\multicolumn{2}{|l|}{\text{Grows astronomically fast!}} & & \multicolumn{2}{|l|}{\text{Grows astronomically slow!}}
%		\end{array}
%		\end{equation*}
		
		\begin{tikzpicture}
		\begin{axis}[
%		symbolic x coords={a small bar,a medium bar,a large bar},
    yticklabels={,,},
		xtick=data]
		\addplot coordinates {
			(0, 1)
			(1, 3)
			(2, 7)
			(3, 61)
			(4, 9999999)
			
		};
		\end{axis}
		\end{tikzpicture}

%TODO Fix this graph? Graph for alpha? Figure with caption?

\end{frame}


\begin{frame}
\frametitle{Introduction: Computing $\alpha(n)$}
\uncover<1->{
	Although it grows slowly, $\alpha(n)$ is hard to \emph{compute} for large $n$ \\
	because it is entangled with the explosively-growing $\Ack(k)$.
}

\bigskip

\uncover<2->{
\textbf{Naive Approach:} starting at $k=0$, calculate $\Ack(k)$,
compare it to $n$, \\ and increment $k$ until $n \le \Ack(k)$.
}

\bigskip

\uncover<3->{
\textbf{Time complexity:} $\Omega(\Ack(\alpha(n)))$.
\\ Computing $\alpha(100) \mapsto^{*} 4$ requires
$\Ack(4) = 2^{2^{2^{65536}}}-3$ steps!
}

\bigskip

\uncover<4->{
\textbf{Unsatisfying workaround:} Hardcode. Return $4$ for most inputs.
}

\bigskip

\uncover<5->{
\imppar{\text{\textbf{Our Goal.} Compute $\alpha$ for \emph{all inputs} without computing $\Ack$.}}
}

\end{frame}

\subsection*{Relation to hyperoperations}

%\begin{frame}
%\frametitle{Introduction: Ackermann \emph{vs} Hyperoperations}
%The Ackermann function is easy to define, but hard to
%understand.
%
%We see it as
%a sequence of $n$-indexed functions $\Ack_n \triangleq \lambda b.A(n,b)$, where for each $n>0$, $\Ack_n$ is the result of applying the previous $\Ack_{n-1}$ $b$ times.
%
%%with a
%%\href{https://github.com/inv-ack/inv-ack/blob/7270e64a2600b771f2b1b1b151f7d13fb2ae6c97/repeater.v\#L161-L177}{\emph{kludge}}. %Linked by Linh
%
%\bigskip
%
%The hierarchical structure resembles that of \emph{hyperoperations}.
%
%%To better understand the Ackermann function as a hierarchy and this kludge, we explore the closely-related hyperoperations.
%
%\end{frame}


\begin{frame}
\frametitle{Introduction: Ackermann \emph{vs} Hyperoperation}

\pause 
The Ackermann function is easy to define, but hard to
understand.

\bigskip

\pause 	
Let's treat the first argument as an ``index''. 
\\Define $\Ack_n \triangleq \lambda b.A(n,b)$. 
\\Then, for $n>0$, $\Ack_n$ is the result of applying the previous $\Ack_{n-1}$ $b$ times.

%with a
%\href{https://github.com/inv-ack/inv-ack/blob/7270e64a2600b771f2b1b1b151f7d13fb2ae6c97/repeater.v\#L161-L177}{\emph{kludge}}. %Linked by Linh

\bigskip

\pause 
The hierarchical structure resembles that of \textcolor{red}{\emph{hyperoperations}}.

%To better understand the Ackermann function as a hierarchy and this kludge, we explore the closely-related hyperoperations.

\end{frame}


\begin{frame}
\frametitle{Introduction: Ackermann \emph{vs} Hyperoperation}

The hyperoperations, Ackermann levels, and their \emph{inverses}:

\begin{table}[t]
	\begin{centermath}
		\begin{array}{c@{\hskip 0.5em}|@{\hskip 1em}c@{\hskip 1em}c@{\hskip 1em}|@{\hskip 1em}c@{\hskip 1em}c}
%			  & \multicolumn{2}{|@{\hskip 0.5em}c@{\hskip 0.5em}|}{\text{Main hierarchies}} & \multicolumn{2}{|@{\hskip 0.5em}c@{\hskip 0.5em}|}{\text{Inverses hierarchies}} \\
			n & a [n] b & \Ack_n(b) & a \angle{n} b & \alpha_n(b)\\
			\hline
			0 & 1 + b & 1 + b & b - 1 & b - 1 \\
			1 & a + b & 2 + b & b - a & b - 2 \\
			2 & a \cdot b & 2b + 3 & \left\lceil \frac{b}{a} \right\rceil & \left\lceil \frac{b-3}{2} \right\rceil \\
			3 & a^b & 2^{b + 3} - 3 & \left\lceil \log_a ~ b \right\rceil & \left\lceil \log_2 ~ (b + 3)\right\rceil - 3 \\
			[2pt]
			4 & \underbrace{a^{.^{.^{.^a}}}}_b & \underbrace{2^{.^{.^{.^2}}}}_{b+3} - 3 & \log^*_a ~ b & \log^*_2 ~ (b + 3) - 3
		\end{array}
	\end{centermath}
	\label{table: hyperop-ack-inv}
\end{table}

\pause
Connection?
\pause
\\ \href{https://github.com/inv-ack/inv-ack/blob/7270e64a2600b771f2b1b1b151f7d13fb2ae6c97/repeater.v\#L161-L177}{Aha!} 
$\Ack_n(b) = {\color{red}2}[n](b{\color{red}+3}) {\color{red}- 3}$
\pause
\quad and \quad $\alpha_n(b) = {\color{red}2}\angle{n}(b{\color{red}+3}) {\color{red}- 3}$.

\end{frame}


%\begin{frame}
%\frametitle{}
%\end{frame}
%
%
%\begin{frame}
%\frametitle{}
%\end{frame}