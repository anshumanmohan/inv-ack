\begin{frame}
\frametitle{The Ackermann Function}
	
	The Ackermann-P\'eter function is defined as:
	\begin{equation*}
	A(n, m) \triangleq \begin{cases}
	m + 1 & \text{ when } n = 0 \\
	A(n-1, 1) & \text{ when } n > 0, m = 0 \\
	A\big(n-1, A(n, m-1)\big) & \text{ otherwise}
	\end{cases}
	\end{equation*}
	
	\pause 
	The \emph{diagonal} Ackermann function is $\Ack(n)~\triangleq~A(n, n)$.
	
	\bigskip
	
	\pause 
	First few values of $\Ack(n)$:
	
  \begin{minipage}{0.5\linewidth}
		\begin{equation*}
	\begin{array}{r|ccccc}
	 n & 0 & 1 & 2 & 3 & 4 \\ \hline
	 \Ack(n) & 1 & 3 & 7 & 61 & 2^{2^{2^{65536}}} - 3 \topspace{3pt}
	\end{array}
	\end{equation*}
  \end{minipage}
  \quad \pause 
  \begin{minipage}{0.4\linewidth}
  	\textcolor{red}{Astronomical growth!}
  \end{minipage}

\end{frame}


%\begin{frame}
%\frametitle{Initial values for $\Ack(n)$ and $\alpha(n)$}
%\begin{columns}[T]
%	\begin{column}{0.4\textwidth}
%		
%		\begin{equation*}
%		\begin{array}{|ll|}
%		n & \Ack(n) \\
%		0 & 1 \\
%		1 & 3 \\
%		2 & 7 \\
%		3 & 61 \\
%		4 & 2^{2^{2^{65536}}} - 3
%		\end{array}
%		\end{equation*}
%		
%		Grows astronomically fast!
%	\end{column}
%
%  \begin{column}{0.5\textwidth}
%  	
%  	\begin{equation*}
%  	\begin{array}{|ll|}
%  	n & \alpha(n) \\
%  	0, 1 & 0 \\
%  	2, 3 & 1 \\
%  	4, 5, 6, 7 & 2 \\
%  	8, 9, \ldots, 61 & 3 \\
%  	62, \ldots, 2^{2^{2^{65536}}} - 3 & 4
%  	\end{array}
%  	\end{equation*}
%  	
%  	Grows astronomically slowly!
%  \end{column}
%\end{columns}
%\end{frame}


% \begin{frame}
% \frametitle{Introduction: Growth Patterns}

%  First few values of $\Ack(n)$ and $\alpha(n)$:
		
% %		\begin{equation*}
% %		\begin{array}{|ll|c|ll|}
% %		n & \Ack(n) & \hspace{3em} & n & \alpha(n) \\[3pt]
% %		0 & 1 & & 0, 1 & 0 \\[3pt]
% %		1 & 3 & & 2, 3 & 1 \\[3pt]
% %		2 & 7 & & 4, 5, 6, 7 & 2 \\[3pt]
% %		3 & 61 & & 8, 9, \ldots, 61 & 3 \\[3pt]
% %		4 & 2^{2^{2^{65536}}} - 3 & & 62, 63 \ldots, 2^{2^{2^{65536}}} - 3 & 4 \\[5pt]
% %		\multicolumn{2}{|l|}{\text{Grows astronomically fast!}} & & \multicolumn{2}{|l|}{\text{Grows astronomically slow!}}
% %		\end{array}
% %		\end{equation*}
		
% 		\begin{tikzpicture}
% 		\begin{axis}[
% %		symbolic x coords={a small bar,a medium bar,a large bar},
%     yticklabels={,,},
% 		xtick=data]
% 		\addplot coordinates {
% 			(0, 1)
% 			(1, 3)
% 			(2, 7)
% 			(3, 61)
% 			(4, 9999999)
			
% 		};
% 		\end{axis}
% 		\end{tikzpicture}

% %TODO Fix this graph? Graph for alpha? Figure with caption?

% \end{frame}


\begin{frame}
\frametitle{The Inverse Ackermann Function}

The \emph{inverse Ackermann function}, $\alpha$, maps $n$ to the smallest~$k$ for
which~$n \le \Ack(k)$, \emph{i.e} \impinline{$\alpha(n) \triangleq \min\left\{k\in \mathbb{N} : n \le \Ack(k)\right\}$}.

\smallskip

\pause 
$\alpha(n)$ grows slowly but is hard to \emph{compute} for large $n$
\\ because it is entangled with the explosively-growing $\Ack(k)$.

\bigskip


%\textbf{Naive Approach:} starting at $k=0$, calculate $\Ack(k)$,
%compare it to $n$, \\ and increment $k$ until $n \le \Ack(k)$.
\pause 
\textbf{Naive Approach:} Compute $\Ack(0), \Ack(1), \ldots$ until $n\le \Ack(k)$. Return $k$.

\bigskip

\pause 
\textbf{Time complexity:} $\Omega(\Ack(\alpha(n)))$.
\\ Computing $\alpha(100) \mapsto^{*} 4$ requires at least
$\Ack(4) = 2^{2^{2^{65536}}}$ steps!

\bigskip

\pause 
\textbf{Engineering hack:} Hardcode with lookup tables. $n > 61 \implies \text{ans} = 4$.

\bigskip

\pause 
\imppar{\text{\textbf{Our Goal.} Compute $\alpha$ for \emph{all inputs} without computing $\Ack$.}}
\end{frame}

%\begin{frame}
%\frametitle{Introduction: Ackermann \emph{vs} Hyperoperations}
%The Ackermann function is easy to define, but hard to
%understand.
%
%We see it as
%a sequence of $n$-indexed functions $\Ack_n \triangleq \lambda b.A(n,b)$, where for each $n>0$, $\Ack_n$ is the result of applying the previous $\Ack_{n-1}$ $b$ times.
%
%%with a
%%\href{https://github.com/inv-ack/inv-ack/blob/7270e64a2600b771f2b1b1b151f7d13fb2ae6c97/repeater.v\#L161-L177}{\emph{kludge}}. %Linked by Linh
%
%\bigskip
%
%The hierarchical structure resembles that of \emph{hyperoperations}.
%
%%To better understand the Ackermann function as a hierarchy and this kludge, we explore the closely-related hyperoperations.
%
%\end{frame}


\begin{frame}[fragile]
\frametitle{Our Solution}

\vspace{-1em}
\lstset{style=myTinyStyle}
% Linked by A
\begin{mdframed}[backgroundcolor=lightgray, roundcorner=10pt,leftmargin=0, rightmargin=0, innerleftmargin=0, innertopmargin=-5,innerbottommargin=-5, outerlinewidth=0, linecolor=lightgray]
\begin{lstlisting}
Require Import Omega Program.Basics.

`\href{https://github.com/inv-ack/inv-ack/blob/7270e64a2600b771f2b1b1b151f7d13fb2ae6c97/inv_ack_standalone.v#L6-L11}{Fixpoint cdn\_wkr}` (a : nat) (f : nat -> nat) (n b : nat) :=
 match b with 0 => 0 | S b' =>
  if (n <=? a) then 0 else S (cdn_wkr f a (f n) k')
 end.

`\href{https://github.com/inv-ack/inv-ack/blob/7270e64a2600b771f2b1b1b151f7d13fb2ae6c97/inv_ack_standalone.v#L14}{Definition countdown\_to}` a f n := cdn_wkr a f n n.

`\href{https://github.com/inv-ack/inv-ack/blob/7270e64a2600b771f2b1b1b151f7d13fb2ae6c97/inv_ack_standalone.v#L32-L38} {Fixpoint inv\_ack\_wkr}` (f : nat -> nat) (n k b : nat) :=
 match b with 0 => 0 | S b' =>
  if (n <=? k) then k else let g := (countdown_to f 1) in
                      inv_ack_wkr (compose g f) (g n) (S k) b
 end.

`\href{https://github.com/inv-ack/inv-ack/blob/7270e64a2600b771f2b1b1b151f7d13fb2ae6c97/inv_ack_standalone.v#L42-L46}{Definition inv\_ack\_linear}` n :=
 match n with 0 | 1 => 0 | _ => 
  let f := (fun x => x - 2) in inv_ack_wkr f (f n) 1 (n - 1)
 end.
\end{lstlisting}
\end{mdframed} 
\end{frame}


\begin{frame}
\frametitle{Ackermann \emph{vs} Hyperoperation}

The Ackermann function is easy to define, but hard to
understand.

\bigskip

\pause 	
Let's index by the first argument. \\ \smallskip
Define $\Ack_n \triangleq \lambda b.A(n,b)$. \\ \smallskip
Then, for $n>0$, $\Ack_n$ is the result of applying the previous $\Ack_{n-1}$ $b$ times.

%with a
%\href{https://github.com/inv-ack/inv-ack/blob/7270e64a2600b771f2b1b1b151f7d13fb2ae6c97/repeater.v\#L161-L177}{\emph{kludge}}. %Linked by Linh

\bigskip

\pause 
The hierarchical structure resembles that of \textcolor{red}{\emph{hyperoperations}}.

%To better understand the Ackermann function as a hierarchy and this kludge, we explore the closely-related hyperoperations.
\end{frame}


\begin{frame}
\frametitle{Introduction: Ackermann \emph{vs} Hyperoperation}

The hyperoperations, Ackermann levels, and their \emph{inverses}:

\begin{table}[t]
	\begin{centermath}
		\begin{array}{c@{\hskip 0.5em}|@{\hskip 1em}c@{\hskip 1em}c@{\hskip 1em}|@{\hskip 1em}c@{\hskip 1em}c}
%			  & \multicolumn{2}{|@{\hskip 0.5em}c@{\hskip 0.5em}|}{\text{Main hierarchies}} & \multicolumn{2}{|@{\hskip 0.5em}c@{\hskip 0.5em}|}{\text{Inverses hierarchies}} \\
			n & a [n] b & \Ack_n(b) & a \angle{n} b & \alpha_n(b)\\
			\hline
			0 & 1 + b & 1 + b & b - 1 & b - 1 \\
			1 & a + b & 2 + b & b - a & b - 2 \\
			2 & a \cdot b & 2b + 3 & \left\lceil \frac{b}{a} \right\rceil & \left\lceil \frac{b-3}{2} \right\rceil \\
			3 & a^b & 2^{b + 3} - 3 & \left\lceil \log_a ~ b \right\rceil & \left\lceil \log_2 ~ (b + 3)\right\rceil - 3 \\
			[2pt]
			4 & \underbrace{a^{.^{.^{.^a}}}}_b & \underbrace{2^{.^{.^{.^2}}}}_{b+3} - 3 & \log^*_a ~ b & \log^*_2 ~ (b + 3) - 3
		\end{array}
	\end{centermath}
	\label{table: hyperop-ack-inv}
\end{table}

\pause
Connection?
\pause
\\ \href{https://github.com/inv-ack/inv-ack/blob/7270e64a2600b771f2b1b1b151f7d13fb2ae6c97/repeater.v\#L161-L177}{Aha!} 
$\Ack_n(b) = {\color{red}2}[n](b{\color{red}+3}) {\color{red}- 3}$
\pause
\quad and \quad $\alpha_n(b) = {\color{red}2}\angle{n}(b{\color{red}+3}) {\color{red}- 3}$.

\end{frame}


%\begin{frame}
%\frametitle{}
%\end{frame}
%
%
%\begin{frame}
%\frametitle{}
%\end{frame}