\subsection*{The Ackermann and Inverse Ackermann functions}
\begin{frame}
\frametitle{The Ackermann and Inverse Ackermann functions}
\label{defn: ack}
	The Ackermann-P\'eter function is defined as follows:
	\begin{equation}
	\label{eq:ackermann}
	A(n, m) \triangleq \begin{cases}
	m + 1 & \text{ when } n = 0 \\
	A(n-1, 1) & \text{ when } n > 0, m = 0 \\
	A\big(n-1, A(n, m-1)\big) & \text{ otherwise}
	\end{cases}
	%A(m, n) \triangleq \begin{cases}
	%n + 1 & \text{ if } m = 0 \\
	%A(m-1, 1) & \text{ if } m > 0, n = 0 \\
	%A(m-1, A(m, n-1)) & \text{ if } m > 0, n > 0
	%\end{cases}
	\end{equation}
	The one-variable \emph{diagonal} Ackermann function is $\Ack(n)~\triangleq~A(n, n)$.\\[5pt]
	
	Its \emph{inverse}, $\alpha(n)$, is the smallest~$k$ for
	which~$n \le \Ack(k)$, \emph{i.e}:
	\begin{equation*}
	\alpha(n) \triangleq \min\left\{k\in \mathbb{N} : n \le \Ack(k)\right\}
	\end{equation*}
\end{frame}


\begin{frame}
\frametitle{Initial values for $\Ack(n)$ and $\alpha(n)$}
\begin{columns}[T]
	\begin{column}{0.5\textwidth}
		TODO: Value table for $\Ack(n)$
		
		Grows astronomically fast!
	\end{column}

  \begin{column}{0.5\textwidth}
  	TODO: Value table for $\alpha(n)$
  	
  	Grows astronomically slowlys!
  \end{column}
\end{columns}
\end{frame}


\begin{frame}
\frametitle{Computing $\alpha(n)$}
Despite growing extremely slowly, $\alpha(n)$ is hard to compute for large $n$ due to the explosive growth of $\Ack(k)$.

\bigskip

\textbf{The Naive Approach:} starting at $k=0$, calculate $\Ack(k)$,
compare~it~to~$n$, and increment $k$ until $n \le \Ack(k)$.

\bigskip

\textbf{Time complexity:} $\Omega(\Ack(\alpha(n)))$,
so \emph{e.g.} computing $\alpha(100) \mapsto^{*} 4$ in this way requires
$\Ack(4) = 2^{2^{2^{65536}}}-3$ steps!
\end{frame}

\subsection*{The hyperoperation/Ackermann hierarchy}

\begin{frame}
\frametitle{The hierarchy of Ackermann levels}
The Ackermann function is easy to define, but hard to
understand.

We see it as
a sequence of $n$-indexed functions $\Ack_n \triangleq \lambda b.A(n,b)$, where for each $n>0$, $\Ack_n$ is the result of applying the previous $\Ack_{n-1}$ $b$ times,

with a
\href{https://github.com/inv-ack/inv-ack/blob/7270e64a2600b771f2b1b1b151f7d13fb2ae6c97/repeater.v\#L161-L177}{\emph{kludge}}. %Linked by Linh

\bigskip

To better understand the Ackermann function as a hierarchy and this kludge, we explore the closely-related hyperoperations.

\end{frame}


\begin{frame}
\frametitle{The Ackermann hierarchy and hyperoperations}
TODO: Polish, add names of levels to this table without overflowing the page

\begin{table}[t]
	\begin{centermath}
		\begin{array}{c@{\hskip 0.5em}|@{\hskip 1em}c@{\hskip 1em}c@{\hskip 1em}|@{\hskip 1em}c@{\hskip 1em}c}
			n & a [n] b & \Ack_n(b) & a \angle{n} b & \alpha_n(b)\\
			\hline
			0 & 1 + b & 1 + b & b - 1 & b - 1 \\
			1 & a + b & 2 + b & b - a & b - 2 \\
			2 & a \cdot b & 2b + 3 & \left\lceil \frac{b}{a} \right\rceil & \left\lceil \frac{b-3}{2} \right\rceil \\
			3 & a^b & 2^{b + 3} - 3 & \left\lceil \log_a ~ b \right\rceil & \left\lceil \log_2 ~ (b + 3)\right\rceil - 3 \\
			[2pt]
			4 & \underbrace{a^{.^{.^{.^a}}}}_b & \underbrace{2^{.^{.^{.^2}}}}_{b+3} - 3 & \log^*_a ~ b & \log^*_2 ~ (b + 3) - 3
		\end{array}
	\end{centermath}
	\label{table: hyperop-ack-inv}
\end{table}

The kludge: $\Ack_n(b) = 2[n](b+3) - 3$ and $\alpha_n(b) = 2\angle{n}(b+3) - 3$.

\end{frame}


%\begin{frame}
%\frametitle{}
%\end{frame}
%
%
%\begin{frame}
%\frametitle{}
%\end{frame}