
\subsection{Overview}

\begin{frame}
\frametitle{Review: Binary Encoding System}

\begin{tabular}{l|c|c}
	Attribute & Unary & Binary Encoding \\ \hline
	Type in Coq & \texttt{nat} & \texttt{N} \topspace{1pt} \\[6pt]
	Number $n$ & $\Theta(n)$ bits & $\Theta\big(\log_2n\big)$ bits \\[7pt]
	Compare $x, y$ & $\Theta\big(\min\{x, y\}\big)$ time & $\Theta\big(\min\{ \log_2x, \log_2y\}\big)$ time \\[9pt]
	Successor & \multirow{2}{*}{$\Theta(1)$ time} & Worse case: $\Theta\big(\log_2x\big)$ time \\[1pt]
	function  &             & Amortized: $\Theta(1)$ time \\
	                    
\end{tabular}

\bigskip

Binary encoding provides faster computations for many operations.

\smallskip

\pause 
What about inverse Ackermann? \imppar{\textbf{Goal:} $\bigO \big(\log_2n\big)$}.

\end{frame}


\subsection{Countdown in Binary}


\begin{frame}[fragile]
\frametitle{Translating Countdown}

\textbf{The worker.}
\begin{lstlisting}
`\href{https://github.com/inv-ack/inv-ack/blob/7270e64a2600b771f2b1b1b151f7d13fb2ae6c97/bin_countdown.v#L104-L109}{Fixpoint bin\_cdn\_wkr}` (f : N -> N) (a n : N) (b : nat) : N :=
  match b with O => 0 | S b' =>
    if (n <=? a) then 0 else 1 + bin_cdn_wkr f a (f n) b'
  end.
\end{lstlisting}

\smallskip

\pause 
\textbf{Observation.} Recursive argument is budget $b$, \\which should still be in \texttt{nat}.

\bigskip

\textbf{Issue.} $b$ cannot be $n$ due to slow binary $\rightarrow$ unary conversion.

\bigskip

\pause 
\textbf{Solution.} Use $b \approx \log_2n$ and contractions that contract ``fast enough''. 
\emph{i.e.} Functions that halve their inputs.
\end{frame}



\begin{frame}[fragile]
\frametitle{Binary Contractions and Countdown}

\textbf{Binary contractions.} $f$ is a binary contraction strict above $a$ if $\forall n, f(n)\le n$ and $f(n)\le \lfloor \frac{n+ a}{2} \rfloor$ when $n > a$.

\smallskip

$f$ shrinks $n$ past $a$ within $\lfloor\log_2(n - a) \rfloor + 1$ steps.

\bigskip

\pause 
New Countdown computation:
% Linked by A
\begin{lstlisting}
`\href{https://github.com/inv-ack/inv-ack/blob/7270e64a2600b771f2b1b1b151f7d13fb2ae6c97/bin_countdown.v#L111-L112}{Definition bin\_countdown\_to}` (f : N -> N) (a n : N) : N :=
bin_cdn_wkr f a n (nat_size (n - a)).
\end{lstlisting}

where \texttt{nat\_size x} computes $\lfloor \log_2x \rfloor + 1$ as a \texttt{nat}.

\bigskip

\pause 
Applicable for all $\alpha_i$ for $i \ge 3$.
 
\end{frame}


\subsection{Inverse Ackermann in Binary}

\begin{frame}[fragile]
\frametitle{Translating the $\alpha$ Hierarchy}

Must start with a strict binary contraction.

\smallskip

More hard-coding to skip those that are not fast enough.

\bigskip

\pause 
\begin{lstlisting}
`\href{https://github.com/inv-ack/inv-ack/blob/7270e64a2600b771f2b1b1b151f7d13fb2ae6c97/bin_inv_ack.v#L55-L60}{Fixpoint bin\_alpha}` (m : nat) (x : N) : N :=
match m with
  | 0%nat => x - 1          
  | 1%nat => x - 2
  | 2%nat => N.div2 (x - 2) 
  | 3%nat => N.log2 (x + 2) - 2
  | S m'  => bin_countdown_to (bin_alpha m') 1 (bin_alpha m' x)
end.
\end{lstlisting}

\end{frame}


\begin{frame}[fragile]
\frametitle{Translating the Inverse Ackermann Function}

Worker function:

\begin{lstlisting}
`\href{https://github.com/inv-ack/inv-ack/blob/7270e64a2600b771f2b1b1b151f7d13fb2ae6c97/bin_inv_ack.v#L326-L333}{Fixpoint bin\_inv\_ack\_wkr}` (f : N -> N) (n k : N) (b : nat) : N :=
match b with 0%nat  => k | S b' => if n <=? k then k else
  let g := (bin_countdown_to f 1) in
    bin_inv_ack_wkr (compose g f) (g n) (N.succ k) b'
  end.
\end{lstlisting}

\pause 
Same idea: use logarithmic size budget. More hard-coding.

\pause 
\begin{lstlisting}
`\href{https://github.com/inv-ack/inv-ack/blob/7270e64a2600b771f2b1b1b151f7d13fb2ae6c97/bin_inv_ack.v#L335-L342}{Definition bin\_inv\_ack}` (n : N) : N :=
if      (n <=? 1) then 0
else if (n <=? 3) then 1
else if (n <=? 7) then 2
else let f := (fun x => N.log2 (x + 2) - 2)
       in bin_inv_ack_wkr f (f n) 3 (nat_size n).
\end{lstlisting}

\end{frame}


\begin{frame}
\frametitle{Time Complexity of Binary $\alpha$: A Sketch}

Similar to $\alpha$ on unary inputs, $\runtime_{\alpha}(n) \approx \runtime_{\alpha_k}(n)$ for $k\triangleq \alpha(n)$.
\pause
\begin{equation*}
\begin{aligned}
& \runtime_{\alpha_{i+1}}(n) \ \lessapprox \ 4n + \bigO\big(\log_2n\big) + 4\log_2n + \bigO\big(\log_2^{(2)}n\big) + \ldots \\
& \quad = 4n + \bigO\left(\log_2n + \log_2^{(2)}n + \log_2^{(3)}n + \ldots\right)
= 4n + \bigO\big(\log_2n\big)
\end{aligned}
\end{equation*}

\smallskip

\pause
Manual computation: $\runtime_{\alpha_3}(n)\le 2\log_2n + \log_2\log_2n + 3$.

\smallskip

\pause
Suppose $\runtime_{\alpha_i}(n) \le 2\log_2n + \bigO\left(\log_2^{(2)}n\right)$. \pause Then
\begin{equation*}
\begin{aligned}
& \runtime_{\alpha_{i+1}}(n) \ \lessapprox \ 2\log_2n + \bigO\left(\log_2^{(2)}n\right) + 2\log_2^{(2)}n + \bigO\big(\log_2^{(3)}n\big) + \ldots \\
& \ = 2\log_2n + \bigO\left(\log_2^{(2)}n + \log_2^{(3)}n + \log_2^{(4)}n + \ldots\right)
= 2\log_2n + \bigO\left(\log_2^{(2)}n\right)
\end{aligned}
\end{equation*}
\pause
By induction, $\runtime_{\alpha_i}(n) \lessapprox 2\log_2n + \bigO\left(\log_2^{(2)}n\right)$ for each $i$, and so is $\runtime_{\alpha}(n)$.

\end{frame}